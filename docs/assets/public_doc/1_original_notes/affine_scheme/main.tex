\documentclass{article}


\usepackage{../../../latex_styles/color_template}
\usepackage{../../../latex_styles/symbol_def}


\title{
	Some words on affine schemes
}
\author{Nguyen Ngoc Khanh}

\begin{document}
\maketitle

\section{AFFINE SCHEME}

\begin{definition}[affine scheme]
	An affine scheme $(X, \mc{O})$ is a locally ringed space that is isomorphic to the spectrum of a ring
\end{definition}

Given any ring $A$, the spectrum of $A$ as a set is the collection of all primes of $A$ and denoted by $\Spec A$. Endow $X = \Spec A$ with the a topology where every closed set is defined by the collection of primes containing some ideal $I$
$$
	V(I) = \set{\mf{p} \in \Spec A: \mf{p} \supseteq I}
$$

The topology is called Zariski topology. For every $f \in A$, the set 
$$
	X_f = \Spec A - V(f) = \set{\mf{p} \in \Spec A: \mf{p} \not\ni f}
$$

of all primes not containing $f$ is an open set. Turn out, $\set{X_f: f \in A}$ generates the Zariski topology. Moreover, there is a sheaf $\mc{O}$ on $X$ that makes it a locally ringed space (a sheaf of rings that every stalk is a local ring) defined on each basic open set $X_f$ by localization of $A$ at $f$
$$
	\mc{O}(X_f) = A_f 
$$

and $\mc{O}(X) = A$. Given any $f, g \in A$ so that $X_f \supseteq X_g$, that is $g \in \sqrt{(f)}$ ($g^n = fh$ for some $h \in A$ and $n \geq 1$), $f$ is a unit in $A_g$, universal property gives a unique restriction map $\mc{O}(X_f) \to \mc{O}(X_g)$
\begin{align*}
	A_f &\to A_g \\
	\frac{x}{f^m} &\mapsto x \tuple*{\frac{h}{g^n}}^m
\end{align*}

These make $(\Spec A, \mc{O})$ a locally ringed space.


\textbf{Examples:}

Let $k$ be a field

\begin{itemize}
	\item $\Spec k = \set{(0)}$ is a singleton set
	
	\item $\Spec \Z = \set{(0), (2), (3), (5), ...}$
	
	\item $\Spec \Z / 6\Z = \set{(2), (3)}$
	
	\item $k[x, y]$ is the ring of polynomials defined on $xy$ plane. By Nullstellensatz, the maximal ideals in $k[x, y]$ are precisely
	$$
		(x - a, y - b)
	$$
	for all $a, b \in k$. Those maximal ideals are called \textbf{closed points} and correspond to points on $xy$ plane. The other prime ideals are called \textbf{generic points} and correspond to other geometric objects. Let $\mf{p} = (f(x, y))$ be a prime ideal in $k[x, y]$, then the closure of $\mf{p}$ consists of $\mf{p}$ itself and all closed points that are the zeros of $f(x, y) = 0$. In particular, $(0)$ corresponds to the "classical plane" $xy$, $(x^2 + y^2 - 1), (y^2 - x^3)$ correspond to some "classical lines" on the $xy$ plane. Moreover, the height of prime ideals $(0), (x^2 + y^2 - 1), (y^2 - x^3)$ is the codimension of the corresponding geometric shapes: $\hgt (0) = 0, \hgt (y^2 - x^3) = 1, \hgt (x - 1, y + 2) = 2$. The dimension of the whole space is defined as its Krull dimension which is $2$ in this case.
	
	\item $k[x, y] / (xy)$ is the ring of polynomials defining on $xy = 0$. The geometric picture consists of two lines cross each other. The spectrum is more or less the same with the case of $k[x, y]$ and its dimension is still $2$.
	
	\item $k[x, y] / (x^2 + y^2 - 1)$ is the ring of polynomials defining on $x^2 + y^2 - 1$ which is the unit circle. Since $(x^2 + y^2 - 1)$ is of height $1$, the Krull dimension $k[x, y] / (x^2 + y^2 - 1)$ is $1$ that coincides with the geometric dimension.


	\item $k[\epsilon] / (\epsilon^2) = \set{a + b \epsilon: a, b \in k}$ is the ring of dual numbers. $\epsilon$ is nilpotent, hence belongs to all prime ideals. $(\epsilon)$ is the unique maximal ideal of $k[\epsilon] / (\epsilon^2)$. 
	\note{TODO - apparently, this have something to do with $\Hom(\Spec k[\epsilon] / (\epsilon^2) ,  X) \cong T_x X$ or function that is everywhere zero but not zero}
	
	\note{TODO - other cases below}
	
	\item $k[x, y] / \mf{p}$ for arbitrary prime $\mf{p}$
	
	\item $k[x] \times k[y]$ is the product of two polynomial rings. \note{TODO}

	\item $k[x]$ when k is not alg closed
	
	\item $k[x]_{(x)}$ - all rational functions that is defined around zero
	
	\item $A / B$ for any quotient of rings is just subspace
	
	\item $k[x]$ is always infinitely many irreducibles



\begin{remark}
	In the affine scheme $(X, \mc{O}) \cong \Spec A$ defined as above, elements of $A$ are called \textbf{functions}, primes of $A$ are called \textbf{points}, sending function $f \in A$ into $f(\mf{p}) \in \mc{O}(X_\mf{p}) = A_\mf{p}$ by the canonical map $A \to A_\mf{p}$ is function evaluation.
\end{remark}


\section{SCHEME}

\begin{definition}[scheme]
	A scheme $(X, \mc{O})$ is a locally ringed space such that each point $x \in X$ admits a local neighbourhood $(U, \mc{O}\vert_U)$ being an affine scheme.
\end{definition}


\end{itemize}












\end{document}
