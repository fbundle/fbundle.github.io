\chapter{AFFINE SCHEME}

\section{DEFINITION OF AFFINE SCHEME}

\begin{definition}[ringed space, locally ringed space]
	A ringed space $(X, \ms{O})$ is a topological space $X$ together with a sheaf of rings $\ms{O}$. A ringed space is a locally ringed space if for every point $x \in X$, the stalk $\ms{O}_x$ is a local ring.
\end{definition}

\begin{definition}[affine scheme]
	An affine scheme is a locally ringed space that is isomorphic to the spectrum of some ring $A$
\end{definition}

\section{CONSTRUCT AFFINE SCHEME FROM RING}

Let $A$ be a ring, let $X = \Spec A$ be the set of prime ideals in $A$
$$
	\Spec A = \set{\text{prime } \mf{p} \subseteq A}
$$

We put a topology on $\Spec A$ generated by the basis of open sets
$$
	D(f) = \set{\mf{p} \in \Spec A: f \notin \mf{p}}
$$

for every $f \in A$. The constructed topology is called Zariski's topology and the open subsets of the form $D(f)$ for some $f \in A$ are called \textbf{principal open subsets}. We put a sheaf of rings $\ms{O}$ on $\Spec A$ generated by
$$
	\ms{O}(D(f)) = A_f
$$

If $D(g) \subseteq D(f)$, that is $g \in \sqrt{(f)}$, $g^n = fh$ for some $h \in A$ and $n \geq 1$. Since $f$ is a unit in $A_g$, the restriction map is well-defined and unique
\begin{align*}
	A_f &\to A_g \\
	\frac{x}{f^m} &\mapsto x \tuple*{\frac{h}{g^n}}^m
\end{align*}

Under this construction, $(X, \ms{O})$ is an affine scheme. An element $f \in A$ is called \textbf{function}, a element $x \in X$ is called \textbf{point}, when refering $x$ as a prime ideal in $A$, we write $\mf{p}_x \subseteq A$, and function evaluation is equivalent to sending $f$ to the residue field of stalk $\ms{O}_X = A_{\mf{p}_x}$.

When $A$ is a polynomial ring of $n$ variables over ring $R$, we write $\A^n_R = \Spec A$

\section{SOME EXAMPLES OF AFFINE SCHEME}

\textbf{example:} (integers) Let $A = \Z$, then
$$
	X = \Spec \Z = \set{(2), (3), (5), ..., (0)}
$$

for each prime number $p \in \Z$, $(p)$ is a closed point. $(0)$ is a generic point, and closure of $(0)$ is the whole space $X$. 

\textbf{example:} (field) Let $k$ be a field. Let $A = k$, then 
$$
	X = \Spec k = \set{(0)}
$$

is a singleton set.

\textbf{example:} (polynomial ring of two variables over an algebraically closed field) Let $A$ be an arbitrary ring, maximal ideals are closed points and other non-maximal primes are generic points. Moreover, the closure of a point $\mf{p}$ is the set of prime ideals containing $\mf{p}$
$$
	V(\mf{p}) = \set{\mf{q} \in \Spec A: \mf{p} \subseteq \mf{q}}
$$

In particular, let $k$ be an algebraically closed field, let $A = k[x, y]$, by Nullstellensatz, the set of closed points are
$$
	\set{(x-a, y-b): a \in k, b \in k}
$$

$X$ admits other generic points $(0)$ and $(f)$ for every irreducible $f \in k[x, y]$. By dimensionality argument, the prime $(f)$ is of height $1$ and the closure of $(f)$ consists of $(f)$ and $(x-a, y-b)$ for $(a, b) \in k^2$ in the vanishing set of $f$. Similarly, closure of $(0)$ is the whole space.

\section{MORPHISM OF AFFINE SCHEMES}

\begin{remark}[inverse image, direct image]
	Let $\psi: X \to Y$ be a continuous map. The inverse image functor $\psi^{-1}$ and direct image functor $\psi_*$ is an adjoint pair between the category of sheaves on $X$ and the category of sheaves on $Y$. Let $\ms{F}$ and $\ms{G}$ be a sheaf on $X$ and a sheaf on $Y$ respectively, then
	$$
		\Hom_{\Sh(X)}(\psi^{-1} \ms{G}, \ms{F}) \cong \Hom_{\Sh(Y)}(\ms{G}, \psi_* \ms{F})
	$$
	
	The direct image functor $\psi_*$ is defined as follows: for every open subset $V \subseteq Y$, then
	$$
		(\psi_* \ms{F})(V) = \ms{F}(\psi^{-1}(U))
	$$
	
	The inverse image functor $\psi^{-1}$ is defined as follows: for every open subset $U \subseteq X$, then
	$$
		(\psi^{-1} \ms{G})(U) = \colim_{V \subseteq Y: \psi(U) \subseteq V} \ms{G}(V)
	$$
\end{remark}

\begin{definition}[morphism of ringed spaces]
	A morphism of ringed spaces $(X, \ms{O}_X) \to (Y, \ms{O}_Y)$ is defined by a continuous map $\psi: X \to Y$ and a morphism of sheaves of rings $\psi^\flat: \ms{O}_Y \to \psi_* \ms{O}_X$. By adjunction between $\psi^{-1}$ and $\psi_*$, this is equivalent to a morphism of sheaves of rings $\psi^\#: \psi^{-1} \ms{O}_Y \to \ms{O}_X$. Moreover, $\psi^\#$ induces a map on stalks
	$$
		\psi^\#_x: (\psi^{-1} O_Y)_x = \ms{O}_{Y, \psi(x)} \to \ms{O}_{X, x}
	$$
	
\end{definition}

\begin{remark}[compatible germs]
	Given a sheaf of sets $\ms{F}$ on $X$, for every open subset $U \subseteq X$, the natural map from sections into product of stalks
	$$
		\ms{F}(U) \hookrightarrow \prod_{x \in U} \ms{F}_x
	$$
	is injective. An element $(s_x)_{x \in U} \subseteq \prod_{x \in U} \ms{F}_x$ in the image if this map is called compatible germs.
\end{remark}

From the equivalence between set of sections and set of compatible germs, one can identify a morphism of ringed spaces by its topological space map $\psi: X \to Y$ and its map on stalks $\psi^\#_x: (\psi^{-1} O_Y)_x = \ms{O}_{Y, \psi(x)} \to \ms{O}_{X, x}$.

\begin{definition}[morphism of locally ringed spaces, morphism of affine schemes]
	A morphism of ringed spaces $(\psi, \psi^\#): (X, \ms{O}_X) \to (Y, \ms{O}_Y)$ is a morphism of locally ringed spaces if for every $x \in X$, the induced map on stalks $\psi^\#_x: (\psi^{-1} O_Y)_x = \ms{O}_{Y, \psi(x)} \to \ms{O}_{X, x}$ is a local homomorphism. A morphism of affine schemes is a morphism of locally ringed spaces.
\end{definition}

We obtain the category of affine schemes, denoted by $\AffSch$. The local homomorphism condition of the map between stalks ensures that any zero function $g \in \ms{O}_{Y, \psi(x)}$ at $\psi(x) \in Y$ will be sent to a zero function $\psi^\#_x (g) \in \ms{O}_{X, x}$ at $x \in X$

\begin{proposition}[equivalence between commutative rings and affine schemes]
	The functor $\Spec$ from the opposite category of commutative rings into the category of affine schemes is fully faithful and essentially surjective.
	$$
		\Spec: \CRing^{\op} \xrightarrow{\sim} \AffSch
	$$
\end{proposition}

\section{CONSTRUCT MORPHISM OF AFFINE SCHEMES FROM MAP OF RINGS}


Let $\phi: A \to B$ be a map of rings, let $X = \Spec B$ and $Y = \Spec A$ be the corresponding affine schemes. Then, the corresponding morphism of locally ringed space $(\psi, \psi^\flat): X \to Y$ is defined as follows: for every $x \in X$
\begin{align*}
	\psi: X &\to Y \\
			x &\mapsto \phi^{-1}(x)
\end{align*}

The local homomorphism $\psi^\#_x: \ms{O}_{Y, \psi(x)} = A_{\psi(x)} \to B_{x}$ is induced from $\phi: A \to B$
\begin{center}
	\begin{tikzcd}
		A \arrow[r, "\phi"] \arrow[d]      & B \arrow[d] \\
		A_{\psi(x)} \arrow[r, "\psi^\#_x"] & B_x        
	\end{tikzcd}
\end{center}


\section{SOME EXAMPLES OF MORPHISM OF AFFINE SCHEMES}


\textbf{example:} (quotient) Let $A$ be a ring and $\mf{a} \subseteq A$ be an ideal. The map $A \twoheadrightarrow A / \mf{a}$ induces an injective map of sets
$$
	\Spec A / \mf{a} \hookrightarrow \Spec A
$$

As sets, we have $\Spec A / \sqrt{\mf{a}} = \Spec A / \mf{a} = \Spec A / \mf{a}^n$ for every $n \geq 1$. However, as affine schemes, in general, they are different, $A / \sqrt{\mf{a}}$ is reduced, i.e. has no nilpotent while $\Spec A / \mf{a}$ and $\Spec A / \mf{a}^n$ might have nilpotents.

Furthermore, when $\mf{a} = \mf{p}_x = x$ is prime, quotient by $\mf{p}_x$ is the action of taking closed subscheme \footnote{meaning will be revealed later} of $X$

\begin{quote}
	\textit{keep all primes containing $\mf{p}_x$. equivalently, keep all points contained in $x$}
\end{quote}


\textbf{example:} (localization) Let $A$ be a ring and $S \subseteq A$ be a multiplicatively closed subset. Localization at $S$ induces an injective map of sets
$$
	\Spec S^{-1} A \hookrightarrow \Spec A
$$

Furthermore, when $A - S = \mf{p}_x = x$ is prime, localization at $\mf{p}_x$ is the action of taking open subscheme \footnote{meaning will be revealed later} of $X$

\begin{quote}
	\textit{keep all primes contained in $\mf{p}_x$. equivalently, keep all points containing $x$}
\end{quote}

\textbf{example:} (disjoint union) Let $A, B$ be rings, then the projection $A \times B \to A$ corresponds to the monomorphism of affine schemes
$$
	\Spec A \hookrightarrow \Spec A \amalg \Spec B \cong \Spec A \times B
$$

Categorically, $A \times B$ is product in rings and $\Spec A \amalg \Spec B$ is coproduct in affine schemes

\textbf{example:} (non-trivial fiber) Consider the ring map $\phi: k[u] \to k[x, y]$ defined by $u \mapsto y^2 - x$. It induces a morphism of affine schemes

\begin{align*}
	\psi: \Spec k[x, y] &\to \Spec k[u] \\
	(0) &\mapsto (0) \\
	(x - a, y - b) &\mapsto (u - (b^2 - a)) \\
	(y^2 - x) &\mapsto (u) \\
	(f) &\mapsto (0)
\end{align*}

for every $a, b \in k$ and $f$ is irreducible other than multiple of $y^2 - x$. \note{(check this carefully)}. 
