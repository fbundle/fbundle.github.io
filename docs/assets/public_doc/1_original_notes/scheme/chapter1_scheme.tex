\chapter{SCHEME}

\section{AFFINE SCHEME}

\subsection{DEFINITION OF AFFINE SCHEME}

\begin{definition}[ringed space, locally ringed space]
	A ringed space $(X, \ms{O})$ is a topological space $X$ together with a sheaf of rings $\ms{O}$. A ringed space is a locally ringed space if for every point $x \in X$, the stalk $\ms{O}_x$ is a local ring.
\end{definition}

\begin{definition}[affine scheme]
	An affine scheme is a locally ringed space that is isomorphic to the spectrum of some ring $A$
\end{definition}

\subsection{CONSTRUCT AFFINE SCHEME FROM A RING}

Let $A$ be a ring. Define $X = \Spec A$ by the set of prime ideals in $A$
$$
	X = \Spec A = \set{\text{prime } \mf{p} \subseteq A}
$$

We put a topology on $\Spec A$ generated by the basis of open sets
$$
	D_f = \set{\mf{p} \in \Spec A: f \notin \mf{p}}
$$

for every $f \in A$. The constructed topology is called Zariski's topology. We put a sheaf of rings $\ms{O}$ on $\Spec A$ generated by
$$
	\ms{O}(D_f) = A_f
$$

If $D_g \subseteq D_f$, that is $g \in \sqrt{(f)}$, $g^n = fh$ for some $h \in A$ and $n \geq 1$. Since $f$ is a unit in $A_g$, the restriction map is well-defined and unique
\begin{align*}
	A_f &\to A_g \\
	\frac{x}{f^m} &\mapsto x \tuple*{\frac{h}{g^n}}^m
\end{align*}

An element $f \in A$ is called \textbf{function}, a prime ideal $\mf{p} \in \Spec A$ is called \textbf{point}, and function evaluation is equivalent to sending $f$ to the residue field of stalk $\ms{O}_\mf{p} = A_\mf{p}$

\subsection{SOME EXAMPLES OF AFFINE SCHEME}

Let $A = \Z$, then
$$
	X = \Spec \Z = \set{(2), (3), (5), ..., (0)}
$$

for each prime number $p \in \Z$, $(p)$ is a closed point. $(0)$ is a generic point, and closure of $(0)$ is the whole space $X$. 

Let $k$ be an algebraically closed field. Let $A = k$, then 
$$
	X = \Spec k = \set{(0)}
$$

is a singleton set.

Let $A$ be an arbitrary ring, maximal ideals are closed points and other non-maximal primes are generic points. Moreover, the closure of a point $\mf{p}$ is the set of prime ideals containing $\mf{p}$
$$
	V(\mf{p}) = \set{\mf{q} \in \Spec A: \mf{p} \subseteq \mf{q}}
$$

In particular, let $A = k[x, y]$, by Nullstellensatz, the set of closed points are
$$
	\set{(x-a, y-b): a \in k, b \in k}
$$

$X$ admits other generic points $(0)$ and $(f)$ for every irreducible $f \in k[x, y]$. By dimensionality argument, the prime $(f)$ is of height $1$ and the closure of $(f)$ consists of $(f)$ and $(x-a, y-b)$ for $(a, b) \in k^2$ in the vanishing set of $f$. Similarly, closure of $(0)$ is the whole space.

\subsection{QUOTIENT AND LOCALIZATION}

Let $A$ be a ring and $\mf{p} \in \Spec A$

Recall the map $A \twoheadrightarrow A / \mf{p}$, it induces an injective map 
$$
	\Spec A / \mf{p} \hookrightarrow \Spec A
$$

Informally, quotient by $\mf{p}$ is the action of taking closed subscheme

\begin{quote}
	\textit{keep all (geometrically inside points = algebraically outside primes containing $\mf{p}$)}
\end{quote}

Recall the map $A \to A_\mf{p}$, it induces an injective map 
$$
	\Spec A_\mf{p} \hookrightarrow \Spec A
$$

Informally, localizing at $\mf{p}$ is the action of taking quotient

\begin{quote}
	\textit{keep all (geometrically outside points = algebraically inside primes contained in $\mf{p}$)}
\end{quote}

\subsection{MORPHISM OF AFFINE SCHEMES}

\begin{definition}[inverse image, direct image]
	Let $\pi: X \to Y$ be a continuous map. The inverse image functor $\pi^{-1}$ and direct image functor $\pi_*$ is an adjoint pair between the category of sheaves on $X$ and the category of sheaves on $Y$. Let $\ms{F}$ and $\ms{G}$ be a sheaf on $X$ and a sheaf on $Y$ respectively, then
	$$
		\hom_{\Sh(X)}(\pi^{-1} \ms{G}, \ms{F}) \cong \hom_{\Sh(Y)}(\ms{G}, \pi_* \ms{F})
	$$
	
	The direct image functor $\pi_*$ is defined as follows: for every open subset $V \subseteq Y$, then
	$$
		(\pi_* \ms{F})(V) = \ms{F}(\pi^{-1}(U))
	$$
	
	The inverse image functor $\pi^{-1}$ is defined as follows: for every open subset $U \subseteq X$, then
	$$
		(\pi^{-1} \ms{G})(U) = \colim_{V \subseteq Y: \pi(U) \subseteq V} \ms{G}(V)
	$$
\end{definition}

\begin{definition}[morphism of ringed spaces]
	A morphism of ringed spaces $(X, \ms{O}_X) \to (Y, \ms{O}_Y)$ is defined by a continuous map $\pi: X \to Y$ and a morphism of sheaves of rings $\pi^\flat: \ms{O}_Y \to \pi_* \ms{O}_X$. By adjunction between $\pi^{-1}$ and $\pi_*$, this is equivalent to a morphism of sheaves of rings $\pi^\#: \pi^{-1} \ms{O}_Y \to \ms{O}_X$
\end{definition}

\begin{definition}[morphism of locally ringed spaces, morphism of affine schemes]
	A morphism of ringed spaces $(\pi, \pi^\#): (X, \ms{O}_X) \to (Y, \ms{O}_Y)$ is a morphism of locally ringed spaces if for every $x \in X$, the induced map on stalks 
	$$
		\pi^\#_x: (\pi^{-1} O_Y)_x = \ms{O}_{Y, \pi(x)} \to \ms{O}_{X, x}
	$$
	is a local homomorphism. A morphism of affine schemes is a morphism of locally ringed spaces.
\end{definition}

The local homomorphism condition of the map between stalks ensures that any zero function $g \in \ms{O}_{Y, \pi(x)}$ at $\pi(x) \in Y$ will be sent to a zero function $\pi^\#_x (g) \in \ms{O}_{X, x}$ at $x \in X$


\begin{proposition}[equivalence between commutative rings and affine schemes]
	The functor $\Spec$ from the opposite category of commutative rings into the category of affine schemes is fully faithful and essentially surjective.
	$$
		\Spec: \CRing^{\op} \xrightarrow{\sim} \AffSch
	$$
\end{proposition}


\subsection{SOME EXAMPLES OF MORPHISM OF AFFINE SCHEMES}

Consider the ring map $\phi: k[x] \to k[x, y]$ defined by $x \mapsto x$. It induces a morphism of affine schemes
$$
	\pi: \Spec k[x, y] \to \Spec k[x]
$$

Let $t = (x - a, y - b) \in \Spec k[x, y]$ be a prime. Preimage of a prime under a ring map is a prime, 
$$
	\pi(t) = \phi^{-1}(t) = (x - a)
$$

The map between stalks at $t$ is
\begin{align*}
	\pi^\#_t: \ms{O}_{\Spec k[x], \pi(t)} &\to \ms{O}_{\Spec k[x, y], t} \\
	\pi^\#_t: k[x]_{(x - a)} &\to k[x, y]_{(x - a, y - b)} \\
					\frac{f}{g} &\mapsto \frac{f}{g}
\end{align*}


Similarly, if $t = (f)$ for some irreducible polynomial $f$, then
$$
	\pi(t) = \phi^{-1}(t) = (0)
$$

When 

\note{TODO}


Consider the ring map $\phi: k[u] \to k[x, y]$ defined by $u \to x + y$
