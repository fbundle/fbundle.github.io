\documentclass{article}
\usepackage{graphicx} % Required for inserting images

% header

%% natbib
\usepackage{natbib}
\bibliographystyle{plain}

%% comment
\usepackage{comment}

% no automatic indentation
\usepackage{indentfirst}

% manually indent
\usepackage{xargs} % \newcommandx
\usepackage{calc} % calculation
\newcommandx{\tab}[1][1=1]{\hspace{\fpeval{#1 * 10}pt}}
% \newcommand[number of parameters]{output}
% \newcommandx[number of parameters][parameter index = x]{output}
% use parameter index = x to substitute the default argument
% use #1, #2, ... to get the first, second, ... arguments
% \tab for indentation
% \tab{2} for for indentation twice

% note
\newcommandx{\note}[1]{\textit{\textcolor{red}{#1}}}
\newcommand{\todo}{\note{TODO}}
% \note{TODO}

%% math package
\usepackage{amsfonts}
\usepackage{amsmath}
\usepackage{amssymb}
\usepackage{tikz-cd}
\usepackage{mathtools}
\usepackage{amsthm}

%% operator
\DeclareMathOperator{\tr}{tr}
\DeclareMathOperator{\diag}{diag}
\DeclareMathOperator{\sign}{sign}
\DeclareMathOperator{\grad}{grad}
\DeclareMathOperator{\curl}{curl}
\DeclareMathOperator{\Div}{div}
\DeclareMathOperator{\card}{card}
\DeclareMathOperator{\Span}{span}
\DeclareMathOperator{\real}{Re}
\DeclareMathOperator{\imag}{Im}
\DeclareMathOperator{\supp}{supp}
\DeclareMathOperator{\im}{im}
\DeclareMathOperator{\aut}{Aut}
\DeclareMathOperator{\inn}{Inn}
\DeclareMathOperator{\Char}{char}
\DeclareMathOperator{\Sylow}{Syl}
\DeclareMathOperator{\coker}{coker}
\DeclareMathOperator{\inc}{in}
\DeclareMathOperator{\Sd}{Sd}
\DeclareMathOperator{\Hom}{Hom}
\DeclareMathOperator{\interior}{int}
\DeclareMathOperator{\ob}{ob}
\DeclareMathOperator{\Set}{Set}
\DeclareMathOperator{\Top}{Top}
\DeclareMathOperator{\Meas}{Meas}
\DeclareMathOperator{\Grp}{Grp}
\DeclareMathOperator{\Ab}{Ab}
\DeclareMathOperator{\Ch}{Ch}
\DeclareMathOperator{\Fun}{Fun}
\DeclareMathOperator{\Gr}{Gr}
\DeclareMathOperator{\End}{End}
\DeclareMathOperator{\Ad}{Ad}
\DeclareMathOperator{\ad}{ad}
\DeclareMathOperator{\Bil}{Bil}
\DeclareMathOperator{\Skew}{Skew}
\DeclareMathOperator{\Tor}{Tor}
\DeclareMathOperator{\Ho}{Ho}
\DeclareMathOperator{\RMod}{R-Mod}
\DeclareMathOperator{\Ev}{Ev}
\DeclareMathOperator{\Nat}{Nat}
\DeclareMathOperator{\id}{id}
\DeclareMathOperator{\Var}{Var}
\DeclareMathOperator{\Cov}{Cov}
\DeclareMathOperator{\RV}{RV}
\DeclareMathOperator{\rank}{rank}

%% pair delimiter
\DeclarePairedDelimiter{\abs}{\lvert}{\rvert}
\DeclarePairedDelimiter{\inner}{\langle}{\rangle}
\DeclarePairedDelimiter{\tuple}{(}{)}
\DeclarePairedDelimiter{\bracket}{[}{]}
\DeclarePairedDelimiter{\set}{\{}{\}}
\DeclarePairedDelimiter{\norm}{\lVert}{\rVert}

%% theorems
\newtheorem{axiom}{Axiom}
\newtheorem{definition}{Definition}
\newtheorem{theorem}{Theorem}
\newtheorem{proposition}{Proposition}
\newtheorem{corollary}{Corollary}
\newtheorem{lemma}{Lemma}
\newtheorem{remark}{Remark}
\newtheorem{claim}{Claim}
\newtheorem{problem}{Problem}
\newtheorem{assumption}{Assumption}
\newtheorem{example}{Example}
\newtheorem{exercise}{Exercise}

%% empty set
\let\oldemptyset\emptyset
\let\emptyset\varnothing

\newcommand\eps{\epsilon}

% mathcal symbols
\newcommand\Tau{\mathcal{T}}
\newcommand\Ball{\mathcal{B}}
\newcommand\Sphere{\mathcal{S}}
\newcommand\bigO{\mathcal{O}}
\newcommand\Power{\mathcal{P}}
\newcommand\Str{\mathcal{S}}


% mathbb symbols
\usepackage{mathrsfs}
\newcommand\N{\mathbb{N}}
\newcommand\Z{\mathbb{Z}}
\newcommand\Q{\mathbb{Q}}
\newcommand\R{\mathbb{R}}
\newcommand\C{\mathbb{C}}
\newcommand\F{\mathbb{F}}
\newcommand\T{\mathbb{T}}
\newcommand\Exp{\mathbb{E}}

% mathrsfs symbols
\newcommand\Borel{\mathscr{B}}

% algorithm
\usepackage{algorithm}
\usepackage{algpseudocode}

% longproof
\newenvironment{longproof}[1][\proofname]{%
  \begin{proof}[#1]$ $\par\nobreak\ignorespaces
}{%
  \end{proof}
}


% for (i) enumerate
% \begin{enumerate}[label=(\roman*)]
%   \item First item
%   \item Second item
%   \item Third item
% \end{enumerate}
\usepackage{enumitem}

% insert url by \url{}
\usepackage{hyperref}

% margin
\usepackage{geometry}
\geometry{
a4paper,
total={190mm,257mm},
left=10mm,
top=20mm,
}


\title{
    compactness
}
\author{Khanh Nguyen}
\date{September 2023}

\begin{document}

\maketitle

\emph{this text presents the idea of local-to-global principle as a definition of compactness}

In mathematics, one often encounters the notion of compactness as the existence in a finite subcover in a cover of a compact set. The notion of compactness seems intuitive at first as it often described as finiteness since it shares many properties with finite sets. In fact, one can consider a finite set is the compact subset of the set of natural numbers as finite sets resemble two notions: discrete and finiteness.

\section{Finite Union Property}

A property is a characteristic that identifies subsets of a given set. \emph{Finite Union Property} is a property that is generalized to all finite union of sets. Formally,

\begin{definition}[Finite Union Property]
	Let $X$ be any set (one might call it the universal set) and property $p$ identify a family $P$ of subsets of $X$. That is, $P = \{ x \in \mathcal{P}(X): p(x) \}$.
   Property $p$ (or family $P$) is said to have \emph{Finite Union Property (FUP)} if
    $$
    	U_1, U_2, ..., U_N \in P \implies \bigcup_{n=1}^N U_n \in P
    $$
    where $N \in \N$
\end{definition}



So, what is so special about \emph{FUP}? Before answer that question, let's see some example of properties that exhibit \emph{FUP}.

\begin{enumerate}
	\item Let $X$ be any non-empty set, property $p_1$ identifies all  finite subsets of $X$ \label{example1}
    \item Let $\N$ be the set of natural numbers, property $p_2$ identifies all the subsets of $\N$ with its maximum element being even, e.g. $\{8\}, \{1, 4\}, \{n \in \N: n \leq 1000 \}, ...$  \label{example2}
    \item Let $f: S \to \R$ be a real-valued continuous function on $S \subseteq \R$, property $p_3$ identifies all open subset of $O \subseteq S$ if the image of $O$ is bounded  \label{example3}
    \item Let $\Omega$ be a sample space, property $p_4$ identifies all events $E \subseteq \Omega$ with probability greater than $0.1$  \label{example4}
\end{enumerate}

Example \ref{example1} and example \ref{example2} are on finite sets while example \ref{example3} and \ref{example4} are on arbitrary sets. So \emph{FUP} is a universal property that works on certain structures.

\section{Compactness}

In order to give a relation to compactness, let's recall the definition of compactness.

\begin{definition}[Compactness]
    \label{def:compact}
    A topological space $(X, \Tau)$ is said to be compact if for every family of open sets $\mathcal{O} = \{O_i: i \in I\}$ where $I$ is an index set such that $\bigcup_{i \in I} O_i = X$, then there exists a finite subset $J \subseteq I$ such that $\bigcup_{j \in J} O_j = X$
\end{definition}

In many textbooks, compactness is often considered as a generalization of finiteness. Indeed, compactness and finiteness share many common properties.

The statements \cite{tao2008compactness} below are true for any finite set $X$

\begin{itemize}
    \item (\textbf{bounded}) If $f: X \to \R$ is a real-valued function on $X$, then $f$ is bounded
    \item (\textbf{maximum}) If $f: X \to \R$ is a real-valued function on $X$, then $f$ attains a maximum value
    \item (\textbf{constant subsequence}) if $x_1, x_2, ..., x_n, ... \in X$ is a sequence of points in $X$, then there exists a constant subsequence $x_{n_1}, x_{n_2}, ...$
\end{itemize}

While for a compact set $X \subseteq \R$, a similar set of statements are true:

\begin{itemize}
    \item (\textbf{bounded}) If $f: X \to \R$ is a \emph{continuous} real-valued function on $X$, then $f$ is bounded
    \item (\textbf{maximum}) If $f: X \to \R$ is a \emph{continuous} real-valued function on $X$, then $f$ attains a maximum value
    \item (\textbf{convergence subsequence}) if $x_1, x_2, ..., x_n, ... \in X$ is a sequence of points in $X$, then there exists a convergence subsequence $x_{n_1}, x_{n_2}, ...$
\end{itemize}

The three statements above exhibit different equivalent notions of compactness in metric space; namely, \emph{local-to-global principle}, \emph{limit point compactness}, \emph{sequential compactness}. With the machinery of \emph{FUP}, we are ready to give a equivalent notion of compactness from \emph{FUP}


\begin{proposition}[Finite Union Compactness]
    \label{prop:compact_new}
    A topological space $(X, \Tau)$ is compact if and only if for every property $p$ on $X$ with \emph{FUP}, for each point $x \in X$, there is an open neighbourhood $O_x$ (not necessarily distinct) that satisfies $p$ implies $X$ satisfies $p$.
\end{proposition}

In the perspective of \emph{local-to-global principle}, \emph{Finite Union Compactness} gives sufficient conditions to generalize $p$ into the whole set if every point has a neighbourhood satisfying $p$. In fact, \emph{Finite Union Compactness} is equivalent to \emph{Compactness}.

Back to the four examples at the beginning. $p_1$ identifies the whole set $X$ if and only if $X$ is finite. If $p_2$ identifies a set $U$, then $U$ must be finite. If $p_3$ identifies $S$, then $S$ must be compact.


Now, we have a better notion of compactness. We will define discreteness as follows

\begin{definition}[Discreteness]
    A set $X$ is said to be discrete if it has a bijective mapping onto a subset of $\N$
\end{definition}

Compactness of discrete sets is then defined as follows

\begin{proposition}[Compactness of discrete sets]
    \label{prop:compact_n}
    A discrete set with the discrete topology is compact if and only if it is finite
\end{proposition}

Equivalently, we can say that any finite set resembles discreteness and compactness. One property enables countability, the other enables \emph{local-to-global principles}.


\section{Proof}

\subsection{Proposition \ref{prop:compact_new}}

\textbf{Compactness $\implies$ Finite Union Compactness}

Let $(X, \Tau)$ be a compact topological space with a property $p$ defined on $X$ that has \emph{FUP}. For each point $x \in X$, there is an open neighbourhood $O_x$ satisfies $p$. Clearly, $\bigcup_{x \in X} O_x = X$. By \emph{Compactness}, there is a subcover $O_{x_1}, O_{x_2}, ..., O_{x_N}$ for $N \in \N$, i.e. $\bigcup_{n=1}^N O_{x_n} = X$. By \emph{FUP}, $p(X) = p(\bigcup_{n=1}^N O_{x_n})$ is true since it is a finite union of sets satisfying $p$


\textbf{Finite Union Compactness $\implies$ Compactness}

Let $(X, \Tau)$ be a compact topological space with an open cover $\mathcal{O} = \{O_i: i \in I\}$ where $I$ is an index set. Let $p$ be the property on $X$ that identifies all open sets that can be covered by a finite number of open sets in $\mathcal{O}$. Clearly, $p$ is a \emph{FUP}.
Invoke Axiom of Choice, for each $x \in X$, choose an open set $O_x \in \mathcal{O}$ containing $x$ (not necessary distinct). Each $O_x$ satisfies $p$. By \emph{Finite Union Compactness}, $p$ is true for the whole set $X$, that is, $X$ can be covered by a finite number of open sets in $\mathcal{O}$, namely a finite subcover of $\mathcal{O}$


\subsection{Proposition \ref{prop:compact_n}} 

\textbf{Finiteness $\implies$ Compactness}

Trivial

\textbf{Compactness $\implies$ Finiteness}

Let $U$ be discrete and compact with discrete topology and let $p$ be a property defined on $U$ that identifies all finite sets. Clearly, each singleton in $U$ satisfies $p$. By \emph{Finite Union Compactness}, $p$ is true for $U$, that is, $U$ is finite.


\bibliography{references}
\end{document}