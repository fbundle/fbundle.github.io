\documentclass{article}
\usepackage{graphicx} % Required for inserting images

\usepackage{amsmath,amsfonts,amsthm}
\newtheorem{theorem}{Theorem}
\newtheorem{lemma}{Lemma}
\newtheorem{definition}{Definition}


\setlength{\parindent}{0pt} % no text indentation


\title{complex}
\author{Khanh Nguyen}
\date{May 2023}

\begin{document}


\maketitle

\emph{this is the note for my learning on complex analysis}

\section{Definition of complex numbers}

\begin{definition}[Real Numbers]
    An ordered field $R$ satisfies least upper bound property: Every non-empty subset of $R$ with an upper bound has a least upper bound.
\end{definition}

\begin{definition}[Complex Numbers]
    The set of complex numbers is defined as $\mathbb{C} = \mathbb{R} \times \mathbb{R}$ with additional structure
    \begin{itemize}
        \item Complex Addition ($+$): $(a_1, b_1) + (a_2, b_2) = (a_1 + a_2, b_1 + b_2)$
        \item Complex Multiplication ($\cdot$): $(a_1, b_1) \cdot (a_2, b_2) = (a_1 a_2 - b_1 b_2, a_1 b_2 + a_2 b_1)$
    \end{itemize}
    $(\mathbb{C}, +, \cdot)$ form the field of complex numbers.
\end{definition}

\begin{theorem}[Real sub-field of Complex Numbers]
    The set $R = \{(a, 0): a \in \mathbb{R}\} \subset \mathbb{C}$ together with Complex Addition and Complex Multiplication is Real Numbers
\end{theorem}

In fact, Given any non-zero complex number $(a, b)$ where $a, b \neq 0$, any set of the form $\{(\alpha a, \alpha b): \alpha \in \mathbb{R}\}$ is also Real Numbers.

In the context of Complex Numbers, we only call the set $R = \{(a, 0): a \in \mathbb{R}\}$ Real Numbers and write them as $a = (a, 0)$

\subsection{Representation of Complex Numbers}

\subsubsection{Standard form}

\begin{theorem}[Standard form of Complex Numbers]
    Any complex number can be written as $a + ib$ where $a, b \in R$ and $i = (0, 1)$.
\end{theorem}

Proof

\begin{align*}
    a + ib  &= (a, 0) + (b, 0) \cdot (0, 1) &\text{(rewrite)}\\
            &= (a, 0) + (b 0 - 0 1, b 1 + 0 0) &\text{(complex multiplication)}\\
            &= (a, 0) + (0, b) &\text{(the field of real numbers)}\\
            &= (a, b) &\text{(complex addition)}\\
\end{align*}

From now on, instead of writing a complex number as $(a, b)$, we will write them in standard form: $a + ib$

\subsubsection{Polar form}

Define the equivalent relation $\sim$ in the set of real numbers with element denoted as $\theta \in \mathbb{R}$

\begin{definition}[$2\pi$ periodic]
$$
    \theta \sim \theta + 2\pi 
$$
\end{definition}

Let the equivalent class $\Theta$ be defined by the equivalent relation $\sim$ on $\mathbb{R}$. Element of $\Theta$ is denoted as $[\theta]$

\begin{definition}[Polar form of Complex Numbers]
    Any non-zero complex number can be written in polar form
$$
    (r, [\theta]) \in (0, \infty) \times \Theta
$$
    where $r = \sqrt{a^2 + b^2}$ and $\theta = \arctan (b/a)$
\end{definition}

Given $(r, [\theta])$ in polar form, we can rewrite it in standard form $a + ib$ where $a = r \cos \theta$, $b = r \sin \theta$


\textbf{Multiplication in polar form}

By the special structure of polar form, multiplication is simpler

$$
    (r_1, [\theta_1]) \cdot (r_2, [\theta_2]) = (r_1 r_2, [\theta_1 + \theta_2])
$$

\subsection{Complex function}
\emph{several functions defined for the set of complex numbers}

\subsubsection{Modulus - Argument - Conjugate}

\textbf{Modulus} $|\cdot|$

$$
    |a + ib| = \sqrt{a^2 + b^2}
$$

Complex Numbers inherits the L2 norm in $\mathbb{R}^2$ making it a normed space.

\textbf{Argument} $\arg$

$$
    \arg (a + ib) = \arctan (b/a)
$$

where $\arctan (b/a)$ is defined to be in the range $[0, 2\pi)$ or $(-\pi, \pi]$

\textbf{Conjugate} $\overline{\cdot}$

$$
    \overline{a + ib} = a - ib
$$

\subsubsection{Inverses}

\noindent
\textbf{Additive inverse}

$$
    - (a + ib) = - a - ib
$$

\textbf{Multiplicative inverse}

$$
    (a + ib)^{-1} = \frac{a - ib}{a^2 + b^2}
$$

\subsubsection{Exponential}

\begin{definition}[Exponential function]
$$
    e^{a + ib} = e^a (\cos b + i \sin b)
$$    
\end{definition}

\begin{theorem}[Polar form as Exponential]
$$
    (r, [\theta]) = r e^{i\theta}
$$
\end{theorem}

Proof

\begin{align*}
    r e^{i\theta}   &= r (\cos \theta + i \sin \theta) &\text{(Exponential)}\\
                    &= r \cos \theta + i r \sin \theta &\text{(the field of complex numbers)}\\
                    &= (r, [\theta]) &\text{(polar form definition)}\\
\end{align*}


From now on, instead of writing a complex number in polar form as $(r, [\theta])$, we will write them as complex exponential: $r e^{i\theta}$

\textbf{Inverse of Exponential}

Exponential function is $2\pi$ periodic in imaginary axis, i.e

$$
    e^{a + ib} = e^{a + i(b+2\pi)}
$$

Hence the inverse of exponential is defined as a multi-function

\begin{definition}[Logarithm]
Defining $\log$ as a multi-function being the inverse of exponential function

\begin{align*}
    \log &: \mathbb{C} \setminus \{ 0 \} \to \mathcal{P}(\mathbb{C}) \\
    \log(z) &= \{x: x \in \mathbb{C} \land e^x = z\} \\
    \log(r e^{i\theta}) &= \{\log r + i (\theta + 2\pi k): k \in \mathbb{Z}\} \\
\end{align*}
\end{definition}

\begin{definition}[Power function] \footnote{in this note, given $f: A \to B$, we write $f(\overline{A}) = \{f(x): x \in \overline{A}\}$ where $\overline{A} \subseteq A$}

$$
    z^n = e^{n \log z}
$$
\end{definition}

Write $z$ in polar form

\begin{align*}
    z^n &= (r e^{i\theta})^n \\
        &= e^{n \log (r e^{i\theta})} \\
        &= e^{n \{ \log r + i (\theta + 2 \pi k): k \in \mathbb{Z} \}} \\
        &= e^{n \log r} e^ {\{ i n (\theta + 2 \pi k): k \in \mathbb{Z} \}} \\
\end{align*}

\textbf{$n$ is a positive integer}

$z^n: \mathbb{C} \setminus \{ 0 \} \to \mathbb{C}$ is a proper function \footnote{one input, one output}

\begin{align*}
    z^n &= e^{n \log r} e^ {\{ i n (\theta + 2 \pi k): k \in \mathbb{Z} \}} \\
        &= r^n e^{i n \theta} \\ 
\end{align*}

\textbf{$m-th$ root, i.e $n = 1/m$ where $m \in \mathbb{N} \setminus \{0\}$}

There are $m$ $m-th$ roots of complex number $z \neq 0$

\begin{align*}
    z^n = \sqrt[m]{z}&= e^{\frac{\log r}{m}} e^ {\{ i (\frac{\theta}{m} + \frac{2 \pi}{m} k): k \in \mathbb{Z} \}} \\
        &= \sqrt[m]{r} \{ e^ {i (\frac{\theta}{m} + \frac{2 \pi}{m} k)}: k \in \mathbb{Z} \cup [0, m) \} \\ 
\end{align*}

\textbf{$n$ is irrational}

There are infinitely many output values of power function


\textbf{Properties of Exponential function and Power function}

Exponential function
\begin{itemize}
    \item $e^{a + b} = e^{a} e^{b}$
    \item $e^{-a} = (e^{a})^{-1}$
\end{itemize}

Power function
\begin{itemize}
    \item $z^{a + b} = z^{a} z^{b}$
    \item $z^{-a} = (z^{a})^{-1}$
    \item $z^{ab} = (z^a)^b$
\end{itemize}



\section{Subspace of $\mathbb{R}^{2 \times 2}$}

The field of complex numbers is isomorphic to a subspace of the vector space of $(2 \times 2)$ matrices with real entries, i.e $\mathbb{R}^{2 \times 2}$ with additional structure as follows

Basis of the complex subspace is $\mathbf{B} = \{r, i\}$ where

$$
    r = \begin{bmatrix}
        1 & 0 \\
        0 & 1 \\
    \end{bmatrix}
    \;
    i = \begin{bmatrix}
        0 & -1 \\
        1 & 0 \\
    \end{bmatrix}
$$

All complex numbers is represented as $z = ar + ib = \begin{bmatrix}
    a & -b \\
    b & a \\
\end{bmatrix}$ where $a, b \in \mathbb{R}$. Furthermore, we define addition as matrix addition and multiplication as matrix multiplication.

\subsection{Polar form}

Let $r = \sqrt{a^2 + b^2}$ and $\theta = \arctan (b/a)$

$$
\begin{bmatrix}
    a & -b \\
    b & a \\
\end{bmatrix} = r \begin{bmatrix}
    \cos \theta & -\sin \theta \\
    \sin \theta & \cos \theta \\
\end{bmatrix}
$$

\emph{Geometrically speaking, a complex number is a composition of a scaling operator and a rotation operator in $\mathbb{R}^2$}







\end{document}


