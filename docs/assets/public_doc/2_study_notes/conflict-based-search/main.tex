\documentclass{article}
\usepackage[utf8]{inputenc}
\usepackage{amssymb}
\usepackage{natbib}
\bibliographystyle{plain}
\usepackage[fleqn]{amsmath}
\usepackage{amsthm}
\newtheorem{theorem}{Theorem}
\newtheorem{lemma}{Lemma}


\title{conflict-based-search}
\author{nguyenngockhanh.pbc }
\date{March 2021}

\begin{document}

\maketitle

    Given a set of resources $R$ and a set of  $k$ agents $A$. Each agent associates with a subset of resources by the cost function $c: A \times P(R) \to \mathbb{R}$. Our goal is to find such $k$ disjoint subsets of $R$ associate with $k$ agents that minimize the total cost. 

    In conflict-based search, we define a conflict tree with each node has 3 properties: \textbf{constraint}, \textbf{assignment} and \textbf{cost}. \textbf{constraint} is a set of constraints, \textbf{assignment} is the least cost function $a: A \to P(R)$ that uniquely assigns each agent to a subset of resources that satisfies the \textbf{constraint} and \textbf{cost} is the cost of \textbf{assignment}.
    
    The branching rules is defined as follow:
    
    (1) if there is no conflict (all subsets are disjoint), the node is a terminal node.
    
    (2) Choose a conflict (a resource in common of more than one agent), branch-out to $m+1$ child nodes where $m$ is the number of agent taking that resource. In each of the first $m$ child nodes, add a constraint assigning the respective agent to take the resource. In last child node, add a constraint preventing all agents to take the resource.
    
    To elaborate more on the branching rule (2), consider a node with properties as follow:
    
    \fbox{\begin{minipage}{30em}
    
    \textbf{constraint}: \{ $c_1$ \}
    
    \textbf{assignment}: $a_1 \to \{r_1, r_2\},  a_2 \to \{r_2, r_3\}, a_3 \to \{r_2, r_4\}$
    
    \textbf{cost}: some real number.
    
    \end{minipage}}
    
    
    Suppose we choose $r_2$ to branch-out, the first child node is:
    
    \fbox{\begin{minipage}{30em}
    
    \textbf{constraint}: \{ $c_1$, (assign $r_2$ to $a_1$) \}
    
    \textbf{assignment}: least cost assignment satisfies the \textbf{constraint}
    
    \textbf{cost}: some real number.

    \end{minipage}}

    The next two child node is constructed in the same manner by replacing the constraint on $a_1$ to $a_2$ and $a_3$ respectively.
    
    The last child node is:
    
    \fbox{\begin{minipage}{30em}
    
    \textbf{constraint}: \{ $c_1$, (do not assign $r_2$ to any of \{$a_1$, $a_2$, $a_3$\}) \}
    
    \textbf{assignment}: least cost assignment satisfies the \textbf{constraint}
    
    \textbf{cost}: some real number.

    \end{minipage}}
    
    By splitting the conflict tree node in this way, there is no duplicate node since $m+1$ child nodes split from a parent node are mutually exclusive. In the case of $m=2$, the branching rules reduced to the method described in \cite{sharon2015conflict}.
    
    Using best first search on the conflict tree guarantees to find the optimal assignment by the following two arguments:
    
    \begin{lemma}[Complete]
    Root node (empty \textbf{constraint}) permits the least cost terminal node if one exists.
    \label{lemma:complete}
    \end{lemma}
    
    \begin{lemma}[Lower bound]
    The cost of a conflict tree node is the lower bound of all terminal nodes it permits.
    \label{lemma:lowerbound}
    \end{lemma}
    
    Lemma \ref{lemma:complete} can be proved by verifying the conflict tree is limited depth and for every branch-out, the least cost terminal node belongs to either one of the branches.
    
    Lemma \ref{lemma:lowerbound} can be proved by contradiction since adding more constraints, the cost of a node cannot decrease.


    \bibliography{references}





\end{document}
