\documentclass{article}
\usepackage{graphicx} % Required for inserting images

% header

%% natbib
\usepackage{natbib}
\bibliographystyle{plain}

%% comment
\usepackage{comment}

% no automatic indentation
\usepackage{indentfirst}

% manually indent
\usepackage{xargs} % \newcommandx
\usepackage{calc} % calculation
\newcommandx{\tab}[1][1=1]{\hspace{\fpeval{#1 * 10}pt}}
% \newcommand[number of parameters]{output}
% \newcommandx[number of parameters][parameter index = x]{output}
% use parameter index = x to substitute the default argument
% use #1, #2, ... to get the first, second, ... arguments
% \tab for indentation
% \tab{2} for for indentation twice

% note
\newcommandx{\note}[1]{\textit{\textcolor{red}{#1}}}
\newcommand{\todo}{\note{TODO}}
% \note{TODO}

%% math package
\usepackage{amsfonts}
\usepackage{amsmath}
\usepackage{amssymb}
\usepackage{tikz-cd}
\usepackage{mathtools}
\usepackage{amsthm}

%% operator
\DeclareMathOperator{\tr}{tr}
\DeclareMathOperator{\diag}{diag}
\DeclareMathOperator{\sign}{sign}
\DeclareMathOperator{\grad}{grad}
\DeclareMathOperator{\curl}{curl}
\DeclareMathOperator{\Div}{div}
\DeclareMathOperator{\card}{card}
\DeclareMathOperator{\Span}{span}
\DeclareMathOperator{\real}{Re}
\DeclareMathOperator{\imag}{Im}
\DeclareMathOperator{\supp}{supp}
\DeclareMathOperator{\im}{im}
\DeclareMathOperator{\aut}{Aut}
\DeclareMathOperator{\inn}{Inn}
\DeclareMathOperator{\Char}{char}
\DeclareMathOperator{\Sylow}{Syl}
\DeclareMathOperator{\coker}{coker}
\DeclareMathOperator{\inc}{in}
\DeclareMathOperator{\Sd}{Sd}
\DeclareMathOperator{\Hom}{Hom}
\DeclareMathOperator{\interior}{int}
\DeclareMathOperator{\ob}{ob}
\DeclareMathOperator{\Set}{Set}
\DeclareMathOperator{\Top}{Top}
\DeclareMathOperator{\Meas}{Meas}
\DeclareMathOperator{\Grp}{Grp}
\DeclareMathOperator{\Ab}{Ab}
\DeclareMathOperator{\Ch}{Ch}
\DeclareMathOperator{\Fun}{Fun}
\DeclareMathOperator{\Gr}{Gr}
\DeclareMathOperator{\End}{End}
\DeclareMathOperator{\Ad}{Ad}
\DeclareMathOperator{\ad}{ad}
\DeclareMathOperator{\Bil}{Bil}
\DeclareMathOperator{\Skew}{Skew}
\DeclareMathOperator{\Tor}{Tor}
\DeclareMathOperator{\Ho}{Ho}
\DeclareMathOperator{\RMod}{R-Mod}
\DeclareMathOperator{\Ev}{Ev}
\DeclareMathOperator{\Nat}{Nat}
\DeclareMathOperator{\id}{id}
\DeclareMathOperator{\Var}{Var}
\DeclareMathOperator{\Cov}{Cov}
\DeclareMathOperator{\RV}{RV}
\DeclareMathOperator{\rank}{rank}

%% pair delimiter
\DeclarePairedDelimiter{\abs}{\lvert}{\rvert}
\DeclarePairedDelimiter{\inner}{\langle}{\rangle}
\DeclarePairedDelimiter{\tuple}{(}{)}
\DeclarePairedDelimiter{\bracket}{[}{]}
\DeclarePairedDelimiter{\set}{\{}{\}}
\DeclarePairedDelimiter{\norm}{\lVert}{\rVert}

%% theorems
\newtheorem{axiom}{Axiom}
\newtheorem{definition}{Definition}
\newtheorem{theorem}{Theorem}
\newtheorem{proposition}{Proposition}
\newtheorem{corollary}{Corollary}
\newtheorem{lemma}{Lemma}
\newtheorem{remark}{Remark}
\newtheorem{claim}{Claim}
\newtheorem{problem}{Problem}
\newtheorem{assumption}{Assumption}
\newtheorem{example}{Example}
\newtheorem{exercise}{Exercise}

%% empty set
\let\oldemptyset\emptyset
\let\emptyset\varnothing

\newcommand\eps{\epsilon}

% mathcal symbols
\newcommand\Tau{\mathcal{T}}
\newcommand\Ball{\mathcal{B}}
\newcommand\Sphere{\mathcal{S}}
\newcommand\bigO{\mathcal{O}}
\newcommand\Power{\mathcal{P}}
\newcommand\Str{\mathcal{S}}


% mathbb symbols
\usepackage{mathrsfs}
\newcommand\N{\mathbb{N}}
\newcommand\Z{\mathbb{Z}}
\newcommand\Q{\mathbb{Q}}
\newcommand\R{\mathbb{R}}
\newcommand\C{\mathbb{C}}
\newcommand\F{\mathbb{F}}
\newcommand\T{\mathbb{T}}
\newcommand\Exp{\mathbb{E}}

% mathrsfs symbols
\newcommand\Borel{\mathscr{B}}

% algorithm
\usepackage{algorithm}
\usepackage{algpseudocode}

% longproof
\newenvironment{longproof}[1][\proofname]{%
  \begin{proof}[#1]$ $\par\nobreak\ignorespaces
}{%
  \end{proof}
}


% for (i) enumerate
% \begin{enumerate}[label=(\roman*)]
%   \item First item
%   \item Second item
%   \item Third item
% \end{enumerate}
\usepackage{enumitem}

% insert url by \url{}
\usepackage{hyperref}

% margin
\usepackage{geometry}
\geometry{
a4paper,
total={190mm,257mm},
left=10mm,
top=20mm,
}


\title{
    MA4271 Homework 3
}
\author{Nguyen Ngoc Khanh - A0275047B}
\date{October 2023}

\begin{document}

\maketitle

\section{Problem}

\begin{problem}
    Prove that the absolute value of the torsion $\tau$ at a point of an asymptotic curve whose curvature is nowhere zero is given by
    \[
        |\tau| = \sqrt{-K}
    \]
    where $K$ is the Gaussian curvature of the surface at the given point.
\end{problem}

Let $\alpha(s)$ be an asymptotic curve on surface $S$ whose curvature is nowhere zero. The normal curvature of $\alpha(s)$ is 
\[
    \langle N, \kappa n \rangle = 0
\]

where $N$ denotes the unit normal vector of surface, $n$ denotes the unit normal vector of $\alpha(s)$, $\kappa$ denotes the curvature of $\alpha$. As $\kappa \neq 0$, $n$ exists everywhere. Torsion of $\alpha$ is defined as $b'(s) = \tau(s) n(s)$. Restrict $N$ on $\alpha$, as $N(s)$ is orthogonal to $n(s)$ and $\alpha'(s) \in T_{\alpha(s)} S$, we have

\[
    N(s) = \lambda (n(s) \times \alpha'(s)) = \lambda b(s) 
\]
where $\lambda$ is either $+1$ or $-1$. Then 
\[
    \tau(s) = ||b'(s)|| = ||N'(s)||
\]

We can write $N'(s)$ in term of Gauss map (differential of $N$ along the direction of tangent vector)

\begin{align*}
    N'(s)
    &= dN_{\alpha(s)} (\alpha'(s)) \\
    &= dN_{\alpha(s)} (e_1 \cos \theta + e_2 \sin \theta) &\text{(let $\alpha'(s) = e_1 \cos \theta + e_2 \sin \theta$)} \\
    &= e_1 k_1 \cos \theta + e_2 k_2 \sin \theta &\text{(eigen direction)} \\
\end{align*}

Then $||N'(s)||^2 = k_1^2 (\cos \theta)^2 + k_2^2 (\sin \theta)^2$. On the other hand, by Euler formula, $0 = k_1 (\cos \theta)^2 + k_2 (\sin \theta)^2$, Then

\begin{align*}
    ||N'(s)||^2
    &= k_1 k_1 (\cos \theta)^2 + k_2 k_2 (\sin \theta)^2 \\
    &= k_1 (- k_2 (\sin \theta)^2) + k_2 (-k_1 (\cos \theta)^2) \\
    &= - k_1 k_2 = -K
\end{align*}
Then
\[
    \tau(s) = ||N'(s)|| = \sqrt{-K}
\]


\begin{problem}
    Suppose that $S_1$ and $S_2$ intersect along a regular curve $C$ and make an angle $\theta(p), p \in C$. Assume that $C$ is a line of curvature of $S_1$. Prove that $\theta(p)$ is constant if and only if $C$ is a line of curvature of $S_2$
\end{problem}

($\impliedby$)
$C = \alpha(s)$ is the line of curvature of both $S_1$ and $S_2$, by Oline Rodrigues, 

\begin{align*}
    N_1'(s) &= \lambda_1(s) \alpha'(s) \\
    N_2'(s) &= \lambda_2(s) \alpha'(s)
\end{align*}

where $N_1(s), N_2(s)$ is the restriction of unit normal vectors of $S_1, S_2$ on $C$, $-\lambda_1(s), -\lambda_2(s)$ are the corresponding principle curvatures

As $\alpha'(s)$ is orthogonal to $N_1(s)$ and $N_2(s)$
\begin{align*}
    \langle N_1'(s), N_2(s) \rangle &= \lambda_1(s) \langle \alpha'(s), N_2(s)\rangle = 0 \\
    \langle N_2'(s), N_1(s) \rangle &= \lambda_2(s) \langle \alpha'(s), N_1(s)\rangle = 0
\end{align*}

Then
\[
    \frac{d}{ds} \langle N_1(s), N_2(s) \rangle = \langle N_1'(s), N_2(s) \rangle + \langle N_2'(s), N_1(s) \rangle = 0
\]

Then $\theta = \langle N_1, N_2 \rangle$ is constant along $C$

($\implies$) $C = \alpha(s)$ is the line of curvature of $S_1$, and $\theta = \langle N_1, N_2 \rangle$ is constant along $C$, similarly by Oline Rodrigues

\[
    N_1'(s) = \lambda_1(s) \alpha'(s)
\]

Then
\[
    \langle N_1'(s), N_2(s) \rangle = \lambda_1(s) \langle \alpha'(s), N_2(s)\rangle = 0
\]

Then 
\[
    \langle N_2'(s), N_1(s) \rangle = \frac{d}{ds} \langle N_1(s), N_2(s) \rangle - \langle N_1'(s), N_2(s) \rangle = 0
\]

As $N_2(s)$ is a unit vector, then $N_2'(s)$ is orthogonal to $N_2(s)$. Since both $\alpha'(s)$ and $N_2'(s)$ are orthogonal two both $N_1(s)$ and $N_2(s)$, then they are parallel, i.e. $N_2'(s) = \lambda(s) \alpha'(s)$. By Oline Rodrigues, $C$ is the line of curvature of $S_2$



\begin{problem}
    Let $\lambda_1, ..., \lambda_m$ be the normal curvatures at $p \in S$ along directions making angles $0, 2\pi/m, ..., (m-1)2\pi/m$ with a principle direction. Prove that
    \[
        \lambda_1 + ... + \lambda_m = mH
    \]
    where $H$ is the mean curvature at $p$
\end{problem}

Denote $\theta_i = 2\pi \frac{i-1}{m}$, by Euler formula,

\begin{align*}
    \sum_{i=1}^m \lambda_i
    &= \sum_{i=1}^m [ k_1 (\cos \theta_i)^2 + k_2 (\sin \theta_i )^2 ] \\
\end{align*}

where $k_1, k_2$ are the principle curvatures \footnote{$k_1$ is either maximum or minimum principle curvature}. If we rotate the $n$ directions by $\frac{\pi}{2}$, we have

\begin{align*}
    \sum_{i=1}^m \lambda_i'
    &= \sum_{i=1}^m [ k_1 (\cos (\theta_i + \pi/2) )^2 + k_2 (\sin (\theta_i + \pi/2) )^2 ] \\
    &= \sum_{i=1}^m [ k_1 (\sin \theta_i )^2 + k_2 (-\cos \theta_i)^2 ] &\text{($\cos (\theta + \pi/2) = \sin \theta, \sin (\theta + \pi/2) = -\cos \theta,$)}\\
\end{align*}

where $\lambda_1', ..., \lambda_m'$ are the corresponding normal curvatures after rotation. Then 
\[
    \left(\sum_{i=1}^m \lambda_i +  \sum_{i=1}^m \lambda_i'\right) = m (k_1 + k_2) = 2mH
\]

and
\[
    \left(\sum_{i=1}^m \lambda_i -  \sum_{i=1}^m \lambda_i'\right) = \sum_{i=1}^m (k_1 - k_2) [(\cos \theta_i)^2 - (\sin \theta_i)^2] = (k_1 - k_2) \sum_{i=1}^m \cos (2 \theta_i)
\]

Note that
\[
    \sum_{i=1}^m \cos (2 \theta_i) = \sum_{i=1}^m \real[e^{\mathbf{i} 2\theta_i}] = \real\left[\sum_{i=1}^m e^{\mathbf{i} 2\theta_i} \right] = \real[0] = 0
\]

where $\mathbf{i}$ denotes the imaginary unit. Hence,
\[
   \lambda_i = \frac{1}{2}  \left(\sum_{i=1}^m \lambda_i +  \sum_{i=1}^m \lambda_i'\right) = mH
\]




\begin{problem}
    Show that a surface which is compact has an elliptic point.
\end{problem}

Let $C$ be a compact regular surface. Pick $c \in \R^3$. For any $x \in C$, the map $\phi(x) = ||x - c||$ is continuous, then it admits a maximum value, namely $r = \max_{x \in C} ||x - c||$. Let $p \in C \cap \Sphere_r(c)$ on the intersection of $C$ and the sphere of radius $r$ centered at $c$.

Let $N$ be any plane containing $p$ that intersects $C$ at more than 1 point \footnote{normal section guarantees the existence of $N$}, then $N$ intersects $\Sphere_r(c)$ at more than 1 point \footnote{because $N$ intersects $C \subseteq \overline{\Ball_r(c)}$ at more than 1 point}, then $N \cap \Sphere_r(c)$ is a sphere of radius $0 < r_N \leq r$, then the curvature of $N \cap C$ is bounded below by a positive number, i.e. $\kappa_N \geq \frac{1}{r_N} \geq \frac{1}{r}$. Therefore, any principle curvature of $C$ at $p$ is bounded below by a positive number, then it must be an elliptic point or hyperbolic point.

 Furthermore, $C$ is contained in one of the half space separated by $T_p \Sphere_r(c)$, then $C$ cannot be a hyperbolic point.





\begin{problem}
    Obtain the asymptotic curves of the one-sheeted hyperboloid $x^2 + y^2 - z^2 = 1$
\end{problem}

In the parameterization
\[
    x(u, v) = \left( \frac{\cos (u+v)}{\cos (u-v)}, \frac{\sin(u+v)}{\cos(u-v)}, \tan(u-v) \right)
\]

Claim: asymptotic curves are the coordinate curves

Let $a = \cos(u+v), b = \sin(u+v), c = \cos(u-v), d = \sin(u-v)$, then we have
\begin{itemize}
    \item $a^2 + b^2 = 1, c^2 + d^2 = 1$
    \item $ac - bd = \cos (2u), ac + bd = \cos (2v)$
    \item $ad + bc = \sin (2u), ad - bc = -\sin  (2v)$
    \item $a_u = -b, a_v = -b$
    \item $b_u = a, b_v = a$
    \item $c_u = -d, c_v = d$
    \item $d_u = c, d_v = -c$
\end{itemize}

Calculation

\begin{align*}
    x &= \left(\frac{a}{c}, \frac{b}{c}, \frac{d}{c} \right) \\
    x_u &= \left(\frac{ad - bc}{c^2}, \frac{ac + bd}{c^2} , \frac{1}{c^2}\right) \\
    x_v &= \left(\frac{-ad -bc}{c^2}, \frac{ac - bd}{c^2}, \frac{-1}{c^2} \right) \\
    N &= x_u \times x_v = \left( \frac{-2a}{c^3}, \frac{2b}{c^3} , \frac{2d}{c^3}\right) \\
    N_u &= \left(\frac{2bc - 6ad}{c^4}, \frac{2ac + 6bd}{c^4}, \frac{2 + 4d^2}{c^4} \right) \\
    N_v &= \left(\frac{2bc + 6ad}{c^4}, \frac{2ac - 6bd}{c^4}, \frac{-2 - 4d^2}{c^4} \right)
\end{align*}

Second fundamental form

\begin{align*}
    e = - \langle N_u, x_u \rangle = 0 \\
    f = - \langle N_u, x_v \rangle > 0 \\ 
    g = - \langle N_v, x_v \rangle = 0
\end{align*}

Then the asymptotic curves are the coordinate curves w.r.t the parameterization





\section{Appendix}









\end{document}