% header

%% natbib
\usepackage{natbib}
\bibliographystyle{plain}

%% comment
\usepackage{comment}

% no automatically indentation
\usepackage{indentfirst}

% manually indent
\usepackage{xargs} % \newcommandx
\usepackage{calc} % calculation
\newcommandx{\tab}[1][1=1]{\hspace{\fpeval{#1 * 10}pt}}
% \newcommand[number of parameters]{output}
% \newcommandx[number of parameters][parameter index = x]{output}
% use parameter index = x to substitute the default argument
% use #1, #2, ... to get the first, second, ... arguments
% \tab for indentation
% \tab{2} for for indentation twice


%% math package
\usepackage{amsfonts}
\usepackage{amsmath}
\usepackage{amssymb}
\usepackage{tikz-cd}


%% operator
\DeclareMathOperator{\tr}{tr}
\DeclareMathOperator{\diag}{diag}
\DeclareMathOperator{\sign}{sign}
\DeclareMathOperator{\grad}{grad}
\DeclareMathOperator{\curl}{curl}
\DeclareMathOperator{\Div}{div}

%% theorems
\newtheorem{axiom}{Axiom}
\newtheorem{definition}{Definition}
\newtheorem{theorem}{Theorem}
\newtheorem{proposition}{Proposition}
\newtheorem{corollary}{Corollary}
\newtheorem{lemma}{Lemma}
\newtheorem{remark}{Remark}
\newtheorem{claim}{Claim}
\newtheorem{problem}{Problem}

%% empty set
\let\oldemptyset\emptyset
\let\emptyset\varnothing

% mathcal symbols
\newcommand\Tau{\mathcal{T}}
\newcommand\Ball{\mathcal{B}}
\newcommand\bigO{\mathcal{O}}

% mathbb symbols
\newcommand\N{\mathbb{N}}
\newcommand\Z{\mathbb{Z}}
\newcommand\Q{\mathbb{Q}}
\newcommand\R{\mathbb{R}}
