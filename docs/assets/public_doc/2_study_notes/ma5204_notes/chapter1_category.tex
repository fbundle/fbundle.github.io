\chapter{CATEGORY THEORY}

this was written when I studied MA5209 - Algebraic Topology by Dr Charmaine Sia

\section{CATEGORY THEORY}

\section{CATEGORICAL CONSTRUCTION OF HOMOLOGICAL ALGEBRA }

\begin{definition}[initial object, terminal object, pointed category, zero map, kernel, cokernel]
	Given a category $C$, an object $0$ is initial if for all $X \in \ob C$, there is only one map in $\Hom(0, X)$, an object $*$ is terminal if for all $X \in \ob C$, there is only one map in $\Hom(X, *)$. Category $C$ is called pointed if it has initial and terminal objects and the unique map $0 \to *$ is an isomorphism. If $C$ is a pointed category, we use the same symbol $0$ for both initial object and terminal object. In a pointed category, there exists a zero map between any two objects $M, N \in \ob C$, defined by
	\begin{center}
		\begin{tikzcd}
			M \arrow[r] \arrow[rd, "0"'] & 0 \arrow[d] \\
			& N          
		\end{tikzcd}
	\end{center}
	the composition of $M \to 0$ and $0 \to N$. Moreover, let $f: M \to N$ be a morphism in $C$, a kernel of $f$ is a map $i: K \to M$ such that $f i = 0$ and such map is universal, that is, if $j: L \to M$ with $f j = 0$, then it factors through $i: K \to M$, a cokernel of $f$ is a map $k: N \to Q$ such that $kf = 0$  and such map is universal, that is, if $l: M \to P$ with $lf = 0$, then it factor through $k: N \to Q$
	\begin{center}
		\begin{tikzcd}
			K \arrow[r, "i"] \arrow[rr, "0", bend left=49]                     & M \arrow[r, "f"] & N & M \arrow[r, "f"] \arrow[rrd, "0"', bend right] \arrow[rr, "0", bend left=49] & N \arrow[r, "k"] \arrow[rd, "l"'] & Q \arrow[d, dashed] \\
			L \arrow[ru, "j"'] \arrow[u, dashed] \arrow[rru, "0"', bend right] &                  &   &                                                                              &                                   & P                  
		\end{tikzcd}
	\end{center}
	A category $C$ has kernels if every morphism has a kernel. A category $C$ has cokernels if every morphism has a cokernel.
\end{definition}

\begin{definition}[preadditive category, $\Ab$-enriched category]
	A category $C$ is called \textbf{preadditive category} (or $\Ab$-enriched category) if for any two objects $M, N \in \ob C$, $\Hom(M, N)$ is an abelian group and composition is bilinear, that is, if $f, g, h$ are morphisms in $C$
	\begin{align*}
		f (g + h) = fg + fh \\
		(f + g) h = fh + gh
	\end{align*}
	
	The collection of morphisms on a preadditive category \textit{resembles} a monoid
\end{definition}

\begin{definition}[additive category]
	A category $C$ is additive if
	\begin{enumerate}
		\item $C$ is pointed
		\item $C$ is preadditive
		\item $C$ admits finite biproduct
	\end{enumerate}
	biproduct is when product and coproduct coincide. finite biproduct is the product/coproduct of finitely many objects
\end{definition}

\begin{definition}[abelian category]
	A category $C$ is abelian if
	\begin{enumerate}
		\item $C$ is additive
		\item every map in $C$ has kernel and cokernel
		\item every monomorphism in $C$ is the kernel of its cokernel
		\item every epimorphism in $C$ is the cokernel of its kernel
	\end{enumerate}
\end{definition}

\begin{remark}
	The category of $R$-modules is an abelian category, given any submodule $N$ of $M$
	\begin{enumerate}
		\item every monomorphism in $C$ is the kernel of its cokernel says that any map $M \to Q$ that sends $N$ to zero factors through $M / N$
		\item every epimorphism in $C$ is the cokernel of its kernel says that any map $L \to M$ that is zero on $M / N$ factor through $N$
	\end{enumerate}
\end{remark}

\begin{definition}[projective class]
	Let $C$ be a \textbf{pointed category with kernels}. A \textbf{projective class} in $C$ is a pair $(\mathcal{P}, \mathcal{E})$ where $\mathcal{P}$ is a collection of objects (called \textbf{projectives}) and $\mathcal{E}$ is a collection of morphisms (called \textbf{epimorphisms}) such that
	\begin{enumerate}
		\item An object $P$ is \textbf{projective} if and only if $P$ has the universal lifting property against every \textbf{epimorphism} $M \to N$, that is, given any \textbf{epimorphism} $M \to N$, if there is a map $P \to N$, then it factors through $M$
		\begin{center}
			\begin{tikzcd}
				M \arrow[r, "epi", two heads] & N                              \\
				& P \arrow[lu, dashed] \arrow[u]
			\end{tikzcd}
		\end{center}
		
		\item A morphism $f: M \to N$ is an \textbf{epimorphism} if and only if every \textbf{projective} has the universal lifting property against $f$, that is, given any \textbf{projective} $P$, if there is a map $P \to N$, then it factors through $M$
		
		\begin{center}
			\begin{tikzcd}
				M \arrow[r, "f", two heads] & N                              \\
				& P \arrow[lu, dashed] \arrow[u]
			\end{tikzcd}
		\end{center}
		
		\item $C$ has enough \textbf{projectives}, that is, given any object $M \in \ob C$, for every \textbf{projective} $P$, there exists an \textbf{epimorphism} $P \to M$.
	\end{enumerate}
	
	A similar definition for \textbf{injective class}
\end{definition}

\begin{remark}
	In the category of $R$-modules, an evidental projective class is the pair $(\mathcal{P}, \mathcal{E})$ where $\mathcal{E}$ is the collection of surjective maps and $\mathcal{P}$ is the collection of projective modules
\end{remark}

\begin{definition}[chain complex, exact sequence]
	In a \textbf{pointed category with kernels}, a \textbf{chain complex} is a sequence such that given any subsequence $A \to B \to C$, $A \to B$ factors through $\ker (B \to C)$, that is, there exists a map $A \to \ker (B \to C)$ such that the diagram below commutes
	\begin{center}
		\begin{tikzcd}
			... \arrow[r] & A \arrow[r] \arrow[d, dashed] & B \arrow[r] & C \arrow[r] & ... \\
			& \ker (B \to C) \arrow[ru]     &             &             &    
		\end{tikzcd}
	\end{center}
	
	Equivalently, the composition $A \to B \to C$ is the zero map. If there is a notion of epimorphism and the map $A \to \ker (B \to C)$ is an epimorphism, then the sequence is called \textbf{exact} at $B$. A sequence is called exact sequence or an acyclic chain complex if it is exact everywhere, possibly except the two ends.
	
	Similar definition for \textbf{cochain complex}
\end{definition}

\begin{definition}[chain map]
	Given two chain complexes $C_\bullet, D_\bullet$ in a \textbf{pointed category with kernels}, for each $n \in \Z$, there is a map $f_n: C_n \to D_n$ such that the diagram below commutes, then $f_\bullet$ is called a \textbf{chain map}
	\begin{center}
		\begin{tikzcd}
			... & C_{n-1} \arrow[l] \arrow[d, "f_{n-1}"] & C_n \arrow[l] \arrow[d, "f_n"] & C_{n+1} \arrow[l] \arrow[d, "f_{n+1}"] & ... \arrow[l] \\
			... & D_{n-1} \arrow[l]                      & D_n \arrow[l]                  & D_{n+1} \arrow[l]                      & ... \arrow[l]
		\end{tikzcd}
	\end{center}
	
	Chain complexes and chain maps form a category and it is called the category of chain complexes.

	Similar definition for \textbf{cochain map}
\end{definition}

\begin{remark}[homology]
	Let $\im f = \ker \coker f$ and $\coim f= \coker \ker f$. The main axiom of abelian category states that the canonical map $\hat{f}: \coim f \to \im f$ is an isomorphism.
	\begin{center}
		\begin{tikzcd}
			& A \arrow[rrr, "f"] \arrow[rd] &                                      &                  & B \arrow[rd] &          \\
			\ker f \arrow[ru] &                               & \coim f \arrow[r, "\hat{f}", dashed] & \im f \arrow[ru] &              & \coker f
		\end{tikzcd}
	\end{center}
	
	Given a complex $L \to M \to N$ in an abelian category if $gf = 0$, one can define homology in three ways
	\begin{center}
		\begin{tikzcd}
			& \im f \arrow[rr, "\phi"] \arrow[rd] &                                         & \ker g \arrow[ld] &   \\
			L \arrow[rr, "f"] \arrow[ru] &                                     & M \arrow[rr, "g"] \arrow[rd] \arrow[ld] &                   & N \\
			& \coker f \arrow[rr, "\psi"]         &                                         & \im g \arrow[ru]  &  
		\end{tikzcd}
	\end{center}
	
	\begin{enumerate}
		\item $\coker (\im f \to \ker g)$
		\item $\ker (\coker f \to \coim g)$
		\item $\im (\ker g \to \coker f)$
	\end{enumerate}

	The first of these corresponds to the usual $\ker / \im$ and it is not very hard to show that all three ways give canonically isomorphic objects in an abelian category. It is essential to require the category to be abelian here, the three possibilities are distinct in a general additive category (with kernels and cokernels).
	\url{https://math.stackexchange.com/a/18112/700122}
\end{remark}



\begin{definition}[chain homotopy]
	Given two chain complexes $C_\bullet, D_\bullet$ in a \textbf{pointed preadditive category with kernels}. Let $f_\bullet, g_\bullet: C_\bullet \to D_\bullet$ be two chain maps. A \textbf{homotopy} from $f_\bullet$ into $g_\bullet$ is a collection of maps $h_n: C_{n-1} \to D_n$ such that $\partial h_{n+1} + h_n \partial = f_n - g_n$
	
	\begin{center}
		\begin{tikzcd}
			... & C_{n-1} \arrow[l, "\partial"'] \arrow[rd, "h_n"] & C_n \arrow[l, "\partial"'] \arrow[rd, "h_{n+1}"] & C_{n+1} \arrow[l, "\partial"'] & ... \arrow[l, "\partial"'] \\
			... & D_{n-1} \arrow[l, "\partial"]                    & D_n \arrow[l, "\partial"]                        & D_{n+1} \arrow[l, "\partial"]  & ... \arrow[l, "\partial"] 
		\end{tikzcd}
	\end{center}
	
	If there exists a homotopy from $f_\bullet$ into $g_\bullet$, then $f_\bullet$ and $g_\bullet$ are called \textbf{homotopic} and denoted by $f_\bullet \sim g_\bullet$. Being homotopic is an equivalence relation.
\end{definition}

\begin{definition}[homotopy equivalent]
	Two chain complexes $C_\bullet, D_\bullet$ are called \textbf{homotopy equivalent} if there exists chain maps $f: C_\bullet \to D_\bullet$ and $g: D_\bullet \to C_\bullet$ so that $gf \sim 1_{C_\bullet}$ and $fg \sim 1_{D_\bullet}$. Being homotopy equivalent is an equivalence relation and each equivalence class is called a \textbf{homotopy type}
\end{definition}

\begin{theorem}[fundamental theorem of homological algebra - FTHA]
	Let $C$ be a \textbf{pointed category with kernels} and $(\mathcal{P}, \mathcal{C})$ be a projective class in $C$. Given $f: M \to M'$ in $C$ and the diagram below
	\begin{center}
		\begin{tikzcd}
			0 & M \arrow[l] \arrow[d, "f"] & P_0 \arrow[l, "\epsilon"'] \arrow[d, "f_0", dashed] & P_1 \arrow[l, "d"'] \arrow[d, "f_1", dashed] & ... \arrow[l, "d"']  \\
			0 & M' \arrow[l]               & P'_0 \arrow[l, "\epsilon'"']                        & P'_1 \arrow[l, "d'"']                        & ... \arrow[l, "d'"']
		\end{tikzcd}
	\end{center}
	
	where both chains are chain complexes, the top chain consists of projectives $P_n$ and the bottom chain is exact. Then,
	\begin{itemize}
		\item There exists a chain map defined by $f_n: P_n \to P'_n$ for every $n \in \N_0$
		\item If $C$ is preadditive, the lift is unique upto chain homotopy.
	\end{itemize}
\end{theorem}

\begin{longproof}
	\begin{enumerate}
		\item The first statement is proved by induction
		\begin{center}
			\begin{tikzcd}
				P_{n-2} \arrow[dd, "f_{n-2}"'] & P_{n-1} \arrow[l] \arrow[dd, "f_{n-1}"'] &                     & P_n \arrow[dd, "f_n", dashed] \arrow[ll] \arrow[ld, dashed] \\
				&                                          & K'_{n-1} \arrow[ld] &                                                             \\
				P'_{n-2}                       & P'_{n-1} \arrow[l]                       &                     & P'_n \arrow[lu, two heads] \arrow[ll]                      
			\end{tikzcd}
		\end{center}
		
		Suppose there exist maps $f_{n-1}: P_{n-1} \to P'_{n-1}$ and $f_{n-2}: P_{n-2} \to P'_{n-2}$. Let $K'_{n-1} = \ker(P'_{n-1} \to P'_{n-2})$.
		
		Since the bottom chain is acyclic, the map $P'_n \to P'_{n-1}$ factors through $K'_{n-1}$ by an epimorphism.
		
		Since the top chain is a chain complex, the composition $P_n \to P_{n-1} \to P'_{n-1} \to P'_{n-2}$ equals $P_n \to P_{n-1} \to P_{n-2} \to P'_{n-2}$ and equals 0 zero, so $P_n \to P_{n-1} \to P'_{n-1}$ factors through $K'_{n-1}$
		
		Since $P_n$ is projective and $P'_n \to K'_{n-1}$ is an epimorphism, $P_n \to K'_{n-1}$ factors through $P'_n$ by a map $f_n: P_n \to P'_n$
		
		\textbf{Base case}: $n=0$, let $P_{n-1} = M, P'_{n-1} = M'$, $P_{n-2} = 0, P'_{n-2} = 0$ and $f_{n-1} = f, f_{n-2} = 0$
		
		\item Let $f^{(1)}_\bullet, f^{(2)}_\bullet: P_\bullet \to P'_\bullet$ be any two lifts from $f: M \to M'$
		
		\begin{center}
			\begin{tikzcd}
				M \arrow[d, "f"'] & P_\bullet \arrow[l, "\epsilon"] \arrow[d, "f^{(1)}_\bullet"', dashed, bend right] \arrow[d, "f^{(2)}_\bullet", dashed, bend left] \\
				M'                & P'_\bullet \arrow[l, "\epsilon'"]                                                                                                
			\end{tikzcd}
		\end{center}
		
		We will prove that $g_\bullet = f^{(1)}_\bullet - f^{(2)}_\bullet$ is chain homotopic to zero, that is to find maps $h_{n+1}: P_n \to P'_{n+1}$ such that $d'h + hd = g$
		
		\begin{center}
			\begin{tikzcd}
				0 \arrow[d, "0"'] & P_0 \arrow[l, "d"'] \arrow[d, "g_0"'] & P_1 \arrow[l, "d"'] \arrow[d, "g_1"'] & ... \arrow[l, "d"'] \\
				0                 & P'_0 \arrow[l, "d'"]                  & P'_1 \arrow[l, "d'"]                  & ... \arrow[l, "d'"]
			\end{tikzcd}
		\end{center}
		
		Suppose there exists map $h_{n-1}: P_{n-2} \to P'_{n-1}$ and $h_{n-2}: P_{n-3} \to P'_{n-2}$ such that
		$$
		g_{n-2} - h_{n-2} d = d' h_{n-1}
		$$
		
		\begin{center}
			\begin{tikzcd}
				P_{n-3} \arrow[rrdd, "h_{n-2}"] &  & P_{n-2} \arrow[rrdd, "h_{n-1}"] \arrow[ll, "d"'] &  & P_{n-1} \arrow[ll, "d"']  \\
				&  &                                                  &  &                           \\
				P'_{n-3}                        &  & P'_{n-2} \arrow[ll, "d'"]                        &  & P'_{n-1} \arrow[ll, "d'"]
			\end{tikzcd}
		\end{center}
		
		Consider the map $g_{n-1} - h_{n-1} d: P_{n-1} \to P'_{n-1}$,
		\begin{align*}
			d'(g_{n-1} - h_{n-1} d) 
			&= d' g_{n-1} - d' h_{n-1} d &\text{(preadditive)}\\
			&= d' g_{n-1} - (g_{n-2} - h_{n-2} d) d &\text{(induction)}\\
			&= d' g_{n-1} - g_{n-2} d &\text{(preadditive, $dd=0$)}\\
			&= 0 &\text{($g_\bullet$ is a chain map)}
		\end{align*}
		
		Let $K'_{n-1} = \ker(d': P'_{n-1} \to P'_{n-2})$.
		
		Since the bottom chain is acyclic, the map $d': P'_n \to P'_{n-1}$ factors through $K'_{n-1}$ by an epimorphism.
		
		\begin{center}
			\begin{tikzcd}
				P_{n-2} \arrow[rrdd, "h_{n-1}"] &  & P_{n-1} \arrow[ll, "d"'] \arrow[d, dashed] \arrow[rrdd, "h_n", dashed] &  &                                                      \\
				&  & K'_{n-1} \arrow[d]                                                     &  &                                                      \\
				P'_{n-2}                        &  & P'_{n-1} \arrow[ll, "d'"]                                              &  & P'_n \arrow[llu, two heads, dashed] \arrow[ll, "d'"]
			\end{tikzcd}
		\end{center}
		
		As $d'(g_{n-1} - h_{n-1} d) = 0$, $g_{n-1} - h_{n-1} d$ factors through $K'_{n-1}$, that is, $g_{n-1} - h_{n-1} d$ equals the composition $P_{n-1} \to K'_{n-1} \to P'_{n-1}$
		
		Since $P_{n-1}$ is projective and $P'_n \to K'_{n-1}$ is an epimorphism, $P_{n-1} \to K'_{n-1}$ factors through $P'_n$ by a map $h_n: P_{n-1} \to P'_n$, that is, the $d' h_n$ equals the composition $P_{n-1} \to P'_n \to K'_{n-1} \to P'_{n-1}$ and equals the composition $P_{n-1} \to K'_{n-1} \to P'_{n-1}$, hence
		
		$$
		d' h_n = g_{n-1} - h_{n-1} d
		$$
		
		\textbf{Base case}: $n=0$, let $P_{n-2} = 0, P'_{n-2} = 0$, $P_{n-1} = M, P'_{n-1} = M'$, $h_{n-1} = 0$, then 
		\begin{align*}
			d'(g_{n-1} - h_{n-1} d) 
			&= 0 &\text{($d': P'_{n-1} \to P'_{n-2}$ is the zero map $M' \to 0$)}
		\end{align*}
		
	\end{enumerate}
\end{longproof}

\begin{definition}[resolution, projective resolution]
	Let $M$ be an object in a \textbf{pointed category with kernels}. A \textbf{resolution} of $M$ is an exact sequence
	\begin{center}
		\begin{tikzcd}
			0 & M \arrow[l] & P_0 \arrow[l, "\epsilon"'] & P_1 \arrow[l, "d"'] & ... \arrow[l, "d"']
		\end{tikzcd}
	\end{center}
	
	If $P_n$ are projectives in a projective class $(\mathcal{P}, \mathcal{E})$, then the sequence is called $\mathcal{P}$-\textbf{projective resolution}.
	
	Similar definition for \textbf{injective resolution}
\end{definition}

\begin{corollary}
	Let $M$ be an object in a \textbf{pointed preadditive category with kernels}. Any two projective resolutions of $M$ are of the same chain homotopy type.
\end{corollary}

\begin{definition}[additive functor]
	Let $C, D$ be \textbf{preadditive categories}, a functor $F: C \to D$ is additive if for every $M, N \in \ob C$,
	$$
	\Hom(M, N) \to \Hom(F(M), F(N))
	$$
	
	is a homomorphism of abelian . In other words, if $f, g: M \to N$, then $F(f + g) = F(f) + F(g)$. Moreover, since $F$ is a functor if $h: N \to Q$, then $F(hf) = F(h) F(f)$. Hence, additive functor on a preadditive category resembles a morphism between two monoids
\end{definition}

\begin{remark}[additive functor preserves chain complex]
	Additive functor preserves chain complex. That is, if $C_\bullet$ is a chain complex, then
	\begin{center}
		\begin{tikzcd}
			... & C_{n-1} \arrow[l, "d"']    & C_n \arrow[l, "d"']    & C_{n+1} \arrow[l, "d"']    & ... \arrow[l, "d"']  \\
			... & F C_{n-1} \arrow[l, "Fd"'] & F C_n \arrow[l, "Fd"'] & F C_{n+1} \arrow[l, "Fd"'] & ... \arrow[l, "Fd"']
		\end{tikzcd}
	\end{center}
	
	the bottom sequence is also a chain complex.
	$$
	(Fd)(Fd) = F(dd) = F(0) = 0
	$$
\end{remark}

\begin{remark}[additive functor preserves chain homotopy]
	Additive functor preserves chain homotopy. That is, if $f, g: C_\bullet \to D_\bullet$ are chain homotopic by a chain homotopy $h$, then, $F(h)$ is a chain homotopy from $Ff$ to $Fg$
	
	\begin{align*}
		(Fd) (Fh) + (Fh) (Fd)
		&= F(dh) + F(hd) &\text{($F$ is a functor)} \\
		&= F(dh + hd) &\text{($F$ is additive)} \\
		&= F(f - g) &\text{($f \simeq g$ by $h$)} \\
		&= F(f) - F(g) &\text{($F$ is a functor)}
	\end{align*}
	
\end{remark}

\begin{remark}
	a functor between additive categories is additive if and only if it preserves finite coproducts
	- Mac Lane's “Categories for the working mathematician”
\end{remark}

\note{TODO - https://math.stackexchange.com/questions/793029/do-covariant-functors-preserve-direct-sums}

\begin{comment}
	\begin{definition}[right exact functor]
		Let $C, D$ be \textbf{pointed preadditive categories with kernels}, a projective class $(\mathcal{P}, \mathcal{E})$ and a short exact sequence
		$$
		0 \to A \to B \to C \to 0
		$$
		
		A functor $F: C \to D$ is called left exact if
		$$
		0 \to F(A) \to F(B) \to F(C)
		$$
		
		is exact
		
	\end{definition}
	
	\note{TODO - on the behavior of functor on zero object}
	
	\begin{remark}
		In an \textbf{abelian category}, every left exact functor is additive.
	\end{remark}
\end{comment}

\begin{definition}[left derived functor]
	Let $C$ be a \textbf{pointed preadditive category with kernels}, a projective class $(\mathcal{P}, \mathcal{E})$. Let $F: C \to \Ab$ be an additive functor, then the \textbf{left derived functor} of $F$ with respect to $(\mathcal{P}, \mathcal{E})$ are $L_n F: C \to \Ab$ defined by
	
	$$
	(L_n F) (X) = H_n(F P_\bullet)
	$$
	
	where $X \leftarrow P_0 \leftarrow P_1 \leftarrow ...$ is a $\mathcal{P}$-projective resolution.
\end{definition}

\begin{remark}
	One can define the dual notion of right derived functor that is left derived functor by cohomology of injective resolution.
\end{remark}

\begin{remark}
	As $F$ is additive and any two projective resolutions of an object are chain homotopic, the left derived functor is well-defined.
\end{remark}

\begin{remark}
	$\Tor_n^A(M, -)$ is a left derived functor in the cateogry of $R$-modules
\end{remark}

\begin{proposition}
	For any short exact sequence
	$$
		0 \to L \to M \to N \to 0
	$$
	
	There exists a natural long exact sequence
	\begin{center}
		\begin{tikzcd}
			&                 & L_2 FN \arrow[lld] &   \\
			L_1 FL \arrow[r] & L_1 FM \arrow[r] & L_1 FN \arrow[lld] &   \\
			FL \arrow[r] & FM \arrow[r] & FN \arrow[r]   & 0
		\end{tikzcd}
	\end{center}
\end{proposition}

\begin{remark}
	If $F$ is an exact functor, then $L_1 F N = 0$. $L_n$ measures how far $F$ from being exact.
\end{remark}

\begin{proof}
	\note{Proof using fundamental lemma of HA}
\end{proof}

\section{UNIVERSAL COEFFICIENT THEOREM FOR CHAIN COMPLEXES OF $R$-MODULES}

\begin{remark}[$\RMod$ is abelian]
	If $R$ is a commutative ring, the category of $R$-modules is abelian
\end{remark}

\begin{remark}[projective module, projective class in $\RMod$]
	In the category $\RMod$, there is a projective class $(\mathcal{P}, \mathcal{E})$ defined by epimorphism being surjective homomorphism. Then, the following are equivalent
	\begin{enumerate}
		\item $P \in \ob \RMod$ is projective
		\item Every short exact sequence $0 \to M \to N \to P \to 0$ splits
		\item $P$ is a direct summand of a free $R$-module, that is, there exists $Q \in \ob \RMod$ such that $P \oplus Q$ is a free $R$-module.
	\end{enumerate}    
\end{remark}

\begin{proof}
	\note{TODO}
\end{proof}

\begin{definition}[Tor functor]
	In the category $\RMod$, define $\Tor_n: \RMod \times \RMod \to \Ab$ such that $\Tor(-, M)$ is the $n$-th left derived functor of $(- \times M)$
	$$
	\Tor_n(N, M) = (L_n (- \otimes M))(N) = H_n(P_\bullet \otimes M)
	$$
	
	where $N \leftarrow P_0 \leftarrow P_1 \leftarrow ...$ is a projective resolution of $N$
\end{definition}

\begin{remark}[some properties of $\Tor$]
	\begin{align*}
		\Tor_n(A, B) &= \Tor_n(B, A) \\
		\Tor_n (\varinjlim_\alpha A_\alpha, B) &= \varinjlim_\alpha \Tor_n (A_\alpha, B) \\
		\Tor_n(P, B) &= 0 &\text{(if $P$ is projective)} \\
	\end{align*}
	\note{TODO}
\end{remark}

\begin{definition}[direct sum of chain complexes of $R$-module - product, coproduct, biproduct]
	In the category $\Ch(\RMod)$, the direct sum $\oplus: \Ch(\RMod) \times \Ch(\RMod) \to \Ch(\RMod)$ is defined as follows:
	$$
	(C_\bullet \oplus D_\bullet)_n = C_n \oplus D_n
	$$
	
	where $C_\bullet, D_\bullet \in \ob \Ch(\RMod)$ and the boundary map $\partial: (C_\bullet \oplus D_\bullet)_n \to (C_\bullet \oplus D_\bullet)_{n-1}$ is defined by
	\begin{align*}
		\partial:   (C \oplus D)_n &\to (C \oplus D)_{n-1} \\
		c \oplus d &\mapsto \partial c \oplus \partial d
	\end{align*}
	
\end{definition}

\begin{definition}[tensor product of chain complexes of $R$-module]
	In the category $\Ch(\RMod)$, the tensor product $\otimes: \Ch(\RMod) \times \Ch(\RMod) \to \Ch(\RMod)$ is defined as follows:
	$$
	(C_\bullet \otimes D_\bullet)_n = \bigoplus_{p + q = n} C_p \otimes D_q
	$$
	
	where $C_\bullet, D_\bullet \in \ob \Ch(\RMod)$ and the boundary map $\partial: (C_\bullet \otimes D_\bullet)_n \to (C_\bullet \otimes D_\bullet)_{n-1}$ is the linear extension of $\partial: C_p \otimes D_q \to (C_\bullet \otimes D_\bullet)_{n-1}$ where
	$$
	\partial (c \otimes d) = \partial c \otimes d + (-1)^{p} c \otimes \partial d
	$$
\end{definition}

\begin{proof}
	\note{TODO - bilinear chain map factors through tensor product}
\end{proof}

\begin{lemma}[on the spot categorical lemma]
	Let $F, G: C \to D$ be functors that are natural to the identity functor. Let $f: A \to B$ be a morphism in $C$, if $FA = GA$ and $FB = GB$, then $Ff = Gf$
	\note{if $C = D$, then it produces a bunch of commutative squares, easy to prove}
	
	\begin{center}
		\begin{tikzcd}
			A \arrow[d, "f"] & FA \arrow[r, "="] \arrow[d, "Ff"] & GA \arrow[d, "Gf"] \\
			B                & FB \arrow[r, "="]                 & GB                
		\end{tikzcd}
	\end{center}
\end{lemma}

\begin{theorem}[universal coefficient theorem - UCT]
	In the category $\RMod$ where $R$ is a PID. Let $C_\bullet$ be a chain complex of free $R$-modules and $N$ be an $R$-module. There is a natural short exact sequence (the map $H_n(C_\bullet) \otimes N \to H_n(C_\bullet \times N)$ is natural)
	\begin{center}
		\begin{tikzcd}
			0 \arrow[r] & H_n(C_\bullet) \otimes N \arrow[r] & H_n(C_\bullet \otimes N) \arrow[r] & {\Tor_1(H_{n-1}(C_\bullet), N)} \arrow[r] & 0
		\end{tikzcd}
	\end{center}
	
	and this sequence splits (but not naturally)
\end{theorem}

\begin{longproof}
	We have the short exact sequence of chain complexes of free $R$-modules
	\begin{center}
		\begin{tikzcd}
			0 \arrow[r] & Z_\bullet \arrow[r, hook] & C_\bullet \arrow[r] & B_{\bullet -1} \arrow[r, two heads] & 0
		\end{tikzcd}
	\end{center}
	
	where $Z_n = \ker(\partial: C_n \to C_{n-1})$ and $B_n = \im (\partial: C_{n+1} \to C_n)$ and the boundary maps on $Z_\bullet$ and $B_{\bullet-1}$ are zeros. Note that, $B_{n-1}$ is free since it is a submodule of free $R$-module $C_n$, then the sequence splits. Then, the sequence below is exact and splits
	
	\begin{center}
		\begin{tikzcd}
			0 \arrow[r] & Z_\bullet \otimes N \arrow[r, hook] & C_\bullet \otimes N \arrow[r, two heads] & B_{\bullet -1} \otimes N \arrow[r] & 0
		\end{tikzcd}
	\end{center}
	
	where the boundary maps on $Z_\bullet \otimes N$ and $B_{\bullet-1} \otimes N$ are zeros and boundary map on $C_\bullet \otimes N$ is $\partial \otimes 1: c \otimes n \mapsto \partial c \otimes n$
	
	That induces a long exact sequence
	
	\begin{center}
		\begin{tikzcd}
			& ... \arrow[r]                              & H_{n+1}(B_{\bullet-1} \otimes N) \arrow[lld, "(i_n)_*"'] \\
			H_n(Z_\bullet \otimes N) \arrow[r]     & H_n(C_\bullet \otimes N) \arrow[r] & H_n(B_{\bullet-1} \otimes N) \arrow[lld, "(i_{n-1})_*"'] \\
			H_{n-1}(Z_\bullet \otimes N) \arrow[r] & ...                                        &                                                                 
		\end{tikzcd}
	\end{center}
	
	where the connecting homomorphism $(i_n)_*$ is induced by inclusion map $i_n: B_{n-1} \otimes N \to Z_n \otimes N$
	
	\begin{center}
		\begin{tikzcd}
			&                         & B_{n-1} \otimes N \arrow[lld, "i_n"', hook] \arrow[ld, hook] \\
			Z_n \otimes N & C_n \otimes N \arrow[l] &                                                             
		\end{tikzcd}
	\end{center}
	
	That induces a short exact sequence
	
	\begin{center}
		\begin{tikzcd}
			0 \arrow[r] & \coker(i_n)_* \arrow[r] & H_n(C_\bullet \otimes N) \arrow[r] & \ker(i_{n-1})_* \arrow[r] & 0
		\end{tikzcd}
	\end{center}
	
	We have another short exact sequence
	
	\begin{center}
		\begin{tikzcd}
			0 \arrow[r] & B_n \arrow[r, "j", hook] & Z_n \arrow[r, two heads] & H_n(C_\bullet) \arrow[r] & 0
		\end{tikzcd}
	\end{center}
	
	Then, the sequence below is exact (\note{see the motivation of left derived functor})
	\begin{center}
		\begin{tikzcd}
			... \arrow[rr]                                       &  & {\Tor_1(Z_n, N)} \arrow[r]          & {\Tor_1(H_n(C_\bullet), N)} \arrow[llld] &   \\
			B_n \otimes N \arrow[rr, "j \otimes 1 = i_n"', hook] &  & Z_n  \otimes N \arrow[r, two heads] & H_n(C_\bullet) \otimes N \arrow[r]       & 0
		\end{tikzcd}
	\end{center}
	
	Note that, both $i_n$ and $(i_n)_*$ are induced from the inclusion $B_n \to Z_n$ and we have a natural transformation of functors from the category of pairs of a chain complex and an $R$-module to $\Ab$,
	\begin{align*}
		(B_\bullet, N) &\mapsto H_n(B_\bullet) \otimes N \\
		(B_\bullet, N) &\mapsto H_n(B_\bullet \otimes N)
	\end{align*}
	
	so $(i_n)_* = i_n$
	
	
	\begin{center}
		\begin{tikzcd}
			B_n \arrow[rr, hook]                                                               &  & Z_n                                          \\
			B_n \otimes N = H_n(B_\bullet) \otimes N \arrow[rr, "j \otimes 1 = i_n", hook] \arrow[d, "\cong"'] &  & Z_n  \otimes N = H_n(Z_\bullet) \otimes N \arrow[d, "\cong"] \\
			H_n(B_\bullet \otimes N) \arrow[rr, "(i_n)_*"']                                    &  & H_n(Z_\bullet  \otimes N)                   
		\end{tikzcd}
	\end{center}
	
	As $Z_n$ is free, $\Tor_1(Z_n, N) = 0$,  then 
	\begin{align*}
		\coker(i_n)_* &= H_n(C_\bullet) \otimes N \\
		\ker (i_n)_* &= \Tor_1(H_n(C_\bullet), N)
	\end{align*}
	
	We have the short exact sequence
	
	\begin{center}
		\begin{tikzcd}
			0 \arrow[r] & H_p(C_\bullet) \otimes N \arrow[r] & H_n(C_\bullet \otimes N) \arrow[r] & \Tor_1(H_{n-1}(C_\bullet), N) \arrow[r] & 0
		\end{tikzcd}
	\end{center}
	
	The split of this sequence is from the map $\beta: H_n(C_\bullet \otimes N) \to \coker(i_n)_*$ induced from projection map $C_n \to Z_n$
\end{longproof}










\section{KÜNNETH THEOREM FOR CHAIN COMPLEXES OF $R$-MODULES}

\begin{theorem}[Künneth theorem]
	In the category $\RMod$ where $R$ is a PID, let $C_\bullet, D_\bullet$ be chain complexes of $R$-modules, and $C_\bullet$ is degree-wise free (each $C_n$ is a free $R$-module). Then, there is a natural short exact sequence (homology cross product is natural)
	\begin{center}
		\begin{tikzcd}
			0 \arrow[r] & \bigoplus_{p+q=n} H_p(C_\bullet) \otimes H_q(D_\bullet) \arrow[r, "\times"] & H_n(C_\bullet \otimes D_\bullet) \arrow[r] & {\bigoplus_{p+q=n-1} \Tor_1 (H_p(C_\bullet), H_q(D_\bullet))} \arrow[r] & 0
		\end{tikzcd}
	\end{center}
	and this sequence splits (but not naturally)
\end{theorem}

\begin{longproof}
	\begin{enumerate}
		\item \textbf{Case 1}: boundary map of $C_\bullet$ is zero
		
		The boundary map $(C_\bullet \otimes D_\bullet)_n \to (C_\bullet \otimes D_\bullet)_{n-1}$ is the linear extension of
		\begin{align*}
			\partial:   C_p \otimes D_q &\to C_p \otimes D_{q-1} \\
			c \otimes d &\mapsto (-1)^{|c|} c \otimes \partial d
		\end{align*}
		
		Hence, we can write the tensor product $C_\bullet \otimes D_\bullet$ as a direct sum of chain complexes
		$$
		C_\bullet \otimes D_\bullet = \bigoplus_p C_p \otimes D_{\bullet - p}
		$$
		
		We have
		\begin{align*}
			H_n(C_\bullet \otimes D_\bullet) 
			&= H_n\tuple*{\bigoplus_p C_p \otimes D_{\bullet-p}} \\
			&= \bigoplus_p H_n(C_p \otimes D_{\bullet-p}) \\
			&= \bigoplus_p C_p \otimes H_n(D_{\bullet - p}) &\text{($C_p$ is free, cons of UCT)}\\
			&= \bigoplus_{p + q = n} C_p \otimes H_q(D_\bullet)&\text{(shifted chain complex)}
		\end{align*}
		
		\item \textbf{Case 2}: $C_\bullet$ is an arbitrary chain complex
		
		We have the short exact sequence of chain complexes of free $R$-modules
		\begin{center}
			\begin{tikzcd}
				0 \arrow[r] & Z_\bullet \arrow[r, hook] & C_\bullet \arrow[r] & B_{\bullet -1} \arrow[r, two heads] & 0
			\end{tikzcd}
		\end{center}
		
		where $Z_n = \ker(\partial: C_n \to C_{n-1})$ and $B_n = \im (\partial: C_{n+1} \to C_n)$ and the boundary maps on $Z_\bullet$ and $B_{\bullet-1}$ are zeros. Note that, $B_{n-1}$ is free since it is a submodule of free $R$-module $C_n$, then the sequence splits. Then, the sequence below is exact and splits (\note{different from the proof of UCT, this uses split, direct sum, tensor product of sequence of chain complexes})
		
		\begin{center}
			\begin{tikzcd}
				0 \arrow[r] & Z_\bullet \otimes D_\bullet \arrow[r, hook] & C_\bullet \otimes D_\bullet \arrow[r, two heads] & B_{\bullet -1} \otimes D_\bullet \arrow[r] & 0
			\end{tikzcd}
		\end{center}
		
		That induces a long exact sequence in homology
		\begin{center}
			\begin{tikzcd}
				& ... \arrow[r]                              & H_{n+1}(B_{\bullet-1} \otimes D_\bullet) \arrow[lld, "(i_n)_*"'] \\
				H_n(Z_\bullet \otimes D_\bullet) \arrow[r]     & H_n(C_\bullet \otimes D_\bullet) \arrow[r] & H_n(B_{\bullet-1} \otimes D_\bullet) \arrow[lld, "(i_{n-1})_*"'] \\
				H_{n-1}(Z_\bullet \otimes D_\bullet) \arrow[r] & ...                                        &                                                                 
			\end{tikzcd}
		\end{center}
		
		where the connecting homomorphism $(i_n)_*$ are induced by inclusion map $i_n: (B_{\bullet-1} \otimes D_\bullet)_{n+1} \to (Z_\bullet \otimes D_\bullet)_n$
		
		\begin{center}
			\begin{tikzcd}
				&                                                 & (B_{\bullet-1} \otimes D_\bullet)_{n+1} \arrow[lld, "i_n"'] \arrow[ld, hook] \\
				(Z_\bullet \otimes D_\bullet)_n & (C_\bullet \otimes D_\bullet)_n \arrow[l, hook] &                                                                           
			\end{tikzcd}
		\end{center}
		
		That induces a short exact sequence
		\begin{center}
			\begin{tikzcd}
				0 \arrow[r] & \coker(i_n)_* \arrow[r] & H_n(C_\bullet \otimes D_\bullet) \arrow[r] & \ker(i_{n-1})_* \arrow[r] & 0
			\end{tikzcd}
		\end{center}
		
		We have another short exact sequence
		
		\begin{center}
			\begin{tikzcd}
				0 \arrow[r] & B_p \arrow[r, "j", hook] & Z_p \arrow[r, two heads] & H_p(C_\bullet) \arrow[r] & 0
			\end{tikzcd}
		\end{center}
		
		Then, the sequence below is exact (\note{see the motivation of left derived functor})
		\begin{center}
			\begin{tikzcd}
				... \arrow[rr]                                                    &  & {\Tor_1(Z_p, H_q(D_\bullet))} \arrow[r]          & {\Tor_1(H_p(C_\bullet), H_q(D_\bullet))} \arrow[llld] &   \\
				B_p \otimes H_q(D_\bullet) \arrow[rr, "j \otimes 1 = i_n"', hook] &  & Z_p  \otimes H_q(D_\bullet) \arrow[r, two heads] & H_p(C_\bullet) \otimes H_q(D_\bullet) \arrow[r]       & 0
			\end{tikzcd}
		\end{center}
		
		Take the direct sum over all pairs $p + q = n$ and note that both $i_n$ and $(i_n)_*$ are induced from the the inclusion $B_p \to Z_p$ we have a natural transformation of functors from the category of pairs of chain complexes to $\Ab$
		\begin{align*}
			(A_\bullet, B_\bullet) &\mapsto H_n(A_\bullet \otimes B_\bullet) \\
			(A_\bullet, B_\bullet) &\mapsto \bigoplus_{p+q=n} H_p(A_\bullet) \otimes H_q(B_\bullet) \\
		\end{align*}
		
		so $(i_n)_* = i_n$
		
		\begin{center}
			\begin{tikzcd}
				B_p \arrow[rr, hook]                                                                        &  & Z_p                                                   \\
				\bigoplus_{p+q=n} B_p \otimes H_q(D_\bullet) \arrow[rr, "j \otimes 1 = i_n", hook] \arrow[d, "="'] &  & \bigoplus_{p+q=n} Z_p  \otimes H_q(D_\bullet) \arrow[d, "="] \\
				H_{n+1}(B_{\bullet-1} \otimes D_\bullet) \arrow[rr, "(i_n)_*"']                                               &  & H_n(Z_\bullet \otimes D_\bullet)                                       
			\end{tikzcd}
		\end{center}
		
	\end{enumerate}
	
	As $Z_p$ is free, $\Tor_1(Z_p, H_q(D_\bullet)) = 0$, then 
	\begin{align*}
		\coker(i_n)_* &= \bigoplus_{p+q=n} H_p(C_\bullet) \otimes H_q(D_\bullet) \\
		\ker (i_n)_* &= \bigoplus_{p+q=n} \Tor_1(H_p(C_\bullet), H_q(D_\bullet))
	\end{align*}
	
	We have the short exact sequence
	
	\begin{center}
		\begin{tikzcd}
			0 \arrow[r] & \bigoplus_{p+q=n} H_p(C_\bullet) \otimes H_q(D_\bullet) \arrow[r, "\times"] & H_n(C_\bullet \otimes D_\bullet) \arrow[r] & {\bigoplus_{p+q=n-1} \Tor_1 (H_p(C_\bullet), H_q(D_\bullet))} \arrow[r] & 0
		\end{tikzcd}
	\end{center}
	
	The split of this sequence is from the map $\beta: H_n(C_\bullet \otimes D_\bullet) \to \coker(i_n)_*$ induced from projection map $(C_\bullet \otimes D_\bullet)_n \to (Z_\bullet \otimes D_\bullet)_n$
	
\end{longproof}




\section{REAL-WORLD APPLICATIONS}

\subsection{UNIVERSAL COEFFICIENT THEOREM FOR TOPOLOGICAL SPACES}

\begin{theorem}[universal coefficient theorem - UCT]
	Let $X$ be a topological space and $C_\bullet(X)$ be the singular chain complex of $X$. The singular chain complex with coefficients $N$ be defined by
	$$
	C_\bullet(X, N) = C_\bullet(X) \otimes N
	$$
	
	The homology group with coefficients $N$ is defined by
	$$
	H_n(X, N) = H_n(C_\bullet(X, N))
	$$
	
	Then, there is a short exact sequence
	\begin{center}
		\begin{tikzcd}
			0 \arrow[r] & H_n(X) \otimes N \arrow[r] & H_n(X, N) \arrow[r] & {\Tor_1(H_{n-1}(X), N)} \arrow[r] & 0
		\end{tikzcd}
	\end{center}
	
	and this sequence splits (but not naturally)
	
\end{theorem}

\subsection{EILENBERG-ZILBER THEOREM}

\begin{remark}[$\Fun(C, \RMod)$]
	Given a category $C$, $\Fun(C, \RMod)$ is a \textbf{pointed preadditive category with kernels} (more precisely, abelian category - will define in the future).
	
	Given a morphism $F \to G$ in $\Fun(C, \RMod)$ (a natural transformation from $F$ to $G$), then the kernel of $F \to G$ is a morphism $K \to F$ such that $K(X)$ is the kernel of $F(X) \to G(X)$ for all $X \in \ob C$
\end{remark}

\begin{remark}[models define projective class in $\Fun(C, \RMod)$]
	Let $\mathcal{M}$ be any set of objects in $C$ (called models), then $\mathcal{M}$ defines a projective class $(\mathcal{P}, \mathcal{E})$ in $\Fun(C, \RMod)$ where a morphism $G \to F$ is an epimorphism (relative to $\mathcal{M}$) if for all $M \in \mathcal{M}$, $G(M) \twoheadrightarrow F(M)$ is surjective. Then, the following are equivalent
	
	\begin{enumerate}
		\item $P \in \ob \Fun(C, \RMod)$ is projective
		\item $P$ is a \textbf{retract of coproduct} of $R \Hom(M, -)$ for some $M \in \mathcal{M}$ where $\Hom(M, -)$ is a functor $\RMod \to \Set$, $R$ is the free $R$-module functor $\Set \to \RMod$. In the case of $R$-module, \textbf{retract of coproduct} is the \textbf{direct summand} of a $R$-module
	\end{enumerate}
\end{remark}

\begin{proof}
	\note{TODO - prove using Yoneda lemma}
\end{proof}

\begin{remark}
	Let $C = \Top$ and model $\mathcal{M} = \set{\Delta^n: n=0, 1, ...}$, then for each $n$, $R \Hom(\Delta^n, -)$ is projective. Note that, $R \Hom(\Delta^n, -)$ is the $n$-singular chain complexes with coefficients in $R$ denoted by $C_n$. Moreover, the sequence below is a projective resolution of the zero-th homology functor $H_0$
	
	\begin{center}
		\begin{tikzcd}
			0 & H_0 \arrow[l] & C_0 \arrow[l, "\epsilon"'] & C_1 \arrow[l, "\partial"'] & ... \arrow[l, "\partial"']
		\end{tikzcd}
	\end{center}
	
	Let $\Ev_X$ be the evaluation functor of topological space $X$, then the left derived functor $\Ev_X$ on $H_0$ is the $n$-th singular homology of $X$
	$$
	(L_n \Ev_X)(H_0) = H_n(C_\bullet(X)) = H_n(X)
	$$
\end{remark}


\begin{theorem}[Eilenberg-Zilber theorem]
	Let $X, Y$ be topological spaces and $C_\bullet$ be the singular chain complex functor from $\Top$ to $\RMod$, then there are two chain maps
	\begin{align*}
		F&: C_\bullet(X \times Y) \to C_\bullet(X) \otimes C_\bullet(Y) \\
		G&: C_\bullet(X) \otimes C_\bullet(Y) \to C_\bullet(X \times Y)
	\end{align*}
	such that $FG$ and $GF$ are chain homotopic to identity. That is, $C_\bullet(X \times Y)$ and $C_\bullet(X) \times C_\bullet(Y)$ are of the same chain homotopy type.
\end{theorem}

\begin{longproof}
	\note{idea of proof}
	
	In the category of $\Top \times \Top$, let model $\mathcal{M} = \set{(\Delta^p, \Delta^q): p, q \geq 0}$. These functors $\Top \times \Top \to \RMod$ are projective
	
	\begin{align*}
		C_n(X \times Y) &= R [\Hom(\Delta^n, X) \times \Hom(\Delta^n, Y)] \\
		C_p(X) \otimes C_q(Y) &= R [\Hom(\Delta^p, X) \times \Hom(\Delta^q, Y)] \\
	\end{align*}
	
	As $H_0(X \times Y) \to H_0(X) \to H_0(Y)$ is an isomorphism, by FTHA, that induces a chain homotopy equivalence.
	
	\begin{center}
		\begin{tikzcd}
			0 & H_0(X \times Y) \arrow[l] \arrow[d, "="'] & C_0(X \times Y) \arrow[l] \arrow[d, dashed]     & C_1(X \times Y) \arrow[l] \arrow[d, dashed]     & ... \arrow[l] \\
			0 & H_0(X) \otimes H_0(Y) \arrow[l]            & (C_\bullet(X) \otimes C_\bullet(Y))_0 \arrow[l] & (C_\bullet(X) \otimes C_\bullet(Y))_1 \arrow[l] & ... \arrow[l]
		\end{tikzcd}
	\end{center}
\end{longproof}

\begin{corollary}
	Same chain homotopy type induces isomorphism in homology, that is
	$$
	H_n(X \times Y) \cong H_n(C_\bullet(X) \otimes C_\bullet(Y))
	$$
\end{corollary}






\subsection{KÜNNETH THEOREM FOR TOPOLOGICAL SPACES}

\begin{theorem}[Künneth theorem]
	In the category $\RMod$ where $R$ is a PID, let $C_\bullet, D_\bullet$ be chain complexes of $R$-modules, and $C_\bullet$ is degree-wise free (each $C_n$ is a free $R$-module). Then, there is a short exact sequence
	\begin{center}
		\begin{tikzcd}
			0 \arrow[r] & \bigoplus_{p+q=n} H_p(C_\bullet) \otimes H_q(D_\bullet) \arrow[r, "\times"] & H_n(C_\bullet \otimes D_\bullet) \arrow[r] & {\bigoplus_{p+q=n-1} \Tor_1 (H_p(C_\bullet), H_q(D_\bullet))} \arrow[r] & 0
		\end{tikzcd}
	\end{center}
	and this sequence splits (but not naturally)
\end{theorem}

\begin{theorem}[Künneth theorem]
	Let $X, Y$ be topological spaces and $R$ be a PID, there is a natural short exact sequence
	\begin{center}
		\begin{tikzcd}
			0 \arrow[r] & \bigoplus_{p+q=n} H_p(X; R) \otimes H_q(Y; R) \arrow[r, "\times"] & H_n(X \times Y) \arrow[r] & {\bigoplus_{p+q=n-1} \Tor_1 (H_p(X; R), H_q(Y; R))} \arrow[r] & 0
		\end{tikzcd}
	\end{center}
	and this sequence splits (but not naturally).
\end{theorem}

\begin{corollary}
	If $H_\bullet(X; R)$ is torsion free over $R$ (\note{of the form $R \oplus R \oplus ...$ without any $R / nR$ term}), then
	$$
	H_n(X \times Y) = \bigoplus_{p+q=n} H_p(X; R) \otimes H_q(Y; R)
	$$
\end{corollary}

Reserve for theory of categories