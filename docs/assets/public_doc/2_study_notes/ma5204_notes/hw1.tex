\chapter{HOMEWORK 1}

\begin{problem}[chapter 1 problem 1]
	Let $x$ be a nilpotent element of a ring $A$. Show that $1 + x$ is a unit of $A$. Deduce that the sum of a nilpotent element and a unit is a unit.
\end{problem}

\begin{proof}
	Let $x \in \eta_R$, then $-x \in \eta_R$, that is $(-x)^n = 0$ for some $n > 0$. We have
	$$
		1 = 1 - (-x)^n = (1 + x)(1 + (-x) + (-x)^2 + ... + (-x)^{n-1})
	$$
	
	Hence, $1 + x$ is a unit. Now let $uv = 1$ for $u, v \in R$, we have
	$$
		(u + x)(v + x) = uv + ux + xv + x^2 = 1 + (ux + xv + x^2)
	$$
	
	Since $\eta_R$ is an ideal, $ux + xv + x^2 \in \eta_R$, therefore, $1 + (ux + xv + x^2)$ is a unit. Let $w \in R$ be the inverse of $1 + (ux + xv + x^2)$, we have
	$$
		(u + x)(v + x) w = 1
	$$
	
	Hence, both $u + x$ and $v + x$ are units.
\end{proof}

\begin{problem}[chapter 1 problem 2]
	Let $A$ be a ring and let $A[x]$ be the ring of polynomials in an indeterminate $x$ with coefficients in $A$. Let $f = a_0 + a_1 x + ... + a_n x^n \in A[x]$. Prove that
	\begin{enumerate}
		\item $f$ is a unit in $A[x]$ $\iff$ $a_0$ is a unit in $A$ and $a_1, ..., a_n$ are nilpotent.
		\item $f$ is nilpotent $\iff$ $a_0, a_1, ..., a_n$ are nilpotent
		\item $f$ is a zero divisor $\iff$ there exists $a \neq 0$ in $A$ such that $af = 0$
		\item $f$ is said to be primitive if $(a_0, a_1, ..., a_n) = (1)$. Prove that if $f, g \in A[x]$, then $fg$ is primitive $\iff$ $f$ and $g$ are primitive
	\end{enumerate}
\end{problem}

\begin{longproof}
	\begin{enumerate}
		\item ($f$ is a unit in $A[x]$ $\iff$ $a_0$ is a unit in $A$ and $a_1, a_2, ..., a_n$ are nilpotent) The statement is true for degree zero polynomials. Suppose $n \geq 1$
		
		($\implies$) Let the inverse of $f(x)$ be polynomial $g(x) = b_0 + b_1 x + b_2 x^2 + ...$ of degree $m$, that is $b_m \neq 0$ and $b_{m+1} = b_{m+2} = ... = 0$, then $f(x) g(x) = 1$ implies $a_0 b_0 = 1$, hence $a_0$ is a unit. We will show that $a_n^{r+1} b_{m-r} = 0$ for all $r=0, 1, ..., m$ by induction. The statement is true when $r=0$ since $a_n b_m = 0$. When $0 < r \leq m$, assume that the statement is true for all $0,1,..., r-1$, that is
		\begin{align*}
			a_n b_m &= 0 \\
			a_n^2 b_{m-1} &= 0 \\
			... \\
			a_n^r b_{m-r+1} &= 0
		\end{align*}
		
		We want to show that $a_n^{r+1} b_{m-r} = 0$. The degree $n+m-r$ coefficient of $f(x) g(x)$ is zero, that is
		$$
			a_n b_{m-r} + a_{n-1} b_{m-r+1} + ... + a_0 b_{m-r+n} = 0
		$$
		
		Multiply both sides by $a_n^r$, we have
		$$
			a_n^{r+1} b_{m-r} + a_{n-1} a_n^r b_{m-r+1} + ... + a_0 a_n^r b_{m-r+n} = 0
		$$
		
		By the induction assumption, we induce that $a_n^{r+1} b_{m-r} = 0$ for all $r=0,1,..., m$. Let $r = m$, then $a_n^{m+1} b_0 = 0$. Since $b_0$ is a unit, $a_n^{m+1} = 0$, that is, $a_n \in \eta_A$. Note that $a_n \in \eta_{A[x]}$ is also nilpotent in the ring $A[x]$. Therefore, $f(x) - a_n x^n$ is a sum of a unit and a nilpotent element which is a unit in $A[x]$. Hence, using the same proof, $a_{n-1} \in \eta_A$. Inductively, all $a_1, a_2, ..., a_n \in \eta_A$
		
		($\impliedby$) If $a_0$ is a unit and $a_1, ..., a_n \in \eta_A$, then $a_0$ is also a unit in $A[x]$ and $a_1 x, ..., a_n x^n \in \eta_{A[x]}$. Therefore, $f(x) = a_0 + a_1 x + ... + a_n x^n$ is a sum of a unit and a nilpotent element which is a unit in $A[x]$
 		
 		\item ($f$ is nilpotent $\iff$ $a_0, a_1, ..., a_n$ are nilpotent)
 		
 		
 		($\implies$) If $f(x)$ is nilpotent, $1 + f(x) = (1 + a_0) + a_1 x + a_2 x^2 + ...$ is a unit. Then, $1 + a_0$ is a unit and $a_1, ..., a_n \in \eta_A$. Moreover, $f(x)^m = 0$ for some $m > 0$. That implies $a_0^m = 0$ for some $m > 0$. Hence, $a_0$ is also nilpotent.
 		
 		($\impliedby$) If $a_0, a_1, ..., a_n \in \eta_A$, then $a_0^m = a_1^m = ... = a_n^m = 0$ for some $m > 0$. $f(x)^{(n+1)m}$ is a sum of terms, where each term is a product of $(n+1)m$ elements from the set
 		$$
 			\set{a_0, a_1 x, ..., a_n x^n}
 		$$
 		
 		By pigeonhole principle, for every term, there is an element appearing at least $m$ times. Hence, $f(x)^{(n+1)m}$ is a zero polynomial, that is, $f(x) \in \eta_{A[x]}$
 		
 		\item ($f$ is a zero divisor $\iff$ there exists $a \neq 0$ in $A$ such that $af = 0$)
 		
		($\implies$) Let $g(x) = b_0 + b_1 x + b_2 x^2 + ... \in A[x]$ such that $f(x) g(x) = 0$. The degree zero term of $f(x)g(x)$ is zero, that is, $a_0 b_0 = 0$, we will show that $a_r b_0^{r+1} = 0$ for all $r$ by induction. Suppose the statement is true for all $0,1, ..., r-1$, that is
		\begin{align*}
			a_0 b_0 &= 0 \\
			a_1 b_0^2 &= 0 \\
			... \\
			a_{r-1} b_0^r &= 0
		\end{align*}
		
		We want to show that $a_r b_0^{r+1} = 0$. The degree $r$ coefficient of $f(x)g(x)$ is zero, that is
		$$
			a_0 b_r + a_1 b_{r-1} + ... + a_r b_0 = 0
		$$
		
		Multiply both sides by $b_0^r$, we have
		$$
			a_0 b_0^r b_r + a_1 b_0^r b_{r-1} + ... + a_r b_0^{r+1} = 0
		$$
		
		By the induction assumption, we induce that $a_r b_0^{r+1} = 0$ for all $r$. Then, $f(x) b_0^{n+1} = 0$
		
		($\impliedby$) by the premise
		
		\item (if $f, g \in A[x]$, then $fg$ is primitive $\iff$ $f$ and $g$ are primitive)
		
		($\implies$) Let $g(x) = b_0 + b_1 x + ... + b_m x^m$. If $f(x) g(x)$ is primitive, then
		$$
			1 = c_0 (a_0 b_0) + c_1 (a_0 b_1 + a_1 b_0) + c_2 (a_0 b_2 + a_1 b_1 + a_2 b_0) + ... + c_{n+m} (a_n b_m)
		$$
		
		for some $c_0, c_1, ..., c_{n+m} \in A$. Hence, $1$ can be written as a linear combination of the finite set $\set{a_0, a_1, ..., a_n}$ with coefficients in $A$, that is $f(x)$ is primitive. Similarly, $g(x)$ is also primitive.
		
		($\impliedby$) Suppose $f(x)g(x)$ is not primitive, let the maximal ideal containing the ideal generated by coefficients of $f(x)g(x)$ be $\mf{m}$. Then, in $(A / \mf{m})[x]$, $0 = \overline{f(x)g(x)} = \overline{f(x)} \; \overline{g(x)}$. Since $\mf{m}$ is maximal that is prime, $A / \mf{m}$ is a domain, then $(A / \mf{m})[x]$ is a domain. On the other hand, $f(x)$ is primitive, then the coefficients of $f(x)$ generate the whole ring $A$, therefore, there exists an $a_i \notin \mf{m}$, if not $A = (a_0, a_1, ..., a_n) \subseteq \mf{m}$. Hence, $\overline{f(x)} \neq 0$ in $(A / \mf{m})[x]$. Similarly, $\overline{g(x)} \neq 0$ in $(A / \mf{m})[x]$. This contradicts with $(A / \mf{m})[x]$ being a domain.
		
	\end{enumerate}
\end{longproof}

\begin{problem}[chapter 1 problem 8]
	Let $A$ be a ring $\neq 0$. Show that the set of prime ideals of $A$ has minimal elements with respect to inclusion.
\end{problem}

\begin{proof}
	The collection of prime ideals of $A$ is a partially ordered set with respect to inclusion. Moreover, given any chain of prime ideals $\set{\mf{p}_i: i \in I}$ under inclusion, the intersection $\mf{p} = \bigcap_{i \in I} \mf{p}_i$ is a prime ideal and a lowerbound (note that, this is only true for arbitrary collection of prime ideals. $(2)$ and $(3)$ are prime in $\Z$ but $(6) = (2) \cap (3)$ is not prime). Suppose, $xy \in \mf{p}$ but $x \notin \mf{p}$ and $y \notin \mf{p}$. Let $x \notin \mf{p}_x$ and $y \notin \mf{p}_y$. Since the collection is a chain, without loss of generality, assume $\mf{p}_x \subseteq \mf{p}_y$. Therefore, both $x, y \notin \mf{p}_x$ but $xy \in \mf{p} \subseteq \mf{p}_x$. Contradiction. Thus, $\mf{p}$ is prime. By Zorn lemma, there is a minimal prime ideal.
\end{proof}

\begin{problem}[chapter 1 problem 10]
	Let $A$ be a ring, $\eta_A$ is its nilradical. Show that the following are equivalent:
	\begin{enumerate}
		\item $A$ has exactly one prime ideal
		\item every element of $A$ is either a unit or nilpotent.
		\item $A / \eta_A$ is a field
	\end{enumerate}
\end{problem}

\begin{longproof}
	($1 \implies 2$) If $A$ has exactly one prime ideal, namely $\eta_A$ the intersection of all prime ideals. Since any maximal ideal is prime, $\eta_A$ is the unique maximal ideal in $A$. Therefore, any element $x \in A$, if $x \in \eta_A$ then $x$ is nilpotent, if $x \notin \eta_A$ and $x$ is a not unit then $x$ is contained in a maximal ideal other than $\eta_A$. Contradiction.
	
	($2 \implies 3$) If $\bar{x} \in A / \eta_A$ is non-zero for some $x \in A$, then $x \notin \eta_A$, thus $x$ is a unit, hence $\bar{x}$ is a unit. Therefore, $A / \eta_A$ is a field.
	
	($3 \implies 1$)  $A / \eta_A$ is a field, then $\eta_A$ is maximal. Since $\eta_A$ is the intersection of all prime ideals, but it is maximal, it can not be a proper subset of any ideal. Hence, $\eta_A$ is the unique prime ideal of $A$
\end{longproof}

\begin{problem}[chapter 1 problem 12]
	A local ring contains no idempotent $\neq 0, 1$
\end{problem}

\begin{proof}
	Let $A$ be a local ring with $\mf{m}$ be its unique maximal ideal. Suppose $a \neq 0, 1$ such that $a^2 = a$, then $a(a-1) = a^2 - a = 0$, that is, $a$ and $a-1$ are zero divisors. Since $a$ and $a-1$ are not a unit, $a, a-1 \in \mf{m}$ but $1 = a - (a - 1) \notin \mf{m}$ which is a contradiction.
\end{proof}

\begin{problem}[chapter 1 problem 15 - Zariski topology]
	Let $A$ be a ring and let $X$ be the set of all prime ideals of $A$. For each subset $E$ of $A$, let $V(E)$ denote the set of all prime ideals of $A$ which contain $E$. Prove that
	\begin{enumerate}
		\item if $\mf{a}$ is the ideal generated by $E$, then $V(E) = V(\mf{a}) = V(\sqrt{\mf{a}})$
		
		\item $V(0) = X, V(1) = \emptyset$
		
		\item if $(E_i)_{i \in I}$ is any family of subsets of $A$, then 
		$$
			V\tuple*{\bigcup_{i \in I} E_i} = \bigcap_{i \in I} V(E_i)
		$$
		
		\item $V(\mf{a} \cap \mf{b}) = V(\mf{ab}) = V(\mf{a}) \cup V(\mf{b})$ for any ideals $\mf{a}, \mf{b}$ of $A$
	\end{enumerate}
\end{problem}

\begin{longproof}
	\begin{enumerate}
		\item (if $\mf{a}$ is the ideal generated by $E$, then $V(E) = V(\mf{a}) = V(\sqrt{\mf{a}})$)
		
		Since $E \subseteq \mf{a}$, $V(E) \supseteq V(\mf{a})$. By definition of ideal generated by set, $\mf{a}$ is the smallest ideal containing $E$, therefore any prime ideal containing $E$ must contain $\mf{a}$, hence $V(E) \subseteq V(\mf{a})$.
		
		Since $\mf{a} \subseteq \sqrt{\mf{a}}$, $V(\mf{a}) \supseteq V(\sqrt{\mf{a}})$. We want to show the other direction $V(\mf{a}) \subseteq V(\sqrt{\mf{a}})$, that is any prime ideal containing $\mf{a}$ must contain $\sqrt{\mf{a}}$. Let $\mf{b} \supseteq \mf{a}$ be a prime ideal, for any element, $x \in \sqrt{\mf{a}}$, $x^n \in \mf{a} \subseteq \mf{b}$ for some $n > 0$.Then, $x x^{n-1} \in \mf{b}$ therefore, either $x \in \mf{b}$ or $x^{n-1} \in \mf{b}$. The induction argument on $n$ implies $x \in \mf{b}$. Hence, $\mf{b} \supseteq \sqrt{\mf{a}}$
		
		\item ($V(0) = X, V(1) = \emptyset$)
		
		Every prime ideal contains $0$, hence $V(0) = X$. Every prime ideal is proper, hence it cannot contain $1$, then $V(1) = \emptyset$
		
		\item ($V\tuple*{\bigcup_{i \in I} E_i} = \bigcap_{i \in I} V(E_i)$)
		
		Let $\mf{a}$ be an ideal. Then,
		
		$\mf{a} \in V\tuple*{\bigcup_{i \in I} E_i}$ $\iff$ $\mf{a} \supseteq E_i$ for all $i \in I$ $\iff$ $\mf{a} \in V(E_i)$ for all $i \in I$ $\iff$ $\mf{a} \in  \bigcap_{i \in I} V(E_i)$
		
		\item ($V(\mf{a} \cap \mf{b}) = V(\mf{ab}) = V(\mf{a}) \cup V(\mf{b})$ for any ideals $\mf{a}, \mf{b}$ of $A$)
		
		Note that, if $E, F \subseteq A$, then $E \subseteq F \implies V(E) \supseteq V(F)$. Since $\mf{ab} \subseteq \mf{a} \cap \mf{b} \subseteq \mf{a}$, then 
		$$
			V(\mf{ab}) \supseteq V(\mf{a} \cap \mf{b}) \supseteq V(\mf{a}) \cup V(\mf{b})
		$$
		
		We will show that $V(\mf{ab}) \subseteq V(\mf{a}) \cup V(\mf{b})$. Suppose $\mf{p} \in V(\mf{ab})$ but $\mf{p} \notin V(\mf{a}) \cup V(\mf{b})$. $\mf{p} \notin V(\mf{a})$ implies there exists $a \in \mf{a}$ such that $a \notin \mf{p}$. $\mf{p} \notin V(\mf{b})$ implies there exists $b \in \mf{b}$ such that $b \notin \mf{p}$. But $ab \in \mf{ab} \subseteq \mf{p}$. This is a contradiction since $\mf{p}$ is prime.
	\end{enumerate}
\end{longproof}

\begin{problem}[chapter 1 problem 17 - a basis for Zariski topology]
	For each $f \in A$, let $X_f$ denote the complement of $V(f)$ in $X = \Spec A$. The set $X_f$ are open. Show that they form a basis of open sets for the Zariski topology, and that
	\begin{enumerate}
		\item $X_f \cap X_g = X_{fg}$
		\item $X_f = \emptyset \iff f$ is nilpotent
		\item $X_f = X \iff f$ is a unit
		\item $X_f = X_g \iff \sqrt{(f)} = \sqrt{(g)}$
		\item $X$ is quasi-compact
		\item each $X_f$ is quasi-compact
		\item an open subset of $X$ is quasi-compact if and only if it is a finite union of set $X_f$
	\end{enumerate}
\end{problem}

\begin{longproof}
	($X_f$ form a basis for Zariski topology) Given any ideal $I$, the open set $X - V(I)$ can be written as a union of $X_f$ 
	$$
		X - V(I) = X - \bigcap_{f \in I} V(f) = \bigcup_{f \in I} \tuple*{X - V(f)} = \bigcup_{f \in I} X_f
	$$
	
	\begin{enumerate}
		\item ($X_f \cap X_g = X_{fg}$)
		
		\begin{align*}
			X_f \cap X_g
			&= (X - V(f)) \cap (X - V(g)) \\
			&= X - (V(f) \cup V(g)) \\
			&= X - (V((f)) \cup V((g))) \\
			&= X - V((fg)) \\
			&= X - V(fg) \\
			&= X_{fg}
		\end{align*}
		
		\item ($X_f = \emptyset \iff f$ is nilpotent)
		$$
			X_f = \emptyset \iff V(f) = X \iff f \in \mf{p} \text{ for every prime ideal } \mf{p} \iff f \in \eta_A
		$$
		
		\item ($X_f = X \iff f$ is a unit)
		$$
			X_f = X \iff V(f) = \emptyset \iff f \notin \mf{m} \text{ for every maximal ideal } \mf{m} \iff f \text{ is a unit}
		$$
		
		The last $\iff$ is true because $f$ is a unit implies $f$ is not in any maximal ideal and $f$ is not a unit  implies $f$ is contained in some maximal ideal.
		
		\item ($X_f = X_g \iff \sqrt{(f)} = \sqrt{(g)}$)
		
		$$
			X_f = X_g \iff V(f) = V(g) \iff V((f)) = V((g))
		$$
		
		By definition, $\sqrt{(f)} = \bigcap_{\mf{p} \in V((f))} \mf{p}$ and $\sqrt{(g)} = \bigcap_{\mf{p} \in V((g))} \mf{p}$, then 
		$$
			V((f)) = V((g)) \implies \sqrt{(f)} = \sqrt{(g)}
		$$
		
		On the other hand,
		$$
			\sqrt{(f)} = \sqrt{(g)} \implies V(\sqrt{(f)}) = V(\sqrt{(g)}) \implies V((f)) = V((g))
		$$
		
		\item ($X$ is quasi-compact)
		
		It is sufficient to prove that given any open cover by basic open sets $\set{X_{f_i}}_{i \in I}$, then there exists a finite subcover $\set{X_{f_j}}_{j \in J}$ for finite subset $J \subseteq I$. We have 
		$$
			\bigcup_{i \in I} X_{f_i} = \bigcup_{i \in I} (X - V(f_i)) = X - \bigcap_{i \in I} V(f_i)
		$$
		
		That is, $\bigcup_{i \in I} X_{f_i} = X \iff \bigcap_{i \in I} V(f_i) = \emptyset$. Moreover, 
		$$
			\bigcap_{i \in I} V(f_i) = \emptyset \iff \text{there is no prime ideal containing } \set{f_i}_{i \in I} \iff (f_i)_{i \in I} = A
		$$
		
		where $(f_i)_{i \in I}$ denotes the ideal generated by $\set{f_i}_{i \in I}$. The second $\iff$ is due to every prime ideal is contained is a maximal ideal. Then, $(f_i)_{i \in I} = A$ implies
		$$
			1 = \sum_{j \in J} a_j f_j
		$$
		
		for some finite subset $J \subseteq I$. Hence, $(f_j)_{j \in J} = A$. That implies $\bigcup_{j \in J} X_{f_j} = X$ by the same argument for index set $J$
		
		\item (each $X_f$ is quasi-compact)
		
		Let $\set{X_{f_i}}_{i \in I}$ be an open cover for $X_f$ by basic open sets. We have
		$$
			X_f \subseteq \bigcup_{i \in I} X_{f_i} \iff V(f) \supseteq \bigcap_{i \in I} V(f_i) = V((f_i)_{i \in I})
		$$
		
		If $\bigcap_{i \in I} V(f_i) = \emptyset$, this falls back to the previous case. Suppose $V((f_i)_{i \in I}) = \bigcap_{i \in I} V(f_i) \neq \emptyset$, we have
		$$
			\mf{p} \in V((f_i)_{i \in I}) \implies \mf{p} \in V(f) \implies f \in \mf{p}
		$$
		
		Therefore
		$$
			f \in \sqrt{(f_i)_{i \in I}} = \bigcap_{\mf{p} \in V((f_i)_{i \in I})} \mf{p}
		$$
		
		That is, $f^n \in (f_i)_{i \in I}$ for some $n > 0$, then
		$$
			f^n = \sum_{j \in J} a_j f_j
		$$
		
		for some finite subset $J \subseteq I$. As $f^n \in (f_j)_{j \in J}$, then $V(f^n) \supseteq V((f_j)_{j \in J})$, we have
		$$
			\mf{p} \in V((f_j)_{j \in J}) \implies \mf{p} \in V(f^n) \implies f^n \in \mf{p}
		$$
		
		Since $\mf{p}$ is prime, $f^n \in \mf{p} \implies f \in \mf{p}$. Therefore,
		$$
			\mf{p} \in V((f_j)_{j \in J}) \implies \mf{p} \in V(f^n) \implies f^n \in \mf{p} \implies f \in \mf{p} \implies \mf{p} \in V(f)
		$$
		
		Thus, $V(f) \supseteq V((f_j)_{j \in J})$, that is, $X_f$ is covered by a finite subcollection
		$$
				X_f \subseteq \bigcup_{j \in J} X_{f_j}
		$$
		
		\item (an open subset of $X$ is quasi-compact if and only if it is a finite union of set $X_f$)
		
		($\impliedby$) finite union of quasi-compact sets is quasi-compact since we can pick a finite subcollection for each set, the total is still a finite subcollection.
		
		($\implies$) if $U$ is an open set in $X$, then $U$ can be written as $U = \bigcup_{i \in I} X_{f_i}$ since $\set{X_f}$ form a basis for $X$. By compactness of $U$, $U$ can be cover by a finite union $U \subseteq \bigcup_{j \in J} X_{f_j} \subseteq \bigcup_{i \in I} X_{f_i} = U$. Hence, $U = \bigcup_{j \in J} X_{f_j}$
		
	\end{enumerate}
\end{longproof}

\begin{problem}[chapter 2 problem 9]
	Let \begin{tikzcd}0 \arrow[r] & A \arrow[r, "i", hook] & B \arrow[r, "p", two heads] & C \arrow[r] & 0\end{tikzcd} be an exact sequence of $R$-modules. If $A$ and $C$ are finitely generated, then so is $B$
\end{problem}

\begin{proof}
	If $b \in \ker p = \im i$, since $i$ is injective, we can write $i^{-1}(b) = r_1 a_1 + r_2 a_2 + ... + r_n a_n$ where $\set{a_1, a_2, ..., a_n}$ generates $A$ and $r_1, r_2, ..., r_n \in R$. Therefore,
	$$
		b = r_1 i(a_1) + r_2 i(a_2) + ... + r_n i(a_n)
	$$
	
	That is, $\set{i(a_1), i(a_2), ..., i(a_n)}$ generates $\ker p$. Let $\set{c_1, c_2, ..., c_m}$ generates $C$. Since $p$ is surjective, pick $\set{b_1, b_2, ..., b_m} \subseteq B$ so that $p(b_i) = c_i$ for all $i=1, 2, ..., m$. Now, if $b \in B - \ker p$, we can write
	$$
		p(b) = s_1 c_1 + s_2 c_2 + ... + s_m c_m
	$$
	for some $s_1, s_2, ..., s_m \in R$. Let
	$$
		b' = s_1 b_1 + s_2 b_2 + ... + s_m b_m
	$$
	
	Then, $p(b - b') = 0$, that is, $b - b' \in \ker p$, hence $b - b'$ can be written as a linear combination of $\set{i(a_1), i(a_2), ..., i(a_n)}$. Thus, the set $\set{i(a_1), i(a_2), ..., i(a_n)} \cup \set{b_1, b_2, ..., b_m}$ generates $B$
\end{proof}

\begin{problem}[chapter 2 problem 10]
	Let $A$ be a ring and $\mf{a}$ be an ideal contained in the Jacobson radical of $A$. Let $M$ be an $A$-module and $N$ be finitely generated $A$-module, let $u: M \to N$ be a homomorphism. If the induced homomorphism $M / \mf{a}M \to N / \mf{a} N$ is surjective, then $u$ is surjective
\end{problem}

\begin{proof}
	We will show that $N = \mf{a} N + \im u$ so that Nakayama lemma version 2 implies $N = \im u$. Let $\set{y_1, y_2, ..., y_n}$ generates $N$, then $\set{y_1 + \mf{a}N, y_2 + \mf{a}N, ..., y_n + \mf{a}N}$ generates $N / \mf{a}N$. For each $i = 1, 2, ..., n$, since $u^*: M / \mf{a} M \to N / \mf{a} N$ is surjective, there is $x_i \in M$ such that
	$$
		u^*(x_i + \mf{a} M) = y_i + \mf{a} N
	$$
	
	That is, $z_i = u(x_i) - y_i \in \mf{a} N$. Now, for each $y \in N$, we have
	$$
		y = \sum_{i=1}^n a_i y_i = \sum_{i=1}^n a_i (u(x_i) - z_i)
	$$
	
	for some $a_1, a_2, ..., a_n \in A$. $\mf{a} N + \im{u}$ being a submodule of $N$ and $ u(x_i) - z_i \in \mf{a} N + \im{u}$ implies $a_i (u(x_i) - y_i) \in \mf{a} N + \im{u}$. Hence, $y \in \mf{a} N + \im{u}$. Thus, $N \subseteq \mf{a} N + \im{u} \subseteq N$, hence $N = \mf{a} N + \im u$
\end{proof}

\begin{problem}[chapter 2 problem 12]
	Let $M$ be a finitely generated $A$-module and $\phi: M \to A^n$ a surjective homomorphism. Show that $\ker \phi$ is finitely generated.
\end{problem}

\begin{proof}
	Since $\phi: M \to A^n$ is surjective, the first row is exact
	\begin{center}
		\begin{tikzcd}
			0 \arrow[r] & \ker \phi \arrow[r, hook] & M \arrow[r, "\phi", two heads] & A^n \arrow[r]                                        & 0 \\
			&                           &                                & A^n \arrow[u, "1_{A^n}"'] \arrow[lu, "\psi", dashed] &  
		\end{tikzcd}
	\end{center}
	
	$A^n$ is projective since it is free, hence the map $1_{A^n}: A^n \to A^n$ factors through the surjective map $M \to A^n$ by a map $\psi: A^n \to M$. In particular, let $e_1, e_2, ..., e_n$ be the canonical basis for $A^n$, for each $e_i$ pick $u_i \in M$ such that $\phi(u_i) = e_i$. Define the map $\psi: A^n \to M$ by
	\begin{align*}
		\psi: A^n &\to M \\
				e_i &\mapsto u_i
	\end{align*}
	
	so that $\phi \psi = 1_{A^n}$. Thus, the sequence splits, by Five lemma, there is an isomorphism $f: M \to \ker \phi \oplus A^n$. Both $M$ and $A^n$ being finitely generated, so is $\ker \phi$.
	
	Indeed, if $f: M \to N \oplus P$ is an isomorphism with $M$ and $N$ being finitely generated. Let $\set{x_1, x_2, ..., x_m}$ generate $M$ and $\set{y_1, y_2, ..., y_n}$ generate $N$. For each $i = 1, 2, ..., m$, then
	$$
		f(x_i) = \tuple*{\sum_{j=1}^n a_j y_j, p_i}
	$$
	for some $a_1, a_2, ..., a_n \in A$ and $p_i \in P$. Let $p \in P$, then there are some $b_1, b_2, ..., b_m \in A$ such that
	$$
		f\tuple*{\sum_{i=1}^m b_i x_i} = (0, p)
	$$
	
	Then,
	$$
		(0, p)
		= f\tuple*{\sum_{i=1}^m b_i x_i} 
		= \sum_{i=1}^m b_i f(x_i) 
		= \sum_{i=1}^m b_i \tuple*{\sum_{j=1}^n a_j y_j, p_j} 
		= \tuple*{\sum_{i=1}^m \sum_{j=1}^n b_i a_j y_j, \sum_{i=1}^m b_i p_i} 
	$$
	
	Thus, $\set{p_1, p_2, ..., p_m}$ generates $P$.
\end{proof}

\begin{problem}[chapter 3 problem 5]
	Let $A$ be a ring. Suppose that for each prime ideal $\mf{p}$, the local ring $A_\mf{p}$ has no nilpotent element $\neq 0$. Show that $A$ has no nilpotent element $\neq 0$. If each $A_\mf{p}$ is an integral domain, is $A$ necessarily an integral domain?
\end{problem}
\begin{longproof}
	($A$ has no nilpotent element $\neq 0$) Suppose $x \in A$ such that $x \neq 0$ and $x^n = 0$ for some $n > 0$. The ideal $(\set{a \in A: ax = 0})$ is proper since if $r_1 a_1 + r_2 a_2 + ... + r_m a_m = 1$, then $0 = r_1 a_1 x + r_2 a_2 x + ... + r_m a_m x = x$. Let $\mf{p}$ be the maximal ideal of $A$ containing $(\set{a \in A: ax = 0})$. For any $s \in A - \mf{p}$, $\frac{x}{s}$ is nilpotent in $A_\mf{p}$ since
	$$
		\tuple*{\frac{x}{s}}^n = \frac{x^n}{s^n} = \frac{0}{s^n} = \frac{0}{1} = 0
	$$
	Moreover, $\frac{x}{s} \neq 0$ in $A_\mf{p}$ since if $\frac{x}{s} = 0$ in $A_\mf{p}$, then there exists $t \in A - \mf{p}$ so that $tx = 0$, by construction of $\mf{p}$, this is a contradiction.
	
	(If each $A_\mf{p}$ is an integral domain, is $A$ necessarily an integral domain?)	Let $A = \Z_6$, $\Z_6$ is not a domain since $2 \times 3 = 0 \mod 6$. The prime ideals of $\Z_6$ are $\set{(2), (3)}$, we have
	\begin{align*}
		S_2 &= \Z_6 - (2) = \set{1, 3, 5} \\
		S_3 &= \Z_6 - (3) = \set{1, 2, 4, 5}
	\end{align*}
	
	The zeros in $S_2^{-1} A$ are $a / s$ where $s \in S_2$ and $a \in \Z_6$ such that $ta = 0 \mod 6$ for $t \in S_2$, that is
	$$
		\set*{\frac{0}{s}, \frac{2}{s}, \frac{4}{s}: s \in S}
	$$
	
	The zeros in $S_3^{-1} A$ are $a / s$ where $s \in S_3$ and $a \in \Z_6$ such that $ta = 0 \mod 6$ for $t \in S_3$, that is
	$$
		\set*{\frac{0}{s}, \frac{3}{s}: s \in S}
	$$
	
	In $S_2^{-1} A$, if $\frac{a}{s} \frac{b}{r} = \frac{ab}{sr} = 0$, then $ab \in \set{0, 2, 4}$. Hence one of $a$ or $b$ must be in $\set{0, 2, 4}$. In In $S_3^{-1} A$, if $\frac{a}{s} \frac{b}{r} = \frac{ab}{sr} = 0$, then $ab \in \set{0, 3}$. Hence one of $a$ or $b$ must be in $\set{0, 3}$. Thus, both $S_2^{-1} A$ and $S_3^{-1} A$ are domain but $A$ is not.
\end{longproof}

\begin{problem}[chapter 3 problem 6]
	Let $A$ be a ring $\neq 0$ and let $\Sigma$ be the set of all multiplicatively closed subsets $S$ of $A$ such that $0 \notin S$. Show that $\Sigma$ has maximal elements and that $S \in \Sigma$ is maximal if and only if $A - S$ is a minimal ideal of $A$.
\end{problem}
\begin{proof}
	As $\Sigma$ forms a partially ordered set under inclusion and union of arbitrary number of sets in $\Sigma$ is also in $\Sigma$. By Zorn lemma, $\Sigma$ has a maximal element. Let $S \in \Sigma$, there is a minimal prime ideal $\mf{p}_S$ in the ring $S^{-1} R$, let $\mf{p}$ be the extension of $\mf{p}_S$ in $R$ so that $\mf{p} \cap S = \emptyset$. Since $\mf{p}$ is prime, $A - \mf{p}$ is a multiplicatively closed that that contains $S$.

	($S \in \Sigma$ is maximal $\implies$ $A - S$ is a minimal prime ideal of $A$)

	By maximality of $S$, $S = A - \mf{p}$. Suppose there is a prime ideal $\mf{q}$ contained properly in $\mf{p}$, the contraction $\mf{q}^c$ of $\mf{q}$ is contained (not necessarily proper) in the contraction $\mf{p}^c = \mf{p}_S$ of $\mf{p}$. As $\phi_S: \Spec S^{-1} R \to \Spec R$ is injective, the containment is proper, $\mf{q}^c \subsetneq \mf{p}_S$, this contradicts the minimality of $\mf{p}_S$. Hence, $\mf{p} = A - S$ is minimal in $A$
	
	($S \in \Sigma$ is maximal $\impliedby$ $A - S$ is a minimal prime ideal of $A$)
	
	$S$ is contained in a maximal multiplicatively closed set $S_1$ in $\Sigma$. Then, $A - S_1$ is a minimal prime ideal of $A$. Suppose $S$ is a proper subset of $S_1$, then the minimal prime ideal $A - S$ contains properly a smaller prime ideal $A - S_1$, that is a contradiction.
\end{proof}
