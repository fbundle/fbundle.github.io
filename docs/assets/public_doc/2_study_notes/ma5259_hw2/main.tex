\documentclass{article}
\usepackage{graphicx} % Required for inserting images

% header

%% natbib
\usepackage{natbib}
\bibliographystyle{plain}

%% comment
\usepackage{comment}

% indent the first paragraph
\usepackage{indentfirst}


%% math package
\usepackage{amsfonts}
\usepackage{amsmath}
\usepackage{amssymb}


%% operator
\DeclareMathOperator{\tr}{tr}
\DeclareMathOperator{\diag}{diag}
\DeclareMathOperator{\sign}{sign}
\DeclareMathOperator{\grad}{grad}
\DeclareMathOperator{\curl}{curl}
\DeclareMathOperator{\Div}{div}

%% theorems
\newtheorem{axiom}{Axiom}
\newtheorem{definition}{Definition}
\newtheorem{theorem}{Theorem}
\newtheorem{proposition}{Proposition}
\newtheorem{corollary}{Corollary}
\newtheorem{lemma}{Lemma}
\newtheorem{remark}{Remark}
\newtheorem{claim}{Claim}
\newtheorem{problem}{Problem}

%% empty set
\let\oldemptyset\emptyset
\let\emptyset\varnothing

% mathcal symbols
\newcommand\Tau{\mathcal{T}}
\newcommand\Ball{\mathcal{B}}

% mathbb symbols
\newcommand\N{\mathbb{N}}
\newcommand\Z{\mathbb{Z}}
\newcommand\Q{\mathbb{Q}}
\newcommand\R{\mathbb{R}}

\title{ma5259\_hw2}
\author{Khanh Nguyen}
\date{October 2024}

\begin{document}

\maketitle

\section{Q1}

Let $\xi_1, \xi_2, ...$ be a sequence of i.i.d coin tosses with bias $P(\xi_1 = H) = p$. (1) Find the probability that we will see $HH$ before seeing $TT$. (2) Conditioned on $\xi_1 \xi_2 = HT$, find the probability that we will see $HT$ again before seeing either $HH$ or $TT$

\subsection{seeing $HH$ before seeing $TT$}

Let $X_1, X_2, ...$ be a stochastic process with state space $S = (HH, HT, TH, TT)$ defined by $X_i = (\xi_i, \xi_{i+1})$. Then $X_1, X_2, ...$ is a Markov chain with transition matrix $\Pi$ defined by

\begin{align*}
    \Pi: S \times S &\to \R \\
    (HH, HH) &\mapsto p \\
    (HH, HT) &\mapsto 1-p \\
    (HT, TH) &\mapsto p \\
    (HT, TT) &\mapsto 1-p \\
    (TH, HH) &\mapsto p \\
    (TH, HT) &\mapsto 1-p \\
    (TT, TH) &\mapsto p \\
    (TT, TT) &\mapsto 1-p \\
\end{align*}


For $x \in S$, let $T_x = \min \set{n \geq 0: X_n = x}$ be the first time seeing $x$. We want to find $P(T_{HH} < T_{TT})$

We have
\begin{align*}
    P_{HH}(T_{HH} < T_{TT}) &= 1 \\
    P_{TT}(T_{HH} < T_{TT}) &= 0 \\
\end{align*}

For any $x \in S \setminus \set{HH, TT}$, we have 
\begin{align*}
    P_x(T_{HH} < T_{TT})
    &= \sum_{y \in S} P_x(X_1 = y, T_{HH} < T_{TT}) \\
    &= \sum_{y \in S} P_x(X_1 = y) P_x(T_{HH} < T_{TT} | X_1 = y) \\
    &= \sum_{y \in S} \Pi(x, y) P_y(T_{HH} < T_{TT}) \\
\end{align*}

Then
\begin{align*}
    P_{HT}(T_{HH} < T_{TT})
    &= \Pi(HT, TH) P_{TH}(T_{HH} < T_{TT}) + \Pi(HT, TT) P_{TT}(T_{HH} < T_{TT}) \\
    &= p P_{TH}(T_{HH} < T_{TT}) \\
    P_{TH}(T_{HH} < T_{TT})
    &= \Pi(TH, HT) P_{HT}(T_{HH} < T_{TT})  + \Pi(TH, HH) P_{HH}(T_{HH} < T_{TT}) \\
    &= (1-p) P_{HT}(T_{HH} < T_{TT}) + p
\end{align*}

Solve the system of two equations, we have
\begin{align*}
    P_{HT}(T_{HH} < T_{TT}) &= \frac{p^2}{1 - (1-p)p}\\
    P_{TH}(T_{HH} < T_{TT}) &= \frac{p}{1 - (1-p)p} \\
\end{align*}

Hence
\begin{align}
    P(T_{HH} < T_{TT})
    &= \sum_{x \in S} P(X_0 = x, T_{HH} < T_{TT}) \\
    &= \sum_{x \in S} P(X_0 = x) P_x(T_{HH} < T_{TT}) \\
    &= P(X_0 = HH) + P(X_0 = HT) \frac{p^2}{1 - (1-p)p} + P(X_0 = TH) \frac{p}{1 - (1-p)p} \\
    &= p^2 + p(1-p) \frac{p^2}{1 - (1-p)p} + p(1-p) \frac{p}{1 - (1-p)p} \\
    &= \frac{2 p^2 - p^3}{1 - p + p^2}
\end{align}

When $p = 1/2$, $P(T_{HH} < T_{TT}) = 1/2$

\subsection{conditioned on $\xi_1 \xi_2 = HT$, seeing $HT$ again before seeing either $HH$ or $TT$}

For $x \in S$, let $T_x = \min \set{n > 0: X_n = x}$ be the first time seeing $x$ in positive time. Let $A = \set{T_{HT} < T_{HH}, T_{TT}}$, we want to find $P_{HT}(A)$

We have
\begin{align*}
    P_{HT}(A)
    &= \sum_{x \in S} P_{HT}(X_1=x, A) \\
    &= \sum_{x \in S} P_{HT}(X_1 = x) P_{HT}(A | X_1=x) \\
    &= \sum_{x \in S} \Pi(HT, x) P_{HT}(A | X_1=x)
\end{align*}

As $P_{HT}(A | X_1=TT) = 0$, then
\begin{align*}
    P_{HT}(A)
    &= \Pi(HT, TH) P_{HT}(A | X_1=TH) \\
    &= p P_{TH}(A)
\end{align*}

On the other hand,
\begin{align*}
    P_{TH}(A)
    &= \sum_{x \in S} P_{TH}(X_1 = x, A) \\
    &= \sum_{x \in S} P_{TH}(X_1 = x) P_{TH}(A| X_1 = x) \\
    &= \sum_{x \in S} \Pi(TH, x) P_{TH}(A| X_1 = x)
\end{align*}

As $P_{TH}(A| X_1 = HH) = 0$ and $P_{TH}(A| X_1 = HT) = 1$, then
\begin{align*}
    P_{TH}(A)
    &= \Pi(TH, HT) P_{TH}(A| X_1 = HT) \\
    &= 1-p
\end{align*}

Hence,
$$
    P_{HT}(A) = p P_{TH}(A) = p(1-p)
$$

\begin{remark}
    This question is simple enough that we can enumerate all possible sequences length $4$ of coin flips and analyze the probability $\set{HTHH, HTHT, HTTH, HTTT}$
\end{remark}

\section{Q2}

Consider a $3$-state Markov chain with transition $\Pi(1, 1) = 1 - \Pi(1, 2) = p$, $\Pi(2, 1) = 1 - \Pi(2, 3) = q$ and $\Pi(3, 3) = 1$. Let
$$
    G(x, y) = \Exp\bracket*{\sum_{n=0}^\infty 1_{\set{X_n = y}} \bigg\vert X_0 = x} = \sum_{n=0}^\infty \Pi^n(x, y)
$$

denote the expected number of visits to state $y$ given that the Markov chain starts at $x$. Find $G(1, 2)$ and $G(2, 1)$

Transition matrix
$$
    \Pi = \begin{bmatrix}
        p & 1-p & 0 \\
        q & 0 & 1-q \\
        0 & 0 & 1
    \end{bmatrix}
$$

We have
$$
    G(x, y) = \Pi^0(x, y) + \sum_{n=1}^\infty \Pi^n(x, y) = \Pi^0(x, y) + \Pi(x, y) + \sum_{n=2}^\infty \Pi^n(x, y)
$$

We have

\begin{align*}
    &G(x, y) - \Pi^0(x, y) - \Pi(x, y) \\
    &=\sum_{n=2}^\infty \Pi^n(x, y) \\
    &= \sum_{n=2}^\infty P(X_n=y | X_0=x) \\
    &= \sum_{n=2}^\infty \sum_{z \in S} P(X_n=y, X_1 = z| X_0=x) &\text{(marginalize)}\\
    &= \sum_{n=2}^\infty \sum_{z \in S} P(X_1 = z| X_0=x) P(X_n=y | X_1 = z, X_0=x) &\text{(conditional probability)}\\
    &= \sum_{n=2}^\infty \sum_{z \in S} \Pi(x, z) P(X_n=y | X_1 = z) &\text{(Markov)}\\
    &= \sum_{z \in S} \Pi(x, z) \sum_{n=2}^\infty P(X_n=y | X_1 = z) &\text{(Tonelli)}\\
    &= \sum_{z \in S} \Pi(x, z) \sum_{m=1}^\infty \Pi^{m}(z, y) &\text{($m = n-1$)}\\
    &= \sum_{z \in S} \Pi(x, z) (G(z, y) - \Pi^0(z, y))
\end{align*}

If $x = y$, we have
$$
    (1 - \Pi(x, x)) G(x, x) =  1 + \sum_{z \in S - \set{x}} \Pi(x, z) G(z, x)
$$

If $x \neq y$, we have
$$
    (1 - \Pi(x, x)) G(x, y) =  \Pi(x, y) G(y, y) + \sum_{z \in S - \set{x, y}} \Pi(x, z) G(z, y)
$$


Given that $G(3, 1) = G(3, 2) = 0$, we have the following equations
\begin{align*}
    (1 - p) G(1, 1) &= 1 + (1-p) G(2, 1) \\
    (1 - p) G(1, 2) &= (1-p) G(2, 2) \\
    G(2, 1) &= q G(1, 1) \\
    G(2, 2) &= 1 + q G(1, 2) \\
\end{align*}

Solve the system of equations, we have
\begin{align*}
    G(2, 1) &= \frac{q}{(1-p)(1-q)} \\
    G(1, 2) &= \frac{1}{1-q}
\end{align*}

\section{Q3}

Consider a random walk on $\set{0, 1, ..., L}$ with transition matrix $\Pi(i, i + 1) = \Pi(i, i - 1) = \frac{1}{2}$ for each $i \in \set{1, ..., L-1}$ and $\Pi(0, 0) = \Pi(L, L) = 1$. For each $i \in \set{0, 1, ..., L}$, find the return probability $f_{ii} = P(X_n = i \text{ for some $n \in \N$ } | X_0 = i)$


Let $A_i^m = \set{X_n = i \text{ for some } n \in \N: n \geq m}$. It is clear that $f_{00} = f_{LL} = 1$, for any $i \in \set{1, ..., L-1}$, note that $\set{A_i^1, X_1 = i\pm1} = \set{A_i^2, X_1 = i\pm1}$, we have
\begin{align*}
    f_{ii}
    &= P(A_i^1 | X_0 = i) \\
    &= P(A_i^1, X_1 = i-1| X_0 = i) + P(A_i^1, X_1 = i+1| X_0 = i) \\
    &= P(A_i^2, X_1 = i-1| X_0 = i) + P(A_i^2, X_1 = i+1| X_0 = i) \\
    &= P(A_i^2 | X_1 = i-1, X_0 = i) P(X_1 = i-1| X_0 = i) \\ &+ P(A_i^2 | X_1 = i+1, X_0 = i) P(X_1 = i+1| X_0 = i)\\
    &= P(A_i^2 | X_1 = i-1) \Pi(i, i-1) + P(A_i^2 | X_1 = i+1) \Pi(i, i+1) &\text{(Markov property)} \\
    &= P(A_i^1 | X_0 = i-1) \Pi(i, i-1) + P(A_i^1 | X_0 = i+1) \Pi(i, i+1)
\end{align*}

For any $a, b, c \in \set{0, 1, ..., L}$ such that $a < b < c$. Starting from $a$, probability of visiting $a$ before visiting $c$ is $\frac{c-b}{c-a}$, probability of visiting $c$ before visiting $a$ is $\frac{b-a}{c-a}$, therefore
\begin{align*}
    P(A_i^1 | X_0 = i-1) &= \frac{i-1}{i} \\
    P(A_i^1 | X_0 = i+1) &= \frac{L-i-1}{L-i} \\
\end{align*}

Hence, for $i \in \set{1, ..., L}$
$$
    f_{ii} = \frac{1}{2} \tuple*{\frac{i-1}{i} + \frac{L-i-1}{L-i}} = 1 - \frac{L}{2i(L-i)}
$$


\section{Q4}

Let $X = (X_n)_{n \in \N_0}$ be a symmetric random walk on $\Z$. For $x \in \Z$, let $P_x(\cdot)$ respectively $\Exp_x[\cdot]$ denote the probability and expectation for $X$ with initial condition $X_0 = x$. Let $T_y = T_y(X) = \min\set{n \geq 0: X_n = y}$ be the first time $X$ visits the point $y \in \Z$. Let $L \in \N$ and define $T = \min\set{T_0, T_L}$ the first time hitting either $0$ or $L$. Show that for $0 \leq x \leq L$, $P_x(T < \infty) = 1$ and furthermore, $\Exp_x[T] < \infty$.

When $X$ starts at either $0$ or $L$, $T = 0$. Suppose $F = \Z \cap [1, L-1]$ is non-empty and $X$ starts at $x \in F$. Let $T_{F^c} = \min\set{n \geq 0: X_n \notin F}$. Since the only transition from a state in $F$ to a state in $F^c$ is transitioning to $0$ or $L$. Hence, $T = T_{F^c}$. Therefore, there exists constant $C > 0$ and $\rho \in (0, 1)$ such that
$$
    P_x(T \geq n) \leq C \rho^n
$$

Then,
$$
    P_x(T < \infty) = 1 - P_x(T = \infty) = 1 - \lim_{n \to \infty} P_x(T \geq n) = 1
$$

Moreover, we have

\begin{align*}
    \Exp_x[T]
    &= \sum_{n=0}^\infty n P_x(T=n) \\
    &= \sum_{n=1}^\infty P_x(T \geq n) \\
    &\leq \sum_{n=1}^\infty C \rho^n \\
    &= \frac{C \rho}{1-\rho} < \infty
\end{align*}

\section{Q5}

Let $X$ be a simple symmetric random walk on $\Z$ with $X_0 = 0$. Let $a \in \N$
\begin{enumerate}[label=(\alph*)]
    \item By conditioning on the Markov chain's first visit to $a$ and using the symmetry of the random walk, show that
    $$
        P_0\tuple*{\max_{0 \leq i \leq n} X_i \geq a} = P_0(X_n = a) + 2 P_0(X_n \geq a+1) = P_0(X_n \notin [-a, a-1])
    $$

    \item \label{q5b} Deduce from the above identity that for any $a \in \N$
    $$
        P_a(T_0 > n) = P_0\tuple*{\max_{0 \leq i \leq n} X_i \leq a-1} = P_0(X_n \in [-a, a-1])
    $$

    \item Deduce from \ref{q5b} that $E_a[T_0] = \infty$ for all $a \in \N$

    \item Deduce from \ref{q5b} for all $n \in \N$
    $$
        P_0(X_1, ..., X_{2n} \neq 0) = P_0(X_{2n} = 0)
    $$
\end{enumerate}

\subsection{a}

Let $T_a = \min \set{n \geq 0: X_n = a}$ be the first time $X$ visits $a$. Then,
$$
    \set*{\max_{0 \leq i \leq n} X_i \geq a} = \set*{T_a \leq n}
$$

Moreover, we can write $\set*{T_a \leq n}$ as a disjoint union of events
\begin{align*}
    &\set*{T_a \leq n} \\
    &= \set*{T_a \leq n, X_n = a} \amalg \set*{T_a \leq n, X_n > a} \amalg \set*{T_a \leq n, X_n < a} \\
    &= \set{X_n = a} \amalg \set{X_n > a} \amalg \set*{T_a \leq n, X_n < a} \\
\end{align*}

We can write $\set*{T_a \leq n, X_n < a}$ as a disjoint union of events
$$
    \set*{T_a \leq n, X_n < a} = \coprod_{i=0}^n \set*{T_a = i, X_n < a}
$$

We have
\begin{align*}
    P_0(T_a \leq n, X_n < a)
    &= \sum_{i=0}^n P_0(T_a = i, X_n < a) \\
    &= \sum_{i=0}^n P_0(X_n < a | T_a = i) P_0(T_a = i) &\text{(conditional probability)}\\
    &= \sum_{i=0}^n P_0(X_n < a | X_i = a) P_0(T_a = i) &\text{(Markov property)}\\
    &= \sum_{i=0}^n P_0(X_n > a | X_i = a) P_0(T_a = i) &\text{(symmetric random walk)}\\
    &= \sum_{i=0}^n P_0(X_n > a | T_a = i) P_0(T_a = i) &\text{(Markov property)}\\
    &= \sum_{i=0}^n P_0(X_n > a, T_a = i) &\text{(conditional probability)}\\
    &= P_0(X_n > a)
\end{align*}

Therefore,

$$
    P_0\tuple*{\max_{0 \leq i \leq n} X_i \geq a} = P_0(T_a \leq n) = P_0(X_n = a) + 2 P_0(X_n > a)
$$

\subsection{b}

In previous part, $P_0(T_a \leq n)$ is the probability of visiting $a$ within $n$ steps starting from $0$. By symmetry, let $P_a(T_0 \leq n)$ be the probability of visiting $0$ within $n$ steps starting from $0$, then $P_a(T_0 \leq n) = P_0(T_a \leq n)$. We have
$$
    P_a(T_0 > n) = 1 - P_a(T_0 \leq n) = 1 - P_0(T_a \leq n) = 1 - P_0\tuple*{\max_{0 \leq i \leq n} X_i \geq a}
$$

\subsection{c}


We have 
\begin{align*}
    \Exp_a[T_0]
    &= \sum_{n=0}^\infty n P_a(T_0 = n) \\
    &= \sum_{n=1}^\infty P_a(T_0 \geq n) \\
    &= \sum_{n=0}^\infty P_a(T_0 > n) \\
    &= \sum_{n=0}^\infty P_0(X_n \in [-a, a-1]) \\
    &= \Exp_0\bracket*{\sum_{n=0}^\infty 1_{\set{X_n \in [-a, a-1]}}} \\
    &\geq \Exp_0\bracket*{\sum_{n=0}^\infty 1_{\set{X_n=0}}}
\end{align*}

Since symmetric random walk is recurrent, starting from $0$, $X$ visits $0$ infinitely many times. Therefore, $\Exp_a[T_0] \geq \Exp_0\bracket*{\sum_{n=0}^\infty 1_{\set{X_n=0}}} = \infty$

\subsection{d}

Starting at $0$, at time $1$, $X_1 = 1$ or $X_1 = -1$. By symmetry, 
$$
    P_0(X_1, ..., X_{2n} \neq 0) = P_1(X_2, ..., X_{2n} \neq 0) = P_1(T_0 > 2n) = P_0(X_{2n} \in [-1, 0])
$$

Since, $2n$ is even $X_{2n}$ cannot be $-1$, $P_0(X_{2n} \in [-1, 0]) = P_0(X_{2n} = 0)$. Hence
$$
    P_0(X_1, ..., X_{2n} \neq 0) = P_0(X_{2n} = 0)
$$

\section{Q6}

A general random walk $X$ on $\Z$ is a Markov chain on $\Z$ with transition probability $\Pi(x, y) = \Pi(0, y-x) = \mu(y - x)$ for all $x, y \in \Z$ where $\mu$ is a probability measure on $\Z$. In particular, the increment $X_1 - X_0$, $X_2 - X_1$, ... are i.i.d with distribution $\mu$. Find a stationary measure for $X$. Can $X$ be positive recurrent?

\subsection{a stationary measure for $X$}

Let $\nu: \Z \to [0, \infty)$ be a stationary measure on $\Z$ satisfying $\sum_{x \in \Z} \nu(x) > 0$. By stationary, we must have $\nu = \nu \Pi$, that is, for every $y \in \Z$,

$$
    \nu(y) = \sum_{x \in \Z} \nu(x) \Pi(x, y)
$$


We have
\begin{align*}
    \sum_{x \in \Z} \nu(0) \mu(-x)
    = \nu(0) 
    = \sum_{x \in \Z} \nu(x) \Pi(x, 0)
    = \sum_{x \in \Z} \nu(x) \mu(-x)
\end{align*}

Therefore,
$$
    \sum_{z \in \Z} (\nu(x) - \nu(0)) \mu(-x) = 0
$$

A stationary measure for $x$ is one so that $x \mapsto \nu(x) - \nu(0)$ is orthogonal to $x \mapsto \mu(-x)$. In particular, one choice of $\nu$ is the uniform measure, that is $\nu(x) = 1$ for all $x \in \Z$

\subsection{can $X$ be positive recurrent?}

No in general. Because suppose $X$ is positive recurrent, then there exists a unique stationary distribution. Let $p: S \to \R$ be a non-zero vector that is orthogonal to $x \mapsto \mu(-x)$, then we can construct two stationary measures. Let $x \in S$ such that $p(x) > 0$, and take $\nu_1(0) = \nu_2(0) = 1$
\begin{align*}
    \nu_1(x) - \nu_1(0) = p(x) \\
    \nu_2(x) - \nu_2(0) = 2p(x) \\
\end{align*}

then, 
$$
    \nu_2(x) = 1 + 2 (\nu_1(x) - 1)
$$

which is not linear.
\section{Q7}

Let $X$ be a random walk on $\set{0, 1, ..., L}$ with transition matrix $\Pi$ such that $\Pi(0, 1) = \Pi(L, L-1) = 1$, $\Pi(i, i+1) = 1 - \Pi(i, i-1) = p$ for all $1 \leq i \leq L-1$. Find the stationary distribution of this random walk. If $X_0 = 0$, find the expected number of visits to $L$ before returning $0$ as well as the expected time of returning.


We can assume that $p \in (0, 1)$ as if $p=0$ or $p=1$, the answers are trivial. We can also assume that $L \geq 3$ as if $L = 1$ or $L=2$, the answers are trivial.

It is clear that $X$ is irreducible since for any pair of states, there exists a path with positive probability connecting them. As the number of states are finite, therefore, $X$ is recurrent. Therefore, there exists a unique stationary distribution for $X$

\subsection{stationary distribution of $X$}

Let $\mu: \set{0, 1, ..., L} \to \R$ be the stationary measure of $X$. Then, we have $\mu = \mu \Pi$, that is, for every $y \in \set{0, 1, ..., L}$, 
$$
    \mu_y = \sum_{x \in \set{0, 1, ..., L}} \mu_x \Pi(x, y)
$$

We have the following
\begin{align*}
    \mu_0 &= \mu_1 \Pi(1, 0) = \mu_1 (1 - p) \\
    \mu_1 &= \mu_0 \Pi(0, 1) + \mu_2 \Pi(2, 1) = \mu_0 + \mu_2 (1-p) \\
    \mu_L &= \mu_{L-1} \Pi(L-1, L) = \mu_{L-1} p \\
    \mu_{L-1} &= \mu(L-2) \Pi(L-2, L-1) + \mu_L \Pi(L, L-1) = \mu_{L-2} p + \mu_L \\
    \mu_i &= \mu_{i-1} \Pi(i-1, i) + \mu_{i+1} \Pi(i+1, i) = \mu_{i-1} p + \mu_{i+1} (1-p) &\text{(if $i \in \set{2, ..., L-2}$)} 
\end{align*}

For convenient, let $\nu: \set{0, 1, ..., L} \to \R$ be defined by 
$$
    \nu_i = \begin{cases}
        \frac{\mu_0}{p} &\text{if $y = 0$} \\
        \mu_i &\text{if $y \in \set{1, ..., L-1}$} \\
        \frac{\mu_L}{1-p} &\text{if $y = L$} \\
    \end{cases}
$$

We have the relation, for all $i \in \set{1, ..., L-1}$,
$$
    \nu_i = \nu_{i-1} p + \nu_{i+1} (1-p)
$$

Let $r^i$ be a basic solution for $\nu_i$, then
$$
    1 = \frac{1}{r} p + r (1-p)
$$

Solve for $r$ we have
$$
    r_1 = 1 \text{ or } r_2 = \frac{p}{1-p}
$$

Hence, general solution for $\nu_i$ is
$$
    \nu_i = a r_1^i + b r_2^i = a + b \tuple*{\frac{p}{1-p}}^i
$$

Since, $\mu_0 = \mu_1(1-p)$ and $\mu_L = \mu_{L-1} p$, then
$$
    p \nu_0 = \nu_1(1-p) \text{ and } (1-p) \nu_L = \nu_{L-1} p
$$

We must have $a = 0$, then $\nu_i = b r_2^i$. As $\mu: \set{0, 1, ..., L} \to \R$ is a distribution, we must have
\begin{align*}
    1
    &= \mu_0 + \mu_L + \sum_{i=1}^{L-1} \mu_i \\
    &= p \nu_0 + (1-p) \nu_L + \sum_{i=1}^{L-1} \nu_i \\
    &= p b + (1-p) b r_2^L + \sum_{i=1}^{L-1} b r_2^i \\
    &= b \tuple*{p + (1-p) r_2^L + \sum_{i=1}^{L-1} r_2^i} \\
    &= b \tuple*{p + (1-p) r_2^L + \frac{r_2 - r_2^L}{1 - r_2}} \\
    &= b \frac{2p(1-p)(1-r_2^L)}{1 - 2p}
\end{align*}

Therefore,
$$
    b = \frac{1 - 2p}{2p(1-p)(1-r_2^L)} \text{ and } \nu_i = b r_2^i
$$

$$
    \mu_i = \begin{cases}
        p b &\text{if $i = 0$} \\
        b r_2^i &\text{if $i \in \set{1, ..., L-1}$} \\
        (1-p) b r_2^L &\text{if $i = L$} \\
    \end{cases}
$$

\subsection{find the expected number of visits to L before
returning to 0, as well as the expected time of returning}

Let $T_x = \min\set{n > 0: X_n = x}$ be the first time $X$ visits $x \in S$ in positive time, then the cycle trick construction states that if $x$ is recurrent then, $\mu_x: S \to [0,\infty)$ defined by 
$$
    \mu_x(y) = \Exp_x\bracket*{\sum_{n=0}^{T_x - 1} 1_{\set{X_n = y}}} = \sum_{n=0}^\infty P_x(X_n = y, n < T_x)
$$

is a stationary measure and if $\Exp_x[T_x] < \infty$, then $\frac{\mu_x(y)}{\Exp_x[T_x]}$ is a stationary distribution. Since $X$ is recurrent, then stationary distribution exists and unique. That is, $\frac{\mu_x(y)}{\Exp_x[T_x]}$ coincides with the stationary distribution calculated in the previous part.

Hence, we need to calculate $\mu_0(L)$ and $\Exp_0[T_0]$. We have
\begin{align*}
    \Exp_0[T_0]
    &= \sum_{n=1}^\infty n P_0(T_0 = n) \\
    &= \sum_{n=1}^\infty P_0(T_0 \geq n) \\
\end{align*}

if $n = 1$, $P_0(T_0 \geq 1) = 1$ and if $n = 2$
$$
    P_0(T_0 \geq 2) = P_0(X_1 \neq 0) = 1
$$

if $n \geq 3$, without loss of generality, let $X_i$ distributed according to the stationary distribution
\begin{align*}
    P_0(T_0 \geq n)
    &= P_0(X_1 \neq 0, ..., X_{n-2} \neq 0, X_{n-1} \neq 0) \\
    &= P(X_{n-1} \neq 0 | X_{n-2} \neq 0) P_0(X_1 \neq 0, ..., X_{n-2} \neq 0) &\text{(Markov property)}\\
    &= P(X_1 \neq 0 | X_0 \neq 0) P_0(T_0 \geq n-1) &\text{(stationary)}
\end{align*}

We will calculate the constant $P(X_1 \neq 0 | X_0 \neq 0)$

\begin{align*}
    P(X_1 \neq 0 | X_0 \neq 0)
    &= \frac{P(X_1 \neq 0, X_0 \neq 0)}{P(X_0 \neq 0)} \\
    &= \frac{P(X_1 \neq 0) - P(X_1 \neq 0, X_0 = 0)}{P(X_0 \neq 0)} \\
    &= \frac{P(X_1 \neq 0) - (P(X_0 = 0) - P(X_1 = 0, X_0 = 0))}{P(X_0 \neq 0)} \\
    &= \frac{P(X_1 \neq 0) - P(X_0 = 0)}{P(X_0 \neq 0)} \\
    &= \frac{1 - 2pb}{1-pb}
\end{align*}

Therefore, for $n \geq 3$,
$$
    P_0(T_0 \geq n) =  \tuple*{\frac{1 - 2pb}{1-pb}}^{n-2}
$$

Then,
$$
    \Exp_0[T_0] = 1 + 1 + \sum_{n=3}^\infty \tuple*{\frac{1 - 2pb}{1-pb}}^{n-2} = \frac{1}{pb} = \frac{2(1-p)(1-r_2^L)}{1 - 2p}
$$

Then, 
$$
    \mu_0(L) = \Exp_0[T_0] (1 - p) b r_2^L = r_2^{L-1} = \tuple*{\frac{p}{1-p}}^{L-1}
$$

\section{Q8}

At each time $n \in \N$, we light up $\xi_n$ many candles where $(\xi_n)_{n \in \N}$ are i.i.d Poisson random variables with mean $1$. The lifetimes of candles are assume to be i.i.d integer-valued with a common probability mass function $f = (f_k)_{k \in \N}$. Assume that the mean lifetime $\lambda = \sum_{k=1}^\infty k f(k) < \infty$. Let $X_n$ denote the number of candles burning at time $n$. Is $X$ a Markov chain? Is it irreducible? Is it aperiodic? Find the stationary distribution if it exists.

\subsection{is $X$ a Markov chain?}

Let $A_t$ be the number of new candles at time $t \in \N_0$. At time $n$, let $z$ be the lifetime of a candle ($z=0$ means that a candle does not last until next time step), then the number of candles burning at time $t$ is
\begin{align*}
    X_t &= \sum_{i=1}^{A_0} 1_{\set{z_i^0 \geq t}} + ... + \sum_{i=1}^{A_n} 1_{\set{z_i^n \geq t-n}} + ... + \sum_{i=1}^{A_t} 1_{\set{z_i^t \geq 0}} \\
    X_{t+1} &= \sum_{i=1}^{A_0} 1_{\set{z_i^0 \geq t+1}} + ... + \sum_{i=1}^{A_n} 1_{\set{z_i^n \geq t-n+1}} + ... + \sum_{i=1}^{A_t} 1_{\set{z_i^t \geq 1}} + \sum_{i=1}^{A_{t+1}} 1_{\set{z_i^{t+1} \geq 0}}
\end{align*}

Then, 
$$
    X_{t+1} = X_{t} - \tuple*{\sum_{i=1}^{A_0} 1_{\set{z_i^0 = t}} + ... + \sum_{i=1}^{A_n} 1_{\set{z_i^n= t-n}} + ... + \sum_{i=1}^{A_t} 1_{\set{z_i^t = 0}}} + \sum_{i=1}^{A_{t+1}} 1_{\set{z_i^{t+1} \geq 0}}
$$

That is, the number of burning candles at time $t+1$ is the number of candles at time $t$ that have positive remaining times plus a random variable. It is generally not a Markov chain unless we can infer the number of candles at time $t$ that have positive remaining times from the total number of candles at time $t$.

Let's construct a counterexample. Suppose distribution of candle lifetime is
$$
    f(k) = \begin{cases}
        0 &\text{if $k = 0$} \\
        1 &\text{if $k = 1$}
    \end{cases}
$$

Then, 
\begin{align*}
    P(X_2 = \cdot | X_1 = 1, X_0 = 0) &= 1 + Pois(1) \\
    P(X_2 = \cdot | X_1 = 1, X_0 = 1) &= Pois(1) \\
\end{align*}

\subsection{is it irreducible?} 

Yes, it is possible to go from a state to any other state. Let $X_n = N$, there is a positive probability such that $X_{n+1} = N + k$ for $k \in \N$ equals to probability of burning $k$ plus the number of candles at time $n$ with $0$ remaining time. To go to $0$, there is a finite number of steps so that all burning candles turn off, that's a positive probability transition ($(e^{-1})^{\text{time}}$)

\subsection{is it aperiodic?}

Yes, it is aperiodic since there is a positive probability of staying at the same state by burning exactly the number of candles with $0$ remaining time in the next time step.

\subsection{find the stationary distribution if it exists}

We have
$$
X_t = \sum_{i=1}^{A_0} 1_{\set{z_i^0 \geq t}} + ... + \sum_{i=1}^{A_n} 1_{\set{z_i^n \geq t-n}} + ... + \sum_{i=1}^{A_t} 1_{\set{z_i^t \geq 0}}
$$

Since $\infty > \lambda = \sum_{k=1}^\infty kf(k) = \sum_{k=1}^\infty P(z \geq k)$. Therefore, $P(z \geq k) \to 0$ as $k \to \infty$.  


\note{TODO}


\section{Q9}

A birth-death chain is a Markov chain with state space $\N_0$ and transition probabilities $\Pi(0, 1) = p_0$, $\Pi(0, 0) = r_0 = 1 - p_0$ and for each $k \in \N$, we have $\Pi(k, k+1) = p_k$, $\Pi(k, k-1) = q_k$, $\Pi(k, k) = r_k$ with $p_k + q_k + r_k = 1$. Find a necessary and sufficient condition for this Markov chain to be irreducible. Show that when $X$ is irreducible, it is in fact reversible and identify the reversible measure. Find a necessary and sufficient condition for this Markov chain to be positive recurrent.

\subsection{Find a necessary and sufficient condition for this Markov chain to be irreducible}

Let $0 \leq m < n$, since the Markov chain can only transition to the adjacent natural numbers, a valid transition from $m$ to $n$ must go through $n - 1$, therefore, $\Pi(n-1, n) = p_{n-1} > 0$ for all $n > 0$. Similarly, let $0 \leq n < m$, all valid transition from $m$ to $n$ must go through $n+1$, therefore, $\Pi(n+1, n) = q_{n+1} > 0$ for all $n \geq 0$. Hence, there necessary condition is
\begin{align*}
    p_k > 0 &\text{ for all $k \geq 0$} \\
    q_k > 0 &\text{ for all $k \geq 1$}  
\end{align*}

Moreover, it is also the sufficient condition. If $0 \leq m < n$, there exists a path with positive probability
$$
    \Pi(m, m+1) \Pi(m+1, m+2) ... \Pi(n-1, n) = p_m p_{m+1} ... p_{n-1} > 0
$$

If $0 \leq n < m$, there exists a path with positive probability
$$
    \Pi(n, n-1) \Pi(n-1, n-2) ... \Pi(m+1, m) = q_n q_{n-1} ... q_{m+1} > 0
$$

\subsection{Show that when $X$ is irreducible, it is in fact reversible and identify the reversible measure}

When $X$ is irreducible, it satisfies the loop condition since probability of every loop is a product of terms $\Pi(k, k+1) \Pi(k, k-1)$. If we reverse direction, the probability stays the same.

Detailed balance condition for reversible measure $\nu: \N_0 \to \R$
$$
    \nu(k) \Pi(k, k+1) = \nu(k+1) \Pi(k+1, k)
$$

therefore $\frac{\nu(k+1)}{\nu(k)} = \frac{\Pi(k, k+1)}{\Pi(k+1, k)} = \frac{p_k}{q_{k+1}}$. Let $\nu(0) = 1$, then
$$
    \nu(n) = \prod_{k=0}^n \frac{p_k}{q_{k+1}}
$$

\subsection{Find a necessary and sufficient condition for this Markov chain to be positive recurrent}

In order to avoid confusion, we define the following definitions: 
\begin{enumerate}
    \item \textbf{irreducible}: there exists a positive probability path from any state to any state
    \item \textbf{recurrent}: starting from any state, the probability of returning in finite time is $1$
    \item \textbf{positive recurrent}: for all $x \in S$, $\Exp_x[T_x] < \infty$
\end{enumerate}

It easy to see that \textbf{positive recurrent} implies \textbf{recurrent}, since if there exists a probability of escaping, $\Exp_x[T_x] = \infty$. Moreover, \textbf{recurrent} does not imply \textbf{irreducible} as we can make probability of staying in state being $1$, so that $\Exp_x[T_x] = 1$


The question is to find the necessary and sufficient condition so that $X$ is all of \textbf{irreducible} and \textbf{positive recurrent}.

\subsubsection{necessary part}

If $X$ is \textbf{irreducible} and \textbf{recurrent}, then it has a stationary distribution, since the reversible measure is stationary, then it can be normalized. i.e. $\sum_{n \in \N_0} \nu(v) < \infty$. Let $\mu: \N_0 \to \R$ be the stationary distribution
$$
    \mu(v) = \frac{\nu(v)}{\sum_{n \in \N_0} \nu(v)}
$$

We have
$$
\Exp_x[T_x]
    = \sum_{n=1}^\infty n P_x(T_x = n)
    = \sum_{n=1}^\infty P_x(T_x \geq n)
$$

Note that $P_x(T_x \geq 2m-1) = P_x(T_x \geq 2m)$, then 
$$
    \Exp_x[T_x] = 2 \sum_{m=1} P_x(T_x \geq 2m)
$$

We will calculate $P_x(T_x \geq 2m)$,

$$
    P_x(T_x \geq 2) = 1 - P_x(T_x < 2) = P_x(T_x = 0) = r_0
$$

If $m \geq 2$, suppose $X$ is distributed according to the stationary distribution, we have

\begin{align*}
    P_x(T_x \geq 2m)
    &= P_x(T_x \geq 2m-1) \\
    &= P_x(X_1 \neq 0, ..., X_{2m-2} \neq 0) \\
    &= P(X_{2m-2} \neq 0 | X_{2m-3} \neq 0) P_x(X_1 \neq 0, ..., X_{2m-3} \neq 0) \\
    &= P(X_{2m-2} \neq 0 | X_{2m-3} \neq 0) P_x(T_x \geq 2m-2) \\
    &= P(X_{2m-2} \neq 0) P_x(T_x \geq 2m-2) &\text{($\set{X_{2m-3} \neq 0}$ is a sure event)}\\
    &= (1 - \mu(0)) P_x(T_x \geq 2m-2)
\end{align*}

Therefore,
$$
    \Exp_x[T_x] = 2 \sum_{m=1} P_x(T_x \geq 2m) = 2 \sum_{m=1} r_0 (1 - \mu(0))^{m-1}
$$

As long as $\mu(0) < 1$, the series converges. Therefore, the necessary condition
is all of (1) $X$ is \textbf{irreducible} (2) $\sum_{n \in \N_0} \nu(v) < \infty$ and (3) $\mu(0) < 1$.

\subsubsection{sufficient part}

On the other hand, if $X$ is (1) \textbf{irreducible} and (2) $\sum_{n \in \N_0} \nu(v) < \infty$, the stationary distribution is defined. Moreover, if (3) $\mu(0) < 1$ is true, for all $x \in \N_0$, $\Exp_x[T_x] < 0$, then $X$ is \text{positive recurrent}. Therefore, (1) (2) (3) is the sufficient condition.

\section{Question 10}
Consider a knight jumping randomly on a $5 \times 5$ square. At each step, it picks one of the admissible moves with equal probability. Find the stationary distribution for this Markov chain by identifying Markov chain as a random walk on a graph.

The Markov chain is finite state and irreducible, therefore, there exists a stationary distribution. Construct the graph and set all conductances to $1$. So that the reversible measure of a node is the number edges on that node.

We denote the symmetry by the matrix below:

$$
    \begin{bmatrix}
        d & e & f & e & d \\
        e & b & c & b & e \\
        f & c & a & c & f \\
        e & b & c & b & e \\
        d & e & f & e & d
    \end{bmatrix}
$$

The reversible measure $\nu: \set{a,b,c,d,e,f} \to \R$ is as follows
\begin{align*}
    \nu(a) &= 8 \\
    \nu(b) &= 4 \\
    \nu(c) &= 6 \\
    \nu(d) &= 2 \\
    \nu(e) &= 3 \\
    \nu(f) &= 4
\end{align*}

The stationary distribution $\mu: \set{a,b,c,d,e,f} \to \R$ is as follows with $s = \nu(a) + 4 \nu(b) + 4 \nu(c) + 4 \nu(d) + 8 \nu(e) + 4 \nu(f) = 96$
\begin{align*}
    \mu(a) &= \frac{\nu(a)}{s} = \frac{1}{12} \\
    \mu(b) &= \frac{\nu(b)}{s} = \frac{1}{24} \\
    \mu(c) &= \frac{\nu(c)}{s} = \frac{1}{16} \\
    \mu(d) &= \frac{\nu(d)}{s} = \frac{1}{48} \\
    \mu(e) &= \frac{\nu(e)}{s} = \frac{1}{32} \\
    \mu(f) &= \frac{\nu(f)}{s} = \frac{1}{24}
\end{align*}

\section{Question 11}
Suppose that the knight starts at the center
\begin{enumerate}
    \item Find the probability that the knight will reach one of the corners before returning to the center
    \item Find the expected time it takes for the knight to reach one of the corners
\end{enumerate}

\subsection{the knight will reach one of the corners before returning to the center}

Let $D$ be the set of corners, $a$ the center, and $A = \set{a}$. Let $T_A = \min\set{n \geq 0: X_n \in A}$ be the first time $X$ visits $A$, $T_D = \min\set{n \geq 0: X_n \in D}$ be the first time $X$ visits $D$. Since starting from $a$, it must go to one of the 8 squares labelled by $e$, then the probability of hitting $D$ before $A$ is

$$
    \phi(e) = P_e(T_D < T_A)
$$


We have, for any $x \notin A \cup D$,
\begin{align*}
    \phi(x) = P_x(T_D < T_A)
    &= \sum_{y \in S} P_x(X_1 = y, T_D < T_A) \\
    &= \sum_{y \in S} P_x(T_D < T_A| X_1 = y) P(X_1 = y | X_0 = x) \\
    &= \sum_{y \in S} \phi(y) \Pi(x, y) \\
\end{align*}

We have the following system of equations
\begin{align*}
    \phi(a) &= 0 \\
    \phi(b) &= 2 \phi(e) \Pi(b, e) + 2 \phi(c) \Pi(b, c) = \frac{1}{2}\phi(e) + \frac{1}{2}\phi(c) \\
    \phi(c) &= 2 \phi(d) \Pi(c, d) + 2 \phi(f) \Pi(c, f) + 2 \phi(b) \Pi(c, b) = \frac{1}{3}\phi(d) + \frac{1}{3}\phi(f) + \frac{1}{3}\phi(b) \\
    \phi(d) &= 1 \\
    \phi(e) &= \phi(b) \Pi(e, b) + \phi(a) \Pi(e, a) + \phi(f) \Pi(e, f) = \frac{1}{3}\phi(b) + \frac{1}{3}\phi(a) + \frac{1}{3}\phi(f) \\
    \phi(f) &= 2 \phi(e) \Pi(f,e) + 2 \phi(c) \Pi(f, c) = \frac{1}{2}\phi(e) + \frac{1}{2}\phi(c)
\end{align*}

Solving the system of equations, we have $\phi(e) = \frac{1}{3}$

\subsection{the expected time it takes for the knight to reach one of the corners}

Let $D$ be the set of corners, $T_D = \min\set{n \geq 0: X_n \in D}$ be the first time $X$ visits $D$. We want to find
$$
    \Exp_a[T_D]
$$

We have, for any $x \notin D$,
\begin{align*}
    \psi(x)
    &= \Exp_x[T_D] \\
    &= \sum_{y \in S} \Exp_x[T_D 1_{\set{X_1 = y}}] \\
    &= 1 + \sum_{y \in S} \Exp_x[(T_D - 1) 1_{\set{X_1 = y}}] \\
    &= 1 + \sum_{y \in S} P(X_1 = y | X_0 = x) \Exp_x[\Exp_x[T_D - 1 | X_1 = y]] \\
    &= 1 + \sum_{y \in S} P(X_1 = y | X_0 = x) \Exp_y[T_D] \\ 
    &= 1 + \sum_{y \in S} \Pi(x, y) \psi(y) \\ 
\end{align*}

We have the following system of equations
\begin{align*}
    \psi(a) &= 1 + 8 \psi(e) \Pi(a, e) = 1 + \psi(e) \\
    \psi(b) &= 1 + 2 \psi(e) \Pi(b, e) + 2 \psi(c) \Pi(b, c) = 1 + \frac{1}{2} \psi(e) + \frac{1}{2} \psi(c) \\
    \psi(c) &= 1 + 2 \psi(d) \Pi(c, d) + 2 \psi(f) \Pi(c, f) + 2 \psi(b) \Pi(c, b) = 1 + \frac{1}{3} \psi(d) + \frac{1}{3} \psi(f) + \frac{1}{3} \psi(b) \\
    \psi(d) &= 0 \\
    \psi(e) &= 1 + \psi(b) \Pi(e, b) + \psi(a) \Pi(e, a) + \psi(f) \Pi(e, f) = 1 + \frac{1}{3}\psi(b) + \frac{1}{3}\psi(a) + \frac{1}{3}\psi(f) \\
    \psi(f) &= 1 + 2 \psi(e) \Pi(f,e) + 2 \psi(c) \Pi(f, c) = 1 + \frac{1}{2}\psi(e) + \frac{1}{2}\psi(c)
\end{align*}

Solving the system of equations, we have $\psi(a) = 18$

\end{document}
