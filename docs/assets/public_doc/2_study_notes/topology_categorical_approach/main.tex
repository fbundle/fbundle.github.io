\documentclass{report}
\usepackage{graphicx} % Required for inserting images

% header

%% natbib
\usepackage{natbib}
\bibliographystyle{plain}

%% comment
\usepackage{comment}

% no automatic indentation
\usepackage{indentfirst}

% manually indent
\usepackage{xargs} % \newcommandx
\usepackage{calc} % calculation
\newcommandx{\tab}[1][1=1]{\hspace{\fpeval{#1 * 10}pt}}
% \newcommand[number of parameters]{output}
% \newcommandx[number of parameters][parameter index = x]{output}
% use parameter index = x to substitute the default argument
% use #1, #2, ... to get the first, second, ... arguments
% \tab for indentation
% \tab{2} for for indentation twice

% note
\newcommandx{\note}[1]{\textit{\textcolor{red}{#1}}}
\newcommand{\todo}{\note{TODO}}
% \note{TODO}

%% math package
\usepackage{amsfonts}
\usepackage{amsmath}
\usepackage{amssymb}
\usepackage{tikz-cd}
\usepackage{mathtools}
\usepackage{amsthm}

%% operator
\DeclareMathOperator{\tr}{tr}
\DeclareMathOperator{\diag}{diag}
\DeclareMathOperator{\sign}{sign}
\DeclareMathOperator{\grad}{grad}
\DeclareMathOperator{\curl}{curl}
\DeclareMathOperator{\Div}{div}
\DeclareMathOperator{\card}{card}
\DeclareMathOperator{\Span}{span}
\DeclareMathOperator{\real}{Re}
\DeclareMathOperator{\imag}{Im}
\DeclareMathOperator{\supp}{supp}
\DeclareMathOperator{\im}{im}
\DeclareMathOperator{\aut}{Aut}
\DeclareMathOperator{\inn}{Inn}
\DeclareMathOperator{\Char}{char}
\DeclareMathOperator{\Sylow}{Syl}
\DeclareMathOperator{\coker}{coker}
\DeclareMathOperator{\inc}{in}
\DeclareMathOperator{\Sd}{Sd}
\DeclareMathOperator{\Hom}{Hom}
\DeclareMathOperator{\interior}{int}
\DeclareMathOperator{\ob}{ob}
\DeclareMathOperator{\Set}{Set}
\DeclareMathOperator{\Top}{Top}
\DeclareMathOperator{\Meas}{Meas}
\DeclareMathOperator{\Grp}{Grp}
\DeclareMathOperator{\Ab}{Ab}
\DeclareMathOperator{\Ch}{Ch}
\DeclareMathOperator{\Fun}{Fun}
\DeclareMathOperator{\Gr}{Gr}
\DeclareMathOperator{\End}{End}
\DeclareMathOperator{\Ad}{Ad}
\DeclareMathOperator{\ad}{ad}
\DeclareMathOperator{\Bil}{Bil}
\DeclareMathOperator{\Skew}{Skew}
\DeclareMathOperator{\Tor}{Tor}
\DeclareMathOperator{\Ho}{Ho}
\DeclareMathOperator{\RMod}{R-Mod}
\DeclareMathOperator{\Ev}{Ev}
\DeclareMathOperator{\Nat}{Nat}
\DeclareMathOperator{\id}{id}
\DeclareMathOperator{\Var}{Var}
\DeclareMathOperator{\Cov}{Cov}
\DeclareMathOperator{\RV}{RV}
\DeclareMathOperator{\rank}{rank}

%% pair delimiter
\DeclarePairedDelimiter{\abs}{\lvert}{\rvert}
\DeclarePairedDelimiter{\inner}{\langle}{\rangle}
\DeclarePairedDelimiter{\tuple}{(}{)}
\DeclarePairedDelimiter{\bracket}{[}{]}
\DeclarePairedDelimiter{\set}{\{}{\}}
\DeclarePairedDelimiter{\norm}{\lVert}{\rVert}

%% theorems
\newtheorem{axiom}{Axiom}
\newtheorem{definition}{Definition}
\newtheorem{theorem}{Theorem}
\newtheorem{proposition}{Proposition}
\newtheorem{corollary}{Corollary}
\newtheorem{lemma}{Lemma}
\newtheorem{remark}{Remark}
\newtheorem{claim}{Claim}
\newtheorem{problem}{Problem}
\newtheorem{assumption}{Assumption}
\newtheorem{example}{Example}
\newtheorem{exercise}{Exercise}

%% empty set
\let\oldemptyset\emptyset
\let\emptyset\varnothing

\newcommand\eps{\epsilon}

% mathcal symbols
\newcommand\Tau{\mathcal{T}}
\newcommand\Ball{\mathcal{B}}
\newcommand\Sphere{\mathcal{S}}
\newcommand\bigO{\mathcal{O}}
\newcommand\Power{\mathcal{P}}
\newcommand\Str{\mathcal{S}}


% mathbb symbols
\usepackage{mathrsfs}
\newcommand\N{\mathbb{N}}
\newcommand\Z{\mathbb{Z}}
\newcommand\Q{\mathbb{Q}}
\newcommand\R{\mathbb{R}}
\newcommand\C{\mathbb{C}}
\newcommand\F{\mathbb{F}}
\newcommand\T{\mathbb{T}}
\newcommand\Exp{\mathbb{E}}

% mathrsfs symbols
\newcommand\Borel{\mathscr{B}}

% algorithm
\usepackage{algorithm}
\usepackage{algpseudocode}

% longproof
\newenvironment{longproof}[1][\proofname]{%
  \begin{proof}[#1]$ $\par\nobreak\ignorespaces
}{%
  \end{proof}
}


% for (i) enumerate
% \begin{enumerate}[label=(\roman*)]
%   \item First item
%   \item Second item
%   \item Third item
% \end{enumerate}
\usepackage{enumitem}

% insert url by \url{}
\usepackage{hyperref}

% margin
\usepackage{geometry}
\geometry{
a4paper,
total={190mm,257mm},
left=10mm,
top=20mm,
}



\title{topology - a categorical approach}
\author{Khanh Nguyen}
\date{June 2024}

\begin{document}

\maketitle

\chapter{FOUR CONSTRUCTIONS OF TOPOLOGY}

Categorical definitions of some common topologies

\begin{definition}[subspace topology]
    Let $(X, \Tau_X)$ be a topological space and $i: A \to X$ be a monomorphism in $\Set$. The subspace topology $\Tau_A$ on $A$ is defined as the coarsest topology such that the map $i: A \to X$ is continuous. Equivalently, the subset topology on $A$ is characterized by universal property as follows: for any morphism $(Y, \Tau_Y) \to (X, \Tau_X)$ in $\Top$, if there is a lift $Y \to X$ in $\Set$ then there is also a lift $(Y, \Tau_Y) \to (A, \Tau_A)$ in $\Top$ such that the diagram below commutes
    \begin{center}
\begin{tikzcd}
                       & Y \arrow[d] \arrow[ld, dashed] &                                    & {(Y, \Tau_Y)} \arrow[d] \arrow[ld, dashed] \\
A \arrow[r, "i", hook] & X                              & {(A, \Tau_A)} \arrow[r, "i", hook] & {(X, \Tau_X)}                             
\end{tikzcd}
    \end{center}
\end{definition}

\begin{definition}[quotient topology]
    Let $(X, \Tau_X)$ be a topological space and $p: X \to B$ be an epimorphism in $\Set$. The quotient topology $\Tau_B$ on $B$ is defined as the finest topology such that the map $p: X \to B$ is continuous. Equivalently, the quotient topology on $B$ is characterized by universal property as follows: for any morphism $(X, \Tau_X) \to (Y, \Tau_Y)$, if there is a lift $B \to Y$ in $\Set$ then there is also a lift $(B, \Tau_B) \to (Y, \Tau_Y)$ in $\Top$ such that the diagram below commutes
    \begin{center}
\begin{tikzcd}
X \arrow[r, "p", two heads] \arrow[d] & B \arrow[ld, dashed] & {(X, \Tau_X)} \arrow[r, "p", two heads] \arrow[d] & {(B, \Tau_B)} \arrow[ld, dashed] \\
Y                                     &                      & {(Y, \Tau_Y)}                                     &                                 
\end{tikzcd}
    \end{center}
\end{definition}

\begin{definition}[product topology]
    Let $\set{(X_\alpha, \Tau_\alpha)}_\alpha$ be an arbitrary collection of topological spaces, $X = \prod_\alpha X_\alpha$, and $p_\alpha: X \to X_\alpha$ be the natural projection. The product topology $\Tau_X$ on $X$ is defined as the coarsest topology such that every $p_\alpha: X \to X_\alpha$ is continuous. Equivalently, the product topology is characterized by universal property as follows: for any collection of morphisms $\set{(Y, \Tau_Y) \to (X_\alpha, \Tau_\alpha)}_\alpha$, if there is a lift $Y \to X$ in $\Set$ then there is also a lift $(Y, \Tau_Y) \to (X, \Tau_X)$ in $\Top$ such that the diagram below commutes
    \begin{center}
\begin{tikzcd}
                                   & Y \arrow[d] \arrow[ld, dashed] &                                              & {(Y, \Tau_Y)} \arrow[d] \arrow[ld, dashed] \\
X \arrow[r, "p_\alpha", two heads] & X_\alpha                       & {(X, \Tau)} \arrow[r, "p_\alpha", two heads] & {(X_\alpha, \Tau_\alpha)}                 
\end{tikzcd}
    \end{center}
\end{definition}

\begin{definition}[coproduct topology]
    Let $\set{(X_\alpha, \Tau_\alpha)}_\alpha$ be an arbitrary collection of topological spaces, $X = \coprod_\alpha X_\alpha$, and $i_\alpha: X_\alpha \to X$ be the natural inclusion. The coproduct topology $\Tau_X$ on $X$ is defined as the finest topology such that every $i_\alpha: X_\alpha \to$ is continuous. Equivalently, the coproduct topology is characterized by universal property as follows: for any collection of morphisms $\set{(X_\alpha, \Tau_\alpha) \to (Y, \Tau_Y)}_\alpha$, if there is a lift $X \to Y$ in $\Set$ then there is also a lift $(X, \Tau_X) \to (Y, \Tau_Y)$ in $\Top$ such that the diagram below commutes
    \begin{center}
\begin{tikzcd}
X_\alpha \arrow[d] \arrow[r, "i_\alpha", hook] & X \arrow[ld, dashed] & {(X_\alpha, \Tau_\alpha)} \arrow[r, "i_\alpha", hook] \arrow[d] & {(X, \Tau_X)} \arrow[ld, dashed] \\
Y                                              &                      & {(Y, \Tau_Y)}                                                   &                                 
\end{tikzcd}
    \end{center}
\end{definition}

testing birtual keybaord
 
\end{document}
