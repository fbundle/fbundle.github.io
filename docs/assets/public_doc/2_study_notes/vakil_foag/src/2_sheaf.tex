\section{SHEAVES}

\subsection{MOTIVATING EXAMPLE: THE SHEAF OF SMOOTH FUNCTIONS}

\note{TODO}

\subsection{DEFINITION OF SHEAVES AND PRESHEAVES}

\begin{definition}[presheaf, section]
	A presheaf of sets $\ms{F}$ on a topological space $X$ is the following data:
	\begin{enumerate}
		\item for each open $U \subseteq X$, there is a set $\ms{F}(U)$, the elements of $\ms{F}(U)$ are called sections of $\ms{F}$ over $U$
		
		\item for each open set inclusion $U \hookrightarrow V$, there is a restriction map
		\begin{align*}
			\res_{V, U}: \ms{F}(V) &\to \ms{F}(U) \\
											   f &\mapsto f\vert_U
		\end{align*}
	\end{enumerate}
	
	The data is required to satisfy the following:
	\begin{enumerate}
		\item for each open set $U \subseteq X$, the restriction map $\res_{U, U}$ is identity$\res_{U, U} = \id_{\ms{F}(U)}$
		\item if $U \hookrightarrow V \hookrightarrow W$ are inclusions of open sets, then the diagram below commutes
		\begin{center}
			\begin{tikzcd}
				\ms{F}(W) \arrow[r, "\res"] \arrow[rr, "\res"', bend right] & \ms{F}(V) \arrow[r, "\res"] & \ms{F}(U)
			\end{tikzcd}
		\end{center}
	\end{enumerate}
	
	In other words, presheaf is a functor from the category of open sets on $X$ with restriction maps into the category of sets $\Set$
\end{definition}

\begin{definition}[germ, stalk]
	For any $p \in X$, consider the collection of sections of $\ms{F}$ over all open sets containing $p$
	$$
		\set{(f, U): p \subseteq U, f \in \ms{F}(U)}
	$$
	together with an equivalence relation $\sim$ as follows: $(f, U) \sim (g, V)$ if and only if there exists $W \subseteq U \cap V$ containing $p$ and $f\vert_W = g\vert_W$. The equivalence class of $f \in \ms{F}(U)$ is called the germ of $f$ at $p$. The collection of all germs at $p$ is called stalk of $\ms{F}$ at $p$ and denoted by $\ms{F}_p$
\end{definition}

\begin{remark}[filtered colimit, an equivalent definition of stalk]
	(The Stacks project - 4.19 Filtered colimit) A diagram $M: \mc{I} \to \mc{C}$ is called directed (or filtered) if the following conditions hold:
	\begin{enumerate}
		\item the category $\mc{I}$ has at least one object
		\item for every pair of objects $x, y$ of $\mc{I}$, there exists an object $z$ and morphisms $x \to z$ and $y \to z$
		\item for every pair of morphisms $a, b: x \to y$ of $\mc{I}$, there exists a morphism $c: y \to z$ so that $M(c \circ a) = M(c \circ b)$ as morphisms in $\mc{C}$ 
	\end{enumerate}
	
	\note{after some gibberish} - the colimit of $M$ (filtered colimit) always exists. In particular, when $\mc{C}$ is $\Set$, then
	$$
		\colim_{\mc{I}} M = \tuple*{\coprod_{i \in \mc{I}} M_i} / \sim
	$$
	
	where $m_i \sim m_j$ if and only if there exists $k \in \mc{I}$, $\phi_{ik}: i \to k$, $\phi_{jk}: j \to k$, and $M(\phi_{ik})(m_i) = M(\phi_{jk})(m_j)$. In formally, two elements are equal if they eventually become equal. When $\mc{I}$ is the category of open sets on $X$ with restriction maps containing $p$ and $M$ is $\ms{F}$, then stalk of $\ms{F}$ at $p$ is precisely the colimit of the corresponding diagram.
\end{remark}

\begin{remark}
	For any category $\mc{C}$, presheaf of $\mc{C}$ on $X$ is also defined similarly
\end{remark}

\begin{definition}[sheaf]
	A presheaf is called a sheaf if it satisfies two additional axioms: for any open set $U$ with open cover $\set{U_i}_{i \in I}$
	\begin{enumerate}
		\item (identity axiom) if two sections $f, g \in \ms{F}(U)$ have $f\vert_{U_i} = g\vert_{U_i}$ for every $i \in I$, then $f = g$
		
		\item (gluability axiom) if for each $i \in I$, there is a section $f_i \in \ms{F}(U_i)$ so that for every intersection $U = U_i \cap U_j$, $f_i\vert_U = f_j\vert_U$, then there exists a section $f \in \ms{F}(U)$ so that $f\vert_{U_i} = f_i$ for every $i \in I$
	\end{enumerate}
\end{definition}

\begin{remark}
	Some remarks on sheaf
	\begin{enumerate}
		\item Informally, identity axiom says there is at most one way to glue and gluability axiom says there is at least one way to glue.
		
		\item From the two axioms for sheaf, one can show that $\ms{F}(\emptyset)$ must be the final object in the target category. There is also a version of presheaf and sheaf where the empty set $\emptyset$ is not included.
		
		\item germ and stalk on sheaf is germ and stalk on presheaf
		
		\item The axioms of sheaf is equivalent to say $\ms{F}\tuple*{\bigcup_{i \in I} U_i}$ is the limit of the diagram consists of $\set{\ms{F}(U_i)}_{i \in I}$, all finite intersections of them, and restriction maps
		\begin{center}
			\begin{tikzcd}
				\ms{F}\tuple*{\bigcup_{i \in I} U_i} \arrow[d] \arrow[rdd] \arrow[rrd] &                               &                            \\
				\ms{F}(U_\alpha) \arrow[rr] \arrow[rd]                                 &                               & \ms{F}(U_\beta) \arrow[ld] \\
				& \ms{F}(U_\alpha \cap U_\beta) &                           
			\end{tikzcd}
		\end{center}
	\end{enumerate}
\end{remark}

\begin{definition}[restriction of sheaf]
	Suppose $\ms{F}$ is a sheaf on topological $X$ and $U \subseteq X$ is an open set, then the collection of $\set{\ms{F}(V): V \subseteq U}$ forms a sheaf on the subspace $U$ of topological space $X$
\end{definition}

\begin{remark}[skyscraper sheaf]
	Let $X$ be a topological space, $p \in X$, and a set $S$. Let $i_p: p \hookrightarrow X$ be the inclusion, then define $i_{p, *} S$ by
	$$
		(i_{p, *} S)(U) = \begin{cases}
			S &\text{ if } p \in U \\
			\set{\bullet} &\text{ if } p \notin U \\
		\end{cases}
	$$
	together with evident restriction maps, they form a sheaf and called skyscraper sheaf supported at $p$
\end{remark}

\begin{remark}[constant presheaf, locally constant sheaf]
	Let $X$ be a topological space and $S$ be any set, then $\underline{S}_{pre}(U) = S$ for all $U \subseteq S$ together with evident restriction maps, they form a presheaf and called constant presheaf.
	
	Let $X$ be a topological space and $S$ be any set, let $\ms{F}(U)$ be the set of maps $U \to S$ that are locally constant, i.e. for any $p \in U$, there is an open neighbourhood of $p$ where the function is constant, then together with evident restriction maps, they form a sheaf and called (locally) constant sheaf and denoted by $\underline{S}$
\end{remark}

\begin{definition}[pushforward sheaf]
	Let $\pi: X \to Y$ be a continuous map and $\ms{F}$ be a presheaf on $X$. Define the presheaf $\pi_* \ms{F}$ on $Y$ by
	$$
		(\pi_* \ms{F})(V) = \ms{F}(\pi^{-1} V)
	$$
	for every open set $V \subseteq Y$. Then, $\pi_* \ms{F}$ is a presheaf on $Y$ and called pushforward of $\ms{F}$. If $\ms{F}$ is a sheaf on $X$ then $\pi_* \ms{F}$ is a sheaf on $Y$.
\end{definition}

\begin{proposition}[pushforward induces maps of stalks]
	Let $\pi: X \to Y$ be a continuous map and $\ms{F}$ be a presheaf on $X$. Let $\ms{G}$ be the pushforward of $\ms{F}$ on $Y$. If $p \in X$ and $q = \pi(p)$, then there is a natural map of stalks $\ms{G}_q \to \ms{F}_p$ induced from the map of representatives
\end{proposition}

\begin{definition}[ringed space, $\ms{O}_X$-module]
	Let $\ms{O}_X$ be a sheaf of rings on topological space $X$, then $(X, \ms{O}_X)$ is called a ringed space. $\ms{O}_X$ is called the structure sheaf of the ringed space. Sections of the structure sheaf $\ms{O}_X$ over an open set $U \subseteq X$ are called functions on $U$.
	
	A sheaf $\ms{F}$ of abelian groups on ring spaced $(X, \ms{O}_X)$ is called sheaf of $\ms{O}_X$-modules if each $\ms{F}(U)$ is an $\ms{O}_X(U)$-module and the structure compatible with restriction maps, that is, for every restriction of open sets $U \to V$, the diagram below commutes
	\begin{center}
		\begin{tikzcd}
			\ms{O}_X(U) \times \ms{F}(U) \arrow[d, "{(\res, \res)}"'] \arrow[r, "\times"] & \ms{F}(U) \arrow[d, "\res"] \\
			\ms{O}_X(V) \times \ms{F}(V) \arrow[r, "\times"]                              & \ms{F}(V)                  
		\end{tikzcd}
	\end{center}
\end{definition}

\begin{remark}[$\Z$-modules and $\underline{\Z}$-modules]
	a sheaf of $\underline{\Z}$-modules is precisely a sheaf of abelian groups where $\underline{\Z}$ denotes the locally constant sheaf to $\Z$.
\end{remark}

\begin{remark}[vector bundle]
	a vector bundle is precisely a locally free sheaf of $\E$-modules where $\E$ is the sheaf of functions on the base space.
\end{remark}

\subsection{MORPHISMS OF PRESHEAVES AND SHEAVES}

\begin{definition}[morphism of presheaves, morphism of sheaves]
	A morphism $\phi: \ms{F} \to \ms{G}$ of presheaves of sets on $X$ is the data of maps $\phi(U): \ms{F}(U) \to \ms{G}(U)$ for every open set $U \subseteq X$ and compatible with restriction maps, that is, for every restriction of open sets $U \to V$, the diagram below commutes
	\begin{center}
		\begin{tikzcd}
			\ms{F}(U) \arrow[r, "\phi(U)"] \arrow[d, "\res"] & \ms{G}(U) \arrow[d, "\res"] \\
			\ms{F}(V) \arrow[r, "\phi(V)"]                   & \ms{G}(V)                  
		\end{tikzcd}
	\end{center}
	
	A morphism of sheaves of sets is a morphism of the underlying presheaves. Morphisms of (pre)sheaves make (pre)sheaves a category and called category of (pre)sheaves.
	
	Let $\Set_X$, $\Ab_X$, etc denote the categories of sheaves of sets, abelian groups, etc.
\end{definition}

\begin{proposition}[morphisms of (pre)sheaves induce morphisms of stalks]
	If $\phi: \ms{F} \to \ms{G}$ is a morphism of presheaves on $X$. If $p \in X$, then there is a natural map $\phi_p: \ms{F}_p \to \ms{G}_p$ induced from the map of representatives. On other words, there is a functor $\Set_X \to \Set$ and called stalkification functor.
\end{proposition}

\begin{proposition}
	Suppose $\pi: X \to Y$ is a continuous map of topological spaces, the pushforward gives a functor $\pi_*: \Set_X \to \Set_Y$
\end{proposition}

\begin{definition}[sheaf Hom]
	Suppose $\ms{F}$ and $\ms{G}$ are two sheaves of sets on $X$, let $Hom(\ms{F}, \ms{G})$ \footnote{\note{TODO - figure out how to write notation Hom correctly}} be defined as follows: for every open set $U \subseteq X$, let
	$$
		Hom(\ms{F}, \ms{G})(U) := \Hom(\ms{F}\vert_U, \ms{G}\vert_U)
	$$
	be the collection of morphisms of the restriction sheaves $\ms{F}\vert_U$, $\ms{G}\vert_U$. It is a sheaf and called "sheaf Hom"
\end{definition}

\subsubsection{presheaves of abelian groups form an abelian category - kernel and cokernel in category of presheaves}

\begin{definition}[presheaf kernel, cokernel]
	If $\phi: \ms{F} \to \ms{G}$ is a morphism of presheaves. Define the presheaf kernel $\ker_{pre} \phi$ and the presheaf cokernel $\coker_{pre} \phi$ by
	\begin{align*}
		(\ker_{pre} \phi)(U) &= \ker \phi(U) \\
		(\coker_{pre} \phi)(U) &= \coker \phi(U)
	\end{align*}
	for every open set $U$. Presheaf kernel and presheaf cokernel are presheaves and kernel and cokernel in the category of presheaves.
\end{definition}

\begin{remark}
	presheaves of abelian groups form an abelian category and work "open set by open set". subpresheaf, image presheaf, quotient presheaf are then defined "open set by open set"
\end{remark}

\begin{proposition}
	Given a topological space $X$, if $U$ is an open set on $X$, then $\ms{F} \mapsto \ms{F}(U)$ defines an exact functor from presheaves of abelian groups into abelian groups.
\end{proposition}

\begin{proposition}
	A sequence of presheaves $0 \to \ms{F}_1 \to \ms{F}_2 \to ... \to \ms{F}_n \to 0$ is exact if and only if $0 \to \ms{F}_1(U) \to \ms{F}_2(U) \to ... \to \ms{F}_n(U) \to 0$ is exact for every open set $U$.
\end{proposition}

\subsubsection{problem with sheaves}

\begin{proposition}
	If $\phi: \ms{F} \to \ms{G}$ is a morphism of sheaves, then the presheaf kernel $\ker_{pre} \phi$ is a sheaf and the kernel in the category of sheaves.
\end{proposition}

\begin{remark}[presheaf cokernel is \note{not cokernel} in the category of sheaves]
	Let $\Omega \subseteq \C$ be an open set, a holomorphic function $f: \Omega \to \C$ is called admitting holomorphic logarithm if there exists a holomorphic function $g: \Omega \to \C$ so that $\exp g = f$.
	
	Let $X = \C$ with the usual topology, let $\ms{O}_X$ be the sheaf of holomorphism functions and $\ms{F}$ be the presheaf of functions admitting a holomorphic logarithm, then there is an exact sequence of presheaves on $X$
	$$
		0 \to \underline{\Z} \to \ms{O}_X \to \ms{F} \to 0
	$$
	
	where $\underline{\Z}$ is the constant sheaf (to $\Z$) on $X$, for each open set $U \subseteq X$, $\underline{Z}(U) = \Z$. $\underline{\Z} \to \ms{O}_X$ is the natural inclusion, and $\ms{O}_X \to \ms{F}$ is given by $f \mapsto \exp(2 \pi i f)$.
	
	Then, $\ms{F}$ is not a sheaf. In other words, there is a holomorphic function that locally has holomorphic logarithm but does not have global logarithm
\end{remark}

\begin{proof}
	Let $A = \set{z \in \C: 10 < |z| < 11} \cap \set{(x, y) \in \C: -1 < y}$ and $B = \set{z \in \C: 10 < |z| < 11} \cap \set{(x, y) \in \C: y < +1}$, then the identity function $z \mapsto z$ has logarithm on $A$ and $B$ but not on $A \cup B$
\end{proof}

\subsection{PROPERTIES DETERMINED AT THE LEVEL OF STALKS AND SHEAFIFICATION}

\subsubsection{properties determined at the level of stalks}

\begin{proposition}[section is determined by germs]
	Given a sheaf of sets $\ms{F}$ on $X$ and any open set $U \subseteq X$, there exists an injective natural map
	$$
		\ms{F}(U) \hookrightarrow \prod_{p \in U} \ms{F}_p
	$$
\end{proposition}

\begin{proof}
	Let $U \subseteq X$ be an open set. For every $p \in U$, since $\ms{F}_p$ is the colimit of the diagram containing $\ms{F}(V)$ for all $V$ containing $p$, hence there is a map $\ms{F}(U) \to \ms{F}_p$. Therefore, there is a map $\ms{F}(U) \to \prod_{p \in U} \ms{F}_p$. We will show that this map is injective. Suppose two distinct functions $f, g \in \ms{F}(U)$, $f \neq g$ in $\ms{F}(U)$ but $f = g$ in $\prod_{p \in U} \ms{F}_p$. Then, $f = g$ in $\ms{F}_p$ for every $p \in U$. For each $p \in U$, since $f = g$ in $\ms{F}_p$, there exists an open set $V_p \subseteq U$ containing $p$ so that $f = g$ in $\ms{F}(V_p)$. By identity axiom, $\bigcup_{p \in U} V_p = U$ and $f = g$ in $\ms{F}(V_p)$ for every $p$, then $f = g$ in $\ms{F}(U)$. Contradiction.
\end{proof}

\begin{remark}
	this proposition gives the feeling that sheaf is just a "sheaf" of "stalks" stick together
\end{remark}

\begin{definition}[compatible germs]
	An element $(s_p)_{p \in U} \in \prod_{p \in U} \ms{F}_p$ is called compatible germs if for each $p \in U$, there exists an open set $U_p \subseteq U$ containing $p$ and a representative $\tilde{s}_p \in \ms{F}(U_p)$, that is $\tilde{s}_p = s_p$ in $\ms{F}_p$
\end{definition}

\begin{proposition}
	The set of compatible germs on $\prod_{p \in U} \ms{F}_p$ is precisely the image of $\ms{F}(U) \hookrightarrow \prod_{p \in U} \ms{F}_p$
\end{proposition}

\begin{proof}
	content...
\end{proof}