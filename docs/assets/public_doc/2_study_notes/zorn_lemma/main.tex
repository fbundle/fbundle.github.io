\documentclass{article}
\usepackage{graphicx} % Required for inserting images

% header

%% natbib
\usepackage{natbib}
\bibliographystyle{plain}

%% comment
\usepackage{comment}

% indent the first paragraph
\usepackage{indentfirst}


%% math package
\usepackage{amsfonts}
\usepackage{amsmath}
\usepackage{amssymb}


%% operator
\DeclareMathOperator{\tr}{tr}
\DeclareMathOperator{\diag}{diag}
\DeclareMathOperator{\sign}{sign}
\DeclareMathOperator{\grad}{grad}
\DeclareMathOperator{\curl}{curl}
\DeclareMathOperator{\Div}{div}

%% theorems
\newtheorem{axiom}{Axiom}
\newtheorem{definition}{Definition}
\newtheorem{theorem}{Theorem}
\newtheorem{proposition}{Proposition}
\newtheorem{corollary}{Corollary}
\newtheorem{lemma}{Lemma}
\newtheorem{remark}{Remark}
\newtheorem{claim}{Claim}
\newtheorem{problem}{Problem}

%% empty set
\let\oldemptyset\emptyset
\let\emptyset\varnothing

% mathcal symbols
\newcommand\Tau{\mathcal{T}}
\newcommand\Ball{\mathcal{B}}

% mathbb symbols
\newcommand\N{\mathbb{N}}
\newcommand\Z{\mathbb{Z}}
\newcommand\Q{\mathbb{Q}}
\newcommand\R{\mathbb{R}}

\title{Zorn's Lemma}
\author{Nguyen Ngoc Khanh}
\date{October 2023}

\begin{document}

\maketitle



\includegraphics{spanning_tree.png}


\begin{definition}[Partially Order Set]
    A partial order on a set $X$ is a binary relation, denoted by $\leq$ such that
    \begin{enumerate}
        \item reflexivity: $x \leq x$ for all $x \in X$
        \item anti-symmetry: $x = y$ if $x \leq y$ and $y \leq x$ for all $x, y \in X$
        \item transitivity: $x \leq z$ if $x \leq y$ and $y \leq z$ for all $x, y, z \in X$
    \end{enumerate}
    A set $X$ equipped with a partial order $\leq$ is said to be a partially ordered set, denoted by $(X, \leq)$
\end{definition}

\begin{definition}[Linearly Ordered Set]
    A partially ordered set $(X, \leq)$ is said to be a linearly ordered set if every two elements are comparable, that is, $x \leq y$ or $y \leq x$ for all $x, y \in X$. The order $\leq$ is said to be a linear order.
\end{definition}

\begin{definition}[Upper Bound]
    Given a partially ordered set $(X, \leq)$, an element $t \in X$ is said to be an upper bound of a subset $Y \subseteq X$ if $y \leq t$ for all $y \in Y$
\end{definition}

\begin{definition}[Maximal]
    Given a partially ordered set $(X, \leq)$, an element $w \in X$ is said to be maximal if $x \in X$ and $w \leq x$, then $x = w$
\end{definition}





\begin{axiom}[Zorn's Lemma]
    Let $(X, \leq)$ be a non-empty partially ordered set in which every linearly ordered subset has an upper bound. Then, $(X, \leq)$ has a maximal element.
\end{axiom}


\end{document}
