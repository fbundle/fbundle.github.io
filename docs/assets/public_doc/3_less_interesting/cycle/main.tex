\documentclass{article}
\usepackage[utf8]{inputenc}

\title{cycle}
\author{nguyenngockhanh.pbc }
\date{April 2021}

% cases
\usepackage{amsmath}
% symbol
\usepackage{amssymb}



\newtheorem{definition}{Definition}
\newtheorem{theorem}{Theorem}

\begin{document}

    \maketitle

    \textbf{Is there an natural number function that always has a cycle and the cycle length is unbounded?}


    \section{Problem}\label{sec:problem}

    Firstly, let define precisely the notations we are using in this text.

    \begin{definition}(Natural Number Set)
        \begin{equation}
            \label{eq:natural}
            \mathbb{N} = \{ 1, 2, 3, \dots \}
        \end{equation}
    \end{definition}

    \begin{definition}(Natural Number Function)
        All functions that have the domain and codomain $\mathbb{N}$
        \begin{equation}
            \label{eq:function}
            f: \mathbb{N} \to \mathbb{N}
        \end{equation}
    \end{definition}

    Denote $f^m(n)$ where $n \in \mathbb{N}$ be the value of $m$-times recursive call of function $f$ to input $n$.
    More formally,

    \begin{gather*}
        f^1(n) = f(n),\\
        f^{m+1}(n) = f(f^m(n))\\
    \end{gather*}


    \begin{definition}(Cycle number of a Natural Number Function)
        Let $f: \mathbb{N} \to \mathbb{N}$.
        A cycle set of $n$ is the set of all numbers $m \in \mathbb{N}$ such that $f^m(n) = n$

        \begin{equation}
            \label{eq:cycle_set}
            c^{(f)}(n) = \{m : m \in \mathbb{N} \land f^m(n) = n \}
        \end{equation}
        if cycle set is non-empty, define the cycle number as the smallest number in the cycle set
        \begin{equation}
            \label{eq:cycle_number}
            c^{(f)}_{\min}(n) = \min{c^{(f)}(n)}
        \end{equation}
    \end{definition}

    In this text, we ignore the trivial case where cycle number is 1.

    Our main theorem is stated as follows

    \begin{theorem}(Cycle)
        There exists a natural number function such that (1) it has a non-trivial cycle number for every input and (2) the set of all cycle numbers is unbounded.

        \begin{equation}
            \label{eq:cycle_theorem}
            \exists f : \mathbb{N} \to \mathbb{N}, (
            \forall n \in \mathbb{N},
            c^{(f)}_{\min}(n) > 1
            ) \land (
            \forall m_0 \in \mathbb{N},
            \exists n \in \mathbb{N},
            c^{(f)}_{\min}(n) \geq m_0
            )
        \end{equation}
    \end{theorem}

    In other words, this function partitions the natural number set into infinitely number of finite subsets where the cardinality of them are unbounded.
    Each cycle is associated with a subset.


    \section{Proof}\label{sec:proof}

    A simple construction satisfies those properties is as follows:

    Suppose we found partition on $\mathbb{N}$ of ($1^*$) infinitely many finite subsets where ($1^{**}$) each of them has the cardinality of at least 2, ($2^*$) the cardinality of these subsets is unbounded.

    \begin{equation}
        \label{eq:partition}
        \mathbb{P} = \{ P_i \}_{i=1}^\infty = \{ P_1, P_2, P_3, \dots \}
    \end{equation}

    Where we order all elements in each $P_i$, so that for every $P_i$, we have a minimum element, a maximum element and a function to yield the successor element if the input is not the maximum element namely

    \begin{equation}
        \label{eq:succ_func}
        succ_i : P_i \backslash \{ \max{P_i} \} \to P_i \backslash \{ \min{P_i} \}
    \end{equation}

    Define a function $f_i: P_i \to P_i$ that returns minimum element of $P_i$ if the input is the maximum element of $P_i$, otherwise return its successor.

    \begin{equation}
        \label{eq:func_sub}
        f_i(n)=
        \begin{cases}
            \min P_i &\text{if $n = \max P_i$}.\\
            succ_i(n) &\text{otherwise}.
        \end{cases}
    \end{equation}

    This function has a cycle number of the cardinality of $P_i$
    Since $\mathbb{P}$ is a partition, these $P_i$ are disjoint and their union is $\mathbb{N}$.
    We define the function $f: \mathbb{N} \to \mathbb{N}$

    \begin{equation}
        \label{eq:func}
        f(n)= f_i(n) \text{ if } n \in P_i
    \end{equation}

    ($1^* \land 1^{**} \to 1$) For every input $n$, the cycle number is $|P_i| \geq 2$ where $P_i$ is the associated subset. ($2^* \to 2$) Since these subsets are unbounded in size, the set of all cycle numbers of $f$ is also unbounded.

    In order to finish the proof, we will show a partition on $\mathbb{N}$ that satisfies ($1^*$), ($1^{**}$) and ($2^*$).

    Let $S_i$ be the set of all natural numbers in the range $[2^i, 2^{i+1})$ for $i=0, 1, 2, \dots$.
    e.g. $S_0 = \{ 1 \}$, $S_1 = \{ 2, 3 \}$, $S_2 = \{ 4 , 5, 6, 7 \}$, $S_3 = \{ 8, 9, \dots, 15 \}$, etc.

    Our partition is in the form

    \begin{equation}
        \label{eq:partition_concrete}
        \mathbb{P} = \{\{ S_0 \cup S_1 \}\} \cup \{ S_i \}_{i=2}^\infty
    \end{equation}


\end{document}
