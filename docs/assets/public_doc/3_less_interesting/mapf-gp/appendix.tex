\chapter{Appendix}

\section{Informal proof of theorem \ref{theo:theorem1}}
\subsection{problem \ref{prob:problem2} $\to$ problem \ref{prob:problem1}*}
$V_P \to V$: nodes of problem \ref{prob:problem2} become nodes of problem \ref{prob:problem1}*\\
$V_P \to V_P$: nodes of problem \ref{prob:problem2} become POIs of problem \ref{prob:problem1}*\\
$E_P \to E$: edges of problem \ref{prob:problem2} become edges of problem \ref{prob:problem1}*\\
$L \to L$: remains $L$\\

If $S_1$ is the solution problem \ref{prob:problem1}*, $S_1$ is also a valid assignment for problem \ref{prob:problem2} ($|S_1| \geq |S_2|$).

The claim is correct due to this procedure: let any $B$ appears more than once in $S_1$, $A \to B \to C$ becomes $A \to C$.

If $S_2$ is the solution of problem \ref{prob:problem2}, $S_2$ is also a valid assignment for problem \ref{prob:problem1}* ($|S_2| \geq |S_1|$).

Hence, problem \ref{prob:problem1}* is at least as hard as problem \ref{prob:problem2}

\subsection{problem \ref{prob:problem1}* $\to$ problem \ref{prob:problem2}}

$V_P \to V_P$: POIs of problem \ref{prob:problem1}* become nodes of problem \ref{prob:problem2}\\
*: shortest paths of problem \ref{prob:problem1}* become edges of problem \ref{prob:problem2}\\
$L \to L$: remains $L$\\

If $S_2$ is the solution of problem \ref{prob:problem2}, $S_2$ is also a valid assignment for problem \ref{prob:problem1}* ($|S_2| \geq |S_1|$)

If $S_1$ is the solution of problem \ref{prob:problem1}*, $S_1$is also a valid assignment for problem \ref{prob:problem2}

The claim is correct by:

If there is no node appears more than once, trivial.

Otherwise, let any $B$ appears more than once in $S_1$, $|A \to B \to C| \geq |A \to C|$. If the inequality occurs, replacing $|A \to B \to C|$ by $|A \to C|$ yields a shorter solution that conflicts with $S_1$ is the solution of problem \ref{prob:problem1}*. The equality occurs, we replace $|A \to B \to C|$ by $|A \to C|$ until there is no more node appear more than once.

Hence, problem \ref{prob:problem2} is at least as hard as problem \ref{prob:problem1}*

\section{Informal proof of theorem \ref{theo:approximation}}


Consider the program of minimizing a function $f: X \to \mathbb{R}$. In many scenarios, it is hard to find an optimal or it is even hard to compute the value of $f(x)$. Isaac Vandermeulen, Roderich Groß, Andreas Kolling \cite{vandermeulen2019balanced} has introduced a method that find a function $f_1: X \to \mathbb{R}$ such that $(1) f(x) = c(f_1(x)) + v$ where $c$ is a monotonically increasing function and $v$ is a random variable. 

Let some bounds on $v$ as follows: $\alpha$ $\in$ (0, 0.5) and $b_\alpha^-$, $b_\alpha^+$ $\in$ $\mathbb{R}_+$

\[
\mathbb{P}[-b_\alpha^- \leq v] = \mathbb{P}[v \leq +b_\alpha^+] = 1 - \alpha
\]

Let $x^*$ and $x^*_1$ be the optimal values for $f$ and $f_1$.

Consider 3 random events:
\[
(A): f(x^*_1) \leq f(x^*) + b_\alpha^- + b_\alpha^+
\]
\[
(B): f(x^*_1) \leq c(f_1(x^*_1)) + b_\alpha^+
\]
\[
(C): f(x^*) \geq c(f_1(x^*)) - b_\alpha^-
\]

\begin{theorem}[Approximation]:

$\mathbb{P}[A] \geq (1 - \alpha)^2$

\label{theo:approximation}
\end{theorem}


We have $f(x^*) \leq f(x^*_1)$ and $f_1(x^*_1) \leq f_1(x^*)$. $c$ is a monotonically increasing function, so $c(f_1(x^*_1)) \leq c(f_1(x^*))$. 

If $(B)$ holds, 
\[
f(x^*_1) \leq c(f_1(x^*_1)) + b_\alpha^+ \leq c(f_1(x^*)) + b_\alpha^+
\]

If $(C)$ also holds,
\[
f(x^*_1) \leq c(f_1(x^*)) + b_\alpha^+ = (c(f_1(x^*)) - b_\alpha^-) + b_\alpha^- + b_\alpha^+
\]

Hence, 
\[
(A): f(x^*_1) \leq f(x^*) + b_\alpha^- + b_\alpha^+
\]


$(A)$ implies the upper-bound on how good the solution of $f_1$. By assuming $v$, $x^*$ and $x^*_1$ be independent, $\mathbb{P}[B \cap C] = \mathbb{P}[B] \times \mathbb{P}[C]$. Since $B \cap C \to A$, $\mathbb{P}[A] \geq \mathbb{P}[B] \times \mathbb{P}[C]$. From (1), $\mathbb{P}[B] = \mathbb{P}[C] = 1 - \alpha$. So, $\mathbb{P}[A] \geq (1 - \alpha)^2$

In conclusion, if one can the proxy $f_1$, a good solution for $f$ can be obtained with high probability.

The error on this approximation depends on how good we can find a function $f$ and a smoother $c$.

\section{Analysis on constant $\beta$}

Let $\alpha = 0$, $\beta > 0$, consider a graph that all edges have unit length. Objective function is

\[
O_7 = \sum_{k=1}^{K} \frac{x^{(k)T} (A - \alpha D) x^{(k)}}{x^{(k)T} x^{(k)} + \beta |V|}
=
\sum_{k=1}^{K} \frac{\sum_{i \in V_k} \sum_{j \in V_k} A_{ij}}{|V_k| + \beta |V|}
= 
\sum_{k=1}^{K} \frac{|V_k|(|V_k| - 1)}{|V_k| + \beta |V|}
\]

Subject to the constraint

\[
\sum_{k=1}^{K} |V_k| = |V| \text{ and } |V_k| > 0 \text{ } \forall k
\]

Let $x_k = \frac{|V_k|}{|V|}$ be a real variable, we have the problem of minimizing

\[
f(x) = \sum_{k=1}^{K} \frac{x_k (x_k-1)}{x_k+\beta |V|}
\]

Subject to 

\[
c_0(x) = \sum_{k=1}^{K} x - 1 = 0 \text{ and } c_k(x) = -x_k \leq 0 \text{ } \forall k
\]

We have:

\[
\frac{\partial f}{\partial x_k} = |V|(1 - \frac{\beta (\beta + 1/|V|)}{(x_k + \beta)^2})
\]


$\frac{\partial f}{\partial x_k}$ has this property if $\beta > 0$:

\[
\text{(1): } \text{$\frac{\partial f}{\partial x_k}$ is a monotonically increasing function for all $x_k \geq 0$}
\]

Theorem \ref{theo:unique} deduces the unique minimum of $O_7$ at $|V_k| = |V|/K$

\section{Theorem \ref{theo:unique}}

Given the program:

\[
\textbf{Minimize: } f(x)
\textbf{ subject to: } c_0(x) = \sum_{i=1}^{n} x_i = 1 \text{ and } c_i(x) = -x_i \leq 0 \text{ } \forall i
\]


Such that $\frac{\partial f}{\partial x_i}$ has this property:

\[
\text{(1): } \text{$\frac{\partial f}{\partial x_i}$ is a monotonically increasing function for all $x_i \geq 0$}
\]

\[
\frac{\partial f}{\partial x_i}(x_1) < \frac{\partial f}{\partial x_i}(x_2) \text{ } \forall \text{ } 0 \leq x_1 < x_2
\]

\begin{theorem}[Unique solution]
\label{theo:unique}
Program has unique solution at $x_i = \frac{1}{n}$ $\forall i$
\end{theorem}

Lagrangian function is

\[
L(x, \mu, \lambda) = f(x) + \sum_{i=1}^{n} \mu_i c_i(x) + \lambda c_0(x)
\]

If $x*$ is a minimum, KKT conditions:

\[
\textbf{Stationary: } \frac{\partial L}{\partial x}(x*) = 0_n
\]
\[
\textbf{Primal feasibility: } c_0(x*) = 0 \text{ and } c_i(x*) \leq 0 \text{ } \forall i
\]
\[
\textbf{Dual feasibility: } \mu_i \geq 0 \text{ } \forall i
\]
\[
\textbf{Complementary slackness: } \mu_i c_i(x*) = 0 \text{ } \forall i
\]

We have:

\[
\frac{\partial L}{\partial x_k} = \frac{\partial f}{\partial x_k} - \mu_k + \lambda
\]

Consider 2 cases:

\textbf{Case 1:} all $x_i > 0$, due to complementary slackness, all $\mu_i = 0$. So that, all $\frac{\partial f}{\partial x_i}$ must be equal. (1) deduces that the unique solution satisfying KKT conditions is $x_i = \frac{1}{n}$

\textbf{Case 2:} some $x_i = 0$, let $x_{i1} = 0$

\[
\frac{\partial f}{\partial x_i}(x_{i1}) - \mu_{i1} + \lambda = 0
\]

\[
\lambda = \mu_{i1} - \frac{\partial f}{\partial x_i}(0)
\]

Since dual feasibility, $\mu_{i1} \geq 0$, So

\[
\lambda \geq - \frac{\partial f}{\partial x_i}(0)
\]

There is at least one $x_i > 0$, let $x_{i2} > 0$, due to complementary slackness, $\mu_{i2} = 0$, So

\[
\frac{\partial f}{\partial x_i}(x_{i2}) + \lambda = 0
\]

\[
\frac{\partial f}{\partial x_i}(x_{i2}) = - \lambda \leq \frac{\partial f}{\partial x_i}(0)
\]

(1) deduces contradiction.