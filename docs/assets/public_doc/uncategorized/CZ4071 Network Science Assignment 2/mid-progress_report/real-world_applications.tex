\chapter{Real-World Applications}
Due to its flexibility, GCNs can be applied in various domains in the real-world. Examples of such domains are in the application of GCNs in Computer Vision (CV), Natural Language Processing (NLP) and other sciences.

CV has been a hot research area in the past decades. Although classic convolutional neural networks (CNN) have achieved great successes in CV, it is not feasible to encode the intrinsic graph structures in the specific learning tasks. On the other hand, GCNs have been applied and achieved a comparable or even better performance in some CV problems. For example, \cite{cui2018context} proposes a graph CNN to leverage both the semantic graphs of words and spatial scene graph for visual relationship detection. Furthermore, in \cite{chen2017photographic}, it is shown that a GCN model can be used to process the input scene graph and generate the images by a cascaded refinement network for photographic image synthesis. Moreover, GCNs can also be applied in videos. \cite{yan2018spatial} applied GCN to perform action recognition by proposing a spatial-temporal graph convolutional model to eliminate the need of hand-crafted part assignment which achieved a greater expressive power.

GCNs also have applications in NLP. For text classification, citation network can be constructed with the documents as nodes and the citation relationships among them as edges. And node classification can be a straightforward way to classify documents into different categories. In \cite{yao2019graph}, TextGCN performs text classification by modeling a whole corpus to a heterogeneous graph and learn word embedding and document embedding simultaneously, followed by a softmax classifier for text classification. In addition, a syntactic GCN model is developed and it can be used on top of syntactic dependence trees, which is suitable for semantic role labelling \cite{marcheggiani2017encoding} and neural machine translation \cite{bastings2017graph}.

GCNs are also applied in other domains outside of Computer Science.

In chemistry, \cite{zitnik2018modeling} first models drug-protein target interactions and protein-protein interactions into a multimodal graph, and then graph convolutions is applied to predict polypharmacy side effects. 

In material science, \cite{xie2018crystal} proposes a crystal GCN to directly learn material properties from the connections of atoms in the crystal. 

In social sciences, GCNs have been widely used for social recommendation to improve the recommendation performances based on user-item interactions and/or user-user interactions. \cite{wu2019neural} proposes a neural influence diffusion model for better social recommendation by considering the influence of trusted friends.
