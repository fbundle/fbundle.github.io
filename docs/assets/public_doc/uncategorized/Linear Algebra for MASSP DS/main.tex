\documentclass{article}

% Language setting
% Replace `english' with e.g. `spanish' to change the document language
\usepackage[vietnamese]{babel}

% Set page size and margins
% Replace `letterpaper' with `a4paper' for UK/EU standard size
\usepackage[letterpaper,top=2cm,bottom=2cm,left=3cm,right=3cm,marginparwidth=1.75cm]{geometry}

% Useful packages
\usepackage{amsmath}
\usepackage{amssymb}    % for square
\usepackage{enumitem}   % for alphabetically enumeration
\usepackage{graphicx}
\usepackage[colorlinks=true, allcolors=blue]{hyperref}







\title{Linear Algebra questions for MASSP Application form}
\author{Hồ Lê Minh Quân}

\begin{document}
\maketitle

%\begin{abstract}
%Your abstract.
%\end{abstract}



\section{Các câu hỏi}

\subsection{(a) Ma trận đối xứng là gì? (b) Nêu 2 tính chất của ma trận đối xứng.}
\begin{enumerate}[label=(\alph*)]
\item
Ma trận $A$ được gọi là đối xứng khi và chỉ khi: $A = A^T$
\item
    \begin{enumerate}
    \item
    Nếu ma trận $A$ khả nghịch thì $A^{-1}$ đối xứng $\Leftrightarrow$ $A$ đối xứng. \\
    \item
    Với $\forall n \in \N: A$ khả nghịch $\Rightarrow A^n $ khả nghịch. \\
    \item
    Với $A$ là ma trận bất kì, ma trận $X = AA^T$ là một ma trận đối xứng.
    \end{enumerate}
\end{enumerate}



\subsection{Cho X là một ma trận. (a) Tại sao tất cả trị riêng của $X^TX$ đều không âm? (b) Khi nào $X^TX$ nghịch đảo được?}
\begin{enumerate}[label=(\alph*)]
\item Ta có: $|X| = |X^T|$, điều này đúng với mọi ma trận $X$.\\
$\lambda$ là một trị riêng của $X^TX$ $\Leftrightarrow$ Tồn tại ma trận $A$ sao cho: $X^TXA = \lambda A$.\\
Điều này tương đương:\\
\begin{align*}
    & |X^TXA| = \lambda |A|\\
    \Rightarrow & |X^T||X||A| = \lambda |A| \\
    \Rightarrow &  |X^T||X|= \lambda\\
    \Rightarrow &  |X|^2= \lambda \geq 0
\end{align*}
Vậy: Tất cả trị riêng của ma trận $X^TX$ đều không âm. 

\item 
\begin{align*}
    X^TX \text{khả nghịch} & \Leftrightarrow |X^TX| \neq 0\\
    & \Leftrightarrow |X|^2 \neq 0\\
    & \Leftrightarrow |X| \neq 0\\
    & \Leftrightarrow \text{X khả nghịch.}
\end{align*}
   

\end{enumerate}



\subsection{Cho ma trận $X$, và vector $Y$ sao cho tích $XY$ có nghĩa. (a) Tại sao $XY$ nằm trong không gian vector sinh ra bởi các cột của $X$? Cho một ma trận $Y$ sao cho tích $XY$ có nghĩa. (b) Tại sao $rank(X) \geq rank(XY)$?}
\begin{enumerate}[label=(\alph*)]
\item 
\[
    X = \begin{bmatrix} 
    x_{11} & \dots  & x_{1n}\\
    \vdots & \ddots & \vdots\\
    x_{m1} & \dots  & x_{mn} 
    \end{bmatrix}
    \qquad
    Y = \begin{bmatrix} 
    \lambda_{1}\\
    \vdots\\
    \lambda_{m}
    \end{bmatrix}
\]
\begin{center}
    $XY$ & = \begin{bmatrix} 
    \lambda_{1}x_{11} + \lambda{2}x_{12} + \dots + \lambda{n}x_{1n}\\
    \dots \\
    \lambda_{1}x_{m1} + \lambda{2}x_{m2} + \dots + \lambda{n}x_{mn}
    \end{bmatrix} \\
    & = $\lambda_{1}$ \begin{bmatrix} $\x_{11}$\\ \dots \\ $\x_{m1}$   \end{bmatrix}
    + $\lambda_{2}$ \begin{bmatrix} $\x_{12}$\\ \dots \\ $\x_{m2}$    \end{bmatrix}
    + $\dots$
    + $\lambda_{n}$ \begin{bmatrix} $\x_{1n}$\\ \dots \\ $\x_{mn}$    \end{bmatrix}
\end{center} 
Ta thấy $XY$ chính là 1 tổ hợp tuyến tính của $n$ vector là các vector cột của $X$, ứng với bộ $n$ hệ số $Y$, nói cách khác $XY$ là vector sinh bởi các vector cột của $X$, hay $XY$ nằm trong không gian vector sinh bởi các cột của $X$.

\item 
Giả sử $X$ có kích thước ($m$,$n$), $Y$ có kích thước ($n$,$p$) thì $XY$ có kích thước ($m$,$p$).\\
Gọi $S_X$ là không gian sinh bởi các cột của $X$, $S_{XY}$ là không gian sinh bởi các cột của $XY$.\\
Điều phải chứng minh tương đương: 
\begin{center}
    dim(S_X) \geq dim(S_{XY}).\\
\end{center}
Gọi $v$ là vector bất kỳ thuộc $S_{XY}$.\\
Có: 
\begin{center}
    v \in S_{XY} \Rightarrow \exists \lambda = [\lambda_1\space\lambda_2\dots\lambda_p]: v = (XY)\lambda \Leftrightarrow v = X(Y\lambda) \Leftrightarrow v \in S_X. \\
\end{center}
 Điều này đã chứng minh được:
 \begin{align*}
     & S_{XY} \subset S_X \\
     \Leftrightarrow \text{   } & dim(S_{XY}) \leq dim(S_X) \text{(đpcm).}
 \end{align*} 
   

\end{enumerate}


\subsection{(a) Nêu định nghĩa của hàm lồi và hàm lõm (đơn biến). (b) Nêu một số tính chất đặc trưng của hàm lồi.}
\begin{enumerate}[label=(\alph*)]}
\item
Xét tập $I = I(a,b)$ và hàm số $y = f(x)$ trên I.\\
Hàm f được gọi là lồi trên I khi và chỉ khi:
\begin{center}
    \forall x,y \in I, \forall \alpha \in [0;1]: f((1-\alpha)x +\alpha y) \leq (1-\alpha)f(x) + \alpha f(y) \\
\end{center}
Hàm f được gọi là lõm trên I khi và chỉ khi:
\begin{center}
    \forall x,y \in I, \forall \alpha \in [0;1]: f((1-\alpha)x +\alpha y) \geq (1-\alpha)f(x) + \alpha f(y) \\
\end{center}

\item
    \begin{enumerate}
        \item 
        \text{Nếu $f$ khả vi trên $I$ thì $f'$ là hàm đơn điệu tăng trên $I$}.\\
        \item
        \text{Nếu $f$ và $f'$ khả vi trên $I$ thì $f''(x) \geq 0, \forall x \in I$}. \\ 
        \item
        \text{$\forall x,y \in I, f(x+y) \geq f(x)+f(y)$}. \\
        \item
        \text{Hàm chỉ có 1 cực tiểu địa phương, đây cũng là giá trị nhỏ nhất của hàm.}
        
    \end{enumerate}

\end{enumerate}
\end{document}