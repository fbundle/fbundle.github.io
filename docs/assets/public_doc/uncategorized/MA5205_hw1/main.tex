%! TEX program = pdflatex
\documentclass{article}
\usepackage{graphicx}
\usepackage{unicode-math-input}

\InputIfFileExists{local.tex}{}{}

\usepackage{amsmath}
\usepackage{amssymb}

%% math package
\usepackage{amsfonts}
\usepackage{amsmath}
\usepackage{amsthm}
\usepackage{amssymb}
\usepackage{tikz-cd}
\usepackage{mathtools}
\usepackage{comment}
\usepackage{mathrsfs}



%% operator
\DeclareMathOperator{\tr}{tr}
\DeclareMathOperator{\diag}{diag}
\DeclareMathOperator{\sign}{sign}
\DeclareMathOperator{\grad}{grad}
\DeclareMathOperator{\curl}{curl}
\DeclareMathOperator{\Div}{div}
\DeclareMathOperator{\diam}{diam}

%% theorems
\newtheorem{axiom}{Axiom}
\newtheorem{definition}{Definition}
\newtheorem{theorem}{Theorem}
\newtheorem{proposition}{Proposition}
\newtheorem{corollary}{Corollary}
\newtheorem{lemma}{Lemma}
\newtheorem{remark}{Remark}
\newtheorem{claim}{Claim}
\newtheorem{problem}{Problem}

%% empty set
\let\oldemptyset\emptyset
\let\emptyset\varnothing

% mathcal symbols
\newcommand\Tau{\mathcal{T}}
\newcommand\Ball{\mathcal{B}}
\newcommand\bigO{\mathcal{O}}

% mathbb symbols
\newcommand\N{\mathbb{N}}
\newcommand\Z{\mathbb{Z}}
\newcommand\Q{\mathbb{Q}}
\newcommand\R{\mathbb{R}}

% mathrsfs symbols
\newcommand\Borel{\mathscr{B}}


\title{MA5205 Homework 1}
\author{Bui Hong Duc (A0236460J) \and Nguyen Ngoc Khanh (A0275047B) \and Nigel Lim (A0154379U)}
\date{August 2023}

\allowdisplaybreaks

\usepackage{hyperref}

\begin{document}


\ifdefined\LOCAL
\typstmathinputenable{\$}
\typstmathinputprepare[\LOCALoutputfile]{\$}
\else
\catcode`\$\active\xdef${\csname _typstmathinput_handle_formula\endcsname}
\ExplSyntaxOn
\providecommand\typstmathinputtext{\text}
\providecommand\typstmathinputenable[1]{
	\catcode `#1 \active
}
\providecommand\typstmathinputdisable[1] {
	\catcode `#1 = 3 \relax
}
\protected\gdef \_typstmathinput_handle_formula {
	\_typstmathinput_handle_formula_a \empty
}
\protected\gdef \_typstmathinput_set_handle_formula_delimiter #1 {
	\protected\long\gdef \_typstmathinput_handle_formula_a ##1 #1 {
		\begingroup\expandafter\endgroup\csname _typstmathinput_prepared)\detokenize\expandafter{##1}\endcsname
	}
}
\ExplSyntaxOff
\csname _typstmathinput_set_handle_formula_delimiter\endcsname{$}\begingroup\long\def\\#1{\expandafter\gdef\csname _typstmathinput_prepared)\detokenize{#1}\endcsname}
\\{
(T f)(s)
&= ∫_0^∞ f(ˆt s) s⁻¹ K(1, ˆt) ⋅ s dif ˆt \
&= ∫_0^∞ f(ˆt s) K(1, ˆt) dif ˆt \
⟹ ‖T f‖_p
&= (∫_0^∞ (∫_0^∞ f(t s) K(1, t) dif t)^p dif s)^(1\/p) \
&≤ ∫_0^∞ (∫_0^∞ f(t s)^p dif s)^(1\/p) K(1, t) dif t.
}{\begin{align*}(Tf)(s)&=∫^{∞}_{0}f(ˆts)s^{−1}K(1,ˆt)⋅s\hspace{0.16666666666666666em}\mathrm{d}ˆt\\&=∫^{∞}_{0}f(ˆts)K(1,ˆt)\hspace{0.16666666666666666em}\mathrm{d}ˆt\\⟹‖Tf‖_{p}&=\Bigl(∫^{∞}_{0}\Bigl(∫^{∞}_{0}f(ts)K(1,t)\hspace{0.16666666666666666em}\mathrm{d}t\Bigr)^{p}\hspace{0.16666666666666666em}\mathrm{d}s\Bigr)^{1/p}\\&≤∫^{∞}_{0}\Bigl(∫^{∞}_{0}f(ts)^{p}\hspace{0.16666666666666666em}\mathrm{d}s\Bigr)^{1/p}K(1,t)\hspace{0.16666666666666666em}\mathrm{d}t.\end{align*}}
\\{
(T f)(s) 
&= ∫_0^∞ f(t) K(s, t) dif t \
&= ∫_0^∞ f(t) s⁻¹ K(1, t/s) dif t.
}{\begin{align*}(Tf)(s)&=∫^{∞}_{0}f(t)K(s,t)\hspace{0.16666666666666666em}\mathrm{d}t\\&=∫^{∞}_{0}f(t)s^{−1}K\biggl(1,\frac{t}{s}\biggr)\hspace{0.16666666666666666em}\mathrm{d}t.\end{align*}}
\\{
(∫_0^∞ f(ˆs)^p t⁻¹ dif ˆs)^(1\/p) = t^(-1\/p) ‖f‖_p}{\((∫^{∞}_{0}f(ˆs)^{p}t^{−1}\hspace{0.16666666666666666em}\mathrm{d}ˆs)^{1/p}=t^{−1/p}‖f‖_{p}\)}
\\{
∫_2^∞ C α⁻¹ (∫_{|f|>α\/2} |f(xx)| dif xx) dif α  
= C ∫_{|f|>1} |f(xx)| (∫_2^(2|f(xx)|) α⁻¹ dif α) dif xx 
}{\[∫^{∞}_{2}Cα^{−1}\Bigl(∫_{\{\vert f\vert >α/2\}}\vert f(\mathbf{x})\vert \hspace{0.16666666666666666em}\mathrm{d}\mathbf{x}\Bigr)\hspace{0.16666666666666666em}\mathrm{d}α=C∫_{\{\vert f\vert >1\}}\vert f(\mathbf{x})\vert \Bigl(∫^{2\vert f(\mathbf{x})\vert }_{2}α^{−1}\hspace{0.16666666666666666em}\mathrm{d}α\Bigr)\hspace{0.16666666666666666em}\mathrm{d}\mathbf{x}\]}
\\{
∫_2^∞ ω(α) dif α 
&≤ ∫_2^∞ C α⁻¹ (∫_{|f|>α\/2} |f(xx)| dif xx) dif α  \
&= C ∫_{|f|>1} |f(xx)| (∫_2^(2|f(xx)|) α⁻¹ dif α) dif xx \
&= C ∫_{|f|>1} |f(xx)| (log (2|f(xx)|) - log(2)) dif xx \
&= C ∫_{|f|>1} |f(xx)| log    |f(xx)| dif xx \
&= C ∫_{|f|>1} |f(xx)| log^+  |f(xx)| dif xx \
&≤ C ∫_(ℝ^n) |f(xx)|    log^+ |f(xx)| dif xx.
}{\begin{align*}∫^{∞}_{2}ω(α)\hspace{0.16666666666666666em}\mathrm{d}α&≤∫^{∞}_{2}Cα^{−1}\Bigl(∫_{\{\vert f\vert >α/2\}}\vert f(\mathbf{x})\vert \hspace{0.16666666666666666em}\mathrm{d}\mathbf{x}\Bigr)\hspace{0.16666666666666666em}\mathrm{d}α\\&=C∫_{\{\vert f\vert >1\}}\vert f(\mathbf{x})\vert \Bigl(∫^{2\vert f(\mathbf{x})\vert }_{2}α^{−1}\hspace{0.16666666666666666em}\mathrm{d}α\Bigr)\hspace{0.16666666666666666em}\mathrm{d}\mathbf{x}\\&=C∫_{\{\vert f\vert >1\}}\vert f(\mathbf{x})\vert (\operatorname{log}(2\vert f(\mathbf{x})\vert )−\operatorname{log}(2))\hspace{0.16666666666666666em}\mathrm{d}\mathbf{x}\\&=C∫_{\{\vert f\vert >1\}}\vert f(\mathbf{x})\vert \operatorname{log}\vert f(\mathbf{x})\vert \hspace{0.16666666666666666em}\mathrm{d}\mathbf{x}\\&=C∫_{\{\vert f\vert >1\}}\vert f(\mathbf{x})\vert \operatorname{log}^{+}\vert f(\mathbf{x})\vert \hspace{0.16666666666666666em}\mathrm{d}\mathbf{x}\\&≤C∫_{ℝ^{n}}\vert f(\mathbf{x})\vert \operatorname{log}^{+}\vert f(\mathbf{x})\vert \hspace{0.16666666666666666em}\mathrm{d}\mathbf{x}.\end{align*}}
\\{
∫_a^b φ dif f
&= ∫_a^b φ dif (g+h) \
&= ∫_a^b φ dif g +  ∫_a^b φ dif h \ 
&= ∫_a^b φ g' dif x +  ∫_a^b φ dif h quad "(integration by parts)"\
&= ∫_a^b φ (f-h)' dif x +  ∫_a^b φ dif h \
&= ∫_a^b φ f' dif x +  ∫_a^b φ dif h. quad "("h'=0" almost everywhere)"
}{\begin{align*}∫^{b}_{a}φ\hspace{0.16666666666666666em}\mathrm{d}f&=∫^{b}_{a}φ\hspace{0.16666666666666666em}\mathrm{d}(g+h)\\&=∫^{b}_{a}φ\hspace{0.16666666666666666em}\mathrm{d}g+∫^{b}_{a}φ\hspace{0.16666666666666666em}\mathrm{d}h\\&=∫^{b}_{a}φg'\hspace{0.16666666666666666em}\mathrm{d}x+∫^{b}_{a}φ\hspace{0.16666666666666666em}\mathrm{d}h\hspace{1em}\typstmathinputtext{(integration by parts)}\\&=∫^{b}_{a}φ(f−h)'\hspace{0.16666666666666666em}\mathrm{d}x+∫^{b}_{a}φ\hspace{0.16666666666666666em}\mathrm{d}h\\&=∫^{b}_{a}φf'\hspace{0.16666666666666666em}\mathrm{d}x+∫^{b}_{a}φ\hspace{0.16666666666666666em}\mathrm{d}h.\hspace{1em}(h'=0\typstmathinputtext{ almost everywhere)}\end{align*}}
\\{ abs(‖f_k‖_p - ‖f‖_p) ≤ ‖f-f_k‖_p }{\[\vert ‖f_{k}‖_{p}−‖f‖_{p}\vert ≤‖f−f_{k}‖_{p}\]}
\\{ ∫_(ℝ^n) ∫_ℝ g(xx, a) dif α dif xx
&= ∫_(ℝ^n) ∫_ℝ χ_{2<α<2 |f(xx)|} ⋅ C α⁻¹ |f(xx)| dif xx dif α 
\ &= ∫_{|f(xx)>1|} ∫_2^(2 |f(xx)|) C α⁻¹ |f(xx)| dif xx dif α. }{\begin{align*}∫_{ℝ^{n}}∫_{ℝ}g(\mathbf{x},a)\hspace{0.16666666666666666em}\mathrm{d}α\hspace{0.16666666666666666em}\mathrm{d}\mathbf{x}&=∫_{ℝ^{n}}∫_{ℝ}χ_{\{2<α<2\vert f(\mathbf{x})\vert \}}⋅Cα^{−1}\vert f(\mathbf{x})\vert \hspace{0.16666666666666666em}\mathrm{d}\mathbf{x}\hspace{0.16666666666666666em}\mathrm{d}α\\&=∫_{\{\vert f(\mathbf{x})>1\vert \}}∫^{2\vert f(\mathbf{x})\vert }_{2}Cα^{−1}\vert f(\mathbf{x})\vert \hspace{0.16666666666666666em}\mathrm{d}\mathbf{x}\hspace{0.16666666666666666em}\mathrm{d}α.\end{align*}}
\\{ ∫_ℝ ∫_(ℝ^n) g(xx, a) dif xx dif α 
&= ∫_ℝ ∫_(ℝ^n) χ_{2<α<2 |f(xx)|} ⋅ C α⁻¹ |f(xx)| dif xx dif α \
&= ∫_2^∞ ∫_{|f|>α\/2}  C α⁻¹ |f(xx)| dif xx dif α, }{\begin{align*}∫_{ℝ}∫_{ℝ^{n}}g(\mathbf{x},a)\hspace{0.16666666666666666em}\mathrm{d}\mathbf{x}\hspace{0.16666666666666666em}\mathrm{d}α&=∫_{ℝ}∫_{ℝ^{n}}χ_{\{2<α<2\vert f(\mathbf{x})\vert \}}⋅Cα^{−1}\vert f(\mathbf{x})\vert \hspace{0.16666666666666666em}\mathrm{d}\mathbf{x}\hspace{0.16666666666666666em}\mathrm{d}α\\&=∫^{∞}_{2}∫_{\{\vert f\vert >α/2\}}Cα^{−1}\vert f(\mathbf{x})\vert \hspace{0.16666666666666666em}\mathrm{d}\mathbf{x}\hspace{0.16666666666666666em}\mathrm{d}α,\end{align*}}
\\{ ∫_ℝ ∫_(ℝ^n) g(xx, a) dif xx dif α = ∫_(ℝ^n) ∫_ℝ g(xx, a) dif α dif xx. }{\[∫_{ℝ}∫_{ℝ^{n}}g(\mathbf{x},a)\hspace{0.16666666666666666em}\mathrm{d}\mathbf{x}\hspace{0.16666666666666666em}\mathrm{d}α=∫_{ℝ^{n}}∫_{ℝ}g(\mathbf{x},a)\hspace{0.16666666666666666em}\mathrm{d}α\hspace{0.16666666666666666em}\mathrm{d}\mathbf{x}.\]}
\\{(dif ˆt)/(dif t) = 1/s}{\(\frac{\hspace{0.16666666666666666em}\mathrm{d}ˆt}{\hspace{0.16666666666666666em}\mathrm{d}t}=\frac{1}{s}\)}
\\{(∫_0^∞ f(t s)^p dif s)^(1\/p)}{\((∫^{∞}_{0}f(ts)^{p}\hspace{0.16666666666666666em}\mathrm{d}s)^{1/p}\)}
\\{0}{\(0\)}
\\{0 ≤ ∫ |(f_k-f) g| ≤ ‖f_k-f‖_p ⋅‖g‖_p'}{\(0≤∫\vert (f_{k}−f)g\vert ≤‖f_{k}−f‖_{p}⋅‖g‖_{p'}\)}
\\{0<p<1}{\(0<p<1\)}
\\{1}{\(1\)}
\\{1/p}{\(\frac{1}{p}\)}
\\{2 f(0)}{\(2f(0)\)}
\\{C}{\(C\)}
\\{C ≥ 2}{\(C≥2\)}
\\{C<2}{\(C<2\)}
\\{C^∞}{\(C^{∞}\)}
\\{C₀^∞(ℝ^n)}{\(C^{∞}_{0}(ℝ^{n})\)}
\\{E}{\(E\)}
\\{F}{\(F\)}
\\{F=P-N}{\(F=P−N\)}
\\{L^1}{\(L^{1}\)}
\\{L^p}{\(L^{p}\)}
\\{N}{\(N\)}
\\{N_Γ₃[φ, F] = ∑ φ(ξ_i) (f(x_i)-f(x_(i-1)))⁻}{\(N_{Γ_{3}}[φ,F]=∑φ(ξ_{i})(f(x_{i})−f(x_{i−1}))^{−}\)}
\\{P}{\(P\)}
\\{P'}{\(P'\)}
\\{P_Γ₃[φ, F] = ∑ φ(ξ_i) (f(x_i)-f(x_(i-1)))⁺}{\(P_{Γ_{3}}[φ,F]=∑φ(ξ_{i})(f(x_{i})−f(x_{i−1}))^{+}\)}
\\{R_Γ₃[φ, F] = P_Γ₃[φ, F] - N_Γ₃[φ, F]}{\(R_{Γ_{3}}[φ,F]=P_{Γ_{3}}[φ,F]−N_{Γ_{3}}[φ,F]\)}
\\{R_Γ₃[φ, F] = ∑ φ(ξ_i) (f(x_i)-f(x_(i-1)))}{\(R_{Γ_{3}}[φ,F]=∑φ(ξ_{i})(f(x_{i})−f(x_{i−1}))\)}
\\{[0, ∞)}{\([0,∞)\)}
\\{[0,+∞)}{\([0,+∞)\)}
\\{[a, b]}{\([a,b]\)}
\\{[l₁, r₁] × ⋯ ×[lₙ, rₙ] ⊆ ℝ^n}{\([l_{1},r_{1}]×⋯×[l_{n},r_{n}]⊆ℝ^{n}\)}
\\{a}{\(a\)}
\\{a<b}{\(a<b\)}
\\{a=r/2, b=r}{\(a=\frac{r}{2},b=r\)}
\\{abs( ‖f_k‖_p^p - ‖f‖_p^p ) ≤ ‖f-f_k‖_p^p}{\(\vert ‖f_{k}‖^{p}_{p}−‖f‖^{p}_{p}\vert ≤‖f−f_{k}‖^{p}_{p}\)}
\\{b}{\(b\)}
\\{b-a>0}{\(b−a>0\)}
\\{boldzero}{\(\mathbf{0}\)}
\\{d ≤ 0}{\(d≤0\)}
\\{d ≥ -1}{\(d≥−1\)}
\\{d ≥ 1}{\(d≥1\)}
\\{f}{\(f\)}
\\{f ∈ L^p}{\(f∈L^{p}\)}
\\{f'(x) x³-f(x) ⋅ 3n x² + 2f(x)}{\(f'(x)x^{3}−f(x)⋅3nx^{2}+2f(x)\)}
\\{f(0) ≠ 0}{\(f(0)≠0\)}
\\{f(x)}{\(f(x)\)}
\\{f(x)=1}{\(f(x)=1\)}
\\{f(xx)}{\(f(\mathbf{x})\)}
\\{f(xx) = f(x₁, …, xₙ) = ∏_(i=1)^n g_(lᵢ, rᵢ)(xᵢ)}{\(f(\mathbf{x})=f(x_{1},…,x_{n})=∏^{n}_{i=1}g_{l_{i},r_{i}}(x_{i})\)}
\\{f(xx)=1}{\(f(\mathbf{x})=1\)}
\\{f(xx)=k(|xx|)}{\(f(\mathbf{x})=k(\vert \mathbf{x}\vert )\)}
\\{f, f_k ∈ L^p}{\(f,f_{k}∈L^{p}\)}
\\{f, f_k, g}{\(f,f_{k},g\)}
\\{f-f_k = χ_((-∞,-k])}{\(f−f_{k}=χ_{(−∞,−k]}\)}
\\{f^*}{\(f^{∗}\)}
\\{f_k = χ_((-k,+∞))}{\(f_{k}=χ_{(−k,+∞)}\)}
\\{f_k → f}{\(f_{k}→f\)}
\\{f≥ 0}{\(f≥0\)}
\\{g}{\(g\)}
\\{g ∈ L^p'}{\(g∈L^{p'}\)}
\\{g ≥ 0}{\(g≥0\)}
\\{g(x)}{\(g(x)\)}
\\{g(x) ≠ 0 ⟺ h(x-a) h(b-x) ≠ 0 ⟺ h(x-a) ≠ 0 ∧ h(b-x) ≠ 0 ⟺ x-a>0 ∧ b-x>0 ⟺ a<x<b}{\(g(x)≠0⟺h(x−a)h(b−x)≠0⟺h(x−a)≠0∧h(b−x)≠0⟺x−a>0∧b−x>0⟺a<x<b\)}
\\{g(x)∈ C^∞}{\(g(x)∈C^{∞}\)}
\\{g(xx, α) = χ_{2<α<2 |f(xx)|} ⋅ C α⁻¹ |f(xx)|}{\(g(\mathbf{x},α)=χ_{\{2<α<2\vert f(\mathbf{x})\vert \}}⋅Cα^{−1}\vert f(\mathbf{x})\vert \)}
\\{g: ℝ^n × ℝ → ℝ}{\(g:ℝ^{n}×ℝ→ℝ\)}
\\{g_(a, b)(x)}{\(g_{a,b}(x)\)}
\\{h}{\(h\)}
\\{h(x) ≠ 0 ⟺ x>0}{\(h(x)≠0⟺x>0\)}
\\{h=f-g}{\(h=f−g\)}
\\{h^((n))(0)=0}{\(h^{(n)}(0)=0\)}
\\{h^((n))(x)}{\(h^{(n)}(x)\)}
\\{h^((n))(x) = (
f'(x) x³-f(x) ⋅ 3n x² + 2f(x)
)/x^(3(n+1)) e^(-x⁻²)}{\(h^{(n)}(x)=\frac{f'(x)x^{3}−f(x)⋅3nx^{2}+2f(x)}{x^{3(n+1)}}e^{−x^{−2}}\)}
\\{h^((n))(x)=f(x)/x^(3n) e^(-x⁻²)}{\(h^{(n)}(x)=\frac{f(x)}{x^{3n}}e^{−x^{−2}}\)}
\\{h^((n-1))}{\(h^{(n−1)}\)}
\\{h^((n-1))(0)=0}{\(h^{(n−1)}(0)=0\)}
\\{h^((n-1))(x) = f(x)/x^(3n) e^(-x⁻²)}{\(h^{(n−1)}(x)=\frac{f(x)}{x^{3n}}e^{−x^{−2}}\)}
\\{i}{\(i\)}
\\{k}{\(k\)}
\\{k →+∞}{\(k→+∞\)}
\\{k ≥ 0}{\(k≥0\)}
\\{k(x)=0 ⟺ h(b-a) ≤ h(x-a) ⟺ b-a ≤ x-a}{\(k(x)=0⟺h(b−a)≤h(x−a)⟺b−a≤x−a\)}
\\{k(x)=1 ⟺ h(b-a)-h(x-a)=h(b-a) ⟺ h(x-a)=0 ⟺ x-a ≤ 0 ⟺ x ≤ a}{\(k(x)=1⟺h(b−a)−h(x−a)=h(b−a)⟺h(x−a)=0⟺x−a≤0⟺x≤a\)}
\\{k(x)=h(h(b-a)-h(x-a))/h(h(b-a))}{\(k(x)=\frac{h(h(b−a)−h(x−a))}{h(h(b−a))}\)}
\\{l_i<r_i}{\(l_{i}<r_{i}\)}
\\{lim_(x → 0⁺) (-d x^(-d-1))/(-2 x⁻³ e^(x⁻²)) = lim_(x → 0⁺) d/2 ⋅ (x^(-(d-2)))/e^(x⁻²)}{\(\operatorname*{lim}_{x→0^{+}}\frac{−dx^{−d−1}}{−2x^{−3}e^{x^{−2}}}=\operatorname*{lim}_{x→0^{+}}\frac{d}{2}⋅\frac{x^{−(d−2)}}{e^{x^{−2}}}\)}
\\{lim_(x → 0⁺) (h^((n-1))(x))/x = 0}{\(\operatorname*{lim}_{x→0^{+}}\frac{h^{(n−1)}(x)}{x}=0\)}
\\{lim_(x → 0⁺) x^(-d) e^(-x⁻²) =0}{\(\operatorname*{lim}_{x→0^{+}}x^{−d}e^{−x^{−2}}=0\)}
\\{lim_(x → 0⁺) x^(-d) e^(-x⁻²)=lim_(x → 0⁺) x^(-d)/e^(x⁻²)}{\(\operatorname*{lim}_{x→0^{+}}x^{−d}e^{−x^{−2}}=\operatorname*{lim}_{x→0^{+}}\frac{x^{−d}}{e^{x^{−2}}}\)}
\\{lim_(x → 0⁺) x^(-d+2) e^(-x⁻²) =0}{\(\operatorname*{lim}_{x→0^{+}}x^{−d+2}e^{−x^{−2}}=0\)}
\\{n}{\(n\)}
\\{n ≥ 0}{\(n≥0\)}
\\{n=0}{\(n=0\)}
\\{n>0}{\(n>0\)}
\\{overline((a, b))=[a, b]}{\(\overline{(a,b)}=[a,b]\)}
\\{overline(B_r (boldzero))}{\(\overline{B_{r}(\mathbf{0})}\)}
\\{p ≥ 1}{\(p≥1\)}
\\{p=∞}{\(p=∞\)}
\\{r>0}{\(r>0\)}
\\{sup}{\(\operatorname*{sup}\)}
\\{x}{\(x\)}
\\{x → 0⁺}{\(x→0^{+}\)}
\\{x ↦ x^p}{\(x↦x^{p}\)}
\\{x ∈ ℝ}{\(x∈ℝ\)}
\\{x ≠ 0}{\(x≠0\)}
\\{x>0}{\(x>0\)}
\\{xx}{\(\mathbf{x}\)}
\\{xx ≠ boldzero}{\(\mathbf{x}≠\mathbf{0}\)}
\\{{f(xx)≠ 0}={xx: |xx|<b}=B_b (xx)=B_r (xx)}{\(\{f(\mathbf{x})≠0\}=\{\mathbf{x}:\vert \mathbf{x}\vert <b\}=B_{b}(\mathbf{x})=B_{r}(\mathbf{x})\)}
\\{{f_k}}{\(\{f_{k}\}\)}
\\{{x: g(x) ≠ 0}}{\(\{x:g(x)≠0\}\)}
\\{{xx: f(xx) ≠ 0}
= {(x₁, …, x_n): g_(lᵢ, rᵢ)(xᵢ) ≠ 0 ∀i ∈{1, …, n}}
= {(x₁, …, x_n): lᵢ<x_i< rᵢ ∀i ∈{1, …, n}}
= (l₁, r₁)×(l₂, r₂) × ⋯ ×(lₙ, rₙ)}{\(\{\mathbf{x}:f(\mathbf{x})≠0\}=\{(x_{1},…,x_{n}):g_{l_{i},r_{i}}(x_{i})≠0∀i∈\{1,…,n\}\}=\{(x_{1},…,x_{n}):l_{i}<x_{i}<r_{i}∀i∈\{1,…,n\}\}=(l_{1},r_{1})×(l_{2},r_{2})×⋯×(l_{n},r_{n})\)}
\\{{ξ_i},{ξ'_i}}{\(\{ξ_{i}\},\{ξ^{\prime }_{i}\}\)}
\\{{φ_k}}{\(\{φ_{k}\}\)}
\\{|P_Γ₃[φ, F] - P'_Γ₃[φ, F]| ≤ ε/2}{\(\vert P_{Γ_{3}}[φ,F]−P^{\prime }_{Γ_{3}}[φ,F]\vert ≤\frac{ε}{2}\)}
\\{|P_Γ₃[φ, F] - P_Γ₄[φ, F]| ≤ ε/2 < ε}{\(\vert P_{Γ_{3}}[φ,F]−P_{Γ_{4}}[φ,F]\vert ≤\frac{ε}{2}<ε\)}
\\{|R_Γ[φ, F]-R'_Γ[φ, F]|<ε/2}{\(\vert R_{Γ[φ,F]}−R^{\prime }_{Γ[φ,F]}\vert <\frac{ε}{2}\)}
\\{|R_Γ₁[φ, P]-R_Γ₂[φ, P]|<ε}{\(\vert R_{Γ_{1}}[φ,P]−R_{Γ_{2}}[φ,P]\vert <ε\)}
\\{|R_Γ₃[φ, F]-R_Γ₄[φ, F]|<ε/2}{\(\vert R_{Γ_{3}}[φ,F]−R_{Γ_{4}}[φ,F]\vert <\frac{ε}{2}\)}
\\{|f-f_k| → 0}{\(\vert f−f_{k}\vert →0\)}
\\{|f-f_k|^p ≤ (|f|+|f_k|)^p}{\(\vert f−f_{k}\vert ^{p}≤(\vert f\vert +\vert f_{k}\vert )^{p}\)}
\\{|f_k|^p}{\(\vert f_{k}\vert ^{p}\)}
\\{|f_k|≤ φ_k}{\(\vert f_{k}\vert ≤φ_{k}\)}
\\{|f|^p}{\(\vert f\vert ^{p}\)}
\\{|xx|≤ a=r/2}{\(\vert \mathbf{x}\vert ≤a=\frac{r}{2}\)}
\\{|{xx ∈ ℝ^n: f^*(xx)>α}| ≤ C α⁻¹ ∫_{|f|>α\/2} |f(xx)| dif xx}{\(\vert \{\mathbf{x}∈ℝ^{n}:f^{∗}(\mathbf{x})>α\}\vert ≤Cα^{−1}∫_{\{\vert f\vert >α/2\}}\vert f(\mathbf{x})\vert \hspace{0.16666666666666666em}\mathrm{d}\mathbf{x}\)}
\\{|⋅|}{\(\vert ⋅\vert \)}
\\{ˆs = t s}{\(ˆs=ts\)}
\\{ˆt = t/s}{\(ˆt=\frac{t}{s}\)}
\\{Γ}{\(Γ\)}
\\{Γ₁, Γ₂}{\(Γ_{1},Γ_{2}\)}
\\{Γ₃}{\(Γ_{3}\)}
\\{Γ₃, Γ₄}{\(Γ_{3},Γ_{4}\)}
\\{Γ₄}{\(Γ_{4}\)}
\\{Γ₄=Γ₃=Γ}{\(Γ_{4}=Γ_{3}=Γ\)}
\\{Γ₅}{\(Γ_{5}\)}
\\{α ∈(2, ∞)}{\(α∈(2,∞)\)}
\\{ε>0}{\(ε>0\)}
\\{ξ}{\(ξ\)}
\\{ξ_i}{\(ξ_{i}\)}
\\{φ = 2^p ⋅ 2 ⋅ |f|^p}{\(φ=2^{p}⋅2⋅\vert f\vert ^{p}\)}
\\{φ ∈ L(E)}{\(φ∈L(E)\)}
\\{φ_k}{\(φ_{k}\)}
\\{φ_k = 2^p (|f|^p+|f_k|^p)}{\(φ_{k}=2^{p}(\vert f\vert ^{p}+\vert f_{k}\vert ^{p})\)}
\\{φ_k → φ}{\(φ_{k}→φ\)}
\\{ω(α) ≤ |{xx ∈ ℝ^n: f^*(xx)>α}|}{\(ω(α)≤\vert \{\mathbf{x}∈ℝ^{n}:f^{∗}(\mathbf{x})>α\}\vert \)}
\\{ω(α) ≤ |{xx ∈ ℝ^n: f^*(xx)>α}| ≤ C α⁻¹ ∫_{|f|>α\/2} |f(xx)| dif xx}{\(ω(α)≤\vert \{\mathbf{x}∈ℝ^{n}:f^{∗}(\mathbf{x})>α\}\vert ≤Cα^{−1}∫_{\{\vert f\vert >α/2\}}\vert f(\mathbf{x})\vert \hspace{0.16666666666666666em}\mathrm{d}\mathbf{x}\)}
\\{ω(α)=|{xx ∈ E: f^*(xx)>α}|}{\(ω(α)=\vert \{\mathbf{x}∈E:f^{∗}(\mathbf{x})>α\}\vert \)}
\\{‖T f‖_p ≤ 
∫_0^∞ t^(-1\/p) K(1, t) ⋅ ‖f‖^p dif t = γ ‖f‖^p}{\(‖Tf‖_{p}≤∫^{∞}_{0}t^{−1/p}K(1,t)⋅‖f‖^{p}\hspace{0.16666666666666666em}\mathrm{d}t=γ‖f‖^{p}\)}
\\{‖f-f_k‖_p^p→ 0}{\(‖f−f_{k}‖^{p}_{p}→0\)}
\\{‖f-f_k‖_p→ 0}{\(‖f−f_{k}‖_{p}→0\)}
\\{‖f-f_k‖_∞ → 0}{\(‖f−f_{k}‖_{∞}→0\)}
\\{‖f_k-f‖_p → 0}{\(‖f_{k}−f‖_{p}→0\)}
\\{‖f_k-f‖_p ⋅ ‖g‖_p' → 0}{\(‖f_{k}−f‖_{p}⋅‖g‖_{p'}→0\)}
\\{‖f_k‖_p → ‖f‖_p}{\(‖f_{k}‖_{p}→‖f‖_{p}\)}
\\{‖f_k‖_p →‖f‖_p}{\(‖f_{k}‖_{p}→‖f‖_{p}\)}
\\{‖f_k‖_p^p → ‖f‖_p^p}{\(‖f_{k}‖^{p}_{p}→‖f‖^{p}_{p}\)}
\\{‖f_k‖_p^p →‖f‖_p^p ⟺ ∫ |f_k|^p → ∫ |f|^p}{\(‖f_{k}‖^{p}_{p}→‖f‖^{p}_{p}⟺∫\vert f_{k}\vert ^{p}→∫\vert f\vert ^{p}\)}
\\{‖f_k‖_∞ =‖f‖_∞ = 1}{\(‖f_{k}‖_{∞}=‖f‖_{∞}=1\)}
\\{→ +∞}{\(→+∞\)}
\\{→ 0}{\(→0\)}
\\{∑ |sup φ(ξ_i) - inf φ(ξ_i)| ⋅ (f(x_i)-f(x_(i-1)))⁺ ≤ ε/2}{\(∑\vert \operatorname*{sup}φ(ξ_{i})−\operatorname*{inf}φ(ξ_{i})\vert ⋅(f(x_{i})−f(x_{i−1}))^{+}≤\frac{ε}{2}\)}
\\{∑ |sup φ(ξ_i) - inf φ(ξ_i)| ⋅ |f(x_i)-f(x_(i-1))| ≤ ε/2}{\(∑\vert \operatorname*{sup}φ(ξ_{i})−\operatorname*{inf}φ(ξ_{i})\vert ⋅\vert f(x_{i})−f(x_{i−1})\vert ≤\frac{ε}{2}\)}
\\{∫ (f_k-f) g → 0 ⟺ ∫ f_k g - ∫ f g → 0 ⟺ ∫ f_k g → ∫ f g}{\(∫(f_{k}−f)g→0⟺∫f_{k}g−∫fg→0⟺∫f_{k}g→∫fg\)}
\\{∫ f g}{\(∫fg\)}
\\{∫ f_k g}{\(∫f_{k}g\)}
\\{∫ |(f_k-f) g|→ 0}{\(∫\vert (f_{k}−f)g\vert →0\)}
\\{∫ |f g|<∞}{\(∫\vert fg\vert <∞\)}
\\{∫ |f-f_k|^p →0}{\(∫\vert f−f_{k}\vert ^{p}→0\)}
\\{∫ φ dif F}{\(∫φ\hspace{0.16666666666666666em}\mathrm{d}F\)}
\\{∫ φ dif P}{\(∫φ\hspace{0.16666666666666666em}\mathrm{d}P\)}
\\{∫ φ_k → 2^p (|f|^p+|f|^p)}{\(∫φ_{k}→2^{p}(\vert f\vert ^{p}+\vert f\vert ^{p})\)}
\\{∫_0^2 ω(α) dif α ≤ ∫_0^2 |E| dif α = 2 |E|}{\(∫^{2}_{0}ω(α)\hspace{0.16666666666666666em}\mathrm{d}α≤∫^{2}_{0}\vert E\vert \hspace{0.16666666666666666em}\mathrm{d}α=2\vert E\vert \)}
\\{∫_E f^* dif xx = ∫_0^∞ ω(α) dif α = ∫_0^2 ω(α) dif α + ∫_2^∞ ω(α) dif α}{\(∫_{E}f^{∗}\hspace{0.16666666666666666em}\mathrm{d}\mathbf{x}=∫^{∞}_{0}ω(α)\hspace{0.16666666666666666em}\mathrm{d}α=∫^{2}_{0}ω(α)\hspace{0.16666666666666666em}\mathrm{d}α+∫^{∞}_{2}ω(α)\hspace{0.16666666666666666em}\mathrm{d}α\)}
\\{∫_E f^*dif xx ≤ 2|E| + C ∫_(ℝ^n) |f|log⁺|f|dif xx ≤ C(|E|+∫_(ℝ^n) |f|log⁺|f|dif xx)}{\(∫_{E}f^{∗}\hspace{0.16666666666666666em}\mathrm{d}\mathbf{x}≤2\vert E\vert +C∫_{ℝ^{n}}\vert f\vert \operatorname{log}^{+}\vert f\vert \hspace{0.16666666666666666em}\mathrm{d}\mathbf{x}≤C(\vert E\vert +∫_{ℝ^{n}}\vert f\vert \operatorname{log}^{+}\vert f\vert \hspace{0.16666666666666666em}\mathrm{d}\mathbf{x})\)}
\\{∫_E |f_k-f|→ 0}{\(∫_{E}\vert f_{k}−f\vert →0\)}
\\{∫_E φ_k → ∫_E φ}{\(∫_{E}φ_{k}→∫_{E}φ\)}
\\{∫_a^b φ dif g}{\(∫^{b}_{a}φ\hspace{0.16666666666666666em}\mathrm{d}g\)}
\\{∫_a^b φ dif h}{\(∫^{b}_{a}φ\hspace{0.16666666666666666em}\mathrm{d}h\)}
\\{≤ (2 max(|f|,|f_k|))^p = 2^p max(|f|^p,|f_k|^p) ≤ 2^p (|f|^p+|f_k|^p)}{\(≤(2\operatorname*{max}(\vert f\vert ,\vert f_{k}\vert ))^{p}=2^{p}\operatorname*{max}(\vert f\vert ^{p},\vert f_{k}\vert ^{p})≤2^{p}(\vert f\vert ^{p}+\vert f_{k}\vert ^{p})\)}
\\{⟺ b≤x}{\(⟺b≤x\)}
\endgroup
\fi
\typstmathinputdisable{\$}

\hfuzz=10pt

\maketitle

\section{Chapter 1}

\subsection{Exercise 1 (o)} %% khanh
% (reviewed by Hong Duc)
Let $E$ be a compact set in $\R^n$ and $f$ be continuous in $E$ relative to $E$. Then the following are true.
\begin{enumerate}
    \item $f$ is bounded on $E$ \label{e1i1}
    \item $f$ attains its supremum and infimum on $E$ \label{e1i2}
    \item $f$ is uniformly continuous on $E$ relative to $E$ \label{e1i3}
\end{enumerate}

\textbf{Proof} % ← could use amsthm but well okay...

$f$ be continuous in $E$ relative to $E$; that is, for all $x_0 \in E$, $f(x_0)$ is finite and either $x_0$ is an isolated point or $x_0$ is a limit point of $E$ and $\lim_{x \to x_0; x \in E} f(x) = f(x_0)$.

This can be written as: Given $f: E \to \R$, for all $x \in E$, for all $\epsilon > 0$, there exists a open ball $\Ball_\delta(x)$ such that $f(E \cap \Ball_\delta(x)) \subseteq \Ball_{\epsilon}(f(x))$. This is equivalent to $f$ being continuous.

The three points are consequences of compactness.

\textbf{Proof of \ref{e1i1}}

$f$ is continuous then $f$ maps compact sets to compact sets, i.e. $f(E)$ is compact. By \emph{Heine-Borel theorem}, $f(E)$ is closed and bounded in $\R$

\textbf{Proof of \ref{e1i2}}

As $f(E)$ is bounded, $\sup f(E)$ and $\inf f(E)$ are finite and limit points of $f(E)$. As $f(E)$ is closed, it contains all of its limit points, including $\sup f(E)$ and $\inf f(E)$

\textbf{Proof of \ref{e1i3} (\emph{Heine–Cantor theorem})}

For all $\epsilon > 0$, we will construct $\delta > 0$ such that $||x - y|| < \delta \implies |f(x) - f(y)| < ε$
for all $x, y \in E$

For all $x \in E$, since $f$ is continuous, there is a ball $\Ball_{\delta_x}(x)$ such that $f(E \cap \Ball_{\delta_x}(x)) \subseteq \Ball_{\epsilon / 2}(f(x))$. The collection $B = \{ \Ball_{\delta_x / 2}(x): x \in E \}$ covers $E$. Since $E$ is compact, there is a finite subcover $B_N = \{ \Ball_{\delta_{x_i} / 2}(x_i): i \in \{1, 2, ..., N \} \}$. Let $\delta = \min \{\delta_{x_i} / 2: i \in \{1, 2, ..., N \} \}$

For any $x, y \in E$ with $||x - y|| < \delta$. $x$ must lie within a ball of $B_N$, namely $\Ball_{\delta_{x_i} / 2}(x_i)$. The distance between $x_i$ and $y$ is

\begin{align*}
    ||y - x_i|| &\leq ||y - x|| + ||x - x_i|| &\text{(triangle inequality)}\\
                &\leq  \delta + \frac{\delta_{x_i}}{2} \\
                &\leq  \frac{\delta_{x_i}}{2} + \frac{\delta_{x_i}}{2} &\text{($\delta = \min \{\delta_{x_i} / 2: i \in \{1, 2, ..., N \} \}$)}\\
                &= \delta_{x_i} 
\end{align*}

Hence, $x, y \in \Ball_{\delta_{x_i}}(x_i)$. By the premise of $f$ being continuous, $\{ f(x), f(y) \} \subseteq \Ball_{\epsilon / 2}(f(x_i))$.
Hence, $|f(x) - f(y)| < \epsilon$.


\subsection{Exercise 15}% khanh
% reviewed by Nigel (fine)
% approved by Hong Duc

Let $\Gamma_1, \Gamma_2$ be any two partitions, we always have $L_{\Gamma_1} \leq L_{\Gamma_1 \cup \Gamma_2} \leq U_{\Gamma_1 \cup \Gamma_2} \leq U_{\Gamma_2}$. Therefore,

\[
    \sup_\Gamma L_\Gamma \leq \inf_\Gamma U_\Gamma
\]

($\implies$)
From exercise 1.r, $f$ is bounded Riemann integrable implies $\sup_\Gamma L_\Gamma = \inf_\Gamma U_\Gamma = A$. Therefore, there exists $\Gamma_1$ and $\Gamma_2$ such that $A + \epsilon/2 > U_{\Gamma_1} \geq U_{\Gamma_1 \cup \Gamma_2} \geq L_{\Gamma_1 \cup \Gamma_2} \geq L_{\Gamma_2} \geq A - \epsilon/2$. Hence, $U_{\Gamma_1 \cup \Gamma_2} - L_{\Gamma_1 \cup \Gamma_2} < \epsilon$

($\impliedby$)

By the premise, $\sup_\Gamma L_\Gamma = \inf_\Gamma U_\Gamma$. From Exercise 1.r, the Riemann integral of $f$ exists.

\begin{comment}
    



Suppose a bounded $f$ is Riemann integrable on $I$, then $\inf_{\Gamma}U_{\Gamma}=\sup_{\Gamma}L_{\Gamma}=A$ where 
\begin{align*}
    U_{\Gamma}=\sum_{k=1}^{m}\sup_{x\in I_{k}} f(x)v(I_{k})\mbox{ , }\quad L_{\Gamma}=\sum_{k=1}^{m}\inf_{x\in I_{k}}f(x)v(I_{k}).
\end{align*}

By definition, we know that $U_{\Gamma}\geq L_{\Gamma}$. Thus, 
\begin{equation*}
    0\leq U_{\Gamma}-L_{\Gamma} = \sum_{i=1}^{m}\left(\sup_{x\in I_{k}}f(x)-\inf_{x\in I_{k}}f(x)\right)v(I_{k})
\end{equation*}
Taking limit of $|\Gamma|\to 0$, we have
\begin{align*}
     \lim_{|\Gamma|\to 0}(U_{\Gamma}-L_{\Gamma}) &= \lim_{|\Gamma|\to 0}\left(\sum_{k=1}^{m}\left(\sup_{x\in I_{k}}f(x)-\inf_{x\in I_{k}}f(x)\right)v(I_{k})\right) \\
     &= A-A \\
     &=0<\epsilon \mbox{ , since }f\mbox{ is Riemann integrable on } I.
\end{align*}
Thus, there is some $Γ$ such that $U_Γ - L_Γ < ϵ$. %TODO added by Hong Duc

Conversely, suppose that given $\epsilon>0$, there is a partition $P$ of $I$ such that $0\leq U_{P}-L_{P}<\epsilon$. Suppose that $f$ is not Riemann-Integrable on $I$. Then
\[
\inf_{P}U_{P}>\sup_{P}L_{P}\implies \inf_{P}U_{P}-\sup_{P}L_{P}>0.
\]
Take $\inf_{P}U_{P}-\sup_{P}L_{P}=2\epsilon$, observe that
\begin{align*}
    U_{P}-L_{P}\geq \inf_{P}U_{P}-\sup_{P}L_{P}=2\epsilon>\epsilon,
\end{align*}
which is a contradiction. Therefore, $f$ is Riemann-Integrable on the interval $I$.

% TODO this part doesn't make sense to me at all -- Hong Duc
%Edited: I have edited the proof for the second part -- Zheng Xian
\end{comment}

\subsection{Exercise 18} % tietze extension theorem
% Hong Duc
% reviewed by khanh - approved

\typstmathinputenable{\$}

If $F$ is empty, the statement is trivial.

Otherwise, the set $ℝ ∖ F$ is open and not the whole of $ℝ$, thus by theorem 1.10 it can be written as a countable disjoint union of (possibly infinite) intervals of the form $(-∞, b)$, $(a, b)$, or $(a,+∞)$ for $a, b ∈ ℝ$.

We define $g$ to equal $f$ in $F$, and outside $F$ it's defined as follows (where $x$ is an arbitrary point in the possibly infinite interval):
\begin{itemize}
    \item on an infinite interval of the form $(-∞, b)$, $g(x) = f(b)$,
    \item on an infinite interval of the form $(a,+∞)$, $g(x)=f(a)$,
    \item on an interval of the form $(a, b)$, $g(x)$ interpolates linearly between the 2 endpoints, specifically $g(x)=f(a) + (f(b)-f(a)) ⋅ (x-a)/(b-a)$. Note that the linear function on the right hand side is equal to $f(a)$ when $x=a$ and equal to $f(b)$ when $x=b$.
\end{itemize}
Thus, because $f$ is continuous relative to $F$, $g$ is continuous on $ℝ$.

For the second part, it's clear that, by construction, in all cases then all values of $|g(x)|$ is $≤ M$ for any $M$ such that $|f(y)|≤ M ∀y ∈ ℝ$.

\typstmathinputdisable{\$}

\section{Chapter 2}

\subsection{Exercise 4}% Nigel, reviewed by Hong Duc
% reviewed by khanh - approved
Suppose $V[f; a, b] > M$; put $V = V[f; a, b]$. Let $0 < \epsilon < V - M$. Then there is a partition $\Gamma = \{a = x_{0}, x_{1}, \dots, x_{n} = b \}$ with $S_{\Gamma}[f; a, b] > M + \epsilon$. 

By pointwise convergence, we can choose $K$ such that for all $0 \leq j \leq n$, $\lvert f_{k}(x_{j}) - f(x_{j}) \rvert < \epsilon / (2n)$, for $k \geq K$. Then for $k \geq K$, $0 \leq j \leq n-1$, 
\begin{align*}
	&\lvert f_{k}(x_{j+1}) - f_{k}(x_{j}) \rvert \\
    &\geq - \lvert f_{k}(x_{j+1}) - f(x_{j+1}) \rvert + \lvert f(x_{j+1}) - f(x_{j}) \rvert - \lvert f(x_{j}) - f_{k}(x_{j}) \rvert \\
	&> \lvert f(x_{j+1}) - f(x_{j}) \rvert - \frac{\epsilon}{n}.
\end{align*}
Thus 
\begin{align*}
	S_{\Gamma}[f_{k}; a, b] &= \sum_{j = 0}^{n-1} \lvert f_{k}(x_{j+1}) - f_{k}(x_{j}) \rvert \\
	&> \sum_{j = 0}^{n-1} \lvert f(x_{j+1}) - f_{k}(x_{j}) \rvert - \frac{\epsilon}{n} \cdot n \\
	&= S_{\Gamma}[f; a, b] - \epsilon.
\end{align*}
Therefore
\begin{equation*}
	S_{\Gamma}[f_{k}; a, b] > M + \epsilon - \epsilon = M,
\end{equation*}
which contradicts $V[f_{k}; a, b] \leq M$.

Thus under the hypotheses of the problem, $V[f; a, b] \leq M$. This completes the proof.

We construct an example of a convergent sequence of functions of bounded variation whose limit is not of bounded variation. Put $a = 0$, $b = 1$.

We define a sequence of functions $f_{k}$:
\[ f_{k}(x) = \begin{cases*}
		1 & if $x \in \{ 2^{-j} \mid 1 \leq j \leq k \}$ \\
		0 & otherwise. 
		\end{cases*} \]
% roughly speaking, f_k is identically zero except at k points, where it's 1
% thus f_k has k "spikes" and variation 2k
% f_k → f, f has "infinitely many spikes" and thus infinite variation

Then $f_{k} \rightarrow f$ pointwise, where
\[ f(x) = \begin{cases*}
		1 & if $x \in \{ 2^{-j} \mid j \in \mathbb{Z}^{+} \}$ \\
		0 & otherwise. 
		\end{cases*} \]

We compute the variation of $f_{k}$. Put $x_{k} = \frac{1}{2}(2^{-k} + 2^{-(k-1)})$.
\begin{align*}
	V[f_{k}; 0, 1] &= V[f_{k}; 0, x_{k}] + V[f_{k}; x_{k}, x_{k-1}] + \cdots + V[f_{k}; x_{3}, x_{2}] + V[f_{k}; x_{2}, 1] \\
	&= 2k,
\end{align*}
as each interval in the sum contains one point in its interior where $f_{k} = 1$, $f_{k} = 0$ elsewhere in the interval, and there are $k$ intervals.

We similarly compute the variation of $f$. For any $n \geq 2$, 
\begin{align*}
	V[f; 0, 1] &= V[f; x_{2}, 1] + \sum_{j = 2}^{n} V[f; x_{j+1}, x_{j}] + V[f; 0, x_{n+1}] \\
	&\geq \sum_{j = 2}^{n} V[f; x_{j+1}, x_{j}] + V[f; 0, x_{n+1}] \\
	&= 2(n-1)
\end{align*}
by the same reasoning as for $f_{k}$.

This holds for arbitrary $n \geq 2$, so $V[f; 0, 1] = +\infty$.

\subsection{Exercise 5}
\iffalse
% khanh

Given any partition $\Gamma = \{a = x_0 < a + \epsilon = x_1 < x_2 < ...< x_m = b \}$ on $[a, b]$
\begin{align*}
    S_\Gamma
    &= |f(a+\epsilon) - f(a)| + S_{\Gamma_1} &\text{($\Gamma_1 = \{x_1, x_2, ..., x_m \}$)}\\
    &\leq |f(a+\epsilon) - f(a)| + M\\
    &\leq |f(a+\epsilon) - f(b)| + |f(b) - f(a)| + M &\text{(triangle inequality)}\\
    &\leq |f(b) - f(a)| + 2M &\text{($\{ a + \epsilon, b \}$ is a partition on $[a + \epsilon, b]$)}\\
\end{align*}

$S_\Gamma$ is bounded, then $f$ is bounded variation

Given $\epsilon > 0$, let $|f(a) - f(a + \epsilon)| > M$. Then

\begin{align*}
    V[f; a, b]
        &= V[f; a, a+\epsilon] + V[f; a + \epsilon, b] \\
        &> M + V[f; a + \epsilon, b] > M \\
\end{align*}

So, in general, it is possible that $V[f; a, b] > M$.
One sufficient condition (not necessary) for $V[f; a, b] \leq M$ is $f$ being continuous at $a$. Given any $\eta > 0$, pick $\epsilon > 0$ such that $x, y \in [a, a+\epsilon] \implies |f(y) - f(x)| < \eta$. Given any partition $\Gamma = \{a = x_0 < a + \epsilon = x_1 < x_2 < ...< x_m = b \}$
% TODO I think what you want is just $x \in [a, a+\epsilon] \implies |f(x) - f(a)| < \eta$? though the condition you stipulated is certainly possible by triangle inequality, it's unnecessary

\begin{align*}
    S_\Gamma
    &= |f(a + \epsilon) - f(a)| + S_{\Gamma_1} &\text{($\Gamma_1 = \{x_1, x_2, ..., x_m \}$)} \\
    &\leq \eta + M \\ 
\end{align*}

So $V[f; a, b] < \eta + M$. Send $\eta$ to $0$ implies, $V[f; a, b] \leq M$




\hrule
\fi

%Hong Duc

\typstmathinputenable{\$}

We can assume $a<b$, otherwise the question is trivial.

We will prove: $V[f; a, b] = X+Y$, where $X=lim_(ε → 0) V[f; a+ε, b]$ and $Y=lim_(ε → 0) sup_(x ∈[a, a+ε]) |f(x)-f(a)|$.

From now on, assume $0<ε ≤ b-a$.

As $ε$ decreases to $0$, $V[f; a+ε, b]$ increases monotonically and bounded above by $M$, thus the limit $X$ exists and is $≤ M$.

For any $ε>0$, then $|f(a+ε)-f(b)|≤ M$, thus $|f(a+ε)-f(a)|≤ M+|f(b)-f(a)|$, thus $f$ is bounded on $(a, b]$, because $f$ is furthermore finite at $a$, then $f$ is bounded on $[a, b]$. Thus $sup_(x ∈[a, a+ε]) |f(x)-f(a)|$ is finite for any $ε>0$.

As $ε$ decreases to $0$, then $sup_(x ∈[a, a+ε]) |f(x)-f(a)|$ decreases monotonically and bounded below by $0$, thus the limit $Y$ exists and is $≥ 0$. (in fact $Y=limsup_(x → a+) |f(x)-f(a)|$)

Pick any $δ>0$.

First we prove $V[f; a, b] ≥ X+Y - δ$. Pick $ε>0$ small enough such that 
$sup_(x ∈[a, a+ε]) |f(x)-f(a)| ≥ Y-δ/4$ and $V[f; a+ε, b]≥ X-δ/2$,
then pick $x ∈[a, a+ε]$ such that $|f(x)-f(a)|≥ Y-δ/2$,
thus $V[f; a, x] ≥ Y-δ/2$, so $V[f; a, b]=V[f; a, x]+V[f; x, a+ε]+V[f; a+ε,b] ≥ Y-δ/2 + X-δ/2 = X+Y-δ$.

Then we prove $V[f; a, b] ≤ X+Y+δ$.
By definition of $Y$, then there exist $ε$ such that for all $0<ε'≤ε$, then $|f(a)-f(a+ε')|<Y+δ$.
Consider any partition $Γ={x₁, x₂, …, x_m}$ of $[a, b]$.
Let $Γ'=Γ ∪ {a+ε} = {y₁, y₂, …, y_n}$ which is another partition of $[a, b]$, then $y₂ ≤ a+ε$.
% TODO $y₂ ≤ a+ε$
% fixed--Hong Duc
Then $S_(Γ') ≤ |f(a)-f(y₂)| + V[f; y₂, b] ≤ Y+δ + V[f; y₂, b]$. Note that $V[f; y₂, b] ≤ X$ because $V[f; a+ε, b]$ monotonically increases to $X$ as $ε → 0+$, so $S_Γ ≤ S_(Γ') ≤ X+Y+δ$. Taking the supremum over all $Γ$, we get $V[f; a, b] ≤ X+Y+δ$.

Since the above holds for all $δ>0$, then $V[f; a, b]=X+Y$ as desired.

Clearly $X ≤ M$.

It is not necessary that $V[f; a, b] ≤ M$. Take for example the function $ f(x)=cases(1" if "x=0, 0" otherwise"), $and $a=0$, $b=1$. It's easy to check $M=0$ satisfies the assumptions, but $X=0$, $Y=1$, $V[f; a, b]=1$.
% TODO to fix the text in the cases

In case $X=M$, then for $V[f; a, b] ≤ M$, then $Y=0$ must hold. But $Y=0$ is clearly equivalent to that $f$ is continuous from the right at $a$. (in conclusion, $f$ is continuous from the right at $a$ ensures that $V[f; a, b] ≤ M$)
\typstmathinputdisable{\$}




\iffalse
Consider the partition $\Gamma=\{a+\epsilon,b\}$ of $[a+\epsilon,b]$, we see that 
\begin{align*}
    |f(b)-f(a+\epsilon)|\leq M &\implies |f(a+\epsilon)|-|f(b)|\leq |f(b)-f(a+\epsilon)|\leq M\\
                               &\implies |f(a+\epsilon)|\leq M+|f(b)|.
\end{align*}
Take another partition $\overline{\Gamma}=\{x_{i}\}_{i=0}^{m}$ where $x_{0}=a,x_{m}=b$ of $[a,b]$. Then,
\begin{align*}
    S_{\overline{\Gamma}}&=|f(x_{1})-f(a)|+\sum_{i=2}^{m}|f(x_{i})-f(x_{i-1})| \\
    &\leq |f(x_{1})|+|f(a)|+V[f;x_{1},b] \\ %TODO ϵ is not defined here, you mean x₁?
    &\leq |f(a)|+|f(b)|+2M.
\end{align*}
Note that $f$ is finite and hence, $|f(a)|, |f(b)|<+\infty$, taking supremum, $V[f;a,b]\leq |f(a)|+|f(b)|+2M$.

$V[f;a,b]$ not necessarily $\leq M$. Consider the function $g$ as such
\[
g(x)=\begin{cases}
    1 & 0<x\leq 1, \\
    0 & x=0.
\end{cases}
\]
Then we can see that for any $\epsilon>0$, $g$ is of bounded variation on $[\epsilon,1]$ where $V[f;\epsilon,1]=0$. However, $g$ is of bounded variation on $[0,1]$ where $V[f;0,1]=1\neq 0$.  

% TODO give an explicit counterexample? -- Hong Duc
% Considered the step function as a counterexample -- Zheng Xian
For $V[f;a,b]\leq M$, we need $f$ to be continuous at $a$. Define a sequence of function $\{f_{k}\}$ such that it is pointwise convergent to $f$ where

\begin{comment}
    -- khanh
    f being continuous at a might be deduced directly from "S_{\overline{\Gamma}}&=|f(x_{1})-f(a)|+\sum_{i=2}^{m}|f(x_{i})-f(x_{i-1})|"
    no need to make a sequence f_k
\end{comment}

\[
f_{k}(x)=\begin{cases}
    f(x) & x\in[a+\epsilon,b] \\
    f(a) & x\in[a,a+\epsilon)
\end{cases},
\]

\begin{comment}
    -- khanh
    f(x) = f(x) ?
    maybe what you meant is f_k(x) = f(x)
\end{comment}


for $\epsilon=\frac{1}{k}$.
Then,
\begin{align*}
    V[f_{k};a,b]&=V\left[f_{k};a,a+\frac{1}{k}\right]+V\left[f_{k};a+\frac{1}{k},b\right] \\
    &=V\left[f_{k};a,a+\frac{1}{k}\right]+V\left[f;a+\frac{1}{k},b\right] \\
    &\leq V\left[f_{k};a,a+\frac{1}{k}\right]+M \\
    &= \left|f\left(a+\frac{1}{k}\right)-f(a)\right|+M \mbox{ , (based on construction).}
    % side note, usually \text{...} is better semantically than \mbox{...}, but okay
\end{align*}
Choose $K_{j}$ such that for all $k\geq K_{j}$
, $|f(a+\frac{1}{k})-f(a)|<\frac{1}{j}$, this is possible
by continuity of $f$ at $a$. Therefore,
\[
V[f_{k};a,b]<\frac{1}{j}+M.
\]
By Exercise $4$, $V[f;a,b]\leq\frac{1}{j}+M$. Since this is true for all $j$, $V[f;a,b]\leq M$.
\fi

\subsection{Exercise 15}% Hong Duc
% reviewing by khanh - approved
% reviewed by Nigel (looks good)
% edited -- Hong Duc

\typstmathinputenable{\$}
\begin{lemma}
    If $f$ is continuous on $[p, q]$, and $M$ is some constant $≥0$ such that $|f(x)|≤ M$ for all $x ∈[a, b]$, and $φ$ is of bounded variation on $[p, q]$, then $|∫_p^q f dif φ| ≤ M ⋅ V[φ; p, q]$.
\end{lemma}
\begin{proof}
    Clearly the integral on the left hand side exists.

    Pick any partition $Γ={x₀, x₁, …, x_m}$ of $[p, q]$ where $x₀=p$ and $x_m=q$, and a selection of points $ξ₁, …, ξ_m$. Then $|R_Γ| = |∑_(i=1)^m f(ξ_i) ⋅ (φ(x_i) - φ(x_(i-1)))| ≤ ∑_(i=1)^m |f(ξ_i)| ⋅ |φ(x_i) - φ(x_(i-1))| ≤ M ⋅ ∑_(i=1)^m |φ(x_i)-φ(x_(i-1))| ≤ M ⋅ V[φ; p, q]$.
\end{proof}

Since $f$ is continuous and $[a, b]$ is compact, $f$ is bounded, let's say $|f(x)|≤ M$ for all $x ∈[a, b]$, and $M≥0$.

We will show that $V[ψ; a, b] ≤ M ⋅ V[φ; a, b]$.

Pick any partition $Γ={x₀, x₁, …, x_m}$ of $[a, b]$. We will show $S_Γ [ψ; a, b] ≤ M ⋅ V[φ; a, b]$.

Indeed, 
$
S_Γ [ψ; a, b] &= ∑_(i=1)^m |ψ(x_i) - ψ(x_(i-1))| \
       &= ∑_(i=1)^m |∫^(x_i)_(x_(i-1)) f dif φ|.
$
For each $i$, let $p=x_(i-1)$ and $q=x_i$, then
$ |∫^q_p f dif φ| ≤ M ⋅ V[φ; p, q]$ by the lemma above.
Adding together over all values of $i$, we get the desired statement.

Thus $ψ$ is of bounded variation on $[a, b]$.

Now, consider $g$ continuous on $[a, b]$. We want to prove $∫_a^b g dif ψ = ∫_a^b g f dif φ$ -- clearly both integrals exist, because both $g$ and $g f$ are continuous, and both $ψ$ and $φ$ are of bounded variation.

Since $[a, b]$ is compact and $g$ is continuous, there exist $N>0$ such that $|g(x)|≤ N$ for all $x ∈[a, b]$.

We know $f$ is uniformly continuous. Fix $ε>0$, there exist $δ>0$ such that for all $x, y ∈[a, b]$, if $|x-y|<δ$ then $|f(x)-f(y)|<ε$.

Consider any partition $Γ={x₀, …, x_m}$ of $[a, b]$ and selection of intermediate points ${ξ₁, …, ξ_m}$, such that $x_i-x_(i-1) < δ$ for all $i$; in other words, $diam Γ < δ$.

Then, the Riemann-Stieltjes sum $R_Γ$ correspond to $∫_a^b g dif ψ$ (the left hand side) is equal to $∑_(i=1)^m g(ξ_i) ⋅ (ψ(x_i) - ψ(x_(i-1)))$, and the sum correspond to $∫_a^b g f dif φ$ (the right hand side) is $∑_(i=1)^m g(ξ_i) f(ξ_i) ⋅ (φ(x_i) - φ(x_(i-1)))$.


The absolute difference of the 2 expressions is:
$
& quad |∑_(i=1)^m
g(ξ_i) ⋅ (ψ(x_i) - ψ(x_(i-1)) - f(ξ_i) ⋅ (φ(x_i) - φ(x_(i-1))))| \
&= |∑_(i=1)^m g(ξ_i) ⋅ ( ∫_(x_(i-1))^(x_i) f dif φ  - f(ξ_i) ⋅ (φ(x_i) - φ(x_(i-1))))| \
&≤ ∑_(i=1)^m N ⋅ |∫_(x_(i-1))^(x_i) (f(x)-f(ξ_i)) dif φ(x)| \
&≤ ∑_(i=1)^m N ⋅ V[φ; x_(i-1), x_i] ⋅ ε.
$

The last inequality is by noting that $|f(x)-f(ξ_i)| ≤ ε$ for all $x ∈ [x_(i-1), x_i]$, and applying the lemma.

Thus, this expression is $≤
N ⋅ ε ⋅ ∑_(i=1)^m  V[φ; x_(i-1), x_i] = N ⋅ ε ⋅ V[φ; a, b]$.

Set $P = N ⋅ ε ⋅ V[φ; a, b]$.
Since $ε$ can be arbitrarily small, we conclude that: for all $P>0$, there exist $δ>0$ such that, for any $Γ$ where $diam Γ < δ$, then the difference between $R_Γ$ for the left integral and $R_Γ$ for the right integral is $≤ P$.

From this, we conclude the two integrals, since they exist, must be equal -- specifically, assume they aren't, that is $|∫_a^b g dif ψ - ∫_a^b g f dif φ| > ε$ for some $ε>0$.
Then we can find $Γ₁$ and $Γ₂$ such that, for any partition $Γ₁'$ finer than $Γ₁$ and $Γ₂'$ finer than $Γ₂$, then
$
cases(
|∫_a^b g dif ψ - R_(Γ₁')| < ε/3,
|∫_a^b g f dif φ - R_(Γ₂')| < ε/3
).
#forcetag("equation-1")
$
(in both cases, the Riemann-Stieltjes sum is for the corresponding integral).
Now pick $P=ε/3$ and construct $δ$ as above, and pick $Γ=Γ₁'=Γ₂'$ being finer than a common refinement of both $Γ₁$ and $Γ₂$,
and with diameter $< δ$. Then,
\begin{itemize}
    \item $diam Γ < δ$ guarantees that the difference between the two Riemann-Stieltjes sums are less than $ε/3$,
    \item $Γ$ is finer than both $Γ₁$ and $Γ₂$ ensures that Equation~\ref{equation-1} holds.
\end{itemize}
This is a contradiction by triangle inequality.
\typstmathinputdisable{\$}

\subsection{Exercise 16}% Nigel
% reviewing by khanh
% reviewed -- Hong Duc

If $f$ has no jump discontinuity, the statement is trivial.
Otherwise, by the definition of jump discontinuity, the points of discontinuity must occur in $(a, b)$.

Suppose the jump discontinuities of $f$ are at $\{ x_{1}, \dotsc, x_{n} \}$, with $x_{i} < x_{i+1}$.

Put $x_{0} = a$ and $x_{n+1} = b$. We can express
\begin{equation*}
	\int_{a}^{b} f d\phi =
    \int_{x_{0}}^{\frac{x_{0} + x_{1}}{2}} 
    f d\phi + \int_{\frac{x_{0} + x_{1}}{2}}^{\frac{x_{1} + x_{2}}{2}} f d\phi + \cdots + \int_{\frac{x_{n-1} + x_{n}}{2}}^{\frac{x_{n} + x_{n+1}}{2}} f d\phi + \int_{\frac{x_{n} + x_{n+1}}{2}}^{x_{n+1}} f d\phi
\end{equation*}
if each of the integrals on the RHS exist.

Observe that $f$ is continuous on $[x_{0}, \frac{x_{0} + x_{1}}{2}]$ and on $[\frac{x_{n} + x_{n+1}}{2}, x_{n+1}]$, while $\phi$ is of bounded variation on them. Thus the first and last integrals on the RHS exist. The remaining integrals are on intervals $I_{k}$ with $f$ having a single jump discontinuity (say at $x_{k}$) in the interior of their intervals,
and continuous elsewhere; $\phi$ continuous at $x_{k}$, and of bounded variation on the interval.

Thus it suffices to prove that for $f$ bounded and continuous on $[a, b]$ except for a jump discontinuity at $c \in (a, b)$, $\phi$ of bounded variation on $[a, b]$ and continuous at $c$, the integral $\int_{a}^{b} f d \phi$ exists.

Fix $\epsilon > 0$. Suppose $\lvert f \rvert \leq M$ on $[a, b]$.
Note $M > 0$ (otherwise there are no jump discontinuities). By continuity of $\phi$ at $c$, choose $\delta > 0$ such that $[c - \delta, c + \delta] \subset [a, b]$ and for $x, y \in (c - \delta, c + \delta)$, $\lvert \phi(x) - \phi(y) \rvert < \epsilon / (24M)$.

Define $f_{1}$ on $[a, c]$ by
\[ f_{1}(x) = \begin{cases*}
		f(x) & if $x < c$ \\
		f(c-) & if $x = c$. 
		\end{cases*} \]

% khanh - is this c^- not c-

Then $f_{1}$ is continuous and agrees with $f$ on $[a, c)$. There is a partition $P' = \{a, x_{1}, \cdots, x_{k-1}, x_{k} = c \}$ such that $U(P, f_{1}, \phi) - L(P, f_{1}, \phi) < \epsilon / 3$ for $P$ finer than $P'$. Refine $P'$ if necessary so that the point $x'$ preceding $c$ in it satisfies $c - \delta < x' < c$.

% khanh - $U(P, f_{1}, \phi) - L(P, f_{1}, \phi) < \epsilon / 3$ I don't recall we have this for non-monotone function \phi

Similarly define $f_{2}$ on $[a, c]$ by
\[ f_{2}(x) = \begin{cases*}
		f(x) & if $x > c$ \\
		f(c+) & if $x = c$. 
		\end{cases*} \]

% khanh - is this c^+ not c+

Then $f_{2}$ is continuous and agrees with $f$ on $(c, b]$. There is a partition $P'' = \{x_{k} = c, x_{k+1}, \cdots, x_{n}, b \}$ such that $U(P, f_{2}, \phi) - L(P, f_{2}, \phi) < \epsilon / 3$ for $P$ finer than $P''$. Refine $P''$ if necessary so that the point $x''$ succeeding $c$ in it satisfies $c < x' < c + \delta$.

Put $P''' = P' \cup P''$. For $P$ finer than $P'''$, with $x'$ the point preceding $c$ in $P'$ and $x''$ the point succeeding $c$ in $P''$, 
\begin{align*}
	U(P, f, \phi) - L(P, f, \phi) &= U(P', f_{1}, \phi) - L(P', f_{1}, \phi) \\
    &+ U(P'', f_{2}, \phi) - L(P'', f_{2}, \phi) \\
	&+ [M_{f}' - m_{f}' - (M_{f_{1}} - m_{f_{1}})][\phi(c) - \phi(x')] \\
	&+ [M_{f}'' - m_{f}'' - (M_{f_{2}} - m_{f_{2}})][\phi(x'') - \phi(c)],
\end{align*}
where $M_{f}' = \sup \{ f(x) \mid x \in [x', c] \}$, $M_{f_{1}} = \sup \{ f_{1}(x) \mid x \in [x', c] \}$, $M_{f}'' = \sup \{ f(x) \mid x \in [c, x''] \}$, $M_{f_{2}} = \sup \{ f_{2}(x) \mid x \in [c, x''] \}$, and $m_{f}'$, $m_{f_{1}}$, $m_{f}''$, $m_{f_{2}}$ are defined analogously, as infima rather than suprema.

By choice of $x'$ and $x''$, $\lvert \phi(c) - \phi(x') \rvert < \epsilon/(24M)$ and $\lvert \phi(x'') - \phi(c) \rvert < \epsilon/(24M)$, so
\begin{align*}
	&\lvert [M_{f}' - m_{f}' - (M_{f_{1}} - m_{f_{1}})][\phi(c) - \phi(x')] \rvert \\
    &+ \lvert [M_{f}'' - m_{f}'' - (M_{f_{2}} - m_{f_{2}})][\phi(x'') - \phi(c)] \rvert \\
	&\leq \epsilon /(24M) \cdot 8M = \epsilon / 3.
\end{align*}

Thus $U(P, f, \phi) - L(P, f, \phi) < \epsilon / 3 + \epsilon / 3 + \epsilon / 3 = \epsilon$, for all partitions $P$ finer than $P'''$. This proves that $\int_{a}^{b} f d\phi$ exists.

\subsection{Exercise 31}% Hong Duc
% reviewed by khanh - approved

For the first statement, for all partition \(Γ\) of \([a,b]\) we can trivially check \(R_{Γ}=f(b)−f(a)\), so the limit is also \(f(b)−f(a)\).

For the second statement, fix a partition \(Γ\). Let \(x_{i−1},x_{i}\) be two consecutive points in the partition, by mean value theorem there exist a choice \(ξ_{i}\) in the segment such that \(f′(ξ_{i})(x_{i}−x_{i−1})=f(x_{i})−f(x_{i−1})\), if such selection is made over all intervals of the partition then \(R_{Γ}=∫^{b}_{a}\hspace{0.16666666666666666em}\mathrm{d}f\); because \(f′\) is Riemann integrable on \([a,b]\), the integral must thus be equal to \(∫^{b}_{a}\hspace{0.16666666666666666em}\mathrm{d}f\).

\begin{comment}
    -- khanh
    better establish that f' is Riemann integrable before making the choice of \xi
\end{comment}

Combine the 2 statements we get the desired result.

\section{Chapter 3}

\subsection{Exercise 9}% Nigel
% reviewed by khanh, approved
% approved by Hong Duc

Put $A_{j} = \bigcup_{k = j}^{\infty} E_{k}$. We have $A_{1} \supset A_{2} \supset \cdots$, and $\lvert A_{1} \rvert_{e} \leq \sum_{k=1}^{\infty} \lvert E_{k} \rvert_{e} < \infty$. 

Therefore putting $A = \cap_{j=1}^{\infty} A_{j} = \lim \sup E_{k}$, we have 
\begin{align*}
\lvert A \rvert_{e} &\leq \lim_{j \rightarrow \infty} \lvert A_{j} \rvert_{e} \\
	&\leq \lim_{j \rightarrow \infty} \sum_{k=j}^{\infty} \lvert E_{k} \rvert_{e} \\
	&= 0,
\end{align*}

The first inequality follows from the fact that $A \subset A_{j}$ for all $j$, whence $\lvert A \rvert_{e} \leq \lvert A_{j} \rvert_{e}$ for all $j$.

The last equality is due to $\sum_{k=1}^{\infty} \lvert E_{k} \rvert_{e} < \infty$.

Thus $\lvert \lim \sup E_{k} \rvert_{e} = \lvert A \rvert_{e} = 0$.

\subsection{Exercise 12}%Hong Duc
% reviewed by Nigel (added some comments)
% reviewed by khanh (looks ok)

We rewrite the statement:
\begin{quote}
    If $A$ and $B$ are measurable subsets of $ℝ¹$, show that $A × B$ is a measurable subset of $ℝ²$ and $|A × B|=|A|⋅|B|$. (Interpret $0 ⋅ ∞$ as $0$.)
\end{quote}

\textbf{Case 1.}
If both $A$ and $B$ are (finite) intervals, the statement is trivial.

\textbf{Case 2.}
If $A$ and $B$ are open sets with finite measure, we write $A=⨆ A_i$ and $B=⨆ B_j$ and $A_i$ and $B_j$ are half-open interval (assume $i$ and $j$ runs over all of $ℕ$ from now on), then $|A|=∑_i |A_i|$ and $|B|=∑_j |B_j|$ are finite (the sum converges). Then $A × B = ⨆ A_i × B_j$ (we see that the $A_i × B_j$ intervals are disjoint, because we can prove that, for any sets $A_i, B_j, A_{i'}, B_{j'}$, if $A_i ∩ A_{i'}=∅$ or $B_j ∩ B_{j'}=∅$, then $(A_i × B_j)∩(A_{i'}× B_{j'})=∅$), because each $A_i × B_j$ is measurable as shown above, thus $A × B$ is measurable and $|A × B|=∑_i ∑_j |A_i| |B_j| = (∑_i |A_i|) (∑_j |B_j|) = |A| |B|$.

\textbf{Case 3.}
If $A$ and $B$ are $G_δ$ sets with finite measure, then pick a sequence of finite-measure open sets $A_i ↘ A$ and $B_i ↘ B$.

This is possible, because, by definition of $G_δ$ sets, $A=⋂_{i ∈ ℤ⁺} A_i$ for each $A_i$ open, because $A$ has finite measure we can pick $C$ open, finite measure, and containing $A$, then the sequence $\{ C ∩ ⋂_{j=1}^i A_i \}_{i ∈ ℤ⁺}$ is open, finite measure, and decreases to $A$. Similar for $B$.

Then we must have that $A_i × B_i ↘ A × B$, so:
\begin{itemize}
    \item $A × B$ is the countable intersection of measurable sets, thus measurable,
    \item because $|A_i||B_i|$ is finite, then we must have $|A_i × B_i|↘|A × B|$, but the left hand side is equal to $|A_i||B_i|$, because $|A_i|↘|A|$ and $|B_i|↘|B|$, then $|A_i||B_i|↘|A||B|$, we get the desired result.
\end{itemize}

\textbf{Case 4.} If $A$ and $B$ are $G_δ$ sets with possibly-infinite measure (assume either $|A|$ or $|B|$ is infinite otherwise we have done it above), then for each $i, j ∈ ℤ$, write $A_i=A ∩[i, i+1)$ and $B_j=B ∩[j, j+1)$.
\begin{itemize}
    \item Note that $[i, i+1)=⋂_{k ∈ ℤ⁺} (i-\frac{1}{k}, i+1)$, thus $[i, i+1)$ is of type $G_δ$, and so is $A_i$. Similar for $B_j$.
    \item Thus $A_i$ and $B_j$ are $G_δ$ sets with finite measure, we can apply the point above, so $A_i × B_j$ are measurable with measure $|A_i||B_j|$.
    \item As $A=⨆_i A_i$ and $B=⨆_j B_j$, then $|A|=∑_i |A_i|$ and $|B|=∑_j |B_j|$. Also, $A × B = ⨆_{i, j} A_i × B_j$, then $A × B$ is measurable and $|A × B|=∑_{i, j} |A_i||B_j|$ -- each term is nonnegative and finite, but the sum may not converge.
\end{itemize}

Now, without loss of generality assume $|A|=∞$.
\begin{itemize}
    \item If $|B|=0$, then $|B_j|=0$ for all $j$, thus all the terms in the sum $∑_{i, j} |A_i||B_j|$ is $0$, thus $|A × B|=0=∞ ⋅ 0=|A|⋅|B|$ as desired.
    \item If $|B|>0$, then there exist some $j₀∈ ℤ$ such that $|B_{j₀}|>0$, then $∑_{i, j} |A_i||B_j| ≥ ∑_i |A_i| |B_{j₀}| = |B_{j₀}| ⋅ (∑_i |A_i|) = |B_{j₀}| ⋅ |A| = ∞$.
\end{itemize}
In all cases, we get $|A × B|=|A|⋅|B|$.

\textbf{Case 5.} $A$ and $B$ are arbitrary measurable sets. Then write $G_A = A ⊔ Z_A$, $G_B = B ⊔ Z_B$, such that $G_A$ and $G_B$ are of type $G_δ$, and $Z_A$ and $Z_B$ has measure zero.

Further write $Z_A  ⊆ T_A$ and $Z_B ⊆ T_B$, where $T_A$ and $T_B$ are of type $G_δ$ and have measure zero.

Then, $|G_A × G_B| = |G_A||G_B|=|A||B|$ by the case above.

Write $C=G_A × G_B$ and $D=(G_A × G_B) ∖ ((ℝ × T_B) ∪(T_A × ℝ))$.

By the part above, $C$ and $D$ are both measurable, and $|C ∖ D|=0$.

We see $D ⊆ A × B ⊆ C$. So $(A × B) ∖ D ⊆ C ∖ D$, because $|C ∖ D|=0$ then $(A × B) ∖ D$ is measurable and have measure zero.

Therefore, $A × B = D ⊔ ((A × B) ∖ D)$ is measurable and have the same measure as $C$ or $D$, which is equal to $|A||B|$. We get the desired result.

\subsection{Exercise 17} %% khanh -- checked by Hong Duc

Give an example that shows the image of a measurable set under a continuous transformation may not be measurable.

\textbf{Proof}

Given a measure zero set $Z$, if there is a continuous mapping $\phi$ that maps $Z$ into a positive measure set $\phi(Z)$, i.e. $|\phi(Z)| > 0$%
. By Corollary 3.39, $\phi(Z)$ contains a non-measurable set, namely $N$. Then the set $\phi^{-1}(N) \cap Z$ is a measure zero set and has its image being non-measurable.

We will construct a continuous mapping that maps \emph{Cantor set} into a positive measure set.
Let $f$ be the \emph{Cantor-Lebesgue function}, define $g$ on $[0, 1]$ onto $[0, 2]$ by
\[
    g(x) = f(x) + x
\]

$g$ is strictly increasing continuous function because $f$ is a increasing continuous and $x$ is strictly increasing continuous. Therefore, $g$ is a homeomorphism. Let $C$ be the \emph{Cantor set}, $O = [0, 1] \setminus C$ is a union of open intervals, so $O$ is open and $C$ is closed. The image of $[0, 1]$ under $g$ can be written as

$$
    g([0, 1]) = g(O \cup  C) = g(O) \cup g(C) = [0, 2]
$$

$g(O)$ and $g(C)$ are disjoint since $O$ and $C$ are disjoint. $g(O)$ is open and $g(C)$ is closed since $g$ is a homeomorphism. Therefore, the two sets $g(O)$ and $g(C)$ are measurable.

Let $\{ I_k\}$ be the collection of open intervals removed in the construction of \emph{Cantor set}. Since $f$ is constant on each $I_k$, so $g(I_k)$ is a translation of $I_k$. Furthermore, $g$ is bijective, implying the intervals $g(I_k)$ are disjoint. So

\begin{align*}
    |g(O)|  &= \left|\bigcup \{ g(I_k)\}\right| \\
            &= \sum |g(I_k)| &\text{(measure additivity of disjoint sets)}\\
            &= \sum |I_k| &\text{(translation preserves measure)}\\
            &= \left|\bigcup \{ I_k\}\right| &\text{(measure additivity of disjoint sets)}\\
            &= |O| \\
            &= |[0, 1]| - |C|   &\text{($[0, 1], C, O$ is measurable)}\\
            &= 1
\end{align*}

Since, $g(O)$ and $g(C)$ are measurable

$$
    |g(C)| = |[0, 2]| - |g(O)| = 1
$$

By Corollary 3.39, $|g(C)| > 0$ contains a non-measurable set, namely $D$. Then, $g^{-1}(D) \cap C$ is a subset of $C$ that has measure zero.

\subsection{Exercise 23}%Hong Duc
% reviewed by khanh, approved

\typstmathinputenable{\$}
Consider $n ∈ ℤ⁺$, write $Z_n = Z ∩ (-n, n)$. We will show ${x²: x ∈ Z_n}$ has measure zero.

For any $ε>0$, cover $Z_n ∖ {0}$ with a countable union of open intervals, $Z_n = ⋃_k I_k$ (assume $k ∈ ℤ⁺$ from now on), such that $∑_k |I_k| < ε$ (this is possible because $Z$ has measure 0, so is $Z_n$). Without loss of generality, assume each interval $I_k$ is either contained in $(0,n)$ or $(-n, 0)$.

Note that for any open interval $(a, b)$ where $0 ≤ a < b ≤ n$, we have ${x²: x ∈(a, b)}= (a², b²)$, the latter has measure $b²-a² = (b-a)(b+a) ≤ 2n ⋅ (b-a) = 2n ⋅ |(a, b)|$. A similar argument applies to show that for any open interval $I ⊆ (-n, 0)$, then $|{x²: x ∈ I}|≤ 2n ⋅ |I|$.

In conclusion, $|{x²: x ∈ Z_n}| ≤ |{0} ∪ ⋃_k {x²: x ∈ I_k}| ≤ ∑_k |{x²: x ∈ I_k}| ≤ ∑_k (2n ⋅ |I_k|) < 2n ⋅ ε$. Since $ε$ can be arbitrarily small, we conclude $|{x²: x ∈ Z_n}|=0$ as desired.

Using the above, because $Z = ⋃_(n ∈ ℤ⁺) Z_n$, then ${x²: x ∈ Z}= ⋃_(n ∈ ℤ⁺) {x²: x ∈ Z_n}$, each term in the countable union on the right has measure zero, thus the union also has measure zero as desired.
\typstmathinputdisable{\$}

\subsection{Exercise 28} %% khanh
% reviewed by Nigel (looks good)
% reviewed by Hong Duc (approved)

If $T: \R^n \to \R^n$ is a Lipschitz transformation, then there is a constant $c' > 0$ such that $|TI| \leq c'|I|$ for every interval $I$.

\textbf{Proof}

If $c = 0$, the problem is trivial, let $c >0$. Set $c'=(2 c \sqrt{n})^n$, then $c'>0$.

Consider the case of a closed cube $Q$ of edge length $d$, then $|Q| = d^n$ and $\diam Q = d \sqrt{n}$. For all $x, y \in Q$, we have $|Tx - Ty| \leq c|x - y|$. Then
\begin{align*}
    \sup \{|Tx - Ty|: x, y \in Q\}  &\leq \sup \{ c|x - y|: x, y \in Q\} \\
    \diam TQ &\leq c \diam Q
\end{align*}

Therefore, $\diam TQ \leq c \diam Q$, that is, $TQ$ is contained in a closed ball of radius $c \diam Q$, which is contained in a closed cube of edge length $2 c \diam Q = 2cd\sqrt{n}$. So by subset property of outer measure, \[|TQ|_e \leq (2 c d\sqrt{n})^n = (2 c \sqrt{n})^n |Q| = c' |Q|.\]

For any $\epsilon > 0$. By Theorem 3.6, let $O$ be an open set containing $I$ with $|O| \leq |I| + \frac{\epsilon}{c'}$.
By theorem 1.11, $O$ can be written as a countable union of non-overlapping closed cubes $Q_k$, i.e. $O = \bigcup_{k=1}^\infty Q_k$. We have


\begin{align*}
    |TI|_e
        &\leq | TO |_e &\text{(subset property of outer measure)} \\
        &= \left| T\bigcup_{k=1}^\infty Q_k \right|_e\\
        &= \left| \bigcup_{k=1}^\infty T Q_k \right|_e \\
        &\leq \sum_{k=1}^\infty |T Q_k|_e &\text{(sub-additivity of outer measure)} \\
        &\leq \sum_{k=1}^\infty c'|Q_k| &\text{(Lipschitz image of cube)} \\
        &= c' \sum_{k=1}^\infty |Q_k|\\
        &= c' |O| &\text{(non-overlapping closed cubes)} \\
        &\leq c' |I| + \epsilon.
\end{align*}

Sending $\epsilon$ to $0$ gives $|TI|_e \leq c' |I|$.

If $I$ is closed, as $T$ is continuous, $TI$ is compact, hence measurable. We have $|TI| \leq c' |I|$. This implies theorem 3.33.
Otherwise, by theorem 3.33, $TI$ is measurable. We have the same conclusion.



\end{document}
