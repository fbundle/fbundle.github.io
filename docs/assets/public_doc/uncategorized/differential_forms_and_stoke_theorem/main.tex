\documentclass{article}
\usepackage{graphicx} % Required for inserting images

% header

%% natbib
\usepackage{natbib}
\bibliographystyle{plain}

%% comment
\usepackage{comment}

% no automatic indentation
\usepackage{indentfirst}

% manually indent
\usepackage{xargs} % \newcommandx
\usepackage{calc} % calculation
\newcommandx{\tab}[1][1=1]{\hspace{\fpeval{#1 * 10}pt}}
% \newcommand[number of parameters]{output}
% \newcommandx[number of parameters][parameter index = x]{output}
% use parameter index = x to substitute the default argument
% use #1, #2, ... to get the first, second, ... arguments
% \tab for indentation
% \tab{2} for for indentation twice

% note
\newcommandx{\note}[1]{\textit{\textcolor{red}{#1}}}
\newcommand{\todo}{\note{TODO}}
% \note{TODO}

%% math package
\usepackage{amsfonts}
\usepackage{amsmath}
\usepackage{amssymb}
\usepackage{tikz-cd}
\usepackage{mathtools}
\usepackage{amsthm}

%% operator
\DeclareMathOperator{\tr}{tr}
\DeclareMathOperator{\diag}{diag}
\DeclareMathOperator{\sign}{sign}
\DeclareMathOperator{\grad}{grad}
\DeclareMathOperator{\curl}{curl}
\DeclareMathOperator{\Div}{div}
\DeclareMathOperator{\card}{card}
\DeclareMathOperator{\Span}{span}
\DeclareMathOperator{\real}{Re}
\DeclareMathOperator{\imag}{Im}
\DeclareMathOperator{\supp}{supp}
\DeclareMathOperator{\im}{im}
\DeclareMathOperator{\aut}{Aut}
\DeclareMathOperator{\inn}{Inn}
\DeclareMathOperator{\Char}{char}
\DeclareMathOperator{\Sylow}{Syl}
\DeclareMathOperator{\coker}{coker}
\DeclareMathOperator{\inc}{in}
\DeclareMathOperator{\Sd}{Sd}
\DeclareMathOperator{\Hom}{Hom}
\DeclareMathOperator{\interior}{int}
\DeclareMathOperator{\ob}{ob}
\DeclareMathOperator{\Set}{Set}
\DeclareMathOperator{\Top}{Top}
\DeclareMathOperator{\Meas}{Meas}
\DeclareMathOperator{\Grp}{Grp}
\DeclareMathOperator{\Ab}{Ab}
\DeclareMathOperator{\Ch}{Ch}
\DeclareMathOperator{\Fun}{Fun}
\DeclareMathOperator{\Gr}{Gr}
\DeclareMathOperator{\End}{End}
\DeclareMathOperator{\Ad}{Ad}
\DeclareMathOperator{\ad}{ad}
\DeclareMathOperator{\Bil}{Bil}
\DeclareMathOperator{\Skew}{Skew}
\DeclareMathOperator{\Tor}{Tor}
\DeclareMathOperator{\Ho}{Ho}
\DeclareMathOperator{\RMod}{R-Mod}
\DeclareMathOperator{\Ev}{Ev}
\DeclareMathOperator{\Nat}{Nat}
\DeclareMathOperator{\id}{id}
\DeclareMathOperator{\Var}{Var}
\DeclareMathOperator{\Cov}{Cov}
\DeclareMathOperator{\RV}{RV}
\DeclareMathOperator{\rank}{rank}

%% pair delimiter
\DeclarePairedDelimiter{\abs}{\lvert}{\rvert}
\DeclarePairedDelimiter{\inner}{\langle}{\rangle}
\DeclarePairedDelimiter{\tuple}{(}{)}
\DeclarePairedDelimiter{\bracket}{[}{]}
\DeclarePairedDelimiter{\set}{\{}{\}}
\DeclarePairedDelimiter{\norm}{\lVert}{\rVert}

%% theorems
\newtheorem{axiom}{Axiom}
\newtheorem{definition}{Definition}
\newtheorem{theorem}{Theorem}
\newtheorem{proposition}{Proposition}
\newtheorem{corollary}{Corollary}
\newtheorem{lemma}{Lemma}
\newtheorem{remark}{Remark}
\newtheorem{claim}{Claim}
\newtheorem{problem}{Problem}
\newtheorem{assumption}{Assumption}
\newtheorem{example}{Example}
\newtheorem{exercise}{Exercise}

%% empty set
\let\oldemptyset\emptyset
\let\emptyset\varnothing

\newcommand\eps{\epsilon}

% mathcal symbols
\newcommand\Tau{\mathcal{T}}
\newcommand\Ball{\mathcal{B}}
\newcommand\Sphere{\mathcal{S}}
\newcommand\bigO{\mathcal{O}}
\newcommand\Power{\mathcal{P}}
\newcommand\Str{\mathcal{S}}


% mathbb symbols
\usepackage{mathrsfs}
\newcommand\N{\mathbb{N}}
\newcommand\Z{\mathbb{Z}}
\newcommand\Q{\mathbb{Q}}
\newcommand\R{\mathbb{R}}
\newcommand\C{\mathbb{C}}
\newcommand\F{\mathbb{F}}
\newcommand\T{\mathbb{T}}
\newcommand\Exp{\mathbb{E}}

% mathrsfs symbols
\newcommand\Borel{\mathscr{B}}

% algorithm
\usepackage{algorithm}
\usepackage{algpseudocode}

% longproof
\newenvironment{longproof}[1][\proofname]{%
  \begin{proof}[#1]$ $\par\nobreak\ignorespaces
}{%
  \end{proof}
}


% for (i) enumerate
% \begin{enumerate}[label=(\roman*)]
%   \item First item
%   \item Second item
%   \item Third item
% \end{enumerate}
\usepackage{enumitem}

% insert url by \url{}
\usepackage{hyperref}

% margin
\usepackage{geometry}
\geometry{
a4paper,
total={190mm,257mm},
left=10mm,
top=20mm,
}


\title{
    Differential Forms and Stoke's Theorem
}
\author{Khanh Nguyen}
\date{August 2023}

\begin{document}

\maketitle

\emph{this is my notes on Differential Forms and Stoke's Theorem from the book: all the mathematics you missed by Thomas A. Garrity}

\section{Volumes of Parallelepipeds}

\begin{theorem}
    In $R^n$, the volume of the parallelepiped spanned by the columns of matrix $A \in \R^{k \times n}$ is
    $$
        \sqrt{\det (A A^T)}
    $$
\end{theorem}

\section{Differential Forms and the Exterior Derivative}

\subsection{Elementary $k$-forms}

In $\R^n$, let $I = \{i_1, i_2, ..., i_k \} \subseteq \{1, 2, ..., n\}$ be an index sequence. The elementary $k$-forms $dx_I$ is defined as the operator measuring the signed volume of the projection of a a parallelepiped into the subspace formed by $e_I = \{e_{i_1}, e_{i_2}, ..., e_{i_k} \}$. \footnote{We also write $dx_I = dx_{i_1} \wedge dx_{i_2} \wedge ... \wedge dx_{i_k}$}

$$
    dx_I(A) = \det \left[ \pi_{e_I}(A) \right]
$$


\subsection{The Vector Space of $k$-forms}

\begin{definition}[Multilinear map]
    A function $f: V_1 \times V_2 \times ... \times V_n \to  W$ is a multilinear map if it is a linear map w.r.t each variable, i.e. for each $k \in \{1, 2, ..., n\}$,
    \begin{itemize}
        \item $f(v_1, ..., a_k + b_k, ..., v_n) = f(v_1, ..., a_k, ..., v_n) + f(v_1, ..., b_k, ..., v_n)$
        \item $f(v_1, ..., \lambda v_k, ..., v_n) = \lambda f(v_1, ..., v_k, ..., v_n)$
    \end{itemize}
\end{definition}

\begin{definition}[Determinant]
    Determinant of an $n \times n$ matrix $A$ is defined as the unique real-valued multilinear map w.r.t each column of $A$
    $$
        det: \R^{n \times n} \to \R
    $$
    with $det(I) = 1$
\end{definition}

\begin{definition}[$k$-forms]
    \label{def:k_form}
    A $k$-form $\omega$ is a real-valued multilinear map w.r.t each column of a $n \times k$ matrix
    $$
        \omega: \R^{n \times k} \to \R
    $$
\end{definition}

By the properties of determinant, each elementary $k$-form is a $k$-form defined in definition \ref{def:k_form}. Furthermore, 

\begin{theorem}
    In $\R^n$, the set of elementary $k$-forms with increasing indices is precisely the basis of the vector space of $k$-forms over the field $\R$ denoted by $\bigwedge^k (\R^n)$. The dimensional of this vector space is $n \choose k$. 
\end{theorem}

\subsection{Rules for Manipulating $k$-forms}

\begin{definition}
    In the symmetric group $S_m$ \footnote{permutation of $m$ elements}, let $k+l = m$ and $\sigma \in S_m$ be the $(k, l)$-shuffle which has the property that
    $$
        \sigma(1) < \sigma(2) < .. < \sigma(k)
    $$
    and 
    $$
        \sigma(k+1) < \sigma(k_2) < .. < \sigma(k+l)
    $$
    The set of all $(k, l)$-shuffles is denoted by $S(k, l)$
\end{definition}

\begin{definition}[Wedge Product]
    Let $A = (A_1, A_2, ..., A_{k+l})$ be an $n \times (k + l)$ matrix. Let $\tau$ be a $k$-form and $\omega$ be an $l$-form, we define the wedge product
    $$
        (\tau \wedge \omega)(A) = \sum_{\sigma \in S(k, l)} (-1)^{\sign(\sigma)} \tau(A_{\sigma(1)}, ..., A_{\sigma(k)}) \omega(A_{\sigma(k+1)}, ..., A_{\sigma(k+l)}))
    $$
    
\end{definition}

\subsection{Differential $k$-forms and the Exterior Derivative}

\begin{definition}
    In the symmetric group $S_n$, let $I = \{i_1, i_2, ..., i_k \} \in S_n$ be the $(k)$-shuffle which has the property that
    $$
        i_1 < i_2 < .. < i_k
    $$
    The set of all $(k)$-shuffles is denoted by $S(k)$
\end{definition}

\begin{definition}[Differential $k$-forms]
    A differential $k$-form is defined as
    $$
        \omega = \sum_{I \in S(k)} f_I dx_I
    $$
    where each $f_I \in C^1(\R^n, \R)$: a differentiable function
\end{definition}

\begin{definition}[Exterior derivative]
    Given a differential $k$-form $\omega = \sum_{I \in S(k)} f_I dx_I$, the exterior derivaive $d\omega$ is
    $$
        d\omega = \sum_{I \in S(k)} df_I \wedge dx_I
    $$
    where $df_I = \sum_{i \in I} \frac{\partial f_I}{\partial x_i} dx_i$
\end{definition}

\begin{proposition}
    For any differential $k$-form $\omega$, we have
    $$
        d(d\omega) = 0
    $$
\end{proposition}

\section{Differential Forms and Vector Fields}

\begin{definition}[$T_0, T_1, T_2, T_3$]
In $\R^3$ with standard coordinates $x, y, z$

Let $T_0$ be the identity map on the space of $0$-form \footnote{$C^1(R^n, R)$}
$$
    T_0(f) = f
$$

Let $T_1$ be the map from the space of $1$-form into $\R^3$
$$
    T_1(f_1 dx + f_2 dy + f_3 dz) = (f_1, f_2, f_3)
$$

Let $T_2$ be the map from the space of $2$-form into $\R^3$
$$
    T_2(f_1 dx \wedge dy + f_2 dy \wedge dz + f_3 dz \wedge dx) = (f_2, f_3, f_1)
$$

Let $T_3$ be the map from the space of $3$-form \footnote{isomorphic to $C^1(\R^n, \R)$} into $C^1(\R^n, \R)$
$$
    T_3(f dx \wedge dy \wedge dz) = f
$$

\end{definition}

\begin{theorem}
    In $\R^3$, let $\omega_k$ denote a differential $k$-form. Then
    \begin{align*}
        T_1(d\omega_0)  &= \grad(T_0(\omega_0)) \\
        T_2(d\omega_1)  &= \curl(T_1(\omega_1)) \\
        T_3(d\omega_2)  &= \Div(T_2(\omega_2)) \\
    \end{align*}
\end{theorem}

In $\R^2$ with coordinates $x_1, x_2, ..., x_n$. There is a single elementary $n$-form, namely $dx_1 \wedge dx_2 \wedge ... \wedge dx_n$. Define the map from space of $n$-forms into $\R$

$$
    T: \bigwedge^n (\R^n) \to \R
$$

by $T(\alpha dx_1 \wedge dx_2 \wedge ... \wedge dx_n) = \alpha$. For $k$-forms, the dual space of $\bigwedge^k(\R^n)$ is isomorphic to $\bigwedge^{n-k}(\R^n)$. Let $\omega_{n-k}$ be a $n-k$-form, the associated linear map in the dual space of $\bigwedge^k(\R^n)$ is

$$
    T_{\omega_{n-k}}: \bigwedge^{k}(\R^n) \to \R
$$

defined by $T_{\omega_{n-k}} (\omega_k) = T(\omega_{n-k} \wedge \omega_k)$. Moreover, dimension of the dual space equals dimension of the original space, i.e. $\dim \bigwedge^{k}(\R^n)= \dim \bigwedge^{n-k}(\R^n)$

\section{Manifolds}


\end{document}