\chapter{Related Work}

A few approximation techniques for the graph clustering problem that inspired our approach will be discussed in this chapter.

\section{Spectral Graph Clustering}

A classic technique for graph clustering is spectral clustering which is introduced comprehensively in the textbook of Fan Chung \cite{chung1997spectral}. The spectral methods typically using several eigen vectors corresponding to the set of smallest eigenvalues on normalized Laplacian matrix that provide an embedding of nodes satisfying the optimality on the chosen objective \footnote{Ratio cut, Normalized cut, Min-Max cut, etc}.

Recently, A. Saade et al introduced Bethe-Hessian matrix \cite{saade2014spectral} for spectral clustering that is provable to  perform down to the detect-ability threshold under stochastic block model \cite{massoulie2014community} which is related to the \emph{non-backtracking operator} firstly published by Florent Krzakala et al \cite{krzakala2013spectral}. Later, another study from Lorenzo \cite{dall2019revisiting} extended A. Saade et al work for degree corrected stochastic block model.

Later on, which is inspired by the \emph{non-backtracking operator}, Fei Jiang at al \cite{jiang2018spectral} published a novel edge embedding framework called \emph{NOBE} which has a superior clustering performance comparing to many state of the art node embedding algorithms.

However, one disadvantage of spectral clustering is that it suffers from the high computational complexity since the algorithm is based on the estimation of eigen vectors. Nonetheless, it might be more applicable in the future with the work on randomized algorithms in numerical linear algebra \cite{kannan2017randomized} and the advancement in engineering \cite{tulloch_2016}.

To wrap up this section, Marina Meila and Jianbo Shi \cite{meila2001learning} \cite{meila2001random}, Pekka Orponen, Satu Elisa Schaeffer, and Vanesa Avalos Gaytan \cite{orponen2008locally} have demonstrated the relationship between spectral clustering and random walk on graph. Hence, exploiting the short random walks in graph is a very promising approach to explore.


\newpage
\section{Graph Embedding}

Graph embedding falls into four different categories \cite{cai2018comprehensive}: \emph{node embedding}, \emph{edge embedding}, \emph{hybrid embedding} and \emph{whole graph embedding}. Nodes, edges or graph are typically represented by a point in Euclidean space. In this section, we will focus on node embedding and hybrid embedding only.

\subsection{Node Embedding}

In \emph{node embedding}, each node is represented as a vector in a small $d$-dimensional Euclidean space. The common proximity measures are Euclidean norm or inner product.


\subsubsection{Matrix Factorization}

Matrix factorization aims to represent the graph as a matrix of neighbourhood proximity and use matrix factorization techniques to obtain the node embedding.

The most well-known technique is Laplacian Eigenmaps which factorizes the Laplacian matrix and get top-$d$ the eigen vectors as the embedding of nodes. 

The other method we wanted to discuss in this section is \emph{Graph Factorization} \cite{ahmed2013distributed}. The authors adopted stochastic gradient descent to approximate a maximum rank-$d$ matrix of the adjacency with a regularization term that penalizes L2 norm.

\begin{equation}
    f(Y, Z, \lambda) = \frac{1}{2} \sum_{(i, j) \in E} (Y_{i, j} - \langle Z_i, Z_j \rangle)^2 + \frac{\lambda}{2} \sum_i ||Z_i||^2
\end{equation}

Where $Y \in R^{|V| \times |V|}$ is the proximity matrix (adjacency matrix) and $Z \in R^{|V| \times d}$, each row vector of $Z$ is an embedding vector of the corresponding node.

\newpage
\subsubsection{Random-walk-based and SkipGram}

Random-walk-based method treats the network as a collection of node walks and the algorithm operates on this collection.

Bryan Perozz et al adopted \emph{SkipGram} \cite{mikolov2013distributed} from natural language processing and proposed \emph{Deepwalk} \cite{perozzi2014deepwalk}. The authors observed a power-law property in the distribution of nodes in short random walks and the distribution of words in documents. Then, they performed random-walk on graph data then used these short walks to train the SkipGram objective.

\begin{equation}
    \frac{1}{T} \sum_{t=1}^T \sum_{-c \leq i \leq +c, i \neq 0} log(P(w_{t+i} | w_t))
\end{equation}

\begin{equation}
    P(w_c | w_t) = \frac{exp(v_c^{'T} v_t)}{\sum_{w_i \in W} exp(v_i^{'T} v_t)}
    \label{eq:softmax}
\end{equation}

Where $w_i$ denotes the $i$-th word in a sentence, $T$ denotes the length of that sentences, $c$ denotes the context size. $P(w_c | w)$ is the probability of a context word given a word. $v_c^{'}$ and $v_t$ are the corresponding context embedding vector and the word embedding vector.

In summary, \emph{SkipGram} maximizes the log-likelihood of the occurrences of context words given a word and \emph{Deepwalk} replaces  word by node in a binary edge network. Their contribution motivates many subsequent studies on the random walks.

Aditya Grover and June Leskovec (\emph{node2vec}) extended \emph{Deepwalk} by providing two tunable parameters $p$ and $q$ guiding the random walk process that balances between depth-first-search and breath-first-search corresponding to homophily as opposed to structural equivalence. Their random walk process was described as: Consider a walk just traversed from $a$ to $b$, let the next node in the walk be $c$, then the transition probability is as below:

\begin{equation}
     P((..., a, b, c) | p, q, (..., a, b)) \propto \left\{
    \begin{array}{ll}
        \frac{1}{p} \;\;\; \text{if $d_{a, c} = 0$} \\
        1  \;\;\; \text{if $d_{a, c} = 1$} \\
        \frac{1}{q} \;\;\; \text{if $d_{a, c} = 2$}
    \end{array}
    \right.
\end{equation}

Where $d_{a, c}$ is the shortest distance from node $a$ to node $c$. By setting $1/p \to 0$, the random walk becomes first-order non-backtracking which is described in \cite{krzakala2013spectral}. By setting $1/q \to 0$, the random walk tends to traverse to the direct neighbours of node $a$. By setting $1/q \to +\infty$, the random walk tends to traverse as far as possible.

In \emph{LINE} \cite{tang2015line}, Jian Tang et al proposed two optimization objectives called first-order proximity and second-order proximity. First-order proximity models the presence of an edge $(v_i, v_j)$ by $p_1(v_i, v_j)$:

\begin{equation}
    p_1(v_i, v_j) = \frac{1}{1 + exp(-u_i^T u_j)}
\end{equation}

In order to produce a good node representation, they minimized the KL-divergence between the empirical distribution edge weights $w_{i, j}$ and the first-order proximity $p_1(v_i, v_j)$ which resulted to:

\begin{equation}
    O_1 = - \sum_{(i, j) \in E} w_{i, j} p_1(v_i, v_j)
\end{equation}

Second-order proximity models the prestige of a node $\lambda_i$ by $p_2(\cdot | v_i)$. Each node plays two role: \emph{vertex} and \emph{context} to over vertices. They defined the probability of \emph{context} $v_j$ over \emph{vertex} $v_j$ as:
\begin{equation}
    p_2(v_j | v_i) = \frac{exp(u_j^{'T} u_i)}{\sum_{k=1}^{|V|} exp(u_k^{'T} u_i)}
\end{equation}

\begin{equation}
    p_2(\cdot | v_i) = \prod_{v_j \in V} p_2(v_j | v_i)
\end{equation}

Similar to first-order proximity, they minimized the KL-divergence between the empirical distribution prestige of a node $\lambda_i$ and the second proximity $p_2(\cdot | v_i)$. They chose the prestige of a node by its degree so that the objective was reduced to:

\begin{equation}
    O_2 = - \sum_{directed (i, j) \in E} w_{i, j} p_2(v_i | v_j)
\end{equation}

Finally, in order to manage the weighted graph, they proposed an edge sampling scheme where probability is proportional to the edge weight.


It also is worth to mention the two techniques to approximate the full softmax (equation \ref{eq:softmax}) has been used in \emph{SkipGram} model: \emph{hierarchical softmax} \cite{morin2005hierarchical} and \emph{negative sampling} \cite{gutmann2012noise} \cite{mikolov2013distributed}. \emph{Hierarchical softmax} handles the calculation of full softmax by a binary tree that reduces the time complexity of the calculation on the probability of vertex given context from $O(V)$ to $O(logV)$. \emph{Negative sampling} on the other hand, approximate the probability of vertex given context by sampling the negative edges according to a noise distribution $P_n(v)$. The \emph{negative sampling} objective is used to replace all $P(w_c | w)$ in the $SkipGram$ model.

\begin{equation}
    log \sigma (v_c^{'T} v) + \sum_{i=1}^K \mathbf{E}_{v_i \sim P_n(v)}[log \sigma (-v_i^{'T} v)] 
\end{equation}


\newpage
\subsubsection{Neural Network}

The study of neural network on graph has emerged in recent years. Many proposed solution to graph problems are inspired by Convolutional Neural Network (CNN).

In \emph{Graph Convolutional Network} (\emph{GCN}) \cite{kipf2016semi}, each layer weighted aggregates a node neighbourhood embedding vector \footnote{the weight depends on the respective kernel}, then filtered by a linear layer. The propagation rule is described as below:

\begin{equation}
    H^{(l+1)} = \sigma (\widetilde{D}^{-1/2} \widetilde{A} \widetilde{D}^{-1/2} H^{(l)} W^{(l)} + b^{(l)})
\end{equation}

Where $\widetilde{A} = A + I$, $A$ is the adjacency. $\widetilde{D}_{ij} = \sum_j \widetilde{A}_{ij}$. $H^{(l)} \in R^{|V| \times d^{(l)}}$ is the representation at layer $(l)$, $d^{(l)}$ is the dimension at layer $(l)$. $W^{(l)}$ and $b^{(l)}$ are weight and bias terms at layer $(l)$. $\sigma(\cdot)$ is the activation.

Later on, Difan Zou et al \cite{zou2019layer} improved the calculation on \emph{GCN} by making use of selective sampling for each layer. \emph{LADIES}. In node classification task, stochastic gradient descent only use a subset of nodes to train the model. Hence, in order to save the calculation step, they used a sampled kernel matrix which only has the corresponding neighbours that are related to the set of labeled nodes.

In \emph{Gaussian-Induced Convolution}, Jiatao Jiang \cite{jiang2019gaussian} proposed a novel method that utilizes both first-order statistics (mean) and second-order statistics (variance). The two presented types of layer are Convolution: Edge-Induced GMM and Coarsening: vertex-induced GMM. In edge induced GMM, inspired by the work on \emph{Fisher vector} \cite{sanchez2013image}, each vertex embedding is calculated based on the derivative of probability density of the receptive field w.r.t a gaussian mixture. It exploits the local information w.r.t to a node using the concept of receptive field in graph. In vertex induced GMM, vertices are partitioned using EM algorithm on GMM. It reduced the computational complexity by looking to the graph in a more global view as merging the nodes into clusters. It is important to highlight that, in vertex partitioning using GMM, the kernel trick has been used in order to reduce the main calculation of the algorithm.

Despite the success of neural network, neural network on graph is typically suitable for supervised learning tasks. In order to solve the unsupervised tasks where no label is provided, an autoencoder often takes place with careful designed regularization constraints on the hidden embedding which typically requires the expert experience.

\newpage
\subsection{Hybrid Embedding}

Hybrid embedding discussed in this section is the embedding of nodes and clusters together.

The two hybrid embedding techniques is being discussed in this section are ComE  and ComE+  which both are the work of Sandro Cavallar et al.

In \emph{ComE} \cite{cavallari2017learning}, the node embedding (LINE \cite{tang2015line}) and community embedding are jointly optimized together. Community embedding objective used in \emph{ComE} was the likelihood of a gaussian mixture.

\begin{equation}
    O_3 = -\frac{\beta}{K} \sum_{i=1}^{|V|} log \sum_{k=1}^{K} \pi_{ik} \mathcal{N}(\phi_i | \psi_k, \Sigma_k)
    \label{eq:come_comm}
\end{equation}

It is noteworthy that, maximizing community embedding log likelihood of GMM was relaxed using log concavity. The objective at equation \ref{eq:come_comm} was replaced by:

\begin{equation}
    O_3^{'} = -\frac{\beta}{K} \sum_{i=1}^{|V|} \sum_{k=1}^{K} log \pi_{ik} \mathcal{N}(\phi_i | \psi_k, \Sigma_k)
\end{equation}


In \emph{ComE+} \cite{cavallari2019embedding}, they extended the work of \emph{ComE} by employing an infinite number of clusters model \emph{IGM} \cite{rasmussen2000infinite} using stick-breaking process. Then, they adopted the variational inference approach \cite{blei2006variational} to cast the maximum posterior problem of detecting community into a minimization of the KL divergence between variational distribution and the posterior.

Hybrid embedding usually has their advantages on their predefined problem. Hence, they are not applicable to produce a meaningful embedding for the general problem. \cite{grover2016node2vec}

\newpage
\section{Dynamic Graph Clustering}

With regard to Dynamic Graph Clustering, Giulio Rossetti et al \cite{rossetti2018community} published a comprehensive survey on techniques have been used. They classified operations in dynamic graph clustering evolution are as follow:

\begin{itemize}
    \item Birth (New): First appearance of a cluster.
    \item Death: The vanishing of a cluster.
    \item Merge (Join): Two or more clusters merge into one.
    \item Split: One cluster splits into two or more components.
    \item Resurgence: A community vanishes for a short period and comes back.
    \item Grow: A cluster accepts nodes.
    \item Contraction: A cluster rejects nodes.
    \item Continue: A cluster remains unchanged.
\end{itemize}

\newpage
\section{Relation to our work}

These presented methods have inspired our approach in different ways. For example, spectral clustering has settled a very clear intuition to the problem, the concept of receptive field in neural network brought an idea of the selective customers to calculate the probability (will be discussed in Method), the idea of jointly optimizing two objectives in \emph{ComE} has inspired the idea of the greedy clustering algorithm using both information from node embedding and graph structure, the idea of infinite clusters in \emph{ComE+} has inspired us about a random process that does not limit the number of clusters that is more suitable in to dynamic set up. Finally, a concept in network science, preferential attachment where a newly created node is more probable to connect to a node with a higher degree which is very similar to Chinese Restaurant Process where a new customer is more probable to connect to a table with more customers already sit guided us to the distance dependent Chinese Restaurant Process which is the main clustering algorithm in this project.

