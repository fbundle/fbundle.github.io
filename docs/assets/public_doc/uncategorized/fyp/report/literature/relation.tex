These presented methods have inspired our approach in different ways. For example, spectral clustering has settled a very clear intuition to the problem, the concept of receptive field in neural network brought an idea of the selective customers to calculate the probability (will be discussed in Method), the idea of jointly optimizing two objectives in \emph{ComE} has inspired the idea of the greedy clustering algorithm using both information from node embedding and graph structure, the idea of infinite clusters in \emph{ComE+} has inspired us about a random process that does not limit the number of clusters that is more suitable in to dynamic set up. Finally, a concept in network science, preferential attachment where a newly created node is more probable to connect to a node with a higher degree which is very similar to Chinese Restaurant Process where a new customer is more probable to connect to a table with more customers already sit guided us to the distance dependent Chinese Restaurant Process which is the main clustering algorithm in this project.