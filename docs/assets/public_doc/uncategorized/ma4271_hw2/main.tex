\documentclass{article}
\usepackage{graphicx} % Required for inserting images

% header

%% natbib
\usepackage{natbib}
\bibliographystyle{plain}

%% comment
\usepackage{comment}

% no automatic indentation
\usepackage{indentfirst}

% manually indent
\usepackage{xargs} % \newcommandx
\usepackage{calc} % calculation
\newcommandx{\tab}[1][1=1]{\hspace{\fpeval{#1 * 10}pt}}
% \newcommand[number of parameters]{output}
% \newcommandx[number of parameters][parameter index = x]{output}
% use parameter index = x to substitute the default argument
% use #1, #2, ... to get the first, second, ... arguments
% \tab for indentation
% \tab{2} for for indentation twice

% note
\newcommandx{\note}[1]{\textit{\textcolor{red}{#1}}}
\newcommand{\todo}{\note{TODO}}
% \note{TODO}

%% math package
\usepackage{amsfonts}
\usepackage{amsmath}
\usepackage{amssymb}
\usepackage{tikz-cd}
\usepackage{mathtools}
\usepackage{amsthm}

%% operator
\DeclareMathOperator{\tr}{tr}
\DeclareMathOperator{\diag}{diag}
\DeclareMathOperator{\sign}{sign}
\DeclareMathOperator{\grad}{grad}
\DeclareMathOperator{\curl}{curl}
\DeclareMathOperator{\Div}{div}
\DeclareMathOperator{\card}{card}
\DeclareMathOperator{\Span}{span}
\DeclareMathOperator{\real}{Re}
\DeclareMathOperator{\imag}{Im}
\DeclareMathOperator{\supp}{supp}
\DeclareMathOperator{\im}{im}
\DeclareMathOperator{\aut}{Aut}
\DeclareMathOperator{\inn}{Inn}
\DeclareMathOperator{\Char}{char}
\DeclareMathOperator{\Sylow}{Syl}
\DeclareMathOperator{\coker}{coker}
\DeclareMathOperator{\inc}{in}
\DeclareMathOperator{\Sd}{Sd}
\DeclareMathOperator{\Hom}{Hom}
\DeclareMathOperator{\interior}{int}
\DeclareMathOperator{\ob}{ob}
\DeclareMathOperator{\Set}{Set}
\DeclareMathOperator{\Top}{Top}
\DeclareMathOperator{\Meas}{Meas}
\DeclareMathOperator{\Grp}{Grp}
\DeclareMathOperator{\Ab}{Ab}
\DeclareMathOperator{\Ch}{Ch}
\DeclareMathOperator{\Fun}{Fun}
\DeclareMathOperator{\Gr}{Gr}
\DeclareMathOperator{\End}{End}
\DeclareMathOperator{\Ad}{Ad}
\DeclareMathOperator{\ad}{ad}
\DeclareMathOperator{\Bil}{Bil}
\DeclareMathOperator{\Skew}{Skew}
\DeclareMathOperator{\Tor}{Tor}
\DeclareMathOperator{\Ho}{Ho}
\DeclareMathOperator{\RMod}{R-Mod}
\DeclareMathOperator{\Ev}{Ev}
\DeclareMathOperator{\Nat}{Nat}
\DeclareMathOperator{\id}{id}
\DeclareMathOperator{\Var}{Var}
\DeclareMathOperator{\Cov}{Cov}
\DeclareMathOperator{\RV}{RV}
\DeclareMathOperator{\rank}{rank}

%% pair delimiter
\DeclarePairedDelimiter{\abs}{\lvert}{\rvert}
\DeclarePairedDelimiter{\inner}{\langle}{\rangle}
\DeclarePairedDelimiter{\tuple}{(}{)}
\DeclarePairedDelimiter{\bracket}{[}{]}
\DeclarePairedDelimiter{\set}{\{}{\}}
\DeclarePairedDelimiter{\norm}{\lVert}{\rVert}

%% theorems
\newtheorem{axiom}{Axiom}
\newtheorem{definition}{Definition}
\newtheorem{theorem}{Theorem}
\newtheorem{proposition}{Proposition}
\newtheorem{corollary}{Corollary}
\newtheorem{lemma}{Lemma}
\newtheorem{remark}{Remark}
\newtheorem{claim}{Claim}
\newtheorem{problem}{Problem}
\newtheorem{assumption}{Assumption}
\newtheorem{example}{Example}
\newtheorem{exercise}{Exercise}

%% empty set
\let\oldemptyset\emptyset
\let\emptyset\varnothing

\newcommand\eps{\epsilon}

% mathcal symbols
\newcommand\Tau{\mathcal{T}}
\newcommand\Ball{\mathcal{B}}
\newcommand\Sphere{\mathcal{S}}
\newcommand\bigO{\mathcal{O}}
\newcommand\Power{\mathcal{P}}
\newcommand\Str{\mathcal{S}}


% mathbb symbols
\usepackage{mathrsfs}
\newcommand\N{\mathbb{N}}
\newcommand\Z{\mathbb{Z}}
\newcommand\Q{\mathbb{Q}}
\newcommand\R{\mathbb{R}}
\newcommand\C{\mathbb{C}}
\newcommand\F{\mathbb{F}}
\newcommand\T{\mathbb{T}}
\newcommand\Exp{\mathbb{E}}

% mathrsfs symbols
\newcommand\Borel{\mathscr{B}}

% algorithm
\usepackage{algorithm}
\usepackage{algpseudocode}

% longproof
\newenvironment{longproof}[1][\proofname]{%
  \begin{proof}[#1]$ $\par\nobreak\ignorespaces
}{%
  \end{proof}
}


% for (i) enumerate
% \begin{enumerate}[label=(\roman*)]
%   \item First item
%   \item Second item
%   \item Third item
% \end{enumerate}
\usepackage{enumitem}

% insert url by \url{}
\usepackage{hyperref}

% margin
\usepackage{geometry}
\geometry{
a4paper,
total={190mm,257mm},
left=10mm,
top=20mm,
}


\title{
    MA4271 Homework 2
}
\author{Nguyen Ngoc Khanh - A0275047B}
\date{September 2023}

\begin{document}

\maketitle

\section{Problem}

\begin{problem}
    Is the set $A = \{(x, y, z) \in \R^3: x^2 + y^2 \leq 1, z = 0\}$ a regular surface? Is the set $B = \{(x, y, z) \in \R^3: x^2 + y^2 < 1, z = 0\}$ a regular surface?
\end{problem}

($A$)
Put $p = (1, 0, 0)$. For any open neighbourhood $V \subseteq \R^3$ containing $p$, $p$ is a boundary point of $A$ then $p$ is also a boundary point of $V \cap A$. Therefore, $V \cap A$ containing a boundary point hence, it is not homeomorphic to any open set in $\R^2$. Therefore, $A$ is not a regular surface.

($B$)
Let $D = \{(u, v): u^2 + v^2 < 1\} \subseteq \R^2$ be the unit disk in $\R^2$. There is a single parameterization $f: D \to B$ of $B$ as defined by 
\[
    f(u, v) = (u, v, 0)
\]
$f$ is a homeomorphism, smooth, and its differential is one-to-one.

\[
    df = 
    \begin{bmatrix}
    1 & 0\\
    0 & 1\\
    0 & 0
    \end{bmatrix}
\]

Therefore, $B$ is a regular surface.

\begin{problem}
    Let two points $p(t)$ and $q(t)$ move with the same speed, $p$ starting from $(0, 0, 0)$ and moving along $z$ axis and $q$ starting at $(a, 0, 0), a \neq 0$ and moving parallel to the $y$ axis. Show that the line joining $p(t)$ and $q(t)$ describes a set in $\R^3$ given by
    \[
        y(x-a) + zx = 0
    \]
    Moreover, is this a regular surface?
\end{problem}

Write $p(t)$ and $q(t)$ as follows

\begin{align*}
    p(t) &= (0, 0, t) \\
    q(t) &= (a, t, 0)
\end{align*}

Any point $r = (x, y, z)$ on the line containing $p(t)$ and $q(t)$ has the form

\begin{align*}
    r(\alpha, t) = (x, y, z)
    &= p(t) + \alpha(q(t) - p(t)) \\
    &= (0, 0, t) + \alpha ((a, t, 0) - (0, 0, t)) \\
    &= (\alpha a, \alpha t, (1 - \alpha) t)
\end{align*}

For each $(\alpha, t) \in \R^2$, $r(\alpha, t) = (x, y, z)$ satisfies $y(x-a) + zx = 0$. On the other hand, for each $(x, y, z)$ satisfies $y(x-a) + zx = 0$, there is a unique $(\alpha, t)$ such that $r(\alpha, t) = (x, y, z)$. Therefore, the set $S = \{(x, y, z) \in \R^3: y(x-a) + zx\}$ describes the line joining $p(t)$ and $q(t)$.

The parameterization $(\alpha, t) \mapsto (x, y, z)$ is a homeomorphism, smooth and the differential is

\[
    df = 
    \begin{bmatrix}
    a & 0\\
    t & \alpha\\
    -t & 1-\alpha
    \end{bmatrix}
\]

This differential is one-to-one everywhere since $\alpha$ and $1 - \alpha$ cannot be zero at the same time (span of row space is $2$). Therefore, $y(x-a) + zx = 0$ is a regular surface. \footnote{there is another argument for regular surface using regular value}

\begin{problem}
    Let $S^2$ and $H$ be defined as following
    \begin{align*}
        S^2 &= \{(x, y, z) \in \R^3: x^2 + y^2 + z^2 = 1 \} \\
        H   &= \{(x, y, z) \in \R^3: x^2 + y^2 - z^2 = 1\}
    \end{align*}

    Denote by $N = (0, 0, 1)$ and $S = (0, 0, -1)$ the north and south poles of $S^2$, respectively. Let $F: S^2 \setminus \{N, S \} \to H$ be defined as following: for each $p \in S^2 \setminus \{N, S \}$, let the perpendicular from $p$ to $z$ axis meet $Oz$ at $q$. Consider the half line $l$ starting at $q$ and containing $p$. Then $F(p) = l \cap H$.
    Prove that $F$ is smooth
\end{problem}

Since both $S^2$ and $H$ are surfaces of revolution (invariant under rotation) on the $z$ axis. With an appropriate change of parameters, let $p = (0, y, z), y > 0$, then $F(p) = (0, \sqrt{1+z^2}, z)$. We will construct a parameterization for each $p$ and $F(p)$ \footnote{Informally, we will take the intersection between $l$ and the plane $y = 1$ }

Let $f_1: \Ball_{\delta} {(0, z)} \subseteq \R^2 \to S^2$ be a parameterization on a neighbourhood of $p$ with $\delta$ small enough such that $1 - v^2 > 0$

\begin{align*}
    f_1(u, v) &= \left( u a(u, v), a(u, v), v \right) \\
    df_1 &= \begin{bmatrix}
    a + u \frac{\partial a}{\partial u} &  u \frac{\partial a}{\partial v}\\
    \frac{\partial a}{\partial u} & \frac{\partial a}{\partial v} \\
    0 & 1
    \end{bmatrix}
\end{align*}

where $a(u, v) = \sqrt{\frac{1-v^2}{u^2+1}} > 0$, $f_1$ is a homeomorphism, smooth, and its differential is one-to-one ($a > 0$ so the first row and the last row of $df_1$ is linearly independent)

Let $f_2: \Ball_{\delta} {(0, z)} \subseteq \R^2 \to H$ be a parameterization on a neighbourhood of $F(p)$.

\begin{align*}
    f_2(u, v) &= \left( u b(u, v), b(u, v), v \right) \\
    df_2 &= \begin{bmatrix}
    b + u \frac{\partial b}{\partial u} &  u \frac{\partial b}{\partial v}\\
    \frac{\partial b}{\partial u} & \frac{\partial b}{\partial v} \\
    0 & 1
    \end{bmatrix}
\end{align*}

where $b(u, v) = \sqrt{\frac{1+v^2}{u^2+1}} > 0$, $f_2$ is a homeomorphism, smooth, and its differential is one-to-one ($b > 0$ so the first row and the last row of $df_2$ is linearly independent)

Under these two parameterizations, $f_2^{-1} \circ F \circ f_1$ is the identity map which is smooth. Hence, $F$ is smooth.

\begin{problem}
    Let $a \neq 0$, $b \neq 0$, and $c \neq 0$. Show that each of the equations
    \begin{align*}
        x^2 + y^2 + z^2 &= ax \\
        x^2 + y^2 + z^2 &= by \\
        x^2 + y^2 + z^2 &= cz
    \end{align*}
    define a regular surface and that they all intersect orthogonally.
\end{problem}

Let $f(x, y, z) = x^2 + y^2 + z^2 - ax$, $0$ is a regular value since $f_x(0) = a \neq 0$. Therefore, $x^2 + y^2 + z^2 = ax$ defines a regular surface. Similar proofs for the other two cases.

\begin{lemma}
    \label{lemma:gradient}
    A regular surface defined by $f(x, y, z) = 0$ has its unit normal vector being the normalized gradient.
\end{lemma}

Consider $f(x, y, z) = x^2 + y^2 + z^2 - ax$ and $g(x, y, z) = x^2 + y^2 + z^2 - by$. The gradient of each function is

\begin{align*}
    Df &= (2x - a, 2y , 2z) \\
    Dg &= (2x, 2y - b, 2z)
\end{align*}

Then 

\begin{align*}
    Df \cdot Dg
    &= (4x^2 - 2ax) + (4y^2 - 2yb) + 4z^2 \\
    &= 2 [(x^2 + y^2 + z^2 - ax) + (x^2 + y^2 + z^2 - by)]
\end{align*}

Let $p = (x, y, z)$ on the intersection of two regular surfaces $f$ and $g$ (the intersection is non-empty since it contains $(0, 0, 0)$ ). Then

\[
    (Df \cdot Dg) |_p = 0
\]

Hence, the two normal vectors are orthogonal. Similar proof for the other two cases.

\begin{problem}
    Show that the area $A$ of a bounded region of the surface $z = f(x, y)$ is
    \[
        A = \int\int_{Q} \sqrt{1 + f_x^2 + f_y^2} dxdy
    \]
    where $Q$ is the normal projection of $P$ onto the $xy$ plane
\end{problem}

$P$ can be parameterized by 

\[
    x(u, v) = (u, v, f(u, v))
\]

where $(u, v) \in Q$. The area element is

\[
    ||x_u \times x_v|| = ||(1, 0, f_u) \times (0, 1, f_v)|| = ||(-f_u, -f_v, 1)|| = \sqrt{1 + f_u^2 + f_v^2}
\]

Hence, area is

\[
    A = \int_Q ||x_u \times x_v|| = \int_Q \sqrt{1 + f_u^2 + f_v^2}
\]

\section{Appendix}

\textbf{Proof of Lemma \ref{lemma:gradient}}

At $p \in S$ where $S$ is a regular surface defined by $f(x, y, z) = 0$. There exists a neighbourhood $V$ of $p$ where $V$ is a graph of a smooth function. Let $V$ be defined by $h: U \subseteq \R^2 \to \R^3$ as $h(x, y) = (x, y, g(x, y))$ on an open set $U \subseteq \R^2$. Consider the function $f \circ h: U \to \R$ where $(f \circ h)(x, y) = 0$ for all $(x, y) \in U$

\begin{align*}
    0
    &= D(f \circ h) \\
    &= (Df) \circ (Dh) \\
    &= (f_x, f_y, f_z) \begin{bmatrix}
    1 & 0\\
    0 & 1\\
    g_x & g_y
    \end{bmatrix}
\end{align*}

Therefore, $g_x = -f_x / f_z$ and $g_y = - f_y / f_z$. On the other hand, we can construct the (scaled) normal vector at $p$ by

\begin{align*}
    \alpha n
    &= h_x \times h_y \\
    &= (1, 0, g_x) \times (0, 1, g_y) \\
    &= (-g_x, -g_y, 1) \\
    &= (f_x, f_y, f_z) / f_z \\
    &= Df / f_z
\end{align*}









\end{document}