\chapter{INTEGRAL DEPENDENCE}

\section{INTEGRAL DEPENDENCE}

\begin{remark}[ring extension]
	If ring map $A \to B$ is injective, we usually write $A \subset B$ or $A \hookrightarrow B$. The ring map $A \to B$ is called a ring extension.
\end{remark}

\begin{definition}[integral]
	Given a ring extension $A \hookrightarrow B$, an element $x \in B$ is integral over $A$ if it satisfies a monic polynomial with coefficients in $A$, that is
	$$
		x^n + a_1 x^{n-1} + ... + a_n = 0
	$$
	
	for some $a_1, a_2, ..., a_n \in A$
\end{definition}

\begin{remark}
	Some examples of integral: $\Z \hookrightarrow \Q$, the integral elements over $\Z$ is $\Z$
\end{remark}

\begin{proposition}
	Given a ring extension $A \hookrightarrow B$, let $x \in B$, the following are equivalent:
	\begin{enumerate}
		\item $x$ is integral over $A$
		
		\item the ring $A[x] \subseteq B$ is finitely generated $A$-module
		
		\item $A[x]$ is contained in a subring $C$ of $B$ such that $C$ is also a finitely generated $A$-module
		
		\item there exists a faithful $A[x]$-module $M$ which is finitely generated as an $A$-module
		An $R$-module $M$ is faithful if and only if $\ann_R(M) = \set{r \in R: rM = 0} = 0$ if and only if $R \to \Hom_R(M, M)$ is injective.
	\end{enumerate}
\end{proposition}

\begin{proof}
	($1 \implies 2$) If $x$ is integral over $A$, then 
	$$
		x^n = - (a_1 x^{n-1} + ... + a_n)
	$$
	
	for some $a_1, a_2, ..., a_n \in A$. That is $x^n$ can be written as a polynomial of degree $\leq n-1$, hence $A[x]$ is generated by $1, x, x^2, ..., x^{n-1}$ as an $A$-module.
	
	($2 \implies 3$) take $C = A[x]$
	
	($3 \implies 4$) take $M = C$ which is a faithful $A[x]$-module since $yC = 0 \implies y1 = 0$
	
	($4 \implies 3$) Consider the $A[x]$-module endomorphism
	\begin{align*}
		\phi: M &\to M \\
					m &\mapsto xm
	\end{align*}
	
	By Nakayama lemma useless version for ideal $A$ in $A[x]$, then there is an equation in $\Hom_{A[x]}(M, M)$
	$$
		\phi^n + a_1 \phi_{n-1} + ... + a_n = 0
	$$
	for some $a_1, a_2, ..., a_n \in A$. Since $M$ is faithful, the map $A[x] \to \Hom_{A[x]}(M, M)$ is injective. Taking the preimage of $\phi^n + a_1 \phi_{n-1} + ... + a_n$ under $A[x] \to \Hom_{A[x]}(M, M)$ is
	$$
		x^n + a_1 x^{n-1} + ... + a_n = 0
	$$
\end{proof}

\begin{corollary}
	Given a ring extension $A \hookrightarrow B$ and $x_1, x_2, ..., x_n \in B$ are integral over $A$, then $A[x_1, x_2, ..., x_n] \subseteq B$ is a finitely generated $A$-module
\end{corollary}

\begin{proof}
	Prove by induction. Base case $n=1$ is from the previous proposition. Suppose the statement is true for $n - 1$, since $x_2, ..., x_n$ are integral over $A$, they are also integral over $A[x_1]$. Moreover, $A[x_1] \hookrightarrow B$ is also a ring extension, hence $A[x_1][x_2, ..., x_n] \subseteq B$ is a finitely generated $A[x_1]$-module, that is also a finitely generated $A$-module.
\end{proof}

\begin{definition}[integral closure, integrally closed, integral ring extension]
	Given a ring extension $A \hookrightarrow B$, the subset $C \subseteq B$ of integral elements over $A$ is a subring of $B$.
	
	\begin{enumerate}
		\item $C$ is called the integral closure of $A$ in $B$, denoted by $A^{icl \subset B}$
		
		\item if $C = B$, $A \hookrightarrow B$ is called integral ring extension
		
		\item if $C = A$, $A$ is called integrally closed in $B$
	\end{enumerate}
\end{definition}

\begin{proof}
	If $x, y$ are integral over $A$, then $C = A[x, y]$ is a finitely generated $A$-module and $x \pm y$ and $xy$ are elements of $C$, hence $A[x \pm y], A[xy] \in C$. By ($3 \implies 1$), $x \pm y$ and $xy$ are integral over $A$
\end{proof}

\begin{remark}
	Some example of integral closure
	\begin{enumerate}
		\item $\Z \hookrightarrow \Q$, then $\Z^{icl \subset \Q} = \Z$
		\item $\Z \hookrightarrow \Q[\sqrt{5}]$, then $\Z^{icl \subset \Q[\sqrt{5}]} = \Z\bracket*{\frac{1 + \sqrt{5}}{2}}$
	\end{enumerate}
\end{remark}

\begin{proposition}[transitivity of integral dependence]
	If $A \hookrightarrow B$ and $B \hookrightarrow C$ are integral ring extensions, then $A \hookrightarrow C$ is an integral ring extension
\end{proposition}

\begin{proof}
	Let $x \in C$, since $B \to C$ is integral ring extension, then
	$$
		x^n + b_1 x^{n-1} + ... + b_n = 0
	$$
	
	for some $b_1, ..., b_n \in B$. Since $A \to B$ is integral ring extension, then the subring $B' = A[b_1, ..., b_n]$ of $B$ is a finitely generated $A$-module. Note that, $x$ is integral with respect to the ring extension $B' \hookrightarrow B$, hence $B'[x]$ is a finitely generated $B'$-module. $B'$ is a finitely generated $A$-module, hence $B'[x]$ is a finitely generated $A$-module. Thus
	$$
		A[x] \subseteq B'[x] \subseteq C
	$$
	
	Hence, $x$ is integral with respect to the ring extension $A \to C$
\end{proof}

\begin{remark}
	Given ring extension $A \hookrightarrow B$, if $x$ is integral over $A$ in $B$, then $A[x]$ is a subring of the integral closure of $A$ in $B$ and $A \hookrightarrow A[x]$ is an integral ring extension. In other words, integral closure of $A$ in $B$ is the union of subrings $A[x]$
\end{remark}

\begin{proof}
	For any $y \in A[x]$, $A[y]$ is contained in $A[x]$ and $A[x]$ is a finitely generated $A$-module, by $3$, $y$ is integral over $A$ in $A[x]$
\end{proof}

\begin{corollary}[integral closure is idempotent]
	Let $A \hookrightarrow B$ be a ring extension and $C$ be the integral closure of $A$ in $B$, then $C$ is integrally closed in $B$.
\end{corollary}

\begin{proof}
	Let $x \in B$ integral over $C$, then $C \to C[x]$ is also a integral ring extension. Hence $A \to C[x]$ is an integral ring extension. Hence, $x \in C[x]$ is integral over $A$, so $x \in C$.
\end{proof}

\begin{proposition}[integral dependence under quotient and localization]
	Let $A \hookrightarrow B$ be an integral ring extension
	\begin{enumerate}
		\item Let $\mf{b} \subseteq B$ be an ideal, let $\mf{a} = \mf{b}^c = \mf{b} \cap A$ be the contraction of $\mf{b}$
		$$
			A / \mf{a} \hookrightarrow B / \mf{b}
		$$
		
		is also an integral ring extension.
		
		\item Let $S \subseteq A$ be an multiplicatively closed subset, then 
		$$
			S^{-1} A \hookrightarrow S^{-1} B
		$$
		
		is also an integral ring extension.
	\end{enumerate}
\end{proposition}
\begin{longproof}
	(1) Let $\bar{b} \in B / \mf{b}$, lift $b \in B$ satisfies
	$$
		b^n + a_1 b^{n-1} + ... + a_n = 0
	$$
	
	for some $a_1, ..., a_n \in A$, mod $\mf{b}$ gives
	$$
		\bar{b}^n + \bar{a}_1 \bar{b}^{n-1} + ... + \bar{a}_n = 0
	$$
	
	(2) Let $x / s \in S^{-1} B$, for $x \in B$ and $s \in S$, then
	$$
		x^n + a_1 x^{n-1} + ... + a_n = 0
	$$
	
	Then,
	$$
		\frac{x^n}{s^n} + \frac{a_1}{s} \frac{x^{n-1}}{s^{n-1}} + \frac{a_2}{s^2} \frac{x^{n-2}}{s^{n-2}} + ... + \frac{a_n}{s^n} = 0
	$$
	
	in $S^{-1} B$. Hence, $x / s$ is integral in $S^{-1} A$
\end{longproof}

\section{THE LYING-OVER THEOREM \\ THE GOING-UP THEOREM}

\begin{proposition}
	Let $A \hookrightarrow B$ be integral ring extension of domains, then $A$ is a field if and only if $B$ is a field. In this case, $A \hookrightarrow B$ is an algebraic extension.
\end{proposition}

\begin{longproof}
	($\implies$) If $A$ is a field, let $x \in B$ be nonzero, let
	$$
		x^n + a_1 x^{n-1} + ... + a_{n-1} x + a_n = 0
	$$
	for some $a_1, ..., a_n \in A$ be the polynomial of smallest degree that $x$ satisfies. Since $B$ is a domain, if $a_n = 0$, then $(x^{n-1} + a_1 x^{n-2} + ... + a_{n-1}) x = 0$, hence $x^{n-1} + a_1 x^{n-2} + ... + a_{n-1} = 0$ contradicts the minimality of degree. Hence, $a_n \neq 0$. Then
	$$
		y = - a_n^{-1} (x^{n-1} + a_1 x^{n-2} + ... + a_{n-1}) \in B
	$$
	
	is the inverse of $x$
	
	($\impliedby$) If $B$ is a field, let $x \in A$ be nonzero, $x^{-1} \in B$ is integral over $A$, then
	$$
		x^{-n} + a_1 x^{-n+1} + ... + a_n = 0
	$$
	
	for some $a_1, ..., a_n \in A$. Hence
	$$
		x^{-1} = - (a_1 + a_2 x + ... + a_n x^{n-1}) \in A
	$$
\end{longproof}

\begin{corollary}
	Let $A \hookrightarrow B$ be integral ring extension, $\mf{q}$ be a prime ideal of $B$ and $\mf{p} = \mf{q}^c = \mf{q} \cap A$ be the contraction of $\mf{q}$. Then, $\mf{q}$ is maximal if and only if $\mf{p}$ is maximal.
\end{corollary}

\begin{proof}
	$A / \mf{p} \hookrightarrow B / \mf{q}$ is a integral ring extension of domains.
	$$
		\mf{q} \text{ is maximal } \iff B / \mf{q} \text{ is a field } \iff A / \mf{p} \text{ is a field } \iff \mf{p} \text{ is maximal}
	$$
\end{proof}

\begin{proposition}[lying over theorem: part 1]
	Let $A \hookrightarrow B$ be a integral ring extension, given $\mf{q}_1 \subseteq \mf{q}_2$ prime ideals of $B$ such that $\mf{q}_1 \cap A = \mf{q}_2 \cap A = \mf{p}$, then $\mf{q}_1 = \mf{q}_2$
\end{proposition}

\begin{proof}
	The integral ring extension $A \hookrightarrow B$ induces another integral ring extension $A_\mf{p} \hookrightarrow B_\mf{p}$
	
	\begin{center}
		\begin{tikzcd}
			A \arrow[r, hook] \arrow[d] & B \arrow[d] \\
			A_\mf{p} \arrow[r, hook]    & B_\mf{p}   
		\end{tikzcd}
	\end{center}
	Let $\mf{m} = \mf{p} A_\mf{p}$, $\mf{n}_1 = \mf{q}_1 B_\mf{p}$, and $\mf{n}_2 = \mf{q}_2 B_\mf{p}$ Then, $\mf{n}_1 \cap A_\mf{p} = \mf{n}_2 \cap A_\mf{p} = \mf{m}$. Since $A_\mf{p} \to B_\mf{p}$ is an integral ring extension, $\mf{n}_1, \mf{n}_2$ are maximal ideals. Since $\mf{n}_1 \subseteq \mf{n}_2$, $\mf{n}_1 = \mf{n}_2$. Hence, $\mf{q}_1 = \mf{q}_2$
\end{proof}

\begin{theorem}[lying over theorem: main statement]
	Let $A \hookrightarrow B$ be an integral ring extension, $\mf{p} \subseteq A$ be a prime ideal, then there exists prime ideal $\mf{q} \subseteq B$ so that $\mf{q} \cap A = \mf{p}$. In other words, the induced function
	$$
		\Spec B \to \Spec A
	$$
	
	is surjective
\end{theorem}

\begin{proof}
	Given any prime ideal $\mf{p} \subseteq A$, the integral ring extension $A \hookrightarrow B$ induces another integral ring extension $A_\mf{p} \hookrightarrow B_\mf{p}$
	\begin{center}
		\begin{tikzcd}
			A \arrow[r, hook] \arrow[d] & B \arrow[d] \\
			A_\mf{p} \arrow[r, hook]    & B_\mf{p}   
		\end{tikzcd}
	\end{center}
	
	Let $\mf{n}$ be a maximal ideal in $B_\mf{p}$, then $\mf{n} \cap A_\mf{p}$ is the unique maximal ideal in the local ring $A_\mf{p}$, hence $\mf{n} \cap A_\mf{p} = \mf{p} A_\mf{p}$. Let $\mf{q} = \mf{n} \cap B$, $\mf{q}$ is prime since it is a contraction of prime ideal. Moreover, $\mf{q} \cap A = \mf{p}$ since $(\mf{n} \cap B) \cap A = (\mf{n} \cap A_\mf{p}) \cap A$, then the map $\Spec B \to \Spec A$ is surjective.	
\end{proof}

\begin{theorem}[going-up theorem]
	Let $A \hookrightarrow B$ be an integral ring extension. Let $\mf{p}_\bullet$ be a chain of prime ideals in $A$ and $\mf{q}_\bullet$ be a chain of prime ideals in $B$
	\begin{center}
		\begin{tikzcd}
			\Spec B \arrow[d, two heads] & \mf{q}_1 \arrow[r, hook] \arrow[d] & \mf{p}_2 \arrow[r, hook] \arrow[d] & ... \arrow[r, hook] & \mf{p}_m \arrow[d]       &                     &          \\
			\Spec A           & \mf{p}_1 \arrow[r, hook]           & \mf{p}_2 \arrow[r, hook]           & ... \arrow[r, hook] & \mf{p}_m \arrow[r, hook] & ... \arrow[r, hook] & \mf{p}_n
		\end{tikzcd}
	\end{center}
	
	so that $\mf{q}_i \cap A = \mf{p}$ for all $i=1, 2, ..., m$. Then $\mf{q}_\bullet$ is extended the the chain
	\begin{center}
		\begin{tikzcd}
			\Spec B \arrow[d, two heads] & \mf{q}_1 \arrow[r, hook] \arrow[d] & \mf{p}_2 \arrow[r, hook] \arrow[d] & ... \arrow[r, hook] & \mf{p}_m \arrow[d] \arrow[r, hook] & ... \arrow[r, hook] & \mf{q}_n \arrow[d] \\
			\Spec A           & \mf{p}_1 \arrow[r, hook]           & \mf{p}_2 \arrow[r, hook]           & ... \arrow[r, hook] & \mf{p}_m \arrow[r, hook]           & ... \arrow[r, hook] & \mf{p}_n          
		\end{tikzcd}
	\end{center}
	so that $q_i \cap A = \mf{p}_i$ for all $i=1, ..., n$
\end{theorem}

\begin{proof}
	The proof of the general case can be reduced to the case when $m = 1$, $n = 2$.
	\begin{center}
		\begin{tikzcd}
			A \arrow[r, hook] \arrow[d]  & B \arrow[d]  \\
			A / \mf{p}_1 \arrow[r, hook] & B / \mf{q}_1
		\end{tikzcd}
	\end{center}
	Let $\bar{A} = A / \mf{p}_1$ and $\bar{B} = B / \mf{q}_1$, because $\mf{p}_1 = \mf{q}_1 \cap A$, then $\bar{A} \hookrightarrow \bar{B}$ is an integral ring extension extension. Let $\bar{\mf{p}}_2 = \mf{p}_2 / \mf{p}_1 \subseteq \bar{A}$, then $\bar{\mf{p}_2}$ is a prime ideal in $\bar{A}$. $\Spec \bar{B} \twoheadrightarrow \Spec \bar{A}$ is surjective, there exists a prime ideal $\bar{\mf{q}} \subseteq \bar{B}$ so that $\bar{\mf{q}} \cap \bar{A} = \bar{\mf{p}}_2$. Take $\mf{q} = \bar{\mf{q}} \cap B$
\end{proof}

\begin{corollary}
	If $A \hookrightarrow B$ is an integral ring extension, then $\dim A \leq \dim B$
\end{corollary}

\begin{proof}
	omitted
\end{proof}

\section{INTEGRALLY CLOSED DOMAIN \\ THE GOING-DOWN THEOREM}

\begin{proposition}
	Let $A \hookrightarrow B$ be a ring extension, let $C$ be the integral closure of $A$ in $B$, let $S \subseteq A$ be a multiplicative subset of $A$, then $S^{-1} C$ is the integral closure of $S^{-1} A$ in $S^{-1} B$
\end{proposition}

\begin{proof}
	$A \hookrightarrow C$ is integral ring extension, then $S^{-1} A \hookrightarrow S^{-1} C$ is also an integral ring extension. For any $b / s \in S^{-1} B$ being integral over $S^{-1} A$, that is
	$$
		\frac{b^n}{s^n} + \frac{a_1}{s_1} \frac{b^{n-1}}{s^{n-1}} + ... + \frac{a_n}{s_n} = 0
	$$
	for some $a_1, ..., a_n \in A$ and $s_1, ..., s_n \in S$. Let $t = s_1 ... s_n$, multiply both sides by $s^n t^n$, we have
	$$
		(bt)^n + \frac{a_1 s t}{s_1} (bt)^{n-1} + ... + \frac{a_n s^n t^n}{s_n} = 0
	$$
	
	Hence, $bt$ is integral over $A$, that is $bt \in C$. Hence, $b / s = (bt) / (st) \in S^{-1} C$
\end{proof}

\begin{definition}[total field of fractions]
	Let $A$ be a domain, then
	$$
		\Frac(A) = (A - \set{0})^{-1} A
	$$
	is a field and it is called the total field of fractions of $A$
\end{definition}

\begin{definition}[integrally closed domain]
	A domain $A$ is called integrally closed if it is integrally closed in $\Frac(A)$
\end{definition}

\begin{theorem}[being integrally closed is a local property]
	Let $A$ be an domain, the following are equivalent
	\begin{enumerate}
		\item $A$ is integrally closed
		\item $A_\mf{p}$ is integrally closed for all prime ideal $\mf{p}$ in $A$
		\item $A_\mf{m}$ is integrally closed for all maxmal ideal $\mf{m}$ in $A$
	\end{enumerate}
\end{theorem}

\begin{proof}
	Let $K = \Frac(A)$ and $C$ is the integral closure of $A$ in $K$ with $f: A \hookrightarrow C$ is an integral ring extension. Then
	
	$A$ is integrally closed in $C$ if and only if $f$ is surjective if and only if $f_\mf{p}$ is surjective for all prime ideal $\mf{p}$. $f_\mf{p}$ is surjective for all prime ideal $\mf{p}$ if and only if $A_\mf{p}$ is integrally closed for all prime ideal $\mf{p}$. $f_\mf{p}$ is surjective for all prime ideal $\mf{p}$ if and only if $f_\mf{m}$ is surjective for all maximal ideal $\mf{m}$ if and only if $A_\mf{m}$ is integrally closed for all maxmal ideal $\mf{m}$
\end{proof}

\begin{definition}[normal domain]
	A domain $A$ is normal if $A_\mf{p}$ is a integrally closed for every prime ideal $\mf{p}$
\end{definition}

\begin{lemma}
	Let $C$ be the integral closure of $A$ in $B$ and $\mf{a}$ be an ideal in $A$, then the integral closure of $\mf{a}$ in $B$ is $\sqrt{\mf{a} C}$
\end{lemma}

\begin{proposition}
	Let $A \subseteq B$ be domains, $A$ be integrally closed and $x \in B$ be integral over an ideal $\mf{a}$ in $A$, then $x$ is algebraic over $K = \Frac(A)$ and if its minimal polynomial over $K$ is
	$$
		t^n + a_1 t^{n-1} + ... + a_n
	$$
	then $a_1, ..., a_n$ lie in $\sqrt{\mf{a}}$
	\note{wtf is this?}
\end{proposition}

\begin{theorem}[going-down theorem]
	Let $A \hookrightarrow B$ be integral ring extension of domains, $A$ is integrally closed in $K = \Frac(A)$, let $\mf{p}_\bullet$ be a chain of prime ideals in $A$ and $\mf{q}_\bullet$ be a chain of prime ideals in $B$
	\begin{center}
		\begin{tikzcd}
			\Spec B \arrow[d, two heads] & \mf{q}_1 \arrow[d] & \mf{p}_2 \arrow[l, hook] \arrow[d] & ... \arrow[l, hook] & \mf{p}_m \arrow[l, hook] \arrow[d] &                     &                          \\
			\Spec A                      & \mf{p}_1           & \mf{p}_2 \arrow[l, hook]           & ... \arrow[l, hook] & \mf{p}_m \arrow[l, hook]           & ... \arrow[l, hook] & \mf{p}_n \arrow[l, hook]
		\end{tikzcd}
	\end{center}
	so that $q_i \cap A = \mf{p}_i$ for all $i=1, ..., m$. Then $\mf{q}_\bullet$ is extended to the chain
	\begin{center}
		\begin{tikzcd}
			\Spec B \arrow[d, two heads] & \mf{q}_1 \arrow[d] & \mf{p}_2 \arrow[l, hook] \arrow[d] & ... \arrow[l, hook] & \mf{p}_m \arrow[l, hook] \arrow[d] & ... \arrow[l, hook] & \mf{q}_n \arrow[l, hook] \arrow[d] \\
			\Spec A                      & \mf{p}_1           & \mf{p}_2 \arrow[l, hook]           & ... \arrow[l, hook] & \mf{p}_m \arrow[l, hook]           & ... \arrow[l, hook] & \mf{p}_n \arrow[l, hook]          
		\end{tikzcd}
	\end{center}
	so that $q_i \cap A = \mf{p}_i$ for all $i=1, ..., n$
\end{theorem}