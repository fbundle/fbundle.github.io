 \documentclass{article}
\usepackage{graphicx} % Required for inserting images
% header

%% natbib
\usepackage{natbib}
\bibliographystyle{plain}

%% comment
\usepackage{comment}

% no automatic indentation
\usepackage{indentfirst}

% manually indent
\usepackage{xargs} % \newcommandx
\usepackage{calc} % calculation
\newcommandx{\tab}[1][1=1]{\hspace{\fpeval{#1 * 10}pt}}
% \newcommand[number of parameters]{output}
% \newcommandx[number of parameters][parameter index = x]{output}
% use parameter index = x to substitute the default argument
% use #1, #2, ... to get the first, second, ... arguments
% \tab for indentation
% \tab{2} for for indentation twice

% note
\newcommandx{\note}[1]{\textit{\textcolor{red}{#1}}}
\newcommand{\todo}{\note{TODO}}
% \note{TODO}

%% math package
\usepackage{amsfonts}
\usepackage{amsmath}
\usepackage{amssymb}
\usepackage{tikz-cd}
\usepackage{mathtools}
\usepackage{amsthm}

%% operator
\DeclareMathOperator{\tr}{tr}
\DeclareMathOperator{\diag}{diag}
\DeclareMathOperator{\sign}{sign}
\DeclareMathOperator{\grad}{grad}
\DeclareMathOperator{\curl}{curl}
\DeclareMathOperator{\Div}{div}
\DeclareMathOperator{\card}{card}
\DeclareMathOperator{\Span}{span}
\DeclareMathOperator{\real}{Re}
\DeclareMathOperator{\imag}{Im}
\DeclareMathOperator{\supp}{supp}
\DeclareMathOperator{\im}{im}
\DeclareMathOperator{\aut}{Aut}
\DeclareMathOperator{\inn}{Inn}
\DeclareMathOperator{\Char}{char}
\DeclareMathOperator{\Sylow}{Syl}
\DeclareMathOperator{\coker}{coker}
\DeclareMathOperator{\inc}{in}
\DeclareMathOperator{\Sd}{Sd}
\DeclareMathOperator{\Hom}{Hom}
\DeclareMathOperator{\interior}{int}
\DeclareMathOperator{\ob}{ob}
\DeclareMathOperator{\Set}{Set}
\DeclareMathOperator{\Top}{Top}
\DeclareMathOperator{\Meas}{Meas}
\DeclareMathOperator{\Grp}{Grp}
\DeclareMathOperator{\Ab}{Ab}
\DeclareMathOperator{\Ch}{Ch}
\DeclareMathOperator{\Fun}{Fun}
\DeclareMathOperator{\Gr}{Gr}
\DeclareMathOperator{\End}{End}
\DeclareMathOperator{\Ad}{Ad}
\DeclareMathOperator{\ad}{ad}
\DeclareMathOperator{\Bil}{Bil}
\DeclareMathOperator{\Skew}{Skew}
\DeclareMathOperator{\Tor}{Tor}
\DeclareMathOperator{\Ho}{Ho}
\DeclareMathOperator{\RMod}{R-Mod}
\DeclareMathOperator{\Ev}{Ev}
\DeclareMathOperator{\Nat}{Nat}
\DeclareMathOperator{\id}{id}
\DeclareMathOperator{\Var}{Var}
\DeclareMathOperator{\Cov}{Cov}
\DeclareMathOperator{\RV}{RV}
\DeclareMathOperator{\rank}{rank}

%% pair delimiter
\DeclarePairedDelimiter{\abs}{\lvert}{\rvert}
\DeclarePairedDelimiter{\inner}{\langle}{\rangle}
\DeclarePairedDelimiter{\tuple}{(}{)}
\DeclarePairedDelimiter{\bracket}{[}{]}
\DeclarePairedDelimiter{\set}{\{}{\}}
\DeclarePairedDelimiter{\norm}{\lVert}{\rVert}

%% theorems
\newtheorem{axiom}{Axiom}
\newtheorem{definition}{Definition}
\newtheorem{theorem}{Theorem}
\newtheorem{proposition}{Proposition}
\newtheorem{corollary}{Corollary}
\newtheorem{lemma}{Lemma}
\newtheorem{remark}{Remark}
\newtheorem{claim}{Claim}
\newtheorem{problem}{Problem}
\newtheorem{assumption}{Assumption}
\newtheorem{example}{Example}
\newtheorem{exercise}{Exercise}

%% empty set
\let\oldemptyset\emptyset
\let\emptyset\varnothing

\newcommand\eps{\epsilon}

% mathcal symbols
\newcommand\Tau{\mathcal{T}}
\newcommand\Ball{\mathcal{B}}
\newcommand\Sphere{\mathcal{S}}
\newcommand\bigO{\mathcal{O}}
\newcommand\Power{\mathcal{P}}
\newcommand\Str{\mathcal{S}}


% mathbb symbols
\usepackage{mathrsfs}
\newcommand\N{\mathbb{N}}
\newcommand\Z{\mathbb{Z}}
\newcommand\Q{\mathbb{Q}}
\newcommand\R{\mathbb{R}}
\newcommand\C{\mathbb{C}}
\newcommand\F{\mathbb{F}}
\newcommand\T{\mathbb{T}}
\newcommand\Exp{\mathbb{E}}

% mathrsfs symbols
\newcommand\Borel{\mathscr{B}}

% algorithm
\usepackage{algorithm}
\usepackage{algpseudocode}

% longproof
\newenvironment{longproof}[1][\proofname]{%
  \begin{proof}[#1]$ $\par\nobreak\ignorespaces
}{%
  \end{proof}
}


% for (i) enumerate
% \begin{enumerate}[label=(\roman*)]
%   \item First item
%   \item Second item
%   \item Third item
% \end{enumerate}
\usepackage{enumitem}

% insert url by \url{}
\usepackage{hyperref}

% margin
\usepackage{geometry}
\geometry{
a4paper,
total={190mm,257mm},
left=10mm,
top=20mm,
}


\title{ma5209 assignment 1}
\author{Nguyen Ngoc Khanh - A0275047B}
\date{March 2024}

\begin{document}
\maketitle

\section{Problem 1}
Define a category $\Ho(\Top)$ in the following way. For objects, take the class of topological spaces. A morphism from $X$ to $Y$ is a homotopy class of continuous maps from $X$ to $Y$. Show that there are unique notions of composition and identity for which the evident "function" from objects and morphisms in $\Top$ to those of $\Ho(\Top)$ constitute a functor. What is an isomorphism in $\Ho(\Top)$? If $A$ is a set and for each $\alpha \in A$ we are given a space $X_\alpha$, construct the product of $X_\alpha$'s and coproduct of $X_\alpha$'s in $\Ho(\Top)$.

Similarly, let $\Ch$ be the category of chain complexes and chain maps. Define $\Ho(\Ch)$ and the functor $\Ch \to \Ho(\Ch)$. Explain why the singular chain complex functor and the $n$-th homology functor define functors on $\Ho(\Top) \to \Ho(\Ch) \to \Ab$ where $\Ab$ is the category of abelian groups and homomorphisms.

\subsection{Definition of $\Ho(\Top)$}
As being homotopic is an equivalence relation, define the identity and composition in $\Ho(\Top)$ as follows:

\begin{itemize}
    \item identity: the identity map of an object $X$ in $\Ho(\Top)$ is defined as the homotopy class of $1: X \to X$, namely $[1]$
    \item composition: let $[f]: X \to Y, [g]: Y \to Z$ be two morphisms in $\Ho(\Top)$ with representatives $f: X \to Y, g: Y \to Z$ that are two morphisms in $\Top$. Then the composition is defined by
    $$
        [g] [f] = [gf]
    $$
    where $[gf]$ denotes the homotopy class of the composition $gf$ in $\Top$ 
\end{itemize}

We will prove that $\Ho(\Top)$, identity, and composition form a category by verifying the following:
\begin{enumerate}
    \item composition is well-defined
    \item $[1]$ is the identity of $X$ in $\Ho(\Top)$
    \item composition satisfies associativity
\end{enumerate}

\begin{longproof}
\begin{enumerate}
    \item composition is well-defined:
    
    Let $f_1: X \to Y, g_1: Y \to Z$ be two other representatives of $[f], [g]$, we will show that $g_1 f_1$ are homotopic to $gf$. Let $F: X \times I \to Y$ be the homotopy from $f$ to $f_1$, $G: Y \times I \to Z$ be the homotopy from $g$ to $g_1$, define $H: X \times I \to Z$ by
    $$
        H(x, t) = G F_1 (x, t) = G(F(x, t), t)
    $$
    where $F_1: X \times I \to Y \times I$ is defined by $F_1(x, t) = (F(x, t), t)$. Now, $H$ is continuous because both $G$ and $F_1$ are continuous. The continuity of $F_1$ is as follows: as any open set in $Y \times I$ is generated by the pair $O_Y \times O_I$ where $O_Y \subseteq Y$ and $O_I \subseteq I$ are two open subsets. We have $F_1^{-1}(O_Y \times O_I) = F_1^{-1}(O_Y \times I) \cap F_1^{-1}(Y \times O_I) = F^{-1}(O_Y) \cap X \times O_I$

    \item $[1]$ is the identity of $X$ in $\Ho(\Top)$:

    Given $[f]: X \to Y$, then $[f][1_Y] = [f 1_Y] = [f]$ and $[1_X][f] = [1_X f] = [f]$. The equality is due to $1_X, 1_Y$ being the identity in $\Top$
    
    \item composition satisfies associativity:

    This is due to associativity of composition in $\Top$
    $$
        [h]([g][f]) = [h][gf] = [h(gf)] = [(hg)f] = [hg][f] = ([h][g])[f]
    $$
\end{enumerate}
\end{longproof}

Define the evident "functor" $F: \Top \to \Ho(\Top)$ as follows
\begin{itemize}
    \item on objects: $X \mapsto X$
    \item on morphisms: $f \mapsto [f]$ where $f: X \to Y$ is a continuous map from $X$ to $Y$ and $[f]$ is the homotopy class of $f$
\end{itemize}

We will prove that $F$ is indeed a functor by verifying the following
\begin{enumerate}
    \item $F(1_X) = 1_{F(X)}$ where $1_X: X \to X$ is the identity map of $X$ in $\Top$ and $1_{F(X)}$ is the identity map of $F(X)$ in $\Ho(\Top)$
    \item $F(gf) = F(g)F(f)$ where $f: X \to Y, g: Y \to Z$ are morphisms in $\Top$
\end{enumerate}

\begin{longproof}

\begin{enumerate}
    \item $F(1_X) = 1_{F(X)}$:

    This is true by definition of identity in $\Ho(\Top)$
    
    \item $F(gf) = F(g)F(f)$:

    This is true by definition of composition in $\Ho(\Top)$

    $$
        F(gf) = [gf] = [g][f] = F(g) F(f)
    $$
\end{enumerate}    
\end{longproof}

\subsection{Isomorphism in $\Ho(\Top)$}

An isomorphism $[f]: X \to Y$ in $\Ho(\Top)$ is a morphism such that there exists $[g]: Y \to X$ such that $[f][g] = 1$ and $[g][f] = 1$. That is, $f$ is a homotopy equivalence.

\subsection{Product in $\Ho(\Top)$}
The product space $\prod_{\alpha} X_\alpha$ (Cartesian product of sets with product topology) is the product in $\Top$. Define the following objects and morphisms

\begin{center}
\begin{tikzcd}
                                                                                          & X_\alpha                                     &                                                                                                       & F(X_\alpha)                                               \\
W \arrow[r, "h"', dashed] \arrow[ru, "f_\alpha"] \arrow[r, "h_1"', dashed, bend right=71] & \prod_\alpha X_\alpha \arrow[u, "p_\alpha"'] & F(W) \arrow[ru, "F(f_\alpha)"] \arrow[r, "F(h)"', dashed] \arrow[r, "F(h_1)"', dashed, bend right=71] & F\tuple*{\prod_\alpha X_\alpha} \arrow[u, "F(p_\alpha)"']
\end{tikzcd}
\end{center}

Given any $F(W) \in \Ho(\Top)$, there is a $F(h)$ such that the diagram commutes. We will prove the uniqueness of $F(h)$. Suppose there is another map $F(h_1)$ that makes the diagram commutes, we will prove that $F(h) = F(h_1)$. Indeed, for every $\alpha \in A$.
$$
    F(p_\alpha h) = F(p_\alpha) F(h) = F(f_\alpha) = F(p_\alpha) F(h_1) = F(p_\alpha h_1)
$$

That is, $p_\alpha h$ is homotopic to $p_\alpha h_1$ for every $\alpha \in A$ then there exists a continuous map $H_\alpha: W \times I \to X_\alpha$ for every $\alpha \in A$ such that $H_\alpha(w, 0) = p_\alpha h (w)$, $H_\alpha(w, 1) = p_\alpha h_1 (w)$. Construct $H: W \times I \to \prod_\alpha X_\alpha$ as follows: (evaluation map)
$$
    H(w, t) = \prod_\alpha H_\alpha(w, t) = \tuple*{H_\alpha(w, t)}_\alpha
$$

This is a homotopy from $h$ to $h_1$, hence $F(h) = F(h_1)$, that is, $F(h)$ is unique
$$
    \prod_{\alpha \in A} F(X_\alpha) = F \tuple*{\prod_{\alpha \in A} X_\alpha}
$$

\subsection{Coproduct in $\Ho(\Top)$}

The disjoint union space $\coprod_{\alpha} X_\alpha$ (disjoint union of sets with disjoint union topology) is the coproduct in $\Top$. Define the following objects and morphisms

\begin{center}
\begin{tikzcd}
X_\alpha \arrow[d, "i_\alpha"] \arrow[rd, "f_\alpha"]                                 &   & F(X_\alpha) \arrow[rd, "F(f_\alpha)"] \arrow[d, "F(i_\alpha)"']                                       &      \\
\coprod_\alpha X_\alpha \arrow[r, "h", dashed] \arrow[r, "h_1"', dashed, bend right=74] & W & F\tuple*{\coprod_\alpha X_\alpha} \arrow[r, "F(h)", dashed] \arrow[r, "F(h_1)"', dashed, bend right=74] & F(W)
\end{tikzcd}
\end{center}

Given any $F(W) \in \Ho(\Top)$, there is a $F(h)$ such that the diagram commutes. We will prove the uniqueness of $F(h)$. Suppose there is another map $F(h_1)$ that makes the diagram commutes, we will prove that $F(h) = F(h_1)$. Indeed, for every $\alpha \in A$
$$
    F(h i_\alpha) = F(h) F(i_\alpha) = F(f_\alpha) = F(h_1) F(i_\alpha) = F(h_1 i_\alpha)
$$

That is $h i_\alpha$ is homotopic to $h_1 i_\alpha$ for every $\alpha \in A$ then there exists a continuous map $H_\alpha: X_\alpha \times I \to W$ for every $\alpha \in A$ such that $H_\alpha(x_\alpha, 0) = h i_\alpha (x_\alpha)$, $H_\alpha(x_\alpha, 1) = h_1 i_\alpha (x_\alpha)$. Construct $H: \coprod_\alpha X_\alpha \times I \to W$ as follows:
$$
    H(x, t) = H_\alpha(x, t) \text{ if } x \in X_\alpha
$$

This is a homotopy from $h$ to $h_1$, hence $F(h) = F(h_1)$. That is, $F(h)$ is unique
$$
    \coprod_{\alpha \in A} F(X_\alpha) = F \tuple*{\coprod_{\alpha \in A} X_\alpha}
$$

\subsection{Definition of $\Ho(\Ch)$}

Define the objects and morphisms in $\Ho(\Ch)$
\begin{itemize}
    \item objects: chain complexes
    \item morphisms: chain maps
\end{itemize}

As being chain homotopic is an equivalence relation, define the identity and composition in $\Ho(\Ch)$
\begin{itemize}
    \item identity: the identity map of a chain complex $X$ in $\Ho(\Top)$ is defined as the chain homotopy class of the identity chain map $1: X \to X$, namely $[1]$
    \item composition: let $[f]: X \to Y, [g]: Y \to Z$ be two morphisms in $\Ho(\Ch)$ with representatives $f: X \to Y, g: Y \to Z$ that are two morphisms in $\Ch$. Then the composition is defined as
    $$
        [g][f] = [gf]
    $$
    where $[gf]$ denotes the chain homotopy class of $gf$ in $\Ch$ 
\end{itemize}

We will prove that $\Ho(\Ch)$, identity, and composition form a category by verifying the following:
\begin{enumerate}
    \item composition is well-defined
    \item $[1]$ is the identity of $X$ in $\Ho(\Ch)$
    \item composition satisfies associativity
\end{enumerate}

\begin{longproof}
\begin{enumerate}
    \item composition is well-defined:
    
    Let $f_1: X \to Y, g_1: Y \to Z$ be two other representatives of $[f], [g]$, we will show that $g_1 f_1$ is chain homotopic to $gf$.

    \begin{center}
        \begin{tikzcd}
        X_{n-1} \arrow[rrdd, "h_f"'] &  & X_n \arrow[ll, "\partial"'] \arrow[rrdd, "h_f"] \arrow[dd, "f-f_1"'] &  & X_{n+1} \arrow[ll, "\partial"'] \\
                                     &  &                                                                      &  &                                 \\
        Y_{n-1} \arrow[rrdd, "h_g"'] &  & Y_n \arrow[ll, "\partial"'] \arrow[rrdd, "h_g"] \arrow[dd, "g-g_1"'] &  & Y_{n+1} \arrow[ll, "\partial"'] \\
                                     &  &                                                                      &  &                                 \\
        Z_{n-1}                      &  & Z_n \arrow[ll, "\partial"]                                           &  & Z_{n+1} \arrow[ll, "\partial"] 
        \end{tikzcd}
    \end{center}

    \begin{align*}
        gf
        &= (g_1 + \partial h_g + h_g \partial)(f_1 + \partial h_f + h_f \partial) &\text{$(f, f_1), (g, g_1)$ are chain homotopic}\\
        &= g_1 (f_1 + \partial h_f + h_f \partial) + (\partial h_g + h_g \partial)(f_1 + \partial h_f + h_f \partial) \\
        &= g_1 f_1 + g_1 \partial h_f + g_1 h_f \partial + (\partial h_g + h_g \partial)(f_1 + \partial h_f + h_f \partial) \\
        &= g_1 f_1 + \partial g_1 h_f + g_1 h_f \partial + (\partial h_g + h_g \partial)(f_1 + \partial h_f + h_f \partial) &\text{($\partial g_1 = g_1 \partial$)}
    \end{align*}

    \begin{align*}
        (\partial h_g + h_g \partial)(f_1 + \partial h_f + h_f \partial)
        &= \partial h_g f_1 + \partial h_g \partial h_f + \partial h_g h_f \partial + h_g \partial f_1 + h_g \partial \partial h_f + h_g \partial h_f \partial \\
        &= \partial h_g f_1 + \partial h_g \partial h_f + \partial h_g h_f \partial + h_g \partial f_1 + h_g  \partial h_f \partial &\text{($\partial \partial = 0$)} \\
        &= (\partial h_g f_1 + \partial h_g h_f \partial + h_g \partial f_1) + (\partial h_g \partial h_f + h_g \partial h_f \partial) &\text{($+$ is commutative, associative)} \\
        &= [\partial h_g (f_1 + h_f \partial) + h_g (f_1 + h_f \partial) \partial] + (\partial h_g \partial h_f + h_g \partial h_f \partial) &\text{($\partial f_1 = f_1 \partial$)}
    \end{align*}
    Therefore, let $h_{gf} = g_1 h_f + h_g (f_1 + h_f \partial) + h_g \partial h_f$, then
    $$
        gf = g_1 f_1 + \partial h_{gf} + h_{gf} \partial
    $$
    
    \item $[1]$ is the identity of $X$ in $\Ho(\Ch)$:

    Given $[f]: X \to Y$, then $[f][1_Y] = [f 1_Y] = [f]$ and $[1_X][f] = [1_X f] = [f]$. The equality is due to $1_X, 1_Y$ being the identity in $\Ch$
    
    \item composition satisfies associativity:

    This is due to associativity of composition in $\Ch$
    $$
        [h]([g][f]) = [h][gf] = [h(gf)] = [(hg)f] = [hg][f] = ([h][g])[f]
    $$

\end{enumerate}
\end{longproof}

Define the "functor" $F: \Ch \to \Ho(\Ch)$ as follows
\begin{itemize}
    \item on objects: $X \mapsto X$
    \item on morphisms: $f \mapsto [f]$ where $f: X \to Y$ is a chain map from $X$ to $Y$ and $[f]$ is the chain homotopy class of $f$
\end{itemize}

We will prove that $F$ is indeed a functor by verifying the following
\begin{enumerate}
    \item $F(1_X) = 1_{F(X)}$ where $1_X: X \to X$ is the identity map of $X$ in $\Ch$ and $1_{F(X)}$ is the identity map of $F(X)$ in $\Ho(\Ch)$
    \item $F(gf) = F(g)F(f)$ where $f: X \to Y, g: Y \to Z$ are morphisms in $\Ch$
\end{enumerate}

\begin{longproof}

\begin{enumerate}
    \item $F(1_X) = 1_{F(X)}$:

    This is true by definition of identity in $\Ch$
    
    \item $F(gf) = F(g)F(f)$:

    $$
        F(gf) = [gf] = [g][f] = F(g) F(f)
    $$
\end{enumerate}
\end{longproof}

\subsection{Singular chain complex functor and singular homology functor on $\Ho(\Top), \Ho(\Ch)$}

\begin{center}
    \begin{tikzcd}
    \Top \arrow[rr, "C_\bullet"] \arrow[dd, two heads] &  & \Ch \arrow[rr, "H_\bullet"] \arrow[dd, two heads] &  & \Ab \\
                                                       &  &                                                   &  &     \\
    \Ho(\Top) \arrow[rr, dashed]                       &  & \Ho(\Ch) \arrow[rruu, dashed]                     &  &    
    \end{tikzcd}
\end{center}

Define the "functor" $C_\bullet: \Ho(\Top) \to \Ho(\Ch)$ as follows
\begin{itemize}
    \item on objects: same with $C_\bullet: \Top \to \Ch$
    \item on morphisms: $[f] \mapsto f \mapsto C_\bullet(f) \mapsto [C_\bullet(f)]$ where $[f]$ is the homotopy class of a morphism $f$ in $\Top$, $[C_\bullet(f)]$ is the chain homotopy class of a morphism $C_\bullet(f)$ in $\Ch$.
\end{itemize}

We will prove that $C_\bullet: \Ho(\Top) \to \Ho(\Ch)$ is indeed a functor by verifying the following

\begin{enumerate}
    \item $C_\bullet([1]) = [1]$
    \item $C_\bullet([g][f]) = C_\bullet([g]) C_\bullet([f])$ where $f: X \to Y, g: Y \to Z$ are morphisms in $\Top$
\end{enumerate}

\begin{longproof}
\begin{enumerate}
    \item $C_\bullet([1]) = [1]$:

    This is true by the definition of $C_\bullet$
    
    \item $C_\bullet([g][f]) = C_\bullet([g]) C_\bullet([f])$:
    
    \begin{align*}
        C_\bullet([g][f])
        &= C_\bullet([gf]) &\text{(composition in $\Ho(\Top)$)} \\
        &= [C_\bullet(gf)] &\text{($C_\bullet: \Ho(\Top) \to \Ho(\Ch)$ on morphisms $[gf]$)} \\
        &= [C_\bullet(g) C_\bullet(f)] &\text{(functor $C_\bullet: \Top \to \Ch$)} \\
        &= [C_\bullet(g)] [C_\bullet(f)] &\text{(composition in $\Ho(\Ch)$)} \\
        &= C_\bullet([g]) C_\bullet([f]) &\text{($C_\bullet: \Ho(\Top) \to \Ho(\Ch)$ on morphisms $[f]$ and $[g]$)} 
    \end{align*}

\end{enumerate}
\end{longproof}

Define the "functor" $H_n: \Ho(\Ch) \to \Ab$ as follows

\begin{itemize}
    \item on objects: same with $H_n: \Ch \to \Ab$
    \item on morphisms: $[C_\bullet(f)] \mapsto C_\bullet(f) \mapsto H_n(f)$
\end{itemize}

We will prove that $H_n: \Ho(\Ch) \to \Ab$ is indeed a functor by verifying the following

\begin{enumerate}
    \item $H_n([1]) = 1$
    \item $H_n([C_\bullet(g)][C_\bullet(f)]) = H_n([C_\bullet(g)]) H_n([C_\bullet(f)])$ where where $C_\bullet(f): X \to Y, C_\bullet(g): Y \to Z$ are morphisms in $\Ch$
\end{enumerate}

\begin{longproof}

\begin{enumerate}
    \item $H_n([1]) = 1$:

    This is true by the definition of $H_n$
    
    \item $H_n([C_\bullet(g)][C_\bullet(f)]) = H_n([C_\bullet(g)]) H_n([C_\bullet(f)])$:

    \begin{align*}
        H_n([C_\bullet(g)][C_\bullet(f)])
        &= H_n([C_\bullet(g) C_\bullet(f)]) &\text{(composition in $\Ho(\Ch)$)} \\
        &= H_n([C_\bullet(gf)]) &\text{(functor $C_\bullet: \Top \to \Ch$)} \\
        &= H_n(gf) &\text{($H_n: \Ho(\Ch) \to \Ab$ on morphism $[C_\bullet(gf)]$)} \\
        &= H_n(g) H_n(f) &\text{(functor $H_n: \Top \to \Ab$)} \\
        &= H_n([C_\bullet(g)]) H_n([C_\bullet(f)]) &\text{($H_n: \Ho(\Ch) \to \Ab$ on morphism $[C_\bullet(f)]$ and $[C_\bullet(g)]$)}
    \end{align*}
\end{enumerate}
\end{longproof}











\section{Problem 2}
\begin{itemize}
    \item Put a CW structure on the product two finite CW complexes.
    \item Show that the composite of two cofibrations is a cofibration.
\end{itemize}

\subsection{CW structure on the product two finite CW complexes}

Given two cell complexes $X, Y$. We define a CW structure on $Z$ as follows
\begin{center}
    \begin{tikzcd}
    \coprod_{\alpha \in A_n} \partial D^n \arrow[rr, hook] \arrow[dd, "a_n"'] &  & \coprod_{\alpha \in A_n} D^n \arrow[dd, "\overline{a_n}"] &  & \coprod_{\beta \in B_n} \partial D^n \arrow[rr, hook] \arrow[dd, "b_n"'] &  & \coprod_{\beta \in B_n} D^n \arrow[dd, "\overline{b_n}"] \\
                                                                              &  &                                                           &  &                                                                          &  &                                                          \\
    X_{n-1} \arrow[rr, hook]                                                  &  & X_n                                                       &  & Y_{n-1} \arrow[rr, hook]                                                 &  & Y_n                                                     
    \end{tikzcd}
\end{center}

\begin{center}
    \begin{tikzcd}
    \coprod_{\gamma \in C_n} \partial D^n \arrow[rr, hook] \arrow[dd, "c_n"'] &  & \coprod_{\gamma \in C_n} D^n \arrow[dd, "\overline{c_n}"] \\
                                                                              &  &                                                           \\
    Z_{n-1} \arrow[rr, hook]                                                  &  & Z_n                                                      
    \end{tikzcd}
\end{center}

where $C_n = \coprod_{i + j = n} A_i \times B_j$ be the disjoint union of $A_i \times B_j$.

\textbf{Some notes on CW structure:}

$a_n$ is the attaching map, $\overline{a_n}$ is the characteristic map. Note that, attaching map is a restriction of characteristic map on the boundary of $\coprod D^n$. In the interior of $\coprod D^n$, characteristic map is a homeomorphism. $X_n$ is the quotient of $\coprod_{\alpha \in A_n} D^n$ under the equivalence class defined by $a_n$ (or $\overline{a_n}$). Let $a_n^\alpha: \partial D^n \to X_{n-1}, \overline{a_n^\alpha}: D^n \to X_n$ are attaching map and characteristic map corresponding to $\alpha \in A_n$. Similar notations for $Y$ and $Z$.

If $D^n$ is a $n$-dimensional cube, we can show that
\begin{align*}
    D^{i + j} &= D^i \times D^j \\
    \partial D^{i + j} &= \partial D^i \times D^j \cup  D^i \times \partial D^j
\end{align*}

For $\gamma = (\alpha, \beta) \in A_i \times B_j$, define attaching map 
$$
    c_n^{\gamma}: \partial D^{i + j} \to X_{i-1} \times Y_j \cup X_i \times Y_{j-1} \subseteq Z_{n-1}
$$

that maps $\partial D^i \times D^j$ to $X_{i-1} \times Y_j$, maps $D^i \times \partial D^j$ to $X_i \times Y_{j-1}$ as follows:

$$
    c_n^{\gamma}(x, y) = \tuple*{\overline{a_i^\alpha}(x), \overline{b_j^\beta}(y)}
$$

where $(x, y) \in \partial D^i \times D^j \cup  D^i \times \partial D^j$. We are left to prove that $Z = X \times Y$ by verifying $Z_n = \bigcup_{i + j = n} X_i \times Y_j$, that is, $\bigcup_{i + j = n} X_i \times Y_j$ is the pushout of $Z_n$'s diagram and finish the proof by the argument on finiteness of $X, Y, Z$

\begin{longproof}
    Consider one of the pair $(i, j)$, we show that $W_{ij} = X_i \times Y_j$ is the pushout of the diagram below:
    
        \begin{center}
            \begin{tikzcd}
            \coprod_{\gamma \in A_i \times B_j} \partial D^{i+j} \arrow[rr, hook] \arrow[dd, "c_n"'] &  & \coprod_{\gamma \in A_i \times B_j} D^{i+j} \arrow[dd, "\overline{c}_n"] \\
                                                                                                 &  &                                                                      \\
            X_{i-1} \times Y_j \cup X_i \times Y_{j-1} \arrow[rr, hook]                          &  & W_{ij}                                                      
            \end{tikzcd}
        \end{center}
    
    
    Let $(x, y), (x_1, y_1) \in \coprod_{\gamma \in A_i \times B_j} D^i \times D^j$.
    
    $(x, y), (x_1, y_1)$ are in the same equivalence class (under $\overline{c_n}$) if and only if $\overline{a_i^\alpha}(x) = \overline{a_i^\alpha}(x_1)$ and $\overline{b_j^\beta}(y) = \overline{b_j^\beta}(y_1)$ if and only if $x, x_1$ be in the same equivalence class of $\overline{a_n}$ and $y, y_1$ be in the same equivalence class of $\overline{b_n}$, that is, $x, x_1$ identify the same point on $X_i$ and $y, y_1$ identify the same point on $Y_i$. Therefore, the pushout of the diagram is exactly $W_{ij} = X_i \times Y_j$

    Hence, $Z_n = \bigcup_{i + j = n} X_i \times Y_j$ is the pushout of the diagram consists of disjoint union over the finite collection of pairs $(i, j)$

    
    As $X, Y$ are finite, let $X = X_m, Y = Y_n$, as $X_0 \subseteq X_1 \subseteq ... \subseteq X$, $Y_0 \subseteq Y_1 \subseteq ... \subseteq Y$, we have
    $$
        Z = Z_{m+n} = X_m \times Y_n = X \times Y
    $$
\end{longproof}

\subsection{Composite of cofibrations}

\begin{center}
\begin{tikzcd}
                                     &  &                                                                   &  &  & W \\
Z \arrow[rr] \arrow[rrrrru, "h_Z"]   &  & Z \times I \arrow[rrru, "H_Z"', dashed]                           &  &  &   \\
                                     &  &                                                                   &  &  &   \\
Y \arrow[rr, "i_1"'] \arrow[uu, "g"] &  & Y \times I \arrow[uu, "g \times 1"] \arrow[rrruuu, "H_Y", dashed] &  &  &   \\
                                     &  &                                                                   &  &  &   \\
X \arrow[rr, "i_1"'] \arrow[uu, "f"] &  & X \times I \arrow[rrruuuuu, "H_X"'] \arrow[uu, "f \times 1"]      &  &  &  
\end{tikzcd}
\end{center}

Suppose $f: X \to Y, g: Y \to Z$ are cofibrations, there is a homotopy $H_X: X \times I \to W$ and a map $h_Z: Z \to W$. If $gf: X \to Z$ is a cofibration, homotopy extension property states that there exists $H_Z: Z \times I \to W$ such that that diagram commutes.

Indeed, let $h_Y: Y \to W$ be defined by $h_Y = h_Z g$. Since $f: X \to Y$ is a cofibration, given $H_X: X \times I \to W$ and $h_Y: Y \to W$, there exists $H_Y: Y \times I \to W$ such that the diagram commutes. Since $g: Y \to Z$ is a cofibration, given $H_Y: Y \times I \to W$ and $h_Z: Z \to W$, there exists $H_Z: Z \times I \to W$ such that the diagram commutes

\section{Problem 3}
For an invertible linear transformation $f: \R^n \to \R^n$. show that the induced map on $H_n(\R^n, \R^n - \set{0}) \approx \Tilde{H}_{n-1}(\R^n - \set{0}) \approx \Z$ is $\1$ or $-\1$ according to whether the determinant of $f$ is positive or negative.

\subsection{Preliminaries}

We adopt the definition of reduced homology in Hatcher.
\begin{definition}[reduced homology]
    Let $X$ be a non-empty topological spaces and $C_\bullet: \Top \to \Ab$ be singular chain functor. Reduced homology is the homology of the chain complex 
    \begin{center}
        \begin{tikzcd}
        0 & \Z \arrow[l] & C_0(X) \arrow[l, "\epsilon"'] & C_1(X) \arrow[l, "\partial"'] & ... \arrow[l, "\partial"']
        \end{tikzcd}
    \end{center}
    where $\epsilon: C_0(X) \to \Z$ is the augmentation map. The augmented chain complex is denoted by $\Tilde{C}_\bullet(X)$ and the reduced homology is denoted by $\Tilde{H}_\bullet(X)$
\end{definition}

\begin{remark}[relationship with singular homology]
    \begin{align*}
        H_0(X) &= \Tilde{H}_0(X) \oplus \Z \\
        H_n(X) &= \Tilde{H}_n(X) \text{ for } n \geq 1
    \end{align*}
\end{remark}

\begin{remark}[reduced homology of common spaces]
    Reduced homology of common spaces
    \begin{itemize}
        \item $\Tilde{H}_n(*) = 0$: homology of a contractible space is the trivial group
        \item $\Tilde{H}_0(X) = \bigoplus_{i=1}^{n-1} \Z$: if $X$ has $n$ path-components
    \end{itemize}
\end{remark}

\begin{remark}[relative homology on reduced homology]
    Short exact sequence of chains
    \begin{center}
        \begin{tikzcd}
        0 \arrow[r] & \Tilde{C}_n(A) \arrow[r, hook] \arrow[d, "\partial"] & \Tilde{C}_n(X) \arrow[rr, two heads] \arrow[d, "\partial"] &  & \Tilde{C}_n(X) / \Tilde{C}_n(A) \arrow[rr] \arrow[d, "\partial", dashed] &  & 0 \\
        0 \arrow[r] & \Tilde{C}_{n-1}(A) \arrow[r, hook]                   & \Tilde{C}_{n-1}(X) \arrow[rr, two heads]                   &  & \Tilde{C}_{n-1}(X) / \Tilde{C}_{n-1}(A) \arrow[rr]                       &  & 0
        \end{tikzcd}
    \end{center}
    Note that, this is identical to the short exact sequence of chains for singular homology except $\Tilde{C}_{-1}(A)$ and $\Tilde{C}_{-1}(X)$. The induced long exact sequence

    \begin{center}
        \begin{tikzcd}
                                   &  &                                       &  & ... \arrow[lllld, "\partial"']             \\
        \Tilde{H}_n(A) \arrow[rr, "i"]     &  & \Tilde{H}_n(X) \arrow[rr, "p", two heads]     &  & {H_n(X, A)} \arrow[lllld, "\partial"']     \\
        \Tilde{H}_{n-1}(A) \arrow[rr, "i"] &  & \Tilde{H}_{n-1}(X) \arrow[rr, "p", two heads] &  & {H_{n-1}(X, A)} \arrow[lllld, "\partial"'] \\
        ...                        &  &                                       &  &                                           
        \end{tikzcd}
    \end{center}
    
\end{remark}


\begin{definition}[$\partial: H_{n+1}(C) \to H_n(A)$]
    Definition of the connecting homomorphism $\partial: H_{n+1}(C) \to H_n(A)$
    \begin{center}
        \begin{tikzcd}
        n+1  &                                                          & b \arrow[r, "p", maps to] \arrow[d, "\partial", maps to]          & c \arrow[d, "\partial", maps to] \\
        n:   & a \arrow[r, "i", maps to] \arrow[d, "\partial", maps to] & \partial b \arrow[r, "p", maps to] \arrow[d, "\partial", maps to] & 0                                \\
        n-1: & \partial a \arrow[r, "i", maps to]                       & \partial^2 b = 0                                                  &                                 
        \end{tikzcd}
    \end{center}
    
    Given $[c] \in H_{n+1}(C)$, (1) take any representative $c \in Z_{n+1}(C)$. As $p: B_{n+1} \to C_{n+1}$ is surjective, (2) take any $b \in B_{n+1}$ such that $pb = c$. As $p \partial b = \partial pb = \partial c = 0$ and $\ker (p: B_n \to C_n) = \im (i: A_n \to B_n)$, take $a \in A_n$ such that $ia = \partial b$, this choice is unique as $i$ is injective. $i \partial a = \partial i a = \partial^2 b = 0$, as $i$ is an injective homomorphism, $\partial a = 0$, then $a \in Z_n(A)$. The construction is done by $[c] \mapsto [a]$
\end{definition}

Another result from Hatcher:

\begin{lemma}
    \label{lemma1}
    Given two short exact sequences of chain complexes with chain maps $\alpha: A_n \to A'_n, \beta: B_n \to B'_n, \gamma: C_n \to C'_n$, such that the diagram below commutes
    \begin{center}
        \begin{tikzcd}
        0 \arrow[r] & A_\bullet \arrow[r, "i"] \arrow[d, "\alpha"] & B_\bullet \arrow[r, "p"] \arrow[d, "\beta"] & C_\bullet \arrow[r] \arrow[d, "\gamma"] & 0 \\
        0 \arrow[r] & A'_\bullet \arrow[r, "i'"]                   & B'_\bullet \arrow[r, "p'"]                  & C'_\bullet \arrow[r]                    & 0
        \end{tikzcd}
    \end{center}
    Then the induced long exact sequence diagram commutes
    \begin{center}
        \begin{tikzcd}
        ... \arrow[r] & H_{n+1}(A_\bullet) \arrow[r, "i_*"] \arrow[d, "\alpha_*"] & H_{n+1}(B_\bullet) \arrow[r, "p_*"] \arrow[d, "\beta_*"] & H_{n+1}(C_\bullet) \arrow[r, "\partial"] \arrow[d, "\gamma_*"] & H_n(A_\bullet) \arrow[r] \arrow[d, "\alpha*"] & ... \\
        ... \arrow[r] & H_{n+1}(A'_\bullet) \arrow[r, "i'_*"]                     & H_{n+1}(B'_\bullet) \arrow[r, "p'_*"]                    & H_{n+1}(C'_\bullet) \arrow[r, "\partial'"]                     & H_n(A'_\bullet) \arrow[r]                     & ...
        \end{tikzcd}
    \end{center}
    where $H_n: \Ch \to \Ab$ is a functor 
\end{lemma}

\begin{longproof}
    The first two squares commute since $H_n$ is a functor. For the third square, recall the definition of $\partial: H_n(C) \to H_{n-1}(A)$
    $$
        \partial [c] = [a]
    $$
    
    where $c = pb$ and $ia = \partial b$. We have
    \begin{align*}
        \gamma c &= \gamma p b = p' \beta b \\
        i' \alpha a &= \beta i a = \beta \partial b = \partial \beta b
    \end{align*}
    
    then by the definition of connecting homomorphism $\partial: H_n(C') \to H_{n-1}(A')$, we have
    $$
        \partial [\gamma c] = [\alpha a]
    $$
    
    Again, $H_n$ is a functor,
    \begin{align*}
        [\gamma c] &= H_n(\gamma) [c] = \gamma_* [c] \\
        [\alpha a] &= H_n(\alpha) [a] = \alpha_* [a] = \alpha_* \partial [c]
    \end{align*}
    
    That is, the last third square commutes
\end{longproof}

\begin{definition}[degree]
    For $n > 0$, let $f: S^n \to S^n$, then $f_*: \Tilde{H}_n(S^n) \to \Tilde{H}_n(S^n)$ is a multiplication $\Z \to \Z$ of $m$. $m$ is called the degree of $f$
\end{definition}

\begin{lemma}
    \label{lemma2}
    Degree of a refection is $-1$
\end{lemma}

\subsection{Main Proof}
Let's denote $X = \R^n, A = \R^n - \set{0}$. Any linear map $f$ in $GL(\R^n)$ can by transformed into either the identity $1$ or a reflection $r$ by Gaussian elimination, each row operation is either row-swap, row-scale, row-sum which can be written as a smooth map of time $t$, that is, any linear map is homotopic to either $1$ (if $\det f > 0$) or $r$ (if $\det f < 0$). Moreover, the homotopy applies for the case of pair of spaces $(X, A)$

Long exact sequence of $(X, A)$ implies the connecting homomorphism $\partial: H_n(X, A) \to \Tilde{H}_{n-1}(A)$ is an isomorphism.

\begin{center}
    \begin{tikzcd}
                               &  & \Tilde{H}_n(X) = 0 \arrow[rr, "p_*", two heads] &  & {H_n(X, A)} \arrow[lllld, "\partial"'] \\
    \Tilde{H}_{n-1}(A) \arrow[rr, "i_*"] &  & \Tilde{H}_{n-1}(X) = 0                        &  &                                       
    \end{tikzcd}
\end{center}

The diagram below commutes

\begin{center}
\begin{tikzcd}
0 \arrow[r] & \Tilde{C}_\bullet(A) \arrow[r, hook] \arrow[d, "f_\#"] & \Tilde{C}_\bullet(X) \arrow[r, two heads] \arrow[d, "f_\#"] & {\Tilde{C}_\bullet(X, A)} \arrow[r] \arrow[d, "f_\#"] & 0 \\
0 \arrow[r] & \Tilde{C}_\bullet(A) \arrow[r, hook]                   & \Tilde{C}_\bullet(X) \arrow[r, two heads]                   & {\Tilde{C}_\bullet(X, A)} \arrow[r]                   & 0
\end{tikzcd}
\end{center}

where $f_\#$ is induced from $f$ in the level of chain. By Lemma \ref{lemma1} the diagram below commutes, , $f_*$ is induced from $f_*$ in the level of homology

\begin{center}
    \begin{tikzcd}
    {C_n(X, A)} \arrow[d, "f_\#"'] & {H_n(X, A)} \arrow[r, "\partial"] \arrow[d, "f_*"'] & \Tilde{H}_{n-1}(A) \arrow[d, "f_*"] & \Tilde{C}_{n-1}(A) \arrow[d, "f_\#"] \\
    {C_n(X, A)}                    & {H_n(X, A)} \arrow[r, "\partial"]                   & \Tilde{H}_{n-1}(A)                  & \Tilde{C}_{n-1}(A)                  
    \end{tikzcd}
\end{center}

As $\partial$ is an isomorphism between $\Z$ and $\Z$ (isomorphism sends $1$ to either $1$ or $-1$), it suffices to show for the case of reduced homology $\Tilde{H}_{n-1}(A)$

\begin{longproof}
    If $f$ is homotopic to the identity $1$, the induced map in $\Tilde{H}_{n-1}(A)$ is the identity map $1$
    
    If $f$ is homotopic to a reflection $r$, let $g: S^{n-1} \to S^{n-1}$ be the restriction of $r$ ($g$ is a reflection on $S^{n-1}$, $\Tilde{H}_{n-1}(g) = -1$), $i: S^{n-1} \to A$ be the inclusion map, $p: A \to S^{n-1}$ be the deformation retraction of $A$ into $S^{n-1}$.
    
    \begin{center}
        \begin{tikzcd}
        A \arrow[d, "p"'] \arrow[r, "f"] & A                       &  & \Tilde{H}_{n-1}(A) \arrow[d, "\Tilde{H}_{n-1}(p)"'] \arrow[r, "\Tilde{H}_{n-1}(f)"] & \Tilde{H}_{n-1}(A)                                \\
        S^{n-1} \arrow[r, "g"']          & S^{n-1} \arrow[u, "i"'] &  & \Tilde{H}_{n-1}(S^{n-1}) \arrow[r, "\Tilde{H}_{n-1}(g)"']                   & \Tilde{H}_{n-1}(S^{n-1}) \arrow[u, "\Tilde{H}_{n-1}(i)"']
        \end{tikzcd}
    \end{center}
    
    Since the left diagram (diagram in $\Top$) commutes, $\Tilde{H}_{n-1}: \Top \to \Ab$ is a functor, the right diagram (diagram in $\Ab$) commutes. As $i$ and $p$ are homotopy equivalence ($p i \simeq 1, i p \simeq 1$), $\Tilde{H}_{n-1}(i) = \Tilde{H}_{n-1}(p) = 1$, then 
    
    $$
        \Tilde{H}_{n-1}(f) = \Tilde{H}_{n-1}(i) \Tilde{H}_{n-1}(g) \Tilde{H}_{n-1}(p) = 1 (-1) 1 = -1
    $$

    the induced map in $\Tilde{H}_{n-1}(A)$ is $-1$.
\end{longproof}

\section{Problem 4}
A polynomial $f(z)$ with complex coefficients, viewed as a map $\C \to \C$ can always be extended to a continuous map of one-point compactifications $\hat{f}: S^2 \to S^2$. Show that the degree of $\hat{f}$ equals the degree of $f$ as a polynomial. Show also that the local degree of $\hat{f}$ at a root of $f$ is the multiplicity of the root.

\subsection{Preliminaries}

\begin{lemma}
    \label{lemma3}
    On $S^1$ (unit circle in $\C$), $\deg z^n = n$
\end{lemma}

\begin{lemma}[Hatcher proposition 2.33]
    \label{lemma4}
    $\deg \Sigma f = \deg f$ where $\Sigma f: \Sigma S^n \to \Sigma S^n$ is the suspension of $f: S^n \to S^n$ and $\Sigma S^n \cong S^{n+1}$ is the suspension of $S^n$
\end{lemma}


\subsection{Degree of $\hat{f}$}

By lemma \ref{lemma3} and \ref{lemma4}, in $S^2$, $\deg \Sigma z^n = \deg z^n = n$. Moreover, there exists a homotopy from $\Sigma z^n$ to $z^n$ (write $z^n$ in polar coordinate). Hence, in $S^2$, $\deg z^n = n$

Let $f(z) = a_n z^n + ... + a_1 z + a_0$ defined on $S^2$, there exist two maps $H_1: S^2 \times I \to S^2$ and $H_2: S^2 \times I \to S^2$ as follows

\begin{align*}
    H_1(z, t) &= a_n^t z^n\\
    H_2(z, t) &= ta_n z^n + (1-t)f(z)
\end{align*}

Both maps are continuous on $\C \times I$ and $\set{\infty} \times I$, hence they are homotopies $z^n \to a_n z^n$, $a_n z^n \to \hat{f}(z)$. Therefore, in $S^2$, $\deg \hat{f} = \deg z^n = n$

\subsection{Local degree of $\hat{f}$}
Let $\set{x_1, ..., x_n}$ be the roots of $f(z)$. Let disjoint path-connected open sets $\set{U_1, ..., U_n}$ such that $x_i \in U_i$ and $V = \hat{f}\tuple*{\bigcup_{i=1}^n U_i}$, hence, $V$ is also path-connected. By definition, local degree of $\hat{f}$ at $x_i$ is the induced function $\hat{f}_*$

\begin{center}
    \begin{tikzcd}
    {H_2(U_i, U_i - \set{x_i}) = \Z} \arrow[rr, "\hat{f}_*"] &  & {H_2(V, V - \set{0}) = \Z}
    \end{tikzcd}
\end{center}

Given the commutative diagram below
\begin{center}
\begin{tikzcd}
0 \arrow[r] & C_\bullet(U_i - \set{x_i}) \arrow[r, "i"] \arrow[d, "f_\#"] & C_\bullet(U_i) \arrow[r, "p"] \arrow[d, "f_\#"] & {C_\bullet(U_i, U_i - \set{x_i})} \arrow[r] \arrow[d, "f_\#"] & 0 \\
0 \arrow[r] & C_\bullet(V - \set{0}) \arrow[r, "i'"]                      & C_\bullet(V) \arrow[r, "p'"]                    & {C_\bullet(V, V - \set{0})} \arrow[r]                         & 0
\end{tikzcd}
\end{center}

By Lemma \ref{lemma1} and exactness, $\partial, \partial'$ are isomorphisms, the square is commutative, the induced maps are the same.

\begin{center}
\begin{tikzcd}
H_2(U_i) = 0 \arrow[rr, "p_*"] &  & {H_2(U_i, U_i - \set{x_i}) = \Z} \arrow[dd, "\hat{f}_*"'] \arrow[rr, "\partial"] &  & H_1(U_i - \set{x_i}) = \Z \arrow[dd, "\hat{f}_*"] \arrow[rr, "i_*"] &  & H_1(U_i) = 0 \\
                               &  &                                                                                  &  &                                                                     &  &              \\
H_2(V) = 0 \arrow[rr, "p'_*"]  &  & {H_2(V, V - \set{0}) = \Z} \arrow[rr, "\partial'"]                               &  & H_1(V - \set{0}) = \Z \arrow[rr, "i'_*"]                            &  & H_1(V) = 0  
\end{tikzcd}
\end{center}

Hence, local degree of $\hat{f}$ at $x_i$ is the degree of $\hat{f}$ restricted to $U_i - \set{x_i} \to V - \set{0}$. Now write $f(z) = (z - x_i)^{m_i} g(z)$ where $g(z) \neq 0$ on $U_i$ and $m_i$ is the multiplicity of root $x_i$. There exists a map $H_3: (U_i - \set{x_i}) \times I \to V - \set{0}$ as follows

$$
    H_3(z, t) = t (z - x_i)^{m_i} + (1-t)f(z)
$$

As $H_3$ is continuous on its domain, hence it is a homotopy from $(z - x_i)^{m_i}$ to $f(z)$. Therefore, $\deg \hat{f}$ restricted to $U_i - \set{x_i} \to V - \set{0}$ is $m_i$

\section{Problem 5}
Let $X$ be the quotient space of $S^2$ under identifications $x \sim -x$ for $x$ in the equator $S^1$. Compute the homology groups $H_i(X)$. Do the same for $S^3$ with antipodal points of equatorial $S^2 \subset S^3$ identified.

\begin{proposition}[cellular boundary formula]
    $d_n(e^n_\alpha) = \sum_\beta d_{\alpha \beta} e^{n-1}_\beta$ where $d_{\alpha \beta}$ is the degree of the map $S^{n-1}_\alpha \to X^{n-1} \to S^{n-1}_\beta$ that is the composition of the attaching map of $e^n_\alpha$ with the quotient map collapsing $X^{n-1} - e^{n-1}_\beta$ to a point.
\end{proposition}

\subsection{$S^2$ with antipodal points of equatorial identified}

Define the CW structure $X_0 \subseteq X_1 \subseteq X_2 = X_3 = ... = X$ as follows

\begin{itemize}
    \item $X_0$ is a single point
    \item $X_1 \cong S^1$, $a^{(1)}_1: S^0_1 \to X_0$ maps two points of $S^0_1$ to $X_0$
    \item $X_2 = X$, $a^{(2)}_1$ and $a^{(2)}_2$ wind around $X_1$ twice in opposite directions and $D^2_1$ and $D^2_2$ are the northern hemisphere and southern hemisphere
\end{itemize}

The pushout diagrams are as follows (subscripts are used to distinguish between multiple copies)
\begin{center}
\begin{tikzcd}
S^0_1 \arrow[dd, "a^{(1)}_\bullet"] \arrow[rr] &  & D^1_1 \arrow[dd, "c^{(1)}_\bullet"] &  & S^1_1 \amalg S^1_1 \arrow[dd, "a^{(2)}_\bullet"] \arrow[rr] &  & D^2_2 \amalg D^2_2 \arrow[dd, "c^{(2)}_\bullet"] \\
                                               &  &                                     &  &                                                             &  &                                                  \\
X_0 \arrow[rr]                                 &  & X_1 \cong S^1                       &  & X_1 \cong S^1 \arrow[rr]                                    &  & X_2                                             
\end{tikzcd}
\end{center}

We have the cellular chain complex

\begin{center}
\begin{tikzcd}
0 & C^{CW}_0(X) = \Z \arrow[l, "d_0"'] & C^{CW}_1(X) = \Z \arrow[l, "d_1"'] & C^{CW}_2(X) = \Z^2 \arrow[l, "d_2"'] & 0 \arrow[l, "d_3"'] & ... \arrow[l]
\end{tikzcd}
\end{center}

We have

$$
    \Z = H_0(X) = H^{CW}_0(X) = \frac{\ker d_0}{\im d_1} = \frac{\Z}{\im d_1}
$$

then, the map $d_1 = 0$ and $H_0(X) = \Z$ since $\im d_1 = k\Z$ for $k \in \Z$ implies $k=0$. As $a^{(2)}_1, a^{(2)}_2$ composed with the quotient map collapsing $X_1 - e^1_1$ are maps $S^1 \to S^1$ that wind around $S^1$ twice in opposite directions, then $d_{1 1} = +2, d_{1 2} = -2$, and

\begin{align*}
    d_2(e^2_1) &= d_{1 1} e^1_1 = + 2 e^1_1\\
    d_2(e^2_2) &= d_{1 2} e^1_1 = - 2 e^1_1
\end{align*}

That is, $\im d_2 = 2 \Z$ and $\ker d_2 = \Span (1, 1)$. Hence, 
$$
    H_1(X) = \frac{\ker d_1}{\im d_2} = \frac{\Z}{2\Z}
$$

$C^{CW}_3(X) = C^{CW}_4(X) = ... = 0$  as $\mathcal{A}_3 = \mathcal{A}_4 = ... = \emptyset$, then $d_3 = d_4 = ... = 0$. Hence,

\begin{align*}
    H_2(X) &= \frac{\ker d_2}{\im d_3} \cong \frac{\Z}{0} = \Z \\
    H_3(X) &= H_4(X) = ... = 0
\end{align*}

\subsection{$S^3$ with antipodal points of equatorial identified}

Define the CW structure $X_0 \subseteq X_1 \subseteq X_2 \subseteq X_3 = X_4 = ... = X$ as follows

\begin{itemize}
    \item $X_0$ is a single point
    \item $X_1 \cong S^1$, $a^{(1)}_1$ maps two points of $S^0_1$ to $X_0$
    \item $X_2 \cong \R P^2$, $a^{(2)}_1$ winds around $X_1$ twice.
    \item $X_3 = X$, $a^{(3)}_1, a^{(3)}_2$ are maps from $S^2$ to $X_2 \cong \R P^2$ identifying antipodal points on $S^2$ to the same point on $\R P^2$ and $D^3_1, D^3_2$ are northern hemisphere and southern hemisphere.
\end{itemize}

The pushout diagrams are as follows (subscripts are used to distinguish between multiple copies)

\begin{center}
\begin{tikzcd}
S^0_1 \arrow[dd, "a^{(1)}_\bullet"] \arrow[rr] &  & D^1_1 \arrow[dd, "c^{(1)}_\bullet"] & S^1_1 \arrow[dd, "a^{(2)}_\bullet"] \arrow[rr] &  & D^2_1 \arrow[dd, "c^{(2)}_\bullet"] & S^2_1 \amalg S^2_2 \arrow[dd, "a^{(3)}_\bullet"] \arrow[rr] &  & D^3_1 \amalg D^3_2 \arrow[dd, "c^{(3)}_\bullet"] \\
                                               &  &                                     &                                                &  &                                     &                                                             &  &                                                  \\
X_0 \arrow[rr]                                 &  & X_1 \cong S^1          & X_1 \cong S^1 \arrow[rr]          &  & X_2 \cong \R P^2                    & X_2 \cong \R P^2 \arrow[rr]                                 &  & X_3                                             
\end{tikzcd}
\end{center}

We have the cellular chain complex

\begin{center}
\begin{tikzcd}
0 & C^{CW}_0(X) = \Z \arrow[l, "d_0"'] & C^{CW}_1(X) = \Z \arrow[l, "d_1"'] & C^{CW}_2(X) = \Z \arrow[l, "d_2"'] & C^{CW}_3(X) = \Z^2 \arrow[l, "d_3"'] & 0 \arrow[l, "d_4"'] & ... \arrow[l]
\end{tikzcd}
\end{center}

Similar to previous part, $d_1 = 0$ and $H_0(X) = \Z$. As $a^{(2)}_1$ composed with the quotient map collapsing $X_1 - e^1_1$ is map $S^1 \to S^1$ that wind around $S^1$ twice, then $d_{1 1} = +2$, and

$$
    d_2(e^2_1) = d_{1 1} e^1_1 = +2 e^1_1
$$

That is, $\im d_2 = 2\Z$ and $\ker d_2 = 0$. Hence
$$
    H_1(X) = \frac{\ker d_1}{\im d_2} = \frac{\Z}{2\Z}
$$

As $\ker d_2 = 0$, then $d_3 = 0$ because $\im d_3 \subseteq \ker d_2$, then $\ker d_3 = \Z^2$. Hence,

$$
    H_2(X) = \frac{\ker d_2}{\im d_3} = \frac{0}{0} = 0
$$

$C^{CW}_4(X) = C^{CW}_5(X) = ... = 0$  as $\mathcal{A}_4 = \mathcal{A}_5 = ... = \emptyset$, then $d_4 = d_5 = ... = 0$. Hence,

\begin{align*}
    H_3(X) &= \frac{\ker d_3}{\im d_4} \cong \frac{\Z^2}{0} = \Z^2 \\
    H_4(X) &= H_5(X) = ... = 0
\end{align*}



\end{document}
