\documentclass{article}
\usepackage{graphicx} % Required for inserting images
% header

%% natbib
\usepackage{natbib}
\bibliographystyle{plain}

%% comment
\usepackage{comment}

% indent the first paragraph
\usepackage{indentfirst}


%% math package
\usepackage{amsfonts}
\usepackage{amsmath}
\usepackage{amssymb}


%% operator
\DeclareMathOperator{\tr}{tr}
\DeclareMathOperator{\diag}{diag}
\DeclareMathOperator{\sign}{sign}
\DeclareMathOperator{\grad}{grad}
\DeclareMathOperator{\curl}{curl}
\DeclareMathOperator{\Div}{div}

%% theorems
\newtheorem{axiom}{Axiom}
\newtheorem{definition}{Definition}
\newtheorem{theorem}{Theorem}
\newtheorem{proposition}{Proposition}
\newtheorem{corollary}{Corollary}
\newtheorem{lemma}{Lemma}
\newtheorem{remark}{Remark}
\newtheorem{claim}{Claim}
\newtheorem{problem}{Problem}

%% empty set
\let\oldemptyset\emptyset
\let\emptyset\varnothing

% mathcal symbols
\newcommand\Tau{\mathcal{T}}
\newcommand\Ball{\mathcal{B}}

% mathbb symbols
\newcommand\N{\mathbb{N}}
\newcommand\Z{\mathbb{Z}}
\newcommand\Q{\mathbb{Q}}
\newcommand\R{\mathbb{R}}


\title{ma5259\_hw1}
\author{Nguyen Ngoc Khanh - A0275047B}
\date{September 2024}

\begin{document}

\maketitle

\section{Lecture Exercises}

\subsection{Lecture 1 Exercise 1}

If $E, F \subseteq \Omega$ are two events, then we can define two random variables $1_E, 1_F: \Omega \to \R$ with $1_E(\omega) = 1_{w \in E}$ and $1_F(\omega) = 1_{w \in F}$. Show the independence of the events $E$ and $F$ is equivalent to the independence of random variables $1_E$ and $1_F$

\subsubsection{Independence of $E, F$ implies independence of $1_E, 1_F$}

Let $X: \Omega \to \R^2$ be the joint random variable defined by $X(\omega) = (1_E(\omega), 1_F(\omega))$. For any Borel measurable rectangle $A \times B \subseteq \R^2$, we have $A \times B = (A \times \R) \cap (\R \times B)$, therefore

$$
    X^{-1} (A \times B)
    = X^{-1}(A \times \R) \cap X^{-1}(\R \times B)
    = 1_E^{-1} A \cap 1_F^{-1} B
$$

Note that, the independence of $E, F \subseteq \Omega$ implies the following independence:
\begin{align*}
    P(E \cap F) &= P(E) P(F) \\
    P(E \cap F^C) &= P(E) P(F^C) \\
    P(E^C \cap F) &= P(E^C) P(F) \\
    P(E^C \cap F^C) &= P(E^C) P(F^C) \\
\end{align*}

As $1_E^{-1} A$ is either $E$ or $E^C$, $1_F^{-1} A$ is either $F$ or $F^C$, then the two events $1_E^{-1} A$ and $1_F^{-1} B$ are independent for all Borel measurable sets $A, B \subseteq \R$, therefore
$$
    P(1_E \in A, 1_F \in B)
    = P(X^{-1}(A \times B))
    = P(1_E^{-1} A \cap 1_F^{-1} B)
    = P(1_E^{-1} A) P(1_F^{-1} B)
    = P(1_E \in A) P(1_F \in B)
$$

\subsubsection{Independence of $1_E, 1_F$ implies independence of $E, F$}

Similarly, the independence of $1_E, 1_F$ implies for all measurable sets $A, B$
$$
    P(1_E^{-1} A \cap 1_F^{-1} B) = P(1_E \in A, 1_F \in B) = P(1_E \in A) P(1_F \in B) = P(1_E^{-1} A) P(1_F^{-1} B)
$$

Take $A = B = \set{1}$, then
$$
    P(E \cap F) = P(E) P(F)
$$

\subsection{Lecture 1 Exercise 2}

Let $X$ and $Y$ denote the outcome of two independent fair coin tosses. Let $Z := Head$ if $X = Y$ and $Z := Tail$ if $X \neq Y$. Show that $X, Y, Z$ are pairwise independent but not jointly independent.

\subsubsection{$X, Y, Z$ are pairwise independent}

$X, Y$ are independent by the premise. We need to prove the independence of the pair $X, Z$.

Let $X, Y, Z$ be defined on $\mathcal{X}, \mathcal{Y}, \mathcal{Z}$ ($\mathcal{X} = \mathcal{Y} = \mathcal{Z} = \set{H, T}$). Given any $(x, z) \in \mathcal{X} \times \mathcal{Z}$. Define the two disjoint sets
\begin{align*}
    \mathcal{F}_{xz} &= \set{y \in \mathcal{Y}: P(Z=z|X=x, Y=y) = 1} \\
    \mathcal{G}_{xz} &= \set{y \in \mathcal{Y}: P(Z=z|X=x, Y=y) = 0}
\end{align*}

By the premise, we have $\mathcal{F}_{xz} \amalg \mathcal{G}_{xz} = \mathcal{Y}$ and $|\mathcal{F}_{xz}| = 1$, therefore
\begin{align*}
    P(X=x, Z=z)
    &= \sum_{y \in \mathcal{Y}} P(X=x, Y=y, Z=z) &\text{(marginalize)} \\
    &= \sum_{y \in \mathcal{Y}} P(X=x, Y=y) P(Z=z | X=x, Y=y) &\text{(conditional probability)} \\
    &= \sum_{y \in \mathcal{F}_{xz}} P(X=x, Y=y) &\text{(definition of $\mathcal{F}_{xz}, \mathcal{G}_{xz}$)} \\
    &= \sum_{y \in \mathcal{F}_{xz}} P(X=x) P(Y=y) &\text{($X, Y$ are independent)} \\
    &= P(X=x) P(Y=y) &\text{($|\mathcal{F}_{xz}| = 1$)} \\ 
    &= P(X=x) P(Z=z) &\text{($P(Y=y) = P(Z=z) = 1/2$ for all $y, z$)}
\end{align*}

\subsubsection{$X, Y, Z$ are not jointly independent}

We have $P(X=H) = P(Y=H) = P(Z=H) = 1/2$, so $P(X=H)P(Y=H)P(Z=H) = 1/8$ but
\begin{align*}
    P(X=H, Y=H, Z=H)
    &= P(Z=H | X=H, Y=H) P(X=H, Y=H) \\
    &= 1 \cdot (1/2 \cdot 1/2) = 1/4
\end{align*}

\subsection{Lecture 2 Exercise 1}

Let $(\Omega, F, P)$ be a probability space. Show the following are all equivalent given $P$ is finitely additive

\begin{enumerate}
    \item $P$ is countably additive on $F$

    \item For any $A_1 \subseteq A_2 \subseteq ... \subseteq \Omega$ with $A_i \in F$,
    $$
        P \tuple*{\bigcup_{n \in \N} A_n} = \lim_{n \to \infty} P(A_n)
    $$

    \item For any $\Omega \supseteq A_1 \supseteq A_2 \supseteq ...$ with $A_i \in F$,
    $$
        P \tuple*{\bigcap_{n \in \N} A_n} = \lim_{n \to \infty} P(A_n)
    $$

    \item For any $\Omega \supseteq A_1 \supseteq A_2 \supseteq ...$ with $\bigcap_{n \in \N} A_n = \emptyset$, $P(A_n) \searrow 0$ as $n \to \infty$
\end{enumerate}

\subsubsection{$(1 \implies 2)$}

Let $B_1 = A_1$, $B_2 = A_2 \setminus A_1$, $B_3 = A_3 \setminus A_2$, ..., $B_i = A_i \setminus A_{i-1}$. So that, $\set{B_n}_{n \in \N}$ is a collection of disjoint measurable sets, $A_n = \coprod_{i=1}^n B_i$ and $\bigcup_{n \in \N} A_n = \coprod_{n \in \N} B_n$. We have
\begin{align*}
    P \tuple*{\bigcup_{n \in \N} A_n}
    &= P \tuple*{\coprod_{n \in \N} B_n} \\
    &= \sum_{n=1}^\infty P(B_n) &\text{($B_n$ disjoint, $P$ is countably additive)} \\
    &= \lim_{n \to \infty} \sum_{i=1}^n P(B_i) &\text{(infinite sum)} \\
    &= \lim_{n \to \infty} P \tuple*{\coprod_{i = 1}^n B_i} &\text{($B_i$ disjoint, $P$ is finite additive)} \\
    &= \lim _{n \to \infty} P(A_n)
\end{align*}

\subsubsection{$(2 \implies 3)$}

Let $B_n = \Omega \setminus A_n$ so that $B_1 \subseteq B_2 \subseteq ...$ and $\bigcap_{n \in \N} A_n = \Omega \setminus \bigcup_{n \in \N} B_n$. By finite additivity, $P(A_n) = 1 - P(B_n)$, $P \tuple*{\bigcap_{n \in \N} A_n} = 1 - P \tuple*{\bigcup_{n \in \N} B_n}$. By (2), $\lim_{n \to \infty} P(B_n) = P \tuple*{\bigcup_{n \in \N} B_n}$. Hence, $\lim_{n \to \infty} P(A_n)$ exists and

\begin{align*}
    \lim_{n \to \infty} P(A_n)
    &= 1 - \lim_{n \to \infty} P(B_n) \\
    &= 1 - P \tuple*{\bigcup_{n \in \N} B_n} &\text{(by (2))}\\
    &= P \tuple*{\bigcap_{n \in \N} A_n}
\end{align*}
        

\subsubsection{$(3 \implies 4)$}

By (3)

$$
    \lim_{n \to \infty} P(A_n) = P \tuple*{\bigcap_{n \in \N} A_n} = P(\emptyset) = 0
$$

\subsubsection{$(4 \implies 1)$}

Let $\set{B_n}_{n \in \N}$ be a collection of disjoint measurable sets. Let $A_n = \coprod_{i=n+1}^\infty B_i$ so that $\coprod_{n \in \N} B_n = A_0 \supseteq A_1 \supseteq ...$ and $\bigcap_{n \in \N_0} A_n = \emptyset$. By finite additivity, $\sum_{i=1}^n P(B_n) = P\tuple*{\coprod_{i=1}^n B_n} = P(A_1 \setminus A_n) = P(A_1) - P(A_n)$. By (4), $\lim_{n \to \infty} P(A_n) = P\tuple*{\bigcap_{n \in \N_0} A_n} = P(\emptyset) = 0$, then
$$
    \sum_{i=1}^\infty P(B_n) = \lim_{n \to \infty} \sum_{i=1}^n P(B_n) = P(A_1) - \lim_{n \to \infty} P(A_n) = P(A_1) = P \tuple*{\bigcup_{n \in \N} B_n}
$$

\subsection{Lecture 2 Exercise 2}

Let $Z$ be a uniform random variable on $[0, 1]$. How can one generate from $Z$ discrete random variable $X$ with geometric distribution $P(X=n) = 2^{-n}$ for each $n \in \N$?

\begin{proof}[Answer]
    Let $X: [0, 1] \to \N$ be defined by
    $$
        X = \begin{cases}
            n &\text{if $Z \in (2^{-n}, 2 \cdot 2^{-n}]$} \\
            1 &\text{if $Z = 0$}
        \end{cases}    
    $$
    So that if $P(X = 1) = P(Z \in (1/2, 1]) + P(Z = 0) = 2^{-1}$ and if $n > 1$
    $$
        P(X = n) = P(Z \in (2^{-n}, 2 \cdot 2^{-n}]) = 2 \cdot 2^{-n} - 2^{-n} = 2^{-n}
    $$
    
\end{proof}

\subsection{Lecture 3 Exercise 1}

Construct a random variable $X$ and $Y$ such that $\Cov(X, Y) = 0$ but $X, Y$ are not independent. However, show that if $X, Y$ are Bernoulli random variables, then $\Cov(X, Y) = 0$ implies that $X, Y$ are independent

\subsubsection{Construction of $X, Y$ so that $\Cov(X, Y) = 0$ but $X, Y$ are not independent}

Let $\Omega = \set{-1, +1}^2$ with the $\sigma$-algebra $F$ consists of all subsets of $\Omega$ and a uniform probability measure $P: F \to \R$. Let $X: \Omega \to \R$ be defined by $X(a, b) = a$ and $Y: \Omega \to \R$ be defined by
\begin{align*}
    Y: \Omega &\to \R \\
    (-1, -1) &\mapsto 0 \\
    (-1, +1) &\mapsto 0 \\
    (+1, -1) &\mapsto -1 \\
    (+1, +1) &\mapsto +1
\end{align*}
So that $\Exp[X] = 0$, $\Exp[Y] = 0$, $\Cov(X, Y) = \Exp[XY]$. $XY$ have the following values
\begin{align*}
    XY: \Omega &\to \R \\
    (-1, -1) &\mapsto 0 \\
    (-1, +1) &\mapsto 0 \\
    (+1, -1) &\mapsto (+1)(-1) = -1 \\
    (+1, +1) &\mapsto (+1)(+1) = +1 \\
\end{align*}
Hence, $\Cov(X, Y) = \Exp[XY] = 0$. On the other hand, $P(X=+1) = 1/2$, $P(Y=0) = 1/2$, but
$$
    P(X=+1, Y=0) = P(Y=0 | X=+1) P(X=+1) = 0
$$

\subsubsection{$\Cov(X, Y) = 0$ implies $X, Y$ are independent given $X, Y$ are Bernoulli random variables}

Let $X, Y$ be Bernoulli random variables
\begin{align*}
    \Cov(X, Y)
    &= \Exp[XY] - \Exp[X]\Exp[Y] \\
    &= P(X=1, Y=1) - P(X=1) P(Y=1)
\end{align*}
Hence, $\Cov(X, Y) = 0$ implies $P(X=1, Y=1) = P(X=1) P(Y=1)$ implies $P(X=x, Y=y) = P(X=x) P(Y=y)$ for all $x, y \in \set{0, 1}$, that is, $X, Y$ are independent
        

\subsection{Lecture 4 Exercise 1}

\begin{enumerate}
    \item $X \sim N(0, \sigma^2)$ has characteristic function $\phi(t) = e^{- \frac{\sigma^2 t^2}{2}}$. Compute the first 4 moments of $X$

    \item Compute the characteristic of $X \sim Pois(\lambda)$
\end{enumerate}

\subsubsection{$X \sim N(0, \sigma^2)$, $\phi(t) = e^{- \frac{\sigma^2 t^2}{2}}$, compute the first 4 moments}

We have

\begin{align*}
    \phi^{(1)}(t)
    &= \tuple*{e^{- \frac{\sigma^2 t^2}{2}}} \tuple*{-\frac{\sigma^2}{2}} 2t \\
    &= -\sigma^2 t \phi(t) \\
    \phi^{(2)}(t) 
    &= -\sigma^2 \phi(t) - \sigma^2 t \phi^{(1)}(t) \\
    \phi^{(3)}(t)
    &= -\sigma^2 \phi^{(1)}(t) - \tuple*{\sigma^2 \phi^{(1)}(t) + \sigma^2 t \phi^{(2)}(t)} \\
    &= - 2 \sigma^2 \phi^{(1)}(t) - \sigma^2 t \phi^{(2)}(t) \\
    \phi^{(4)}(t)
    &= - 2 \sigma^2 \phi^{(2)}(t) - \tuple*{\sigma^2 \phi^{(2)}(t) + \sigma^2 t \phi^{(3)}(t)} \\
    &= - 3 \sigma^2 \phi^{(2)}(t) - \sigma^2 t \phi^{(3)}(t)
\end{align*}

So,
\begin{align*}
    \phi(0) &= 1 \\
    \phi^{(1)}(0) &= 0 \\
    \phi^{(2)}(0) &= -\sigma^2 \\
    \phi^{(3)}(0) &= 0 \\
    \phi^{(4)}(0) &= (-3 \sigma^2)(-\sigma^2) = 3 \sigma^4
\end{align*}

As $\phi^{(k)}(0) = i^k \Exp[X^k]$, 

\begin{align*}
    \Exp[X] &= 0 \\
    \Exp[X^2] &= \sigma^2 \\
    \Exp[X^3] &= 0 \\
    \Exp[X^4] &= 3 \sigma^4 \\
\end{align*}


\subsubsection{$X \sim Pois(\lambda)$, compute the characteristic function}

Poisson distribution
$$
    p(k) = \frac{\lambda^k e^{-\lambda}}{k!}
$$

Characteristic function
\begin{align*}
    \phi(t)
    &= \int_{\N} e^{itk} d p(k) \\
    &= \sum_{k=0}^\infty e^{itk} p(k) \\
    &= \sum_{k=0}^\infty e^{itk} \frac{\lambda^k e^{-\lambda}}{k!} \\
    &= e^{-\lambda} \sum_{k=0}^\infty \frac{(\lambda e^{it} )^k}{k!} \\
    &= e^{-\lambda} e^{\lambda e^{it}} \\
    &= e^{- \lambda + \lambda e^{it}}
\end{align*}


\section{Homework Questions}

\subsection{Q1}
Let $X_1, X_2$ be two independent exponential random variables with parameters $\lambda_1, \lambda_2 > 0$ respectively. In other words, $P(X_1 > x) = e^{- \lambda_1 x}$ and $P(X_2 > x) = e^{-\lambda_2 x}$ for all $x \geq 0$. Let $X := \min \set{X_1, X_2}$. Show that $X$ is again an exponential random variable with parameter $\lambda = \lambda_1 + \lambda_2$

\begin{proof}

\begin{align*}
    P(X > x)
    &= P(X_1 > x, X_2 > x) &\text{($\set{\min\set{X_1, X_2} > x}$ and $\set{X_1 > x} \cap \set{X_2 > x}$ are the same subset in $\Omega$)}\\
    &= P(X_1 > x) P(X_2 > x) &\text{($X_1, X_2$ are independent)} \\
    &= e^{- \lambda_1 x} e^{-\lambda_2 x} \\
    &= e^{- (\lambda_1 + \lambda_2) x}
\end{align*}

As $P(X > x)$ determines the distribution of $X$ and it matches the exponential distribution with parameter $\lambda = \lambda_1 + \lambda_2$, therefore, $X$ is an exponential random variable with parameter $\lambda = \lambda_1 + \lambda_2$

\end{proof}

\subsection{Q2}

$12$ people stand in a circle. Independently, every pair of neighbours decide to link arms with probability $1/2$. This breaks $12$ people into disjoint groups. What is the expected number of groups this results in?

\begin{proof}[Answer]

    Let $X_1, ..., X_{12}$ be $12$ i.i.d Bernoulli random variables with $p = 1/2$, $X_i$ represents whether there is a missing link between person $i$ and person $i+1$ (or $12$ and $1$). Let $X = X_1 + ... + X_{12}$, then $X$ is the number of missing links. Let $Y$ be the number of connected components, then
    $$
        Y = \begin{cases}
            1 &\text{if $X=0, 1$} \\
            X &\text{otherwise}
        \end{cases}
    $$

    We have
    \begin{align*}
        \Exp[Y]
        &= \sum_{k=1}^{12} k P(Y = k) \\ 
        &= P(Y = 1) + \sum_{k=2}^{12} k P(Y=k) \\
        &= P(Y = 1) + \sum_{k=2}^{12} k P(X=k) &\text{($P(Y=k) = P(X=k)$ if $k \geq 2$)}\\
        &= P(Y = 1) - P(X=1) + \sum_{k=0}^{12} k P(X=k)\\
        &= P(Y = 1) - P(X = 1) + \Exp[X] \\
        &= (P(X = 0) + P(X = 1)) - P(X = 1) + \Exp[X] &\text{($P(Y = 1) = P(X = 0) + P(X = 1)$)}\\
        &= P(X = 0) + \Exp[X] \\
    \end{align*}

    We have $P(X=0) = P(X_1=0, X_2=0, ..., X_{12}=0) = \frac{1}{2^{12}}$ and
    \begin{align*}
        \Exp[X]
        &= \Exp[X_1 + ... + X_{12}] \\
        &= \Exp[X_1] + ... + \Exp[X_{12}] &\text{(independent)} \\
        &= 12 \Exp[X_1] = 6&\text{(identical)}
    \end{align*}

    Hence, $\Exp[Y] = \frac{1}{2^{12}} + 6$

\end{proof}

\subsection{Q3}

We toss a fair coin repeatedly. What is the expected number of coin tosses it takes to observe Head followed by two consecutive Tails?

\begin{proof}[Answer]

Let $X = (X_n)_{n \in \N}$ be a sequence of i.i.d Bernoulli random variables of probability $p=1/2$ each represents a coin toss. Let $Y = (Y_n)_{n \in \N}$ be another sequence such that
$$
    Y_n = (X_n, X_{n+1}, X_{n+2})
$$

Then, $Y$ is a time-homogeneous Markov chain with state space
$$
    S = \set{TTT, TTH, THT, THH, HTT, HTH, HHT, HHH}
$$

and transition matrix $\Pi$ and $Y_1$ is distributed uniformly on $S$. For any $n \in \N$, define $g: \N \to \R$ by
$$
    g(n) = P(Y_1 \neq HTT, Y_2 \neq HTT, ..., Y_n \neq HTT)
$$

Then, by defintion of conditional probability
$$
    g(n) = P(Y_n \neq HTT | Y_1 \neq HTT, ..., Y_{n-1} \neq HTT) g(n-1)
$$

We have

\begin{align*}
    & P(Y_n \neq HTT | Y_1 \neq HTT, ..., Y_{n-1} \neq HTT) \\
    &= P(Y_n \neq HTT | Y_{n-1} \neq HTT) &\text{(Markov property)}\\
    &= 1 - P(Y_n = HTT | Y_{n-1} \neq HTT) &\text{(complement event)}\\
    &= 1 - \frac{P(Y_n = HTT, Y_{n-1} \neq HTT)}{P(Y_{n-1} \neq HTT)} &\text{(conditional probability)}\\
    &= 1 - \frac{P(Y_n = HTT) - P(Y_n = HTT, Y_{n-1} = HTT)}{1 - P(Y_{n-1} = HTT)} &\text{(complement event)}\\
\end{align*}

For all $n \in \N$, $P(Y_n = HTT) = P(X_n=H, X_{n+1}=T, X_{n+2}=T) = 1/8$ and $P(Y_n = HTT, Y_{n-1} = HTT) = 0$, then
$$
    P(Y_n \neq HTT | Y_1 \neq HTT, ..., Y_{n-1} \neq HTT) = 1 - \frac{1/8}{1 - 1/8} = \frac{6}{7}
$$

As $g(1) = P(Y_1 \neq HTT) = 1 - P(Y_1 = HTT) = 7/8$, then, for all $n \in \N$
$$
    g(n) = \frac{7}{8} \tuple*{\frac{6}{7}}^{n-1}
$$

Let $f: \N - \set{1} \to \R$ be defined by
$$
    f(n) = P(Y_1 \neq HTT, Y_2 \neq HTT, ..., Y_{n-1} \neq HTT, Y_n = HTT)
$$

Then $f(n) = g(n-1) - g(n)$ for all $n \in \N - \set{1}$. We extend the domain of $f$ by defining $f(1) = P(Y_n = HTT) = \frac{1}{8}$. Now, on the space $\Hom(\N, S)$ of all sequences of realizations of $Y_i$, we define function $\phi: \Hom(\N, S) \to \N$ by the first time observing $Y_i = HTT$. Then $\Exp[\phi]$ is the expected time to observe $Y_i = HTT$. We can decompose $\Hom(\N, S)$ into a sequence of disjoint events

$$
    \Hom(\N, S) = \set{Y_n \neq HTT: n \in \N} \amalg \set{Y_1 = HTT} \amalg \set{Y_1 \neq HTT, Y_2 = HTT} \amalg \amalg \set{Y_1 \neq HTT, Y_2 \neq HTT, Y_3 = HTT} \amalg ...
$$

with $P(\set{Y_n \neq HTT: n \in \N}) = 0$. Therefore,

\begin{align*}
    \Exp[\phi]
    &= \sum_{n=1}^\infty n f(n) \\
    &= f(1) + \sum_{n=2}^\infty n f(n) \\
    &= f(1) + \sum_{n=2}^\infty n (g(n-1) - g(n)) \\
    &= f(1) + \sum_{n=2}^\infty n g(n-1) - \sum_{n=2}^\infty n g(n) &\text{suppose $\sum_{n=2}^\infty n g(n) < \infty$}\\
    &= f(1) + \tuple*{\sum_{n=2}^\infty (n-1) g(n-1) + \sum_{n=2}^\infty g(n-1)} - \sum_{n=2}^\infty n g(n) \\
    &= f(1) + \sum_{n=1}^\infty g(n) + \sum_{n=1}^\infty n g(n) - \sum_{n=2}^\infty n g(n) \\
    &= f(1) + \sum_{n=1}^\infty g(n) + g(1) \\
    &= f(1) + g(1) + \frac{7}{8} \sum_{n=1}^\infty \tuple*{\frac{6}{7}}^n \\
    &= \frac{1}{8} + \frac{7}{8} + \frac{7}{8} \frac{6/7}{1 - 6/7} = \frac{25}{4} = 6.25
\end{align*}

Expected number of coin tosses $6.25 + 2 = 8.25$

\begin{lemma}
    Let $\alpha \in (0, 1)$, $\sum_{n=1}^\infty n \alpha^n$ converges.
\begin{proof}
    Let $\alpha < \beta < 1$, then there exists $N \in \N$, such that for all $n > N$, $n < \tuple*{\frac{\beta}{\alpha}}^n$. We write 
    $$
        \sum_{n=1}^\infty n \alpha^n = \sum_{n=1}^N n \alpha^n + \sum_{n=N}^\infty n \alpha^n < \sum_{n=1}^N n \alpha^n + \sum_{n=N}^\infty \beta^n < \infty
    $$
\end{proof}
\end{lemma}

\end{proof}

\subsection{Q4}

Let $G_n$ be a random graph with $n$ vertices $\set{1, ..., n}$ where every pair of vertices is connected by an edge independently with probability $p \in (0, 1)$. A set of four vertices $\set{i, j, k, l}$ is said to form a square if there are exactly $4$ edges among them, forming a square.
\begin{enumerate}
    \item What is the expected number of squares in $G_n$?
    \item Find a suitable upperbound on variance of $N_n$ as $n \to \infty$ and use it to prove the weak law of large numbers for $\frac{N_n}{\Exp[N_n]}$. More precisely, show that
    $$
        P\tuple*{\abs*{\frac{N_n}{\Exp[N_n]} - 1} > \eps} \to 0
    $$

    as $n \to \infty$ for every $\eps$
\end{enumerate}

\subsubsection{Expected number of squares}
    
Let $[n] = \set{i \in \N: i < n}$, let $\mathcal{S}_m = \set{S \in \mathcal{P}([n]): |S| = m}$ be the collection of subsets of size $m$ of $n$ vertices. Let $f: \mathcal{S}_4 \to \set{0, 1}$ denote the function that has value $f(S) = 1$ if $S$ is a square. If $S = \set{a, b, c, d}$ then $f(S) = 1$ if and only if one of the following disjoint events happen ($ab=1$ denotes $a-b$ being an edge)
\begin{itemize}
    \item $E_1$: $ab=bc=cd=da=1, ac=bd=0$
    \item $E_2$: $ac=cd=db=ba=1, ad=bc=0$
    \item $E_3$: $ac=cb=bd=da=1, ab=cd=0$
\end{itemize}

Given $S = \set{a, b, c, d} \in \mathcal{S}_4$, then 
\begin{align*}
    \Exp[f(S)]
    &= P(E_1) + P(E_2) + P(E_3) &\text{(disjoint events)}\\
    &= 3 P(E_1) &\text{(symmetry)} \\
    &= 3 (P(ab=1) P(bc=1) P(cd=1) P(da=1) P(ac=0) P(bd=0)) &\text{(independent)} \\
    &= 3 p^4 (1-p)^2
\end{align*}

Let $\mu = \Exp[f(S)]$, let $N_n$ denote the number of squares in $G_n$, then $N_n = \sum_{S \in \mathcal{S}_4} f(S)$ and
\begin{align*}
    \Exp[N_n]
    &= \Exp\bracket*{\sum_{S \in \mathcal{S}_4} f(S)} \\
    &= \sum_{S \in \mathcal{S}_4} \Exp[f(S)] &\text{(linearity of expectation)} \\
    &= |\mathcal{S}_4| \mu = {n \choose 4} 3 p^4 (1-p)^2
\end{align*}
    
\subsubsection{Upper bound on variance of $N_n$ as $n \to \infty$}

Now, we bound the variance of $N_n$, for each $S \in \mathcal{S}_4$, let $\mu = \Exp[f(S)]$ and $\sigma^2 = \Var(f(S))$
\begin{align*}
    \Var(N_n)
    &= \Var\tuple*{\sum_{S \in \mathcal{S}_4} f(S)} \\
    &= \Exp\bracket*{
    \tuple*{\sum_{S \in \mathcal{S}_4} f(S) - \Exp\bracket*{\sum_{S \in \mathcal{S}_4} f(S)}}^2} \\
    &= \Exp\bracket*{
    \tuple*{\sum_{S \in \mathcal{S}_4} f(S) - |\mathcal{S}_4| \mu}^2} \\
    &= \Exp\bracket*{
    \tuple*{\sum_{S \in \mathcal{S}_4} (f(S) - \mu)}^2} \\
    &= \Exp\bracket*{
    \sum_{P \in \mathcal{S}_4} \sum_{Q \in \mathcal{S}_4} (f(P) - \mu)(f(Q) - \mu)} \\
    &= 
    \sum_{P \in \mathcal{S}_4} \sum_{Q \in \mathcal{S}_4} \Exp[(f(P) - \mu)(f(Q) - \mu)] &\text{(linearity of expectation)} \\
    &= 
    \sum_{S \in \mathcal{S}_4} \Exp[(f(S) - \mu)^2] + \sum_{(P, Q) \in \mathcal{S}_4^2: P \neq Q} \Exp[(f(P) - \mu)(f(Q) - \mu)] \\
    &= |\mathcal{S}_4| \sigma^2 + \sum_{(P, Q) \in \mathcal{S}_4^2: P \neq Q} \Exp[(f(P) - \mu)(f(Q) - \mu)]
\end{align*}

Note that, if $n$ is large, for each $P \in \mathcal{S}_4$, there are ${4 \choose 2} {n-4 \choose 2} + {4 \choose 3} {n-4 \choose 1} = O(n^2)$ possible choices of $Q \in \mathcal{S}_4$ so that $Q \neq P$ and $f(P)$ and $f(Q)$ are dependent (either $P$ and $Q$ share $2$ or $3$ vertices). Therefore, there are at most $n O(n^2) = O(n^3)$ dependent pairs of $P, Q$ with $P \neq Q$. For each dependent pair $P, Q$ with $P \neq Q$, by Cauchy–Schwarz
$$
    \Exp[(f(P) - \mu)(f(Q) - \mu)] \leq \sqrt{\Exp[(f(P) - \mu)^2] \Exp[(f(Q) - \mu)^2]} = \sigma^2
$$

Then, 
$$
    \sum_{(P, Q) \in \mathcal{S}_4^2: P \neq Q} \Exp[(f(P) - \mu)(f(Q) - \mu)] = O(n^3) \sigma^2
$$

Note that, $|\mathcal{S}_4| = O(n^4)$, then
$$
    \Var(N_n) = O(n^4) \sigma^2 + O(n^3) \sigma^2 = O(n^4) \sigma^2
$$

which is of the same $n$-order in the independent case. Therefore, it yields the weak law of large numbers for $N_n$

\subsection{Q5}

Tom throws a fair die repeatedly. If Tom throws the die $n$ times, and let $N_n$ denote the number of distinct faces of the die that he has seen, the find the mean and variance of $N_n$. If $Y$ denote the number of times Tom has to throw the die in order to see each face of the die at least once, then find the mean and variance of $Y$ (\note{variable name was changed from $X$ to $Y$})

\subsubsection{mean and variance of $N_n$}

$$
    N_n = \sum_{i=1}^6 f\tuple*{\bigcup_{m=1}^n \set{X_m = i}} = 6 - \sum_{i=1}^6 f\tuple*{\bigcap_{m=1}^n \set{X_m \neq i}}
$$

where $f(A) = 1$ if event $A$ occurs and $0$ otherwise. By linearity of expectation

$$
    \Exp[N_n] = 6 - \sum_{i=1}^6 \Exp\bracket*{f\tuple*{\bigcap_{m=1}^n \set{X_m \neq i}}}
$$

for each $i=1, ..., 6$,
$$
    f\tuple*{\bigcap_{m=1}^n \set{X_m \neq i}} = \begin{cases}
        1 &\text{with probability $P(X_1 \neq i, ..., X_n \neq i)$} \\
        0 &\text{with probability $1 - P(X_1 \neq i, ..., X_n \neq i)$}
    \end{cases}
$$

As $P(X_1 \neq i, ..., X_n \neq i) = \tuple*{\frac{5}{6}}^n$, then $\Exp\bracket*{f\tuple*{\bigcap_{m=1}^n \set{X_m \neq i}}} = \tuple*{\frac{5}{6}}^n$, then

$$
    \Exp[N_n] = 6 - 6 \tuple*{\frac{5}{6}}^n
$$

As $\Var(N_n) = \Var(6 - N_n)$, we have
\begin{align*}
    \Exp[(6 - N_n)^2]
    &= \Exp\bracket*{\tuple*{\sum_{i=1}^6 f\tuple*{\bigcap_{m=1}^n \set{X_m \neq i}}}^2} \\
    &= \Exp\bracket*{\sum_{i=1}^6 \sum_{j=1}^6 f\tuple*{\bigcap_{m=1}^n \set{X_m \neq i}} f\tuple*{\bigcap_{m=1}^n \set{X_m \neq j}} } \\
    &= \sum_{i=1}^6 \Exp\bracket*{f\tuple*{\bigcap_{m=1}^n \set{X_m \neq i}}^2 } + \sum_{i \neq j} \Exp\bracket*{f\tuple*{\bigcap_{m=1}^n \set{X_m \neq i}} f\tuple*{\bigcap_{m=1}^n \set{X_m \neq j}}} \\
\end{align*}

Left term:
$$
    \Exp\bracket*{f\tuple*{\bigcap_{m=1}^n \set{X_m \neq i}}^2 } = \Exp\bracket*{f\tuple*{\bigcap_{m=1}^n \set{X_m \neq i}} } = \tuple*{\frac{5}{6}}^n
$$

Right term: as
$$
    f\tuple*{\bigcap_{m=1}^n \set{X_m \neq i}} f\tuple*{\bigcap_{m=1}^n \set{X_m \neq j}} = \begin{cases}
        1 &\text{with probability $P(X_1 \neq i,j, ..., X_n \neq i,j)$} \\
        1 &\text{with probability $1 - P(X_1 \neq i,j, ..., X_n \neq i,j)$} \\
    \end{cases}
$$

As $P(X_1 \neq i,j, ..., X_n \neq i,j) = \tuple*{\frac{4}{6}}^n$, then
$$
    \Exp\bracket*{f\tuple*{\bigcap_{m=1}^n \set{X_m \neq i}} f\tuple*{\bigcap_{m=1}^n \set{X_m \neq j}}} = \tuple*{\frac{4}{6}}^n
$$

Therefore,

$$
    \Exp[(6 - N_n)^2] = 6 \tuple*{\frac{5}{6}}^n + \frac{6 \cdot 5}{2} \tuple*{\frac{4}{6}}^n
$$

$$
    \Var(N_n) = \Var(6 - N_n) = \Exp[(6 - N_n)^2] - \Exp[6 - N_n]^2 = \tuple*{\frac{5}{6}}^n + \frac{6 \cdot 5}{2} \tuple*{\frac{4}{6}}^n - 6^2 \tuple*{\frac{5}{6}}^{2n}
$$

\subsubsection{mean and variance of $Y$}

We have $\set{Y \leq n}$ is the event where we see $6$ faces in the first $n$ throws. As the distribution is uniform, we will use counting to calculate $P(Y \leq n)$. After the first $n$ throws

\begin{itemize}
    \item number of outcomes: $6^n$
    \item number of ways to see $6$ faces is (number of choice of subsets of size $6$) x (permutation of $6$) (the rest): ${n \choose 6} 6! 6^{n-6}$ 
\end{itemize}

Then 
$$
    P(Y \leq n) = \frac{6! {n \choose 6}  6^{n-6}}{6^n} = \frac{6!}{6^6} {n \choose 6}
$$

Then
$$
    P(Y=n) = P(Y \leq n) - P(Y \leq n-1) = \frac{(n-1) ... (n-5)}{6^5}
$$

\subsection{Q6}

We toss a biased coin repeatedly, where the probability of seeing Head equals $\frac{1}{n}$ for some $n \in \N$. Let $T_n$ denote the number of coin tosses needed to see Head for the first time, and let $N_n$ denote the number of Heads among the first $n$ coin tosses.
\begin{enumerate}
    \item Compute the mean and variance of $T_n$ and $N_n$ respectively.
    
    \item Show that as $n \to \infty$, the distribution of $\frac{T_n}{n}$ converges to the exponential distribution with parameter $1$. In other words, show that for each $x \in \R$
    $$
        P\tuple*{\frac{T_n}{n} > x} \to P(Z > x) = e^x
    $$

    as $n \to \infty$ where $Z$ is an exponential random variable with parameter $1$

    \item Show that as $n \to \infty$, the distribution of $N_n$ converges to the Poisson distribution with parameter $1$. In other words, show that for each $k \in \N_0 = \N \cup \set{0}$
    $$
        P(N_n = k) \to P(W = k) = e^{-1} \frac{1}{k!}
    $$

    as $n \to \infty$ where $W$ is a Poisson random variable with parameter $1$
\end{enumerate}

Let $p = \frac{1}{n}, q = 1 - p$

\subsubsection{mean and variance of $T_n$ and $N_n$}

As $P\tuple*{\bigcap_{i \in \N} \set{X_i = T}} = 0$

\begin{align*}
    P(T_n = k)
    &= P(X_1=T, X_2=T, ..., X_{k-1}=T, X_k = H) \\
    &= P(X_1=T)P(X_2=T) ... P(X_{k-1}=T) P(X_k = H) \\
    &= q^{k-1} p
\end{align*}

$T_n$ follows geometric distribution

\begin{lemma}[geometric series]
    If $x < 1$, then the series below converges
    \begin{align*}
        f(x)    &= \sum_{k=0}^\infty x^k = \frac{1}{1-x} \\
        f'(x)   &= \sum_{k=0}^\infty k x^{k-1} = \frac{1}{(1-x)^2} \\
        f''(x)  &= \sum_{k=0}^\infty k(k-1) x^{k-2} = \frac{2}{(1-x)^3} \\
    \end{align*}

    Furthermore, we can rewrite

    $$
        f''(x) = \sum_{k=1}^\infty k^2 x^{k-1} + f'(x)
    $$
    
\end{lemma}

We have

\begin{align*}
    \Exp[T_n]
    &= \sum_{k=1}^\infty k P(T_n = k) \\
    &= p \sum_{k=1}^\infty k q^{k-1}  \\
    &= p f'(q) = n
\end{align*}


\begin{align*}
    \Exp[T_n^2]
    &= \sum_{k=1}^\infty k^2 P(T_n = k) \\
    &= p \sum_{k=1}^\infty k^2 q^{k-1}  \\
    &= p \tuple*{ f''(q) - f'(q) } \\
    &= 2n^2 - n
\end{align*}

$$
    \Var(T_n) = \Exp[T_n^2] - \Exp[T_n]^2 = n^2 - n
$$

We have

\begin{align*}
    P(N_n = k) = {n \choose k} p^k q^{n-k}
\end{align*}

$N_n$ follows binomial distribution

\begin{lemma}[binomial]
\begin{align*}
    g(n, p, q)
    &= \sum_{k=0}^n k {n \choose k} p^k q^{n-k} \\
    &= \sum_{k=1}^n k {n \choose k} p^k q^{n-k} &\text{(change the range of $k$)}\\
    &= \sum_{k=1}^n n {n-1 \choose k-1} p^k q^{n-k} &\text{($k (n, k) = n (n-1, k-1)$)} \\
    &= np \sum_{k=1}^n {n-1 \choose k-1} p^{k-1} q^{(n-1) - (k-1)} \\
    &= np \sum_{l=0}^m {m \choose l} p^l q^{m-l} \\
    &= np &\text{(Binomial theorem)}
\end{align*}
\end{lemma}

We have

\begin{align*}
    \Exp[N_n] = \sum_{k=0}^n k P(N_n = k) = g(n, p, q) = np
\end{align*}

\begin{align*}
    \Exp[N_n^2]
    &= \sum_{k=0}^n k^2 P(N_n = k) \\
    &= \sum_{k=0}^n k^2 {n \choose k} p^k q^{n-k} \\
    &= \sum_{k=1}^n k^2 {n \choose k} p^k q^{n-k} &\text{(change the range of $k$)} \\
    &= \sum_{k=1}^n kn {n-1 \choose k-1} p^k q^{n-k} &\text{($k (n, k) = n (n-1, k-1)$)} \\
    &= np \sum_{k=1}^n k {n-1 \choose k-1} p^{k-1} q^{(n-1) - (k-1)} \\
    &= np \tuple*{\sum_{k=1}^n (k-1) {n-1 \choose k-1} p^{k-1} q^{(n-1) - (k-1)} + \sum_{k=1}^n {n-1 \choose k-1} p^{k-1} q^{(n-1) - (k-1)} }\\
    &= np \tuple*{g(n-1, p, q) + (p+q)^{n-1}} \\
    &= n^2 p^2 - n p^2 + np
\end{align*}

$$
    \Var(N_n) = \Exp[N_n^2] - \Exp[N_n]^2 = np (1-p)
$$

\subsubsection{$T_n / n$ converges to exponential distribution}

$$
    P(T_n > k) = \sum_{l=k+1}^\infty P(T_n=l) = \sum_{l=k+1}^\infty q^{l-1} p = \frac{q^k}{1-q} p = q^k
$$

For each $x \in \R$

$$
    P\tuple*{\frac{T_n}{n} > x} = P(T_n > \lfloor nx \rfloor) = q^{\lfloor nx \rfloor} 
$$

Let $m = nx$, then we have

$$
    \tuple*{1 - \frac{x}{m}}^m \leq P\tuple*{\frac{T_n}{n} > x} \leq \tuple*{1 - \frac{x}{m}}^{m+1}
$$

As $n \to \infty$, both side converges to $e^{-x}$. Hence, $P\tuple*{\frac{T_n}{n} > x} \to e^{-x}$

\subsubsection{$N_n$ converges to Poisson distribution}

\begin{align*}
    P(N_n = k)
    &= {n \choose k} p^k q^{n-k} \\
    &= \frac{n!}{k! (n-k)!} \frac{1}{n^k} \tuple*{1 - \frac{1}{n}}^{n-k} \\
    &= \frac{1}{k!} (1 + o(1)) \tuple*{1 - \frac{1}{n}}^{n-k} &\text{($o(1) \to 0$ as $n \to \infty$)}\\
\end{align*}

Then,
$$
    \lim_{n \to \infty} P(N_n = k) = \frac{1}{k!} \tuple*{\lim_{n \to \infty} \tuple*{1 - \frac{1}{n}}^n} \tuple*{\lim_{n \to \infty} \tuple*{1 - \frac{1}{n}}^{-k}} = \frac{1}{k!} e^{-1}
$$


\subsection{Q7}

A sequence of biased coins are flipped. The $r$-th coin has probability $\theta_r$ of showing Head where $\theta_r$ is a random variable in $[0, 1]$. Let $S_n$ be the number of Heads among the first $n$ coin tosses.

\begin{enumerate}
    \item Assuming that $(\theta_r)_{r \in \N}$ are i.i.d uniformly distributed on $[0, 1]$. Is there a centering and scaling of $S_n$ such that the Central Limit Theorem holds? If the answer is yes, then find the centering and scaling constants which lead to the standard normal distribution.

    \item What if $(\theta_r)_{r \in \N}$ all equal the same uniform variable $\theta$
\end{enumerate}

\subsubsection{$\theta_r$ are uniformly distributed on $[0, 1]$}


Let $X_r$ be a Bernoulli random variable of parameter $\theta_r$ for the $r$-th coin toss. As $X_r$ are i.i.d and

$$
    S_n = \sum_{r = 1}^n X_r 
$$

the question is to find the mean and variance of $X_r$. Consider the pair of random variable $(X_r, \theta_r): \Omega \to \set{0, 1} \times [0, 1]$, for each partition $\set{0 = x_0 < x_1 < ... < x_n = 1}$, we have

\begin{align*}
    P(X_r = 1)
    &= P(X_r = 1, \theta_r \in [0, 1]) \\
    &= \sum_{i=1}^n P(X_r = 1, \theta_r \in [x_{i-1}, x_i]) \\
    &= \sum_{i=1}^n P(X_r = 1 | \theta_r \in [x_{i-1}, x_i]) P(\theta_r \in [x_{i-1}, x_i]) \\
\end{align*}

As $x_{i-1} \leq P(X_r = 1 | \theta_r \in [x_{i-1}, x_i]) \leq x_i$ and $P(\theta_r \in [x_{i-1}, x_i])$, the sum is a Riemann sum of function $f: [0, 1] \to [0, 1]$ defined by $f(x) = x$. For each partition, the value of Riemann sum is fixed, therefore, the Riemann sum equals its limit

$$
    P(X_r = 1) = \sum_{i=1}^n P(X_r = 1 | \theta_r \in [x_{i-1}, x_i]) P(\theta_r \in [x_{i-1}, x_i]) = \int_0^1 f dx = \frac{1}{2}
$$

Hence, $\Exp[X_r] = \frac{1}{2}$ and $\Var(X_r) = \frac{1}{4}$. Scaling of $S_n$ is

$$
    \frac{S_n - n\Exp[X_r]}{\sqrt{n \Var(X_r)}}
$$

\subsubsection{$\theta_r$ is the same for all $r$}

Using the same method as above, we know $P(X_1 = 1) = P(X_2 = 1) = \frac{1}{2}$, for each partition $\set{0 = x_0 < x_1 < ... < x_n = 1}$, we have

$$
    P(X_1=1, X_2=1) = P(X_1=1, X_2=1, \theta \in [0, 1]) = \sum_{i=1}^n P(X_1=1, X_2=1 | \theta \in [x_{i-1}, x_i]) P(\theta \in [x_{i-1}, x_i])
$$

Similarly, we have $x_{i-1}^2 \leq P(X_1=1, X_2=1 | \theta \in [x_{i-1}, x_i]) \leq x_i^2$, then $P(X_1=1, X_2=1) = \int_0^1 x^2 dx = \frac{1}{3}$. Hence, $X_1$ and $X_2$ are not independent. The central limit theorem on $S_n$ no longer holds.

\subsection{Q8}

In Lecture 5, we compute the probability of Gambler A's run if A and B start with respectively \$$m$ and \$$n$, and they bet on a fair coin. Suppose now that the coin is biased with probability $p \in (0, 1)$ of seeing Head, while A still bets on seeing Head and B bets on seeing Tail. What is the probability of A's ruin as a function of $m$, $n$, and $p$?

\begin{longproof}[Answer]

Let $X_t$ denote the total wealth of $A$ at time $t$, then $X_0 = m$. Let $L = m+n$, let $E$ denote the event for A's ruin, then

$$
    P(E | X_0 = m) = \Exp[P(E | X_0 = m, X_1)] = p P(E | X_1 = m+1) + (1-p) P(E | X_1 = m-1)
$$

Let $f(k) = P(E | X_0 = k)$, we have the recurrence relation for all $1 \leq k \leq L-1$

$$
    f(k) = p f(k+1) + (1-p) f(k-1)
$$

And the boundary conditions: $f(0) = 1, f(L) = 0$. We have the characteristic function

$$
    r = p r^2 + (1-p)
$$

If $p = 1/2$, the case is reduced to Lecture 5, if $p \neq 1/2$, the characteristic function has two distinct roots

$$
    r_1 = \frac{1 + \sqrt{1 - 4p(p-1)}}{2p} \text{ and } r_2 = \frac{1 - \sqrt{1 - 4p(p-1)}}{2p}
$$

$f(k)$ is of the form

$$
    f(k) = A r_1^k + B r_2^k
$$

Using the boundary conditions we have

\begin{align*}
    1 &= A + B \\
    0 &= A r_1^L + B r_2^L
\end{align*}

Solve for $A, B$, we have $P(E) = f(m)$ can be written as a function of $m, n$ and $p$
    
\end{longproof}

\subsection{Q9}

Let $X_1, X_2, ...$ be a sequence of i.i.d Bernoulli random variables with parameter $1/2$. We can think of $(X_i)_{i \in \N}$ as indicator random variables for showing Head for a sequence of i.i.d fair coin tosses. Let us construct a random variable $Y$ from $X_1, X_2, ...$ by defining
$$
    Y = \sum_{n \in \N: X_n = 1} \frac{1}{2^n}
$$

Show that $Y$ is uniformly distributed on $[0, 1]$ if we treat $Y$ as a binary number $0. X_1 X_2 ...$

\begin{proof}

We need to show that $P(Y < y) = y$ for all $y \in (0, 1)$. Let $y = 0. x_1 x_2 ...$ be the binary representation of $y$, let $y_n = 0. x_1 x_2 ..., x_n = \frac{\lfloor 2^n y \rfloor}{2^n}$ be the number from the first $n$ bits of $y$. We have $\set{Y < y} = \bigcup_{n \in \N} \set{Y < y_n}$ and

$$
    \set{Y < y_1} \subseteq \set{Y < y_2} \subseteq ... \subseteq \set{Y < y}
$$

Therefore
$$
    P(Y < y) = \lim_{n \to \infty} P(Y < y_n)
$$

Now, we calculate $P(Y < y_n)$. Partition the space of all sequences into $2^n$ disjoint subsets where two sequences are in the same subset if they have the same first $n$ bits. The construction yields $2^n$ measurable subsets where each subset has the same probability $\frac{1}{2^n}$ and moreover each subset is either less than $y_n$ or greater than or equal $y_n$, there are exactly $2^n y_n$ subsets that are less than $y_n$ (for example, if $y_n = 0.10$, then all sequences starts with $0.00$ or $0.01$ are less than $y_n$), hence
 
$$
    P(Y < y_n) = y_n
$$

As $y_n \to y$ as $n \to \infty$,
$$
    P(Y < y) = \lim_{n \to \infty} P(Y < y_n) = \lim_{n \to \infty} y_n = y
$$



    
\end{proof}

\end{document}
