\subsection{test1}

Inspired by derivation of Ratio cut \cite{gallier}, we introduced method test1.

Let $A$ be symmetric, let $D$ be the degree matrix of $A$, $L = D - A$ be the laplacian matrix.
Let $x^{(1)}$, $x^{(2)}$, ..., $x^{(K)}$ be the indicator vector for each partition such that:
\[
x^{(k)}_i = \{
        \begin{array}{ll}
            1 \;\;\;\; \text{if $i \in V_k$}\\
            0 \;\;\;\; \text{otherwise}
        \end{array}
\]

Since $K$ partitions are disjoint, we always have a set of $K$ orthogonal vectors.

\[
x^{(k_1)T} x^{(k_2)} = 0 \;\;\; \forall k_1 \neq k_2
\]


Ratio cut minimizes the sum of all cuts divided by its corresponding partition volume.

\[
\sum_{k=1}^{K} \frac{x^{(k)T} L x^{(k)}}{x^{(k)T} x^{(k)}}
=
\sum_{k=1}^{K} \frac{cut(V_k, V \setminus V_k)}{|V_k|}
\]

subject to $x^{(k)} \in \{0, 1\}^{|V|}$ and $x^{(k_1)T} x^{(k_2)} = 0$ $\forall k_1 \neq k_2$

Furthermore, Ratio cut extends the domain the indicator vectors to real number.

\[
\sum_{k=1}^{K} \frac{x^{(k)T} L x^{(k)}}{x^{(k)T} x^{(k)}}
\]

subject to $x^{(k)} \in \mathbb{R}^{|V|}$ and $x^{(k_1)T} x^{(k_2)} = 0$ $\forall k_1 \neq k_2$

The problem of minimizing Rayleigh quotients with orthogonal constraints yields $K$-smallest eigen vectors.

In test1, we replaced laplacian matrix $L$ by adjacency matrix $A$. The objective of the formulation is

\[
O_1 = \sum_{k=1}^{K} \frac{x^{(k)T} A x^{(k)}}{x^{(k)T} x^{(k)}}
=
\sum_{k=1}^{K} \frac{\sum_{i=1}^{|V|} \sum_{j=1}^{|V|} x^{(k)}_i x^{(k)}_j A_{ij}}{\sum_{i=1}^{|V|} x^{(k)2}_i}
=
\sum_{k=1}^{K} \frac{\sum_{i \in V_k} \sum_{j \in V_k} A_{ij}}{|V_k|}
= 
\sum_{k=1}^{K} (1 - \frac{1}{|V_k|}) C_a^{(k)}
\]

Which is approximately equal to the sum of average cycle length.

