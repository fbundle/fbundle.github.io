\documentclass{article}
% header

%% natbib
\usepackage{natbib}
\bibliographystyle{plain}

%% comment
\usepackage{comment}

% no automatic indentation
\usepackage{indentfirst}

% manually indent
\usepackage{xargs} % \newcommandx
\usepackage{calc} % calculation
\newcommandx{\tab}[1][1=1]{\hspace{\fpeval{#1 * 10}pt}}
% \newcommand[number of parameters]{output}
% \newcommandx[number of parameters][parameter index = x]{output}
% use parameter index = x to substitute the default argument
% use #1, #2, ... to get the first, second, ... arguments
% \tab for indentation
% \tab{2} for for indentation twice

% note
\newcommandx{\note}[1]{\textit{\textcolor{red}{#1}}}
\newcommand{\todo}{\note{TODO}}
% \note{TODO}

%% math package
\usepackage{amsfonts}
\usepackage{amsmath}
\usepackage{amssymb}
\usepackage{tikz-cd}
\usepackage{mathtools}
\usepackage{amsthm}

%% operator
\DeclareMathOperator{\tr}{tr}
\DeclareMathOperator{\diag}{diag}
\DeclareMathOperator{\sign}{sign}
\DeclareMathOperator{\grad}{grad}
\DeclareMathOperator{\curl}{curl}
\DeclareMathOperator{\Div}{div}
\DeclareMathOperator{\card}{card}
\DeclareMathOperator{\Span}{span}
\DeclareMathOperator{\real}{Re}
\DeclareMathOperator{\imag}{Im}
\DeclareMathOperator{\supp}{supp}
\DeclareMathOperator{\im}{im}
\DeclareMathOperator{\aut}{Aut}
\DeclareMathOperator{\inn}{Inn}
\DeclareMathOperator{\Char}{char}
\DeclareMathOperator{\Sylow}{Syl}
\DeclareMathOperator{\coker}{coker}
\DeclareMathOperator{\inc}{in}
\DeclareMathOperator{\Sd}{Sd}
\DeclareMathOperator{\Hom}{Hom}
\DeclareMathOperator{\interior}{int}
\DeclareMathOperator{\ob}{ob}
\DeclareMathOperator{\Set}{Set}
\DeclareMathOperator{\Top}{Top}
\DeclareMathOperator{\Meas}{Meas}
\DeclareMathOperator{\Grp}{Grp}
\DeclareMathOperator{\Ab}{Ab}
\DeclareMathOperator{\Ch}{Ch}
\DeclareMathOperator{\Fun}{Fun}
\DeclareMathOperator{\Gr}{Gr}
\DeclareMathOperator{\End}{End}
\DeclareMathOperator{\Ad}{Ad}
\DeclareMathOperator{\ad}{ad}
\DeclareMathOperator{\Bil}{Bil}
\DeclareMathOperator{\Skew}{Skew}
\DeclareMathOperator{\Tor}{Tor}
\DeclareMathOperator{\Ho}{Ho}
\DeclareMathOperator{\RMod}{R-Mod}
\DeclareMathOperator{\Ev}{Ev}
\DeclareMathOperator{\Nat}{Nat}
\DeclareMathOperator{\id}{id}
\DeclareMathOperator{\Var}{Var}
\DeclareMathOperator{\Cov}{Cov}
\DeclareMathOperator{\RV}{RV}
\DeclareMathOperator{\rank}{rank}

%% pair delimiter
\DeclarePairedDelimiter{\abs}{\lvert}{\rvert}
\DeclarePairedDelimiter{\inner}{\langle}{\rangle}
\DeclarePairedDelimiter{\tuple}{(}{)}
\DeclarePairedDelimiter{\bracket}{[}{]}
\DeclarePairedDelimiter{\set}{\{}{\}}
\DeclarePairedDelimiter{\norm}{\lVert}{\rVert}

%% theorems
\newtheorem{axiom}{Axiom}
\newtheorem{definition}{Definition}
\newtheorem{theorem}{Theorem}
\newtheorem{proposition}{Proposition}
\newtheorem{corollary}{Corollary}
\newtheorem{lemma}{Lemma}
\newtheorem{remark}{Remark}
\newtheorem{claim}{Claim}
\newtheorem{problem}{Problem}
\newtheorem{assumption}{Assumption}
\newtheorem{example}{Example}
\newtheorem{exercise}{Exercise}

%% empty set
\let\oldemptyset\emptyset
\let\emptyset\varnothing

\newcommand\eps{\epsilon}

% mathcal symbols
\newcommand\Tau{\mathcal{T}}
\newcommand\Ball{\mathcal{B}}
\newcommand\Sphere{\mathcal{S}}
\newcommand\bigO{\mathcal{O}}
\newcommand\Power{\mathcal{P}}
\newcommand\Str{\mathcal{S}}


% mathbb symbols
\usepackage{mathrsfs}
\newcommand\N{\mathbb{N}}
\newcommand\Z{\mathbb{Z}}
\newcommand\Q{\mathbb{Q}}
\newcommand\R{\mathbb{R}}
\newcommand\C{\mathbb{C}}
\newcommand\F{\mathbb{F}}
\newcommand\T{\mathbb{T}}
\newcommand\Exp{\mathbb{E}}

% mathrsfs symbols
\newcommand\Borel{\mathscr{B}}

% algorithm
\usepackage{algorithm}
\usepackage{algpseudocode}

% longproof
\newenvironment{longproof}[1][\proofname]{%
  \begin{proof}[#1]$ $\par\nobreak\ignorespaces
}{%
  \end{proof}
}


% for (i) enumerate
% \begin{enumerate}[label=(\roman*)]
%   \item First item
%   \item Second item
%   \item Third item
% \end{enumerate}
\usepackage{enumitem}

% insert url by \url{}
\usepackage{hyperref}

% margin
\usepackage{geometry}
\geometry{
a4paper,
total={190mm,257mm},
left=10mm,
top=20mm,
}


\title{
    On the set function in $\R^n$ \\
    \large \textit{the ideas of these proofs below are from MA5205 Graduate Analysis I instructed by Dr. Chua Seng Kee at National University of Singapore}
}
\author{Nguyen Ngoc Khanh}
\date{October 2023}

\begin{document}

\maketitle

In $\R^n$, let $u$ be a set function on $\R^n$ which is a non-negative function on a subcollection of the collection of all subsets of $\R^n$, i.e. $u: \mathcal{U} \to [0, +\infty]$, $\mathcal{U} \subseteq \Power(\R^n)$. 

We will construct the smallest set of assumptions on $u$ such that the following limit converges for almost every $x$ in a bounded outer measure $X$, i.e. $|X| < +\infty$
\[
    Du(x) = \lim_{B \searrow x} \frac{u(B)}{|B|} 
\]
where $B \searrow x$ denotes the contraction of $B$ into a singleton $x$, and $|B|$ denotes the Lebesgue measure of $B$.

\begin{assumption}[Closed balls]
    $\overline{\Ball_r(x)} \in \mathcal{U}$ for all $x \in \R^n$ and $r > 0$
\end{assumption}

Consider a simpler version where $B$ is a closed ball of radius $r > 0$ centered at $x$, i.e. $B = \overline{\Ball_r(x)}$. Define $\limsup$, $\liminf$ as follows
\begin{itemize}
    \item $\overline{D}u(x) = \limsup_{B \searrow x} \frac{u(B)}{|B|} = \lim_{r \to 0^+} \left( \sup_{t \in (0, r)} \frac{u(B_t(x))} {|B_t(x)|} \right)$
    \item $\underline{D}u(x) = \liminf_{B \searrow x} \frac{u(B)}{|B|} = \lim_{r \to 0^+} \left( \inf_{t \in (0, r)} \frac{u(B_t(x))} {|B_t(x)|} \right)$
\end{itemize}

The necessary and sufficient condition for the existence of $Du(x)$ is $\overline{D}u(x) = \underline{D}u(x)$. If $\overline{D}u(x) > \underline{D}u(x)$, let $\alpha, \beta \in \Q$ such that $\overline{D}u(x) > \alpha > \beta > \underline{D}u(x) \geq 0$. Then $Du(x)$ exists for almost every $x \in X$ if 
\[
    A_{\alpha \beta} = \{x \in X: \overline{D}u(x) > \alpha > \beta > \underline{D}u(x)\}
\]
has measure zero for all $\alpha, \beta \in \Q$. If $x \in A_{\alpha \beta}$, then given any $\delta > 0$, there exist balls $B^{(\alpha)}, B^{(\beta)}$ centered at $x$ with radius less than $\delta$ such that $\frac{u(B^{(\alpha)})}{|B^{(\alpha)}|} > \alpha > \beta > \frac{u(B^{(\beta)})}{|B^{(\beta)}|}$. That is, the collections $\{B^{(\alpha)}\}$ and $\{B^{(\beta)}\}$ each cover $A_{\alpha \beta}$ finely.

\begin{lemma}[Vitali Covering Theorem for Lebesgue measure]
    A collection $\mathcal{V}$ of closed balls is said to cover a set $E$ finely if for every $x \in E$ and $\eta > 0$, there is a ball $B$ containing $x$ whose radius is less than $\eta$. Suppose $E$ is covered finely by a collection $\mathcal{V}$ of balls such that $0 < \mu_e(E) < +\infty$ \footnote{if $\mu_e(E) = 0$, the result is immediate}. Then, given any $\epsilon > 0$, there is a countable collection $M$ of disjoint balls such that
    \[
        \mu\left(E \setminus \bigcup_{B_k \in M} B_k \right) = 0 \text{ and } \sum_{B_k \in M} \mu(B_k) < \mu_e(E) + \epsilon
    \]
\end{lemma}

\begin{claim}
\label{claim:1}
Given any open ball $U$ contains a subset $E \subseteq A_{\alpha \beta}$, the $u(\overline{U}) \geq \alpha |E|_e$
\end{claim}

Given any open ball $U$ contains a subset $E \subseteq A_{\alpha \beta}$, since $U$ is open, for each $x \in E$, we can find a subcollection of balls $\{B^{(\alpha)}_U\} \subseteq \{B^{(\alpha)} \}$ that finely covers $E$ and is contained in $U$. By Vitali Covering Theorem for Lebesgue measure on finite outer measure $E$ with $\{B^{(\alpha)}_U \}$, given any $\eta > 0$, there exists a finite collection of pairwise disjoint balls $\{B_i\}_{i=1}^N$ contained in $U$ such that $\left|E \setminus \bigcup_{i=1}^N B_i \right|_e < \eta / \alpha$ and $u(B_i) > \alpha |B_i|$

\begin{assumption}[Additivity]
    $A, B \in \mathcal{U} \text{ and } A \cap B = \emptyset \implies A \cup B \in \mathcal{U} \text{ and } u(A \cup B) = u(A) + u(B)$
\end{assumption}

\begin{assumption}[Subset]
    $A, B \in \mathcal{U} \text{ and } A \subseteq B \implies u(A) \leq u(B)$  
\end{assumption}

Then,
\begin{align*}
    u(\overline{U})
    &\geq u \left(\bigcup_{i=1}^N B_i \right) &\text{(subset)} \\
    &= \sum_{i=1}^N u(B_i) &\text{(additivity)} \\
    &> \alpha \sum_{i=1}^N |B_i| &\text{($B_i \in \{B^{(\alpha)}_U\} \subseteq \{B^{(\alpha)} \}$)}\\
    &= \alpha \left| \bigcup_{i=1}^N B_i \right| &\text{(additivity of Lebesgue measure)} \\
    &= \alpha \left| E \cap \bigcup_{i=1}^N B_i \right|_e &\text{(subset of Lebesgue measure)}\\
    &= \alpha \left( |E|_e - \left| E \setminus \bigcup_{i=1}^N B_i \right|_e \right) &\text{(Caratheodory condition)}\\
    &= \alpha |E|_e - \eta &\text{(Vitali Covering Theorem for Lebesgue measure)}\\
\end{align*}

As the inequality is true for all $\eta > 0$, $u(\overline{U}) \geq \alpha|E|_e$


By Vitali Covering Theorem for Lebesgue measure on finite outer measure $A_{\alpha \beta}$ ($|A_{\alpha \beta}|_e \leq |X|_e < +\infty$) with $\{B^{(\beta)} \}$, given any $\epsilon > 0$, there exists a finite collection of pairwise disjoint balls $\{B_i \}_{i=1}^N$ such that $\left| A_{\alpha \beta} \setminus \bigcup_{i=1}^N B_i \right|_e < \epsilon$, $\sum_{i=1}^N |B_i| < |A_{\alpha \beta}|_e + \epsilon$, and $u(B_i) < \beta|B_i|$

Then
\begin{align*}
    |A_{\alpha \beta}|_e + \epsilon
    &> \sum_{i=1}^N |B_i| &\text{(Vitali Covering Theorem for Lebesgue measure)}\\
    &> \frac{1}{\beta} \sum_{i=1}^N \mu(B_i) &\text{($B_i \in \{B^{(\beta)}_U\} \subseteq \{B^{(\beta)} \}$)}\\
    &\geq \frac{\alpha}{\beta} \sum_{i=1}^N |B_i^o \cap A_{\alpha \beta}|_e &\text{(by claim $\ref{claim:1}$ as open ball $B_i^o$ contains $B_i^o \cap A_{\alpha \beta} \subseteq A_{\alpha \beta}$)} \\
    &\geq \frac{\alpha}{\beta} \left| A_{\alpha \beta} \cap \bigcup_{i=1}^N B_i^o\right|_e &\text{(subadditivity of Lebesgue outer measure)} \\
    &\geq \frac{\alpha}{\beta} \left| A_{\alpha \beta} \cap \bigcup_{i=1}^N B_i\right|_e &\text{($|A \cap B|_e \leq |A \cap B^o|_e + |A \cap B^b|_e$ \footnote{$B^b$ denotes the boundary of $B$})} \\
    &\geq \frac{\alpha}{\beta} \left( |A_{\alpha \beta}|_e -  \left| A_{\alpha \beta} \setminus \bigcup_{i=1}^N B_i\right|_e \right) &\text{(Caratheodory condition)} \\
    &> \frac{\alpha}{\beta} |A_{\alpha \beta}|_e - \frac{\alpha}{\beta}\epsilon &\text{(Vitali Covering Theorem for Lebesgue measure)}\\
\end{align*}

Send $\epsilon$ to $0^+$ implies $|A_{\alpha \beta}| \geq \frac{\alpha}{\beta} |A_{\alpha \beta}|$. Hence $|A_{\alpha \beta}| = 0$

\begin{theorem}
    \label{theorem:1}
    Given a set function $u: \mathcal{U} \to [0, +\infty]$ defined for all closed balls $\overline{\Ball_r(x)}, r > 0$ that is equipped with (1) additivity and (2) subset, Then
    \[
        Du(x) = \lim_{B \searrow x} \frac{u(B)}{|B|} 
    \]
    is defined for almost every $x \in X$ where $X \subseteq \R^n$ is any finite outer measure set and $B \searrow x$ denotes the contraction of closed balls $\overline{\Ball_r(x)}, r > 0$ into a singleton $x$
\end{theorem}

Theorem \ref{theorem:1} is true for every $X_r = \Ball_r(0), r \in \N$, then it is also true for $\R^n$.

\begin{corollary}
    Given a set function $u: \mathcal{U} \to [0, +\infty]$ defined for all closed balls $\overline{\Ball_r(x)}, r > 0$ that is equipped with (1) additivity and (2) subset, Then
    \[
        Du(x) = \lim_{B \searrow x} \frac{u(B)}{|B|} 
    \]
    is defined for almost every $x \in \R^n$ set where $B \searrow x$ denotes the contraction of closed balls $\overline{\Ball_r(x)}, r > 0$ into a singleton $x$
\end{corollary}

\begin{remark}
    Some remarks
    \begin{itemize}
        \item If $u$ is the Lebesgue measure, then $Du(x) = 1$ everywhere.
        \item If $u$ is any positive measure on $\R^n$, $Du(x)$ is defined everywhere.
        \item If $f: \R^n \to [0, +\infty]$ is Lebesgue measurable, let $u(E) = \int_E f$, then we have the Lebesgue Differentiation Theorem
    \end{itemize}
\end{remark}

\begin{definition}[Absolute Continuous Set Function]
    A set function $u: \mathcal{U} \to [0, +\infty]$ is defined to be absolute continous if given any $\epsilon > 0$, there exists a $\delta > 0$ such that if $E \in \mathcal{U}$, $|E| < \delta$, then $u(E) < \epsilon$
\end{definition}

\begin{lemma}[Theorem 7.28]
    If $u: \mathcal{U} \to [0, +\infty]$ is an absolute continuous set function, $E \in \mathcal{U}$ with $|E| < +\infty$ and $\lim_{B \searrow x} \frac{u(B)}{|B|} = 0$ for every $x \in E$, then $u(E) = 0$
\end{lemma}

$u: \mathcal{U} \to [0, +\infty]$ is absolute continous, that is, given any $\epsilon > 0$, there exists a $\delta > 0$ such that $E \in \mathcal{U}, |E| < \delta$ then $u(E) < \epsilon$

For each $x \in E$, $\lim_{B \searrow x} \frac{u(B)}{|B|} = 0$, that is, given any $\epsilon > 0$, then there exists a $\delta > 0$ such that for all ball $B$ with radius less than $\delta$, i.e. $r(B) < \delta$, then $u(B) < \epsilon |B|$. Fix $\epsilon > 0$, there is a collection of balls $\{B^{(\epsilon)} \}$ that finely covers $E$ and $u(B) < \epsilon |B|$ for all $B \in \{B^{(\epsilon)} \}$. By Vitali Covering Theorem for Lebesgue measure on finite measure $E$ with $\{B^{(\epsilon)} \}$, given any $\epsilon, \delta > 0$, there exists a finite collection of pairwise disjoint balls $\{B_i \}_{i=1}^N$ such that $|E \setminus \bigcup_{i=1}^N B_i| < \delta$ and $\sum_{i=1}^N |B_i| < |E| + \epsilon$. Hence,

\begin{align*}
    u(E)
    &= u\left(E \setminus \bigcup_{i=1}^N B_i \right) + u\left(E \cap \bigcup_{i=1}^N B_i \right) &\text{$\left( \text{additivity, suppose } E \setminus \bigcup_{i=1}^N B_i, E \cap \bigcup_{i=1}^N B_i \in \mathcal{U} \right)$} \\
    &\leq \epsilon + u\left(E \cap \bigcup_{i=1}^N B_i \right) &\text{(Vitali Covering Theorem for Lebesgue measure, absolute continuous)}\\
    &\leq \epsilon + u\left(\bigcup_{i=1}^N B_i \right) &\text{(subset)}\\
    &= \epsilon + \sum_{i=1}^N u(B_i) &\text{(additivity)} \\
    &\leq \epsilon + \epsilon \sum_{i=1}^N |B_i| &\text{($B_i \in \{B^{(\epsilon)} \}$)} \\
    &\leq \epsilon + \epsilon (|E| + \epsilon) &\text{(Vitali Covering Theorem for Lebesgue measure)}
\end{align*}

As $|E| < +\infty$, send $\epsilon$ to $0^+$ implies $u(E) = 0$

\begin{remark}
    Put $u([a, b]) = |f(b) - f(a)|$, $\lim_{B \searrow x} \frac{u(B)}{|B|} = 0$ implies $f$ being singular, recover \textbf{Theorem 7.28}    
\end{remark}


\begin{corollary}
    \label{corollary:2}
    If $u: \mathcal{U} \to [0, +\infty]$ is an absolute continuous set function, $E \in \mathcal{U}$ with $|E| < +\infty$ and $\lim_{B \searrow x} \frac{u(B)}{|B|} = 0$ for \textbf{almost} every $x \in E$, then $u(E) = 0$
\end{corollary}

Let $F \subseteq E$ with $|F| = 0$ and $\lim_{B \searrow x} \frac{u(B)}{|B|} = 0$ in $E \setminus F$. $u(E) = u(E \setminus F) + u(F) = u(F)$. By absolute continuity of $u$, $u(E) = u(F) = 0$





\begin{theorem}[Theorem 7.29]
    If $u: \mathcal{U} \to [0, +\infty]$ is an absolute continuous set function. Given $E \in \mathcal{U}$ with $|E|_e < +\infty$, given necessary conditions (additivity, subset) then $u'(x) = \lim_{B \searrow x} \frac{u(B)}{|B|}$ is defined almost every $x \in E$. Furthermore,
    \[
        u(E) = \int_E u'
    \]
    for all $E \in \mathcal{U} \cap \mathcal{M}(\R^n)$.
\end{theorem}

$u'$ is defined almost every $x$ in a finite outer measure $E$ is immediate from theorem \ref{theorem:1}. For the second part, let $v(E) = u(E) - \int_E u'$ for all $E \in \mathcal{U} \cap \mathcal{M}(\R^n)$, then 

\[
    \lim_{B \searrow x} \frac{v(E)}{|E|} = \lim_{B \searrow x} \frac{u(E) - \int_E u'}{|E|} = \lim_{B \searrow x} \frac{u(E)}{|E|} - \lim_{B \searrow x} \frac{\int_E u'}{|E|} = 0 \text{ for almost every $x \in E$}
\]

By corollary \ref{corollary:2}, $v(E) = 0$ almost every where.

\begin{remark}
    Put $u([a, b]) = |f(b) - f(a)|$, $f' = u' = \lim_{B \searrow x} \frac{u(B)}{|B|}$, recover \textbf{Theorem 7.29}    
\end{remark}

\end{document}
