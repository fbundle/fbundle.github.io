\documentclass{article}
\usepackage{graphicx} % Required for inserting images

% header

%% natbib
\usepackage{natbib}
\bibliographystyle{plain}

%% comment
\usepackage{comment}

% indent the first paragraph
\usepackage{indentfirst}


%% math package
\usepackage{amsfonts}
\usepackage{amsmath}
\usepackage{amssymb}


%% operator
\DeclareMathOperator{\tr}{tr}
\DeclareMathOperator{\diag}{diag}
\DeclareMathOperator{\sign}{sign}
\DeclareMathOperator{\grad}{grad}
\DeclareMathOperator{\curl}{curl}
\DeclareMathOperator{\Div}{div}

%% theorems
\newtheorem{axiom}{Axiom}
\newtheorem{definition}{Definition}
\newtheorem{theorem}{Theorem}
\newtheorem{proposition}{Proposition}
\newtheorem{corollary}{Corollary}
\newtheorem{lemma}{Lemma}
\newtheorem{remark}{Remark}
\newtheorem{claim}{Claim}
\newtheorem{problem}{Problem}

%% empty set
\let\oldemptyset\emptyset
\let\emptyset\varnothing

% mathcal symbols
\newcommand\Tau{\mathcal{T}}
\newcommand\Ball{\mathcal{B}}

% mathbb symbols
\newcommand\N{\mathbb{N}}
\newcommand\Z{\mathbb{Z}}
\newcommand\Q{\mathbb{Q}}
\newcommand\R{\mathbb{R}}

\title{sheaf\_iso}
\author{Khanh Nguyen}
\date{November 2024}

\begin{document}

\begin{lemma}
    Let $E \to X$ and $F \to X$ be vector bundles over $X$. Then there is a sheaf isomorphism
    $$
        \tau: \mathcal{E}(E) \otimes_\mathcal{E} \mathcal{E}(F) \to \mathcal{E}(E \otimes F)
    $$

    where $\mathcal{E}(E)$ and $\mathcal{E}(F)$ are sheaves induced from sections of vector bundles.
\begin{proof}
    For every $x \in X$, pick $U \subseteq X$ containing $x$ small enough such that $E\vert_U \to U$ and $F\vert_U \to U$ are trivial bundles. Let 
    $$
        \Tau(-) = \mathcal{E}(E)(-) \otimes_{\mathcal{E}(-)} \mathcal{E}(F)(-)
    $$

    be the tensor product of presheaves. We will construct the map $t$ and show that the diagram below commutes
    \begin{center}
\begin{tikzcd}
\Tau(U) \arrow[d, "r^U_V"] \arrow[r, "t"] & \mathcal{E}(E \otimes F)(U) \arrow[d, "r^U_V"] \\
\Tau(V) \arrow[r, "t"]                    & \mathcal{E}(E \otimes F)(V)                   
\end{tikzcd}
    \end{center}

    Let $e = \set{e_1, e_2, ..., e_m}$ and $f = \set{f_1, f_2, ..., f_n}$ be frames of $E$ and $F$ on $U$. Then, every element $\xi \in \Tau(U)$ can be written as
    $$
        \xi = \sum_{i=1}^m \sum_{j=1}^n \xi_{ij} (e_i \otimes_{\mathcal{E}(U)} f_j)
    $$

    where $\xi_{ij} \in \mathcal{E}(U)$. And every element $\eta \in \mathcal{E}(E \otimes F)(U)$ can be written as
    $$
        \eta(x) = \sum_{i=1}^m \sum_{j=1}^n \eta_{ij}(x) (e_i(x) \otimes f_i(x))
    $$

    where $\eta_{ij} \in \mathcal{E}(U)$. Hence, there exists a natural isomorphism of sheaves from $\Tau\vert_U = \Tau^{sh}\vert_U$ to $\mathcal{E}(E \otimes F)\vert_U$ defined on $U \subseteq X$.

\begin{lemma}
    If $\mathcal{F}$ and $\mathcal{G}$ be two sheaves on $X$ and there exists an open cover $\set{U_i}_{i \in I}$ for $X$ such that $\tau_i: \mathcal{F}\vert_{U_i} \to \mathcal{G}\vert_{U_i}$ is an isomorphism of sheaves for all $i \in I$, then there exists an isomorphism of sheaves $\tau: \mathcal{F} \to \mathcal{G}$
\begin{proof}
    Let $V$ be open, without loss of generality, we can assume that there exists an open cover $\set{U_i}_{i \in I}$ for $V$ such that for each $U_i \in \set{U_i}_{i \in I}$, $\tau_i: \mathcal{F}\vert_{U_i} \to \mathcal{G}\vert_{U_i}$ is an isomorphism of sheaves. Let $U \in \set{U_i}$,
    \begin{center}
\begin{tikzcd}
\mathcal{F}(V) \arrow[r, "r^V_U"] \arrow[d, "\tau_V"', dashed] & \mathcal{F}(U) \arrow[d, "\tau_U"] \\
\mathcal{G}(V) \arrow[r, "r^V_U"]                              & \mathcal{G}(U)                    
\end{tikzcd}
    \end{center}

    Let $f \in \mathcal{F}(U)$, define
    $$
        g_U = \tau_U r^V_U f
    $$

    For any $U_i, U_j \in \set{U_i}_{i \in I}$ with $U_i \cap U_j \neq \emptyset$, the diagram below commutes
    \begin{center}
\begin{tikzcd}
\mathcal{F}(V) \arrow[rr, "r^V_{U_i}"] \arrow[rd, "r^V_{U_j}"] \arrow[ddd, "\tau_V"', dashed] &                                                                                 & \mathcal{F}(U_i) \arrow[rd, "r^{U_i}_{U_i \cap U_j}"] \arrow[ddd, "\tau_{U_i}"] &                                                              \\
                                                                                              & \mathcal{F}(U_j) \arrow[ddd, "\tau_{U_j}"] \arrow[rr, "r^{U_j}_{U_i \cap U_j}"] &                                                                                 & \mathcal{F}(U_i \cap U_j) \arrow[ddd, "\tau_{U_i \cap U_j}"] \\
                                                                                              &                                                                                 &                                                                                 &                                                              \\
\mathcal{G}(V) \arrow[rd] \arrow[rr]                                                          &                                                                                 & \mathcal{G}(U_i) \arrow[rd, "r^{U_i}_{U_i \cap U_j}"]                           &                                                              \\
                                                                                              & \mathcal{G}(U_j) \arrow[rr, "r^{U_j}_{U_i \cap U_j}"]                           &                                                                                 & \mathcal{G}(U_i \cap U_j)                                   
\end{tikzcd}
    \end{center}

    Hence, by commutativity, we have
    $$
        r^{U_i}_{U_i \cap U_j} g_{U_i} = r^{U_j}_{U_i \cap U_j} g_{U_j} 
    $$

    By definition of sheaf, there exists a unique map $\tau_V: \mathcal{F}(V) \to \mathcal{G}(V)$ that makes the diagram commutes. Similarly, we can construct a unique map $\tau_V^{-1}: \mathcal{G}(V) \to \mathcal{F}(V)$. It can be verified that $\tau$ defines a natural isomorphism between two functors $\mathcal{F}$ and $\mathcal{G}$
\end{proof}
\end{lemma}

    Back to the proof, use the lemma, we can construct a natural isomorphism
    $$
        \tau: \mathcal{E}(E) \otimes_\mathcal{E} \mathcal{E}(F) \to \mathcal{E}(E \otimes F)
    $$ 
\end{proof}
\end{lemma}

\end{document}
