\documentclass{article}

% Language setting
% Replace `english' with e.g. `spanish' to change the document language
\usepackage[english]{babel}

% Set page size and margins
% Replace `letterpaper' with`a4paper' for UK/EU standard size
\usepackage[letterpaper,top=2cm,bottom=2cm,left=3cm,right=3cm,marginparwidth=1.75cm]{geometry}

% Useful packages
\usepackage{amsmath}
\usepackage{amssymb}


\newtheorem{definition}{Definition}


\title{Two-Sum}
\author{Khanh Nguyen}

\begin{document}
\maketitle


\section{Preliminaries}

Let first define some data structures

Denote $[a, b] = \{a, a+1, a+2, ..., b-1, b\}$ for $a < b \in \mathbb{N}$ be the set of all natural numbers between $a$ and $b$ (inclusive)


\begin{definition}[finite length array]
An array of length $n$ of element from $S$ is defined as a function $f: [1, n] \to S$
\end{definition}

The set all all arrays of length $n$ from $S$ is $S^{[1, n]}$


The set of all arrays from $S$ is $\bigcap_{n=0}^{\infty} S^{\mathbb{N} \cap [1, n]}$ or $S^*$ for short.

\begin{definition}[sub-array]
Let $a \leq b \in \mathbb{N} \cap [1, n]$, the sub-array of $f$ with respect to bounds $(a, b)$ denoted as $f_{(a, b)}: \mathbb{N} \cap [1, b-a+1] \to S$ is defined by $f_{(a, b)}(i) = f(a+i-1) \forall i \in \mathbb{N} \cap [1, b-a+1]$
\end{definition}

Note that $f_{(1, n)} = f$

\begin{definition}[increasing array]
An array $f: \mathbb{N} \cap [1, n] \to \mathbb{N}$ is increasing if and only if $\forall i < j \in \mathbb{N} \cap [1, n], f(i) \leq f(j)$
\end{definition}

\begin{definition}[array membership]
If $\exists i \in \mathbb{N}, x = f(i)$, we write $x \in f$
\end{definition}

\begin{definition}[array length]
The function $len : S^* \to \mathbb{N}$ returns the length of an array
\end{definition}

\section{Two-Sum}

Given an increasing array $f$ of length $n$ and a number $s$, if there exists two numbers $a, b \in f$ such as $a + b = s$, define a function $ts: \mathbb{N} \times \mathbb{N}^* \to \mathbb{N} \times \mathbb{N}$ as $ts(s, f) = (a, b)$ where $a \leq b \in f$ and $a + b = s$

One possible definition of $ts$ is as follow:

\begin{equation}
    ts(s, f) = 
    \begin{cases}
    (f(1), f(n)) &\text{if $f(1) + f(n) = s$} \\
    ts(s, f_{(2, n)}) &\text{if $f(1) + f(n) < s$} \\
    ts(s, f_{(1, n-1)}) &\text{if $s < f(1) + f(n)$} \\
    \end{cases}
\end{equation}

\end{document}